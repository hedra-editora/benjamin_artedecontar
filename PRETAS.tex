%!TEX root=./LIVRO.tex 

\textbf{Walter Benjamim} é um filósofo, crítico literário, tradutor (traduziu, entre outros, Baudelaire, Proust e Balzac) e também ficcionista alemão. Estudou filosofia num ambiente dominado pelo neokantismo, em Berlim, Freiburg, Munique e Berna, onde defendeu tese de doutorado sobre os primeiros românticos alemães. Durante o seu exílio em Paris, nos anos trinta, foi ligado ao Institut für Sozialforschung, embrião da chamada Escola de Frankfurt. Entre seus interlocutores e amigos, encontram-se personalidades marcantes do século XX como Theodor W. Adorno, Hannah Arendt, Bertolt Brecht e Gershom Scholem.

\textbf{\titulo} propõe uma nova tradução anotada do clássico ensaio no qual Walter Benjamin esboça a figura do contador de histórias a partir de um comentário crítico do contista russo Nikolai Leskov; reúne também a pouco conhecida produção ficcional do próprio ensaísta, trazendo alguns contos completamente inéditos em português. O volume inclui ainda peças que Benjamin produziu para o rádio, deslocando a arte tradicional de contar para a cena moderna, e textos híbridos, onde o crítico faz obra de ficção ou o contador de histórias de repente filosofa. Para aprofundar a reflexão que norteia a organização deste livro, o posfácio examina o hibridismo entre imagem ficcional e pensamento filosófico, que atravessa a obra benjaminiana, interrogando-o à luz do confronto entre tradição e modernidade, característico de sua arte de contar histórias.

\textbf{Patrícia Lavelle} é professora do Departamento de Letras da PUC-Rio, atuando no Programa de Pós-graduação em Literatura, Cultura e Contemporaneidade. É também pesquisadora associada ao Centre Georg Simmel da EHESS-Paris,  instituição na qual defendeu doutorado em filosofia e deu aulas. Sua tese foi publicada na França: Religion et histoire. Sur le concept d’expérience chez Walter Benjamin. Paris: Cerf, col. Passages, 2008. Entre outros livros e números de revista, organizou também na França uma edição comentada da Infância em Berlim por volta de 1900 (Paris: L’Herne, 2012). Para a tradicional série “Cahier de l’Herne”, editou o volume Walter Benjamin, reunindo cartas, textos inéditos do autor e 39 ensaios de especialistas (2013). 

\textbf{Georg Otte} é mestre pela Universidade de Trier e doutor pela Universidade Federal de Minas Gerais, em cuja Faculdade de Letras é Professor Titular. Realizou pós-doutorado na Universidade Humboldt de Berlim e no, também berlinense, Centro de Pesquisas Literárias e Culturais (ZFL). Tem experiência na área de Letras, com ênfase em Teoria Literária, dedicando-se ao estudo dos seguintes temas: Walter Benjamin, Hans Blumenberg, Estética e Cultura, Mito e Modernidade.

\textbf{Marcelo Backes} é escritor e tradutor, autor dos romances \emph{O último minuto} (Companhia das Letras, 2013) e \emph{A casa cai} (Companhia das Letras, 2015), entre outras obras. Doutor em Germanística e Romanística pela Universidade de Freiburg, na Alemanha, e professor da Casa do Saber do Rio de Janeiro, verteu ao português mais de 30 obras da literatura alemã, entre clássicos como Heinrich von Kleist e contemporâneos como Ingo Schulze. Organiza a publicação das obras de Arthur Schnitzler pela Editora Record e coordena a coleção de clássicos Fanfarrões, Libertinas \& Outros Heróis pela Civilização Brasileira.

\textbf{Coleção W.Benjamim} é um projeto acadêmico-editorial que envolve pesquisa, tradução e publicação de obras e textos seletos desse importante filósofo, crítico literário e historiador da cultura judeu-alemão, em volumes organizados por estudiosos versados em diferentes aspectos de sua obra, vida e pensamento.

\textbf{Amon Pinho} é doutor pela Universidade de São Paulo e pós-doutor pela Universidade de Lisboa, em cujo Centro de Filosofia é Pesquisador Associado. Professor Associado na Universidade Federal de Uberlândia, tem experiência nas áreas de História e Filosofia, com ênfase em Teoria e Filosofia da História, História da Historiografia (geral e brasileira) e História da Filosofia Contemporânea.

\textbf{Francisco De Ambrosis Pinheiro Machado} é Professor de Filosofia na Universidade Federal de São Paulo (UNIFESP). Fez graduação na Universidade de São Paulo e doutorado pela Universidade Ludwig Maximilian de Munique. Pesquisa sobre filosofia da história, teoria crítica da cultura e estética na obra dos autores vinculados à Teoria Crítica, sobretudo, Walter Benjamin.


