%!TEX root=./LIVRO.tex 

\textbf{Walter Benjamin} (1892-1940) foi um filósofo, crítico literário, tradutor (de Baudelaire, Proust e Balzac, entre outros) e também um ficcionista alemão. Estudou filosofia num ambiente dominado pelo neokantismo, em Berlim, Freiburg, Munique e Berna, onde defendeu tese de doutorado sobre os primeiros românticos alemães. Durante o seu exílio em Paris, nos anos trinta, foi ligado ao Instituto de Pesquisa Social, embrião da chamada Escola de Frankfurt. Entre seus interlocutores e amigos, encontram"-se personalidades marcantes do século~\versal{XX} como Theodor W. Adorno, Hannah Arendt, Bertolt Brecht e Gershom Scholem.

\textbf{\titulo} propõe uma nova tradução anotada do clássico ensaio no qual Walter Benjamin esboça a figura do contador de histórias a partir de um comentário crítico do contista russo Nikolai Leskov; reúne também a pouco conhecida produção ficcional do próprio ensaísta, trazendo o conjunto de seus contos, alguns inéditos em português. O volume inclui ainda peças que Benjamin produziu para o rádio, deslocando a arte tradicional de contar histórias para a cena moderna, e textos híbridos, onde o crítico faz obra de ficção ou o contador de histórias filosofa.

\textbf{Patrícia Lavelle} é Professora do Departamento de Letras da \versal{PUC}"-Rio, atuando no Programa de Pós"-graduação em Literatura, Cultura e Contemporaneidade. É também Pesquisadora Associada à \versal{EHESS}"-Paris, onde defendeu doutorado em Filosofia e deu aulas. Sua tese foi publicada em livro: \emph{Religion et histoire. Sur le concept d’expérience chez Walter Benjamin.} Paris: Cerf, col. Passages, 2008. Entre outros volumes coletivos, organizou \emph{Walter Benjamin}. Paris: L’Herne, col. Cahiers de l’Herne, 2013. 

\textbf{Georg Otte} é Professor Titular na Faculdade de Letras da Universidade Federal de Minas Gerais. Tem experiência na área de Letras, com ênfase em Teoria Literária, dedicando"-se ao estudo dos seguintes temas: Walter Benjamin, Hans Blumenberg, Estética e Cultura, Mito e Modernidade.

\textbf{Marcelo Backes} é escritor e tradutor, doutor em Germanística e Romanística pela Universidade de Freiburg e professor na Casa do Saber do Rio de Janeiro. Verteu mais de 30 obras da literatura alemã. Na Hedra, publicou a sua tradução de \emph{Explosão: romance de etnologia}, de Hubert Fichte.

\textbf{Coleção Walter Benjamin} é um projeto acadêmico"-editorial que envolve pesquisa, tradução e publicação de obras e textos seletos desse importante filósofo, crítico literário e historiador da cultura judeu"-alemão, em volumes organizados por estudiosos versados em diferentes aspectos de sua obra, vida e pensamento. 

\textbf{Amon Pinho} é Professor Associado na Universidade Federal de Uberlândia e Pesquisador Associado no Centro de Filosofia da Universidade de Lisboa. Tem experiência nas áreas de História e Filosofia, com ênfase em Teoria da História e História da Filosofia Contemporânea. Entre os seus temas de eleição, dedica"-se ao estudo da vida e obra de Walter Benjamin.

\textbf{Francisco De Ambrosis Pinheiro Machado} é Professor Associado na Escola de Filosofia, Letras e Ciências Humanas da Universidade Federal de São Paulo (\versal{UNIFESP}). Pesquisa sobre filosofia da história, teoria crítica da cultura e estética na obra dos autores vinculados à Teoria Crítica, sobretudo, Walter Benjamin.


