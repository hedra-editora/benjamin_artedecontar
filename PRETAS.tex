%!TEX root=./LIVRO.tex 

\textbf{Walter Benjamim} \lipsum[1]


\textbf{\titulo} \lipsum[2]

\textbf{Patrícia Lavelle} é professora do Departamento de Letras da PUC-Rio, associada ao Programa de Pós-graduação em Literatura, Cultura e Contemporaneidade. É também pesquisadora associada ao Centre Georg Simmel da Ecole des Hautes Etudes en Sciences Sociales (EHESS) e ao Fonds Ricœur (Institut Protestant de Théologie de Paris).
Em 2006, defendeu doutorado em filosofia na EHESS em Paris, onde morou entre 1999 e 2014. Sua tese, que investiga as fontes kantianas na obra de Walter Benjamin, foi publicada em forma de livro: Religion et histoire. Sur le concept d’expérience chez Walter Benjamin. Paris: Editions du Cerf, col. Passages, 2008. Em 2007, fez pós-doutorado no Walter Benjamin-Archiv de Berlim. Entre outros livros e números de revista, organizou na França uma edição comentada da primeira versão da Berliner Kindheit um 1900 (Paris: L’Herne, 2012). Para a tradicional série “Cahier de l’Herne”, editou o volume Walter Benjamin, obra de 391 páginas reunindo cartas, documentos biográficos, textos inéditos do autor e 39 ensaios de especialistas (Paris: L’Herne, 2013).
Tem artigos publicados em diversas revistas acadêmicas brasileiras e sua dissertação de mestrado, que recebeu prêmio comemorativo do Programa de Pós-graduação em História Social da Cultura da PUC-Rio, foi publicada pela Editora UFMG: Espelho distorcido. Imagens do individuo no Brasil oitocentista (Belo Horizonte, 2003). É também autora de uma coletânea de poemas (Bye bye Babel, Rio de Janeiro: 7Letras, 2018). 

\textbf{XXXX} \lipsum[4]

\textbf{Amon Pinho}...
\textbf{Francisco}…

\textbf{Coleção W.Benjamim} \lipsum[6]


