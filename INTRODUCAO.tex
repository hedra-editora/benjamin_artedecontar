\textsc{Apresentação}

\textsc{Coligir, traduzir, editar Walter} \textsc{Benjamin}

Pela casa editorial Hedra e por este \emph{A} \emph{arte de contar
histórias}, a Coleção Walter Benjamin principia \emph{hic et nunc}, aqui
e agora, a sua publicação. Seu propósito: somar-se às iniciativas que se
têm destacado pelo compromisso de contribuir efetivamente para um
apuramento crescente na recepção da obra e da vida de Walter Benedix
Schöenflies Benjamin (1892-1940). Seu projeto: ir-se compondo em volumes
organizados por estudiosos experimentados no pensar benjaminiano,
orientados invariavelmente pelo critério basilar de, em judiciosas
traduções, disponibilizar ao público leitor interessado obras ou textos
seletos de Walter Benjamin; os mais como os menos conhecidos, já
editados ou ainda inéditos no Brasil, não obstante, por certo, sempre
relevantes para um conhecimento matizado e substantivo do postumamente
aclamado pensador judeu-alemão.

Em \emph{A arte de contar histórias}, cuja organização e posfácio a
Patrícia Lavelle (PUC-Rio e EHESS) se devem, os leitores encontrarão,
logo à partida, uma nova tradução do célebre ensaio de Benjamin sobre o
escritor russo Nikolai Leskov (1831-1895). E na sequência dessas
reflexões acerca dos contadores de história e do declínio secular da sua
arte de contar -- \emph{pari passu} ao esgarçamento da completude de
sentido de que pode ser dotada a experiência (\emph{Erfahrung}) do mundo
e da vida --, certamente se surpreenderão com os contos redigidos por um
Benjamin ele-mesmo contador de histórias, suas narrativas radiofônicas,
seus contos enquanto crítica, isto é, enquanto exercício conjugado de
criticismo literário. Mas não ``só''. Em tais contos e narrativas,
igualmente se depararão com noções, temas e questões recorrentes na
ensaística benjaminiana mais notabilizada e que nessas páginas,
complementarmente, de forma esclarecedora e original se prolongam. De
noções como ``estado de exceção'' e ``presença de espírito'' ao tema da
fisiognomonia -- a arte de conhecer o caráter de um indivíduo, ou mesmo
de uma cidade, por meio da observação interpretativa das suas feições;
ou da questão do esfacelamento das sociedades tradicionais, do seu meio
artesanal e do \emph{taedium vitae} que o caracteriza -- em suas
articulações com o processo de enfraquecimento da faculdade de
intercambiar experiências nos quadros das ascendentes sociedades
industriais --, aos experimentos com haxixe e à embriaguez do surreal,
do mítico e do maravilhoso, nos limiares de uma existência em potencial
reencantada; apenas para pontuarmos alguns exemplos.

Como seria de se esperar em edições desta natureza, os textos que
compõem o presente volume foram todos vertidos dos originais em alemão
-- nas traduções realizadas por Georg Otte (UFMG), Marcelo Backes
(Uni-Freiburg) e Patrícia Lavelle --, sendo que no caso do ensaio sobre
Leskov, optou Patrícia Lavelle, logo quando este livro começou a tomar
forma, em 2014, por traduzir \emph{Erzähler} não mais por ``narrador'',
conforme até ali mormente ocorria, mas por ``contador de histórias'',
nisso acompanhando aspectos da recepção francófona (com
``\emph{conteur}'') e, poderíamos acrescentar, anglófona (com
``\emph{storyteller}'') do referido ensaio.

A esse respeito, observemos a título de breve nota que, embora cientes
das vantagens e limitações próprias a cada uma dessas duas alternativas
translacionais, seria de interesse notarmos as pertinências etimológica
-- desde \emph{Erzähler}\footnote{Para uma aproximação de caráter
  etimológico do substantivo ``\emph{Erzähler}'', requer-se inicialmente
  a consideração do verbo ``\emph{zählen}'', cujo significado primeiro é
  o de ``contar'' no sentido de ``fazer conta'', ``computar'',
  ``enumerar'', ``calcular''. E também a de termos a ele ligados, tais
  como ``\emph{Zähler}'' (``contador'', pessoa ou dispositivo),
  ``\emph{zahlen}'' (``pagar''), ``\emph{Zahler}'' (``pagador'') e
  ``\emph{Zahl}'' (``número'', ``algarismo''), todos pertencentes, de um
  modo ou de outro, a um campo semântico próprio ao universo, digamos,
  contabilístico. Conexo a esse campo semântico de que ``\emph{zählen}''
  é o núcleo, não obstante dele se demarcando por deslocamento, está o
  verbo ``\emph{erzählen}'', que também significa ``contar'', porém não
  mais por meio de números, e sim por meio de palavras, no sentido de
  contar ou relatar um enredo, um caso, uma história. De onde,
  finalmente, o sentido fundamental de ``\emph{Erzähler}'' como contador
  de enredos, de tramas, intrigas, casos, histórias -- factuais, míticas
  ou ficcionais.} -- e histórico-antropológica da opção por ``contador
de histórias'', bem como, e ainda mais, o que há de revelador no
seguinte trecho de um bilhete de Benjamin, escrito em francês e
destinado ao filósofo alemão, exilado como ele em Paris, Paul Ludwig
Landsberg (1901-1944), em missiva de dezembro de 1939, que portava duas
versões, a alemã e a francesa, do ensaio acerca de Leskov: ``Voilà `le
narrateur' (mais il faudrait bien plutôt traduire: Le
conteur)''.\footnote{``Eis `o narrador' (mas seria preciso de
  preferência traduzir: O contador de histórias)''. Cf. Walter Benjamin,
  \emph{Gesammelte Briefe}, vol. VI. Edição de Christoph Gödde e Henri
  Lonitz. Frankfurt a. M.: Suhrkamp, 2000, p. 367.}

Com o objetivo de melhor propiciar uma maior interação entre os
conteúdos do livro e o seu público leitor, foram apensas enquanto
aparato crítico aos escritos aqui coligidos, notas informativas e
explicativas da organizadora e dos tradutores deste volume, bem como dos
editores da Coleção Walter Benjamin, tendo igualmente cabido, no que
concerne aos editores da coleção, a dupla e cuidadosa tarefa de uma
revisão de tradução e preparação de originais, não apenas com a
finalidade de proporcionar ajustes e aperfeiçoamentos no volume em si,
como com a de estabelecer, desde já e projetivamente, uma uniformização
formal e terminológica (léxico-conceitual) entre os volumes que comporão
o conjunto desta iniciativa editorial.

A esse propósito, aliás, pensamos que de outro modo não poderia ser,
haja vista a aspiração que esta coleção despretensiosamente porta de
poder fazer jus à recepção brasileira, e também portuguesa, cada vez
mais apurada, maturada e rigorosa de que tem sido objeto a obra de
Walter Benjamin. Recepção no seio da qual, gradualmente, se foi
estabelecendo e fixando um vocabulário ou léxico em português dos termos
e conceitos benjaminianos, em atenta observância das particularidades do
universo terminológico próprio aos originais em alemão.

Coligir, traduzir e editar Walter Benjamin, sim, mas -- e a despeito de
incidentais imperfeições -- criteriosa e criticamente, sem deixar-se
levar por tentações de fito meramente mercadológico que dele e do
inconformismo característico de seu pensamento poderiam fazer nada mais
do que um reificado e domesticado produto de consumo cultural, inclusive
neste contexto ainda recente de entrada da obra do autor em domínio
público.

Às pessoas e instituições que de vária maneira contribuíram de modo
significativo para o acontecer deste livro e o inaugurar da coleção,
gostaríamos de manifestar, à guisa de conclusão, os nossos
agradecimentos. Ao Instituto Goethe, nas pessoas de Bethe Ferreira de
Souza (sucursal São Paulo) e Andreas Schmohl (\emph{Head Office}
Munique), pela subvenção às translações do alemão para o português,
dentro do Programa de Fomento à Tradução de Livros Alemães em Língua
Estrangeira. À Pró-Reitoria de Pesquisa e Pós-Graduação da Universidade
Federal de Uberlândia em articulação com o Programa de Pós-Graduação em
História, nas pessoas dos Professores Doutores Alexandre Walmott Borges
e Jean Luiz Neves Abreu, respectivamente, pelo apoio concedido à
presente edição, nos quadros do Programa de Auxílio Financeiro às
Atividades Científicas, de Pesquisa e Inovação Tecnológica. Por fim, à
Editora Hedra, na pessoa do seu diretor geral Jorge Sallum, pela
parceria, generosidade, sensibilidade e inteligência editoriais. Ao
abrigo desta casa editora cujo nome remete sempre à beleza reverdecente
das heras, a Coleção Walter Benjamin foi concebida e gestada. Vemos
agora o seu primeiro rebento nascer.

Os editores da coleção,

Amon Pinho

(Universidade Federal de Uberlândia / Centro de Filosofia da
Universidade de Lisboa)

Francisco De Ambrosis Pinheiro Machado

(Universidade Federal de São Paulo)
