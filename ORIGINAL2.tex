%!TEX root=./LIVRO.tex 



\part{O contador de histórias no rádio}

\chapter{No minuto exato\,\footnote[*]{``Auf die Minute'', in \versal{GS} \versal{IV}-2, pp. 761-763.
  Tradução de Marcelo Backes. Texto publicado no \emph{Frankfurter
  Zeitung} em 6 de dezembro de 1934 com o pseudônimo de Detlef Holz.
  [\versal{N}. \versal{E}.]} }

Depois de meses tentando, acabei recebendo da direção de programação da
\versal{D}\ldots{} o encargo de entreter os ouvintes durante vinte minutos com um
relato sobre a minha especialidade, a atividade livreira. Caso meu papo
encontrasse eco, acenaram com a possibilidade de repetir regularmente os
referidos relatos. O chefe de departamento se mostrou amável o
suficiente para chamar minha atenção ao fato de que eram decisivos, além
da estrutura de tais considerações, o modo como eu as apresentaria.
``Principiantes'', disse ele, ``cometem o erro de acreditar que têm de
fazer sua apresentação diante de um público mais ou menos grande, que
apenas por acaso não é visível. Nada é mais errado do que isso. O
ouvinte de rádio é quase sempre um solitário, e mesmo que o senhor
excepcionalmente alcance alguns milhares, serão apenas milhares de
ouvintes solitários. O senhor tem de se comportar, portanto, como se
falasse com uma única pessoa -- ou com várias pessoas isoladas, se
preferir; mas de modo algum a muitas pessoas reunidas. Essa é a primeira
coisa. E há ainda uma segunda: o senhor deve se manter rigorosamente no
tempo previsto. Se não o fizer, nós precisaremos fazê-lo em seu lugar, e
o faremos simplesmente desligando os microfones sem qualquer piedade.
Qualquer atraso, até mesmo o mais mínimo, tem, como sabemos por
experiência, a tendência de se multiplicar no decorrer da programação.
Caso nossa intervenção não ocorra no momento exato, toda a nossa
programação acaba fugindo ao controle\ldots{} Portanto, peço ao senhor que
não se esqueça: apresentação descontraída! E terminar no minuto exato!''

Eu dei toda a atenção a essas orientações e as segui com cuidado; para
mim, muita coisa também dependia da gravação desse meu primeiro
programa. Em casa eu havia lido em voz alta o manuscrito, com o qual me
apresentei na estação à hora prevista, controlando com o relógio o tempo
que demorava. O locutor que me anunciaria recebeu-me com distinção, e eu
por certo poderia tomar como um sinal particular de sua confiança, o
fato de ele ter aberto mão de supervisionar minha estreia de uma cabine
incômoda. De seu anúncio até a despedida, eu era senhor de mim mesmo.
Pela primeira vez eu me encontrava em uma sala de transmissões moderna,
onde tudo está disposto para o mais absoluto conforto do locutor e o
desenvolvimento tranquilo de suas capacidades. Ele pode falar em pé, de
um púlpito, ou então sentar em uma das poltronas espaçosas, tem à sua
escolha as mais diferentes fontes de luz, pode até caminhar de um lado a
outro e levar consigo o microfone. Por fim, um relógio de mesa, cujo
mostrador não marca horas, mas apenas minutos, deixa claro ao locutor
quanto vale o instante naquela câmara vedada. Quando o mostrador
estivesse sobre o quarenta, eu tinha de estar pronto.

Eu lera pouco mais da metade do meu manuscrito quando voltei outra vez
os olhos para o relógio, no qual o ponteiro dos segundos descrevia o
mesmo círculo prescrito ao ponteiro dos minutos, mas com uma velocidade
sessenta vezes maior. Por acaso eu cometera um erro de direção quando
treinara em casa? Ou errara no tempo agora? Uma coisa era certa, dois
terços do meu tempo haviam se passado. Enquanto eu continuava lendo
palavra a palavra do meu discurso com o tom compromissado de antes,
procurava, febril e em silêncio, por uma saída. Só uma decisão ousada
poderia ajudar, parágrafos inteiros tinham de ser sacrificados,
considerações que levassem logo ao final precisavam ser improvisadas em
seu lugar. Afastar-me do meu texto não era desprovido de perigo. Mas não
me restava alternativa. Reuni todas as minhas forças, virei várias
páginas do manuscrito enquanto estendia uma frase mais longa e, por fim,
cheguei, feliz como um aviador em seu campo de voo, ao círculo de ideias
reunido no parágrafo final. Suspirando de alívio, reuni logo depois meus
papéis e, no êxtase da conquista, me afastei do púlpito para vestir meu
sobretudo.

Eis que agora o locutor que me anunciara deveria entrar. Mas ele se fez
esperar, e eu me virei para a porta. Nisso, meu olhar caiu outra vez
sobre o relógio. O ponteiro dos minutos mostrava o trinta e seis!\ldots{}
Ainda faltavam quatro minutos inteiros até o quarenta! O que eu antes
vislumbrara às pressas devia ter sido o \emph{ponteiro dos segundos}!
Agora eu também entendia o fato de o locutor não chegar. No mesmo
instante, porém, o silêncio que ainda há pouco me parecera tão benfazejo
me envolveu como se fosse uma rede. Naquela câmara, cujo funcionamento
era determinado pela técnica e pelo homem que imperava através dela, um
novo horror se estendeu sobre mim, que no entanto era aparentado daquele
mais antigo que nós conhecemos. Eu emprestei a mim mesmo meu ouvido, e
ao encontro dele ecoou de repente nada mais do que o próprio silêncio.
Mas este eu reconheci como sendo o silêncio da morte, que me derrubava
avassaladoramente naquele exato instante em milhares de ouvidos e
milhares de lares ao mesmo tempo.

Um medo indescritível tomou conta de mim, e logo em seguida uma
determinação selvagem. Salvar o que ainda poderia ser salvo, disse
comigo mesmo, e arranquei o manuscrito do bolso do sobretudo, selecionei
a primeira e melhor que encontrei entre as páginas que havia pulado e
recomecei a leitura com uma voz que para mim parecia ser sobrepujada
pelas batidas do coração. Era impossível exigir ideias de mim. E, uma
vez que o trecho do texto que eu havia encontrado era breve, estendi as
sílabas, fiz as vogais vibrarem suas asas, rolei os erres e incluí entre
as frases pausas repletas de reflexão. E assim cheguei ao final mais uma
vez -- desta vez o final correto. O locutor chegou e me dispensou,
diligente, como antes havia me recebido. Minha inquietude continuava, no
entanto. Quando por fim, no dia seguinte, encontrei um amigo o qual eu
sabia que me ouvira, perguntei de passagem sobre a sua impressão. ``Foi
muito bom'', disse ele. ``O problema é que os receptores sempre nos
deixam um pouco na mão. O meu mais uma vez simplesmente ficou fora do ar
durante um minuto.''

\chapter{Caspar Hauser\,\footnote[*]{``Caspar Hauser'', in \versal{GS} \versal{VII}-1,
  pp. 174-180. Tradução de Georg Otte. Escrita para um público
  infanto-juvenil, essa narrativa radiofônica foi difundida em 22 de
  novembro e 17 de dezembro de 1930, tendo o próprio Benjamin como
  locutor. [\versal{N}. \versal{E}.]} }
Hoje, para variar, vou contar para vocês simplesmente uma história. Três
coisas vou adiantar logo: primeiro, cada palavra dela é verdadeira;
segundo, ela é intrigante tanto para adultos quanto para crianças, e as
crianças a entendem tão bem quanto os adultos; terceiro, mesmo se o
personagem principal morre ao final dessa história, ela não tem um
verdadeiro fim. Em compensação, ela tem a vantagem de não ter acabado, e de
que, talvez, nós todos saberemos seu fim um dia.

Se eu começar a contá-la agora, vocês não devem pensar ``mas isso está
começando como qualquer outra história ilustrada para jovens''. Quem
começa a contar de forma tão detalhada e tão sossegada, no entanto, não
sou eu, mas o conselheiro secreto do tribunal de segunda instância,
Anselm von Feuerbach, que, com certeza, não a escreveu para a juventude
mais madura, mas dirigiu seu livro sobre Caspar Hauser, que
foi lido em toda Europa, aos adultos. E, espero, da mesma maneira que vocês vão
escutá-la durante 20 minutos, a Europa inteira a acompanhou sem fôlego
durante cinco anos, de 1828 a 1833. Ela começa assim:

``Em Nuremberg, o segundo dia de Pentecostes é um dos dias de diversão
por excelência, pois a maior parte dos seus habitantes se dispersa no
campo e nas cidades vizinhas. Nesses dias, a cidade, que por si só já é
muito extensa em relação à sua população escassa, torna-se tão
silenciosa e vazia, especialmente com o tempo bom da primavera, que se
parece muito mais com aquela cidade encantada do Saara do que com uma
cidade movimentada de negócios e de comércio. Nessa situação,
principalmente em algumas partes mais distantes do seu centro, as coisas
secretas podem facilmente ocorrer em público, sem deixarem, por isso, de
ser secretas. -- Assim, no segundo dia de Pentecostes, em 26 de maio de
1828, na parte da tarde entre as 4 e as 5 horas, aconteceu o seguinte:
Um cidadão, domiciliado na praça chamada \emph{Unschlittplatz}, ainda
estava na frente de sua casa para, de lá, ir até o chamado Portão Novo
(\emph{Neues Tor}), quando, não muito longe de si, se deparou com um
jovem vestido como um camponês e que ficava parado numa postura
altamente peculiar, esforçando-se, à maneira de uma pessoa embriagada, a
se movimentar para frente, sem conseguir manter-se adequadamente ereto e
governar seus pés. O mencionado cidadão aproximou-se do forasteiro, que
lhe estendeu uma carta endereçada `Ao Senhor Capitão de cavalaria do 4.º
Esquadrão do 6.º Regimento \emph{Chevaux-Léger} de Nuremberg'.'' Acho
que, nesta altura, devo interromper a história, não apenas para explicar
que um regimento \emph{Chevaux-Léger} é um regimento de cavalaria, mas
também para dizer a vocês que essa palavra francesa estava escrita de
forma completamente errada, apenas imitando seu som. Isso é importante,
pois é assim que vocês devem imaginar a ortografia da carta inteira que
Caspar Hauser levava consigo e que vou ler para vocês depois. Assim que
tiverem escutado essa carta, vão entender facilmente por que o capitão
não ficou com o menino por muito tempo, mas tentou se livrar dele da
maneira mais rápida, ou seja, chamando a polícia. Como vocês sabem, a
primeira coisa que se faz quando alguém se dirige à polícia é um
registro. E naquela época em que o capitão, que não sabia o que fazer
com o Caspar Hauser, o entregou à polícia, formaram-se os primeiros
registros do gigantesco processo ``Caspar Hauser'', que hoje está
guardado em 49 volumes no Arquivo do Estado de Munique. Uma coisa desse
processo está bem clara: Caspar Hauser chegou em Nuremberg como uma
pessoa totalmente embrutecida e tosca, cujo vocabulário não abrangia
mais do que 50 palavras e que não entendia nada que lhe era dito e que
só tinha duas respostas a todas as perguntas dirigidas a ele: ``tinham
cabalero'' e ``num sei''. Mas como ele chegou a ter o nome de Caspar
Hauser? É uma história bastante estranha. Quando o capitão o levou para
o posto de polícia, a maioria dos guardas não chegou a um consenso sobre
se a questão era tratá-lo como uma pessoa demente ou um semi-selvagem.
Um ou outro, entretanto, ponderou a possibilidade de que esse rapaz
poderia ser um refinado impostor. E essa posição ganhou, num primeiro
momento, uma certa probabilidade devido ao seguinte fato: tiveram a
ideia de fazer um teste para ver se ele sabia escrever, deram-lhe uma
pena com tinta, colocaram uma folha de papel para ele e o mandaram
escrever algo. Ele parecia ficar feliz com isso, pegou a pena habilmente
entre seus dedos e, para a surpresa de todos, escreveu com traços firmes
e legíveis o nome \emph{Caspar Hauser}. Depois disso, o mandaram
acrescentar o nome do seu lugar de origem. Mas ele não fez outra coisa a
não ser balbuciar novamente seu ``tinham cabalero'' e ``num sei''.

O que esses bons policiais não conseguiram saber na época, ninguém
conseguiu saber até o presente momento; ninguém ficou sabendo de onde
Caspar Hauser surgiu. Mas a conversa daquele posto de polícia, segundo a
qual esse rapaz poderia ser um impostor muito esperto, também se manteve
como boato ou como convicção até o dia de hoje. Vocês ainda vão ouvir
algumas curiosidades que motivaram essa afirmação. Pelo menos eu, como
narrador da história, não quero esconder que a considero como
equivocada. Não se deve procurar no rapaz, mas em um outro lugar a
impostura que deu início a esta história. Para tal, agora preciso ler
para vocês a carta que estava com Caspar Hauser quando chegou em
Nuremberg.

``Magnífico Senhor Capitão! Estou lhe enviando um menino que gostaria de
ser um fiel servidor do seu Rei, ele pediu. Esse menino foi deixado
comigo -- quer dizer: me foi empurrado clandestinamente -- no dia 7 de
outubro de 1812. Eu mesmo sou um pobre trabalhador diarista, tenho dez
filhos, mal tenho como me sustentar, e não consegui saber nada sobre a
mãe dele. Mas também não falei no tribunal que o menino foi deixado
comigo, pois eu pensei que deveria tratá-lo como um filho; dei-lhe uma
educação cristã e desde o ano de 1812 não deixei que saísse de casa para
que ninguém soubesse onde ele cresceu, e ele mesmo não sabe o nome da
minha casa, e o lugar ele também não conhece; o senhor pode lhe
perguntar, ele não vai poder dizer. Caro Senhor Capitão, o senhor não
deve insistir, pois ele não sabe o lugar onde moro, eu o levei de noite,
ele não sabe mais voltar para casa. E ele está sem um tostão porque eu
mesmo não tenho nada. Se o senhor não ficar com ele, deve matá-lo ou
pendurá-lo na chaminé e defumar.''

Ora, junto com essa carta havia um pequeno bilhete que não estava
escrito em letra gótica, como essa carta, mas em letra latina, e ainda
em outro papel, aparentemente com uma caligrafia totalmente diferente.
Supostamente era a carta que acompanhava a criança deixada pela mãe 16
anos atrás. Ali estava escrito que ela era uma moça pobre e que não
teria como alimentar a criança. O pai pertenceria ao Regimento
\emph{Chevaux-Léger} de Nuremberg e que era para mandar o menino também
para esse Regimento assim que fizesse 17 anos. -- No entanto, e aqui se
evidencia pela primeira vez a impostura que fazia parte desse jogo
esdrúxulo: o exame químico mostrou que as duas cartas, aquela de 1828,
supostamente do diarista, e a outra, de 1812, supostamente da mãe, foram
escritas com a mesma tinta. Como vocês podem imaginar, logo não se
acreditou mais numa, nem na outra carta, nem na existência do pretenso
diarista, nem muito menos da pretensa moça pobre.

Nesse meio tempo, colocaram Caspar Hauser primeiro na cadeia municipal
de Nuremberg, considerando-o menos como um prisioneiro do que como uma
curiosidade que representava um dos pontos de atração para os
forasteiros. Do grande número de pessoas ilustres que esse caso
extraordinário havia levado a Nuremberg, havia também o Conselheiro
Anselm von Feuerbach, que chegou a conhecer Caspar Hauser na ocasião e
sobre o qual escreveu o livro cujo começo li para vocês. Foi ele quem
deu a essa história uma virada decisiva, pois foi o primeiro que não
enxergou Caspar Hauser superficialmente, mas o estudou com o mais
profundo interesse. Ele percebeu que a inépcia, o idiotismo e a
ignorância do menino encontravam-se no mais gritante contraste com seus
dons extraordinários e seu caráter nobre. Essa natureza e a excelência
de suas dádivas, mas também certas marcas externas como, por exemplo, as
cicatrizes de vacina -- sendo que, naquele tempo, apenas as famílias
mais ilustres mandavam vacinar seus filhos --, tudo isso fez com que
Feuerbach fosse o primeiro a ponderar a possibilidade de que esse
forasteiro misterioso pudesse ser o filho de uma família da alta
aristocracia, e que ele fora escondido criminosamente por parentes para
privá-lo de sua herança. Em suas especulações, Feuerbach pensava
especificamente na família do Grão-duque de Baden. Suspeitas como essa eram veiculadas inclusive de maneira disfarçada pelos jornais da época e aumentavam ainda mais o interesse pela pessoa de Casper Hauser. Pode-se imaginar como isso deve ter inquietado todos aqueles que supunham que ele tivesse
desaparecido silenciosamente de algum asilo de pobres ou hospício de
Nuremberg. Mas as coisas tomaram outro rumo. Feuerbach, em sua qualidade
de alto funcionário do Estado, tinha uma certa influência e cuidava que
o menino ficasse em um ambiente que satisfizesse sua avidez de aprender,
que havia despertado com grande vivacidade. Em seguida, ele foi acolhido
como um filho na casa do professor Daumer de Nuremberg, um homem bom e
nobre, mas ao mesmo tempo bastante excêntrico. Ele nos deixou não apenas
um livro volumoso sobre Caspar Hauser, mas toda uma biblioteca de obras
extravagantes sobre sabedoria oriental, segredos naturais, curas
milagrosas e ainda sobre magnetismo. O professor fazia algumas
experiências com Caspar Hauser nesse sentido, certamente com muito
cuidado e sensibilidade humana. Segundo as descrições que nos deu disso,
Caspar deve ter mostrado uma sensibilidade muito delicada, clareza de
pensamento, sobriedade e pureza. Seja como for, fazia muito progresso e
logo esteve em condições de tentar descrever sua vida por conta própria.
Nessa ocasião, veio à tona tudo que sabemos até hoje do tempo anterior à
sua aparição em Nuremberg. Parece que passou muitos anos num cárcere
subterrâneo, onde nunca chegou a ver um raio de luz, nem um ser vivo.
Dois cavalinhos de madeira e um cachorro, também de madeira, teriam sido
seus únicos companheiros; água e pão sua única alimentação. Pouco antes
de ter sido tirado de sua prisão, um desconhecido teria feito contato
para visitá-lo e, sempre ficando nas suas costas para não ser visto,
conduzir sua mão e ensinar-lhe, assim, a escrever. É claro que esses
relatos, feitos num alemão tão truncado quanto as anotações, despertaram
muitas dúvidas. Mas causa estranheza que há ao mesmo tempo testemunhos
de que Caspar Hauser, nos seus primeiros meses em Nuremberg, não
tolerava outra coisa a não ser pão e água, nem sequer leite, e que era
capaz de enxergar no escuro. Os jornais não perderam a oportunidade de
noticiar que Caspar Hauser teria começado a trabalhar em sua biografia.
Já naquela época, isso quase significou seu fim, pois, pouco depois de a
notícia se espalhar, encontraram-no, sem consciência e sangrando na
testa, no porão da casa do professor Daumer. Um desconhecido, contou,
teria aplicado um golpe de machado, enquanto estava no abrigo debaixo da
escada. Nunca descobriram esse desconhecido. Mas dizem que, mais ou
menos quatro dias depois disso, um senhor elegante teria abordado uma
mulher fora da cidade para perguntar se o ferido estava vivo ou morto,
para depois acompanhar essa mulher até o portão, onde estava afixado um
aviso da polícia sobre o ferido. Depois de ter lido esse aviso, ele
teria se afastado de uma maneira altamente suspeita, sem retornar para a
cidade.

Ora, se tivéssemos tanto tempo não apenas quanto eu queria, mas vocês
também, espero, queriam ter, poderia apresentar-lhes outra pessoa
notável que apareceu nesse momento na vida de Hauser, um senhor distinto
que o adotou. Não temos como entrar em detalhes sobre a importância
desse senhor, mas apenas quero salientar que, a partir daquele momento,
tratava-se de cuidar melhor da segurança de Hauser, levando-o de
Nuremberg a Ansbach, onde Anselm von Feuerbach passou a ocupar o cargo
de presidente do tribunal. Isso foi em 1831. Caspar Hauser teria mais
dois anos a viver, até ser assassinado em 1833. Como isso se deu, vou
contar para vocês agora, para encerrar.

No decorrer do tempo, ele tinha passado por uma grande mudança. Por mais
que suas capacidades intelectuais tivessem progredido, por mais que seus
dons tivessem se enobrecido, sua evolução mental parou depois de algum
tempo e seu caráter perdeu a pureza. Dizem que no final de sua vida --
sabemos que não passou dos 31 anos --, ele foi um homem um tanto mau e
medíocre, que ganhava discretamente sua vida como escrivão e com
trabalhos em papelão, nos quais mostrava muita habilidade. No mais,
porém, ele não se destacava por uma maior dedicação, nem era
especialmente honesto.

Então, numa manhã de dezembro do ano 1833, aconteceu que um homem o
abordou na rua com as palavras: ``Uma recomendação do senhor jardineiro
da Corte e um convite para visitar, hoje à tarde, o poço artesiano no
parque etc.'' -- Por volta das quatro horas, Caspar Hauser compareceu ao
Jardim da Corte. Não havia ninguém perto do poço artesiano. Ele deu mais
cem passos na mesma direção, quando um homem saiu da mata e estendeu um
saquinho roxo na direção dele, dizendo: ``Dou-lhe esse saquinho de
presente!'' Mal Caspar Hauser tinha tocado o saquinho, sentiu uma
facada. O homem desapareceu, Caspar deixou o saquinho cair e ainda
conseguiu arrastar-se até sua casa. Mas a ferida era mortal. Depois de
três dias, morreu. Ainda haviam feito um interrogatório com ele, mas a
questão se esse desconhecido era o mesmo que tentou matá-lo quatro anos
antes em Nuremberg ficou no escuro, assim como o resto. Por isso, nesse
momento também havia pessoas que afirmavam que Caspar Hauser tinha
esfaqueado a si mesmo. Mas encontraram o saquinho. E este era bastante
enigmático, pois não continha outra coisa a não ser um bilhete dobrado
em que estava escrito, em letra espelhada: ``O Hauser poderá contar a
vocês como sou e de onde venho. Para poupá-lo, vou dizer eu mesmo de
onde venho. Sou da fronteira com a Baviera. Vou até dizer a vocês meu
nome.'' Mas aí seguem-se apenas três letras maiúsculas: \versal{MLO}.

Já disse a vocês que há 49 volumes de processo no Arquivo do Estado em
Munique. Dizem que o Rei Luís~\versal{I}, que se interessava muito pelo assunto,
havia folheado todos. Depois disso, muitos eruditos ainda os
consultaram. A polêmica em torno da questão de se Caspar Hauser era ou não
um príncipe de Baden até hoje não foi esclarecida. Cada ano sai um ou
outro livro no qual se afirma que o enigma finalmente está resolvido.
Podemos fazer uma aposta de 100 contra 1. Quando vocês estiverem
adultos, ainda haverá pessoas que não conseguirão se livrar dessa
história. Quando esbarrarem em um livro dessa natureza, vocês talvez o
lerão para ver se ele tem a solução que a rádio ficou devendo a vocês.

\chapterspecial{O Coração Gelado}{Peça de rádio baseada em Wilhelm Hauff\ \footnote[*]{``Das
  Kalte Herz. Hörspiel nach Wilhelm Hauff'', in \versal{GS} \versal{VII}-1, pp. 316-346.
  Tradução de Georg Otte, com a colaboração de Francisco De Ambrosis
  Pinheiro Machado. Peça radiofônica para crianças escrita em
  colaboração por Benjamin e Ernst Schoen, seu antigo colega de escola,
  que compôs também a música para a emissão. Produzido em 1932, o
  programa foi ao ar em 16 de maio. O texto é uma adaptação de um conto
  de fadas de Wilhelm Hauff. Benjamin, que era colecionador de livros
  infantis, possuía uma edição das obras completas do autor desde 1918,
  como testemunha uma carta na qual ele conta a Ernst Schoen ter
  recebido o volume como presente de aniversário. [\versal{N}. \versal{E}.]}}{Walter Benjamin e Ernst Schoen}{}



\section{Personagens}

\begin{myparindent}{0pt}
\versal{O locutor}

\versal{Peter Munk do Carvão}

\versal{Anão de Vidro}

\versal{Miguel o Holandês}\footnote{Na época, transportava-se a madeira
  simplesmente juntando os troncos em jangadas, que flutuavam rio
  abaixo. Miguel o Holandês trabalha nessa atividade, sendo que o
  apelido de ``Holandês'' vem do fato de a madeira da Floresta Negra ser
  transportada no rio Reno até a Holanda, onde os troncos serviam como
  escoras para as construções em solo pantanoso. [\versal{N}. \versal{T}.]}

\versal{Ezequiel}

\versal{Schlurker}

\versal{O Rei do Tablado}

\versal{Lisbete}

\versal{Mendigo}

\versal{Moendeiro}

\versal{Moendeira}

\versal{Filho do Moendeiro}

\versal{Uma Voz}

\versal{Postilhão}\bigskip

\emph{Prelúdio}

\begin{Parskip}
\versal{Locutor} --- Prezados ouvintes. Começa mais uma vez o nosso Programa da
Juventude e estou pensando em ler outro conto de fadas para vocês. Qual
conto será que vocês vão querer? Vamos dar uma olhada no nosso grande
dicionário no qual estão os nomes de todos os autores de contos de
fadas, como no catálogo de telefone, onde posso escolher um nome. Então:
\versal{A}, de Abracadabra, não serve; vamos folhear mais; \versal{B}, de Bechstein, já é
mais interessante, mas esse já tivemos recentemente.

\emph{Batidas na porta.}

\versal{C} de Celsius, o contrário de Réaumur, \versal{D}, \versal{E}, \versal{F}, \versal{G}.

\emph{Som mais forte das batidas.}

\versal{H} de Hauff, Wilhelm Hauff, este sim, seria o certo para nós.

\emph{Agora estão espancando a porta}.

Que barulho infernal é esse? Aqui na Rádio? Assim não dá para fazer o
nosso Programa da Juventude, ora bolas! Entre! Entre logo!
\emph{Sussurrando:} Vocês estão perturbando o meu Programa da Juventude
-- mas, o que é isso? Que figuras estranhas vocês são! O que vocês
querem?

\versal{Peter Munk do Carvão} --- Somos as personagens do conto de fadas ``O
coração frio'', de Wilhelm Hauff.

\versal{Locutor} --- Do ``Coração frio'' de Wilhelm Hauff? Então é como se vocês
tivessem sido chamados, sejam bem-vindos! Mas como conseguiram entrar?
Não sabem que aqui é uma rádio? E que ninguém pode entrar aqui sem
autorização?

\versal{Peter Munk do Carvão} --- O senhor é o locutor?

\versal{Locutor} --- Claro que sou o locutor!

\versal{Peter Munk do Carvão} --- Então estamos no lugar certo. Entrem todos e
fechem a porta. E agora, quem sabe, poderemos primeiramente nos
apresentar.

\versal{Locutor} --- Sim, mas\ldots{}

\emph{Cada apresentação é acompanhada de uma melodia tocada por uma
caixinha de música}.

\versal{Peter Munk do Carvão} --- Sou o Peter Munk, nascido na Floresta Negra. Me
chamam de Peter Munk do Carvão, porque herdei do meu pai, junto com o
colete de honra com botões de prata e as meias vermelhas para os dias de
festa, a profissão de carvoeiro.

\versal{Anão de Vidro} --- Eu sou o anão de vidro, tenho apenas 3 pés e meio de
altura, mas muito poder sobre o destino dos homens. Se for uma criança
de domingo\footnote{No original \emph{Sonntagskind,} pessoa nascida em
  um domingo que, segundo crença popular alemã, é favorecida pelo
  destino. [\versal{N}. \versal{E}.]}, Sr. Locutor, faça um passeio pela Floresta
Negra e, quando vir um anão na sua frente, com um chapéu pontudo e
largo, com colete e calças turcas e meias vermelhas, faça logo um
desejo, pois isso significa que me achou.

\versal{Miguel o Holandês} --- E eu sou o Miguel Holandês. Meu colete é de linho
escuro, as calças de couro preto estão presas por largos suspensórios
verdes. No bolso carrego um metro de latão e uso botas de madeireiro que
leva as toras rio abaixo, mas tudo isso de tamanho gigantesco, de
maneira que, só para fabricar as botas, seria necessária uma dúzia de
bezerros.

\versal{Ezequiel} --- Sou Ezequiel o Gordo. Chamam-me assim porque a minha cintura
é imensa. Também tenho as riquezas que combinam com a minha barriga. Não
é à toa que me consideram o mais rico do grupo. Todo ano vou duas vezes
a Amsterdã para vender madeira de construção, e, enquanto os outros têm
que voltar a pé, eu subo o Reno com muita pompa.

\versal{Schlurker} --- Sou Schlurker o Altão, o homem mais alto e mais magro da
Floresta Negra inteira, mas também o mais atrevido, porque, mesmo com
todo mundo sentado apertado no bar, preciso de mais espaço do que quatro
gordos.

\versal{Rei do Tablado}, \emph{afetado} --- Permita-me que me apresente, Sr.
Locutor, sou o Rei do Tablado.

\versal{Miguel o Holandês}, \emph{interrompendo} --- Já chega, Rei do Tablado, não
precisa de tanta cerimônia. Sei muito bem de onde vem seu dinheiro e que
antes você era um pobre servo lenhador.

\versal{Lisbete} --- Sou a Sr.\textsuperscript{a} Lisbete, filha de um pobre
sitiante lenhador, mas a mais bela e mais cheia de virtudes da Floresta
Negra inteira e sou casada com o Peter Munk do Carvão.

\versal{Mendigo} --- E eu sou o último de todos, pois sou apenas um pobre mendigo
e por isso faço um papel que, embora importante, é pequeno.

\versal{Locutor} --- Agora já chega de ouvir quem vocês são; a minha cabeça já
está muito confusa. Mas o que vocês querem aqui na Rádio? Por que me
atrapalham no meu trabalho?

\versal{Peter Munk do Carvão} --- Para dizer a verdade, Sr. Locutor, a gente
queria muito visitar a Terra da Voz.

\versal{Locutor} --- A Terra da Voz, Peter Munk do Carvão? Como vou entender isso
agora? Dá para ser mais claro?

\versal{Peter Munk do Carvão} --- Veja, Sr. Locutor, já faz 100 anos que estamos
no livro de contos de fadas de Hauff. Assim, só podemos falar apenas
para uma criança de cada vez. Mas dizem que agora está na moda que as
personagens dos contos de fadas saiam dos livros para ir para a Terra da
Voz, onde podem se apresentar a milhares de crianças de uma única vez.
Queremos fazer isso também, e nos falaram que o Sr. Locutor seria a
pessoa certa para nos ajudar nisso.

\versal{Locutor}, \emph{lisonjeado} --- Isso certamente é verdade, se estiver
falando da Terra da Voz do Rádio.

\versal{Miguel o Holandês}, \emph{grosseiro} --- Claro que estamos falando disso!
Nos deixe entrar em cena então, Sr. Locutor, sem cerimônia.

\versal{Ezequiel}, \emph{grosseiro} --- Não fale besteira, Miguel. Na Terra da Voz
não dá para ver nada!

\versal{Peter Munk do Carvão} --- Com certeza dá para ver alguma coisa na Terra da
Voz, mas não dá para ser visto. E percebi que é isso que lhe incomoda.
Claro, você não é feliz quando não consegue ser visto com suas
correntes, suas joias e seus lenços. Mas pense bem no que recebe em
troca: todas as pessoas que você pode ver do pico mais alto da Floresta
Negra e ainda mais que isso vão poder lhe ouvir sem você ter que
levantar a voz nem um pouco.

\versal{Rei do Tablado} --- Pensando bem, Peter Munk do Carvão, não concordo muito
com você. Na Floresta Negra, tudo bem, lá conheço tudo -- mas na Terra
da Voz, temo que vou errar o caminho e tropeçar o tempo todo por causa
das raízes.

\versal{Ezequiel} --- Raízes! Não há raízes na Terra da Voz!

\versal{Peter Munk do Carvão} --- Não dê crédito a ele, Rei do Tablado. Tenho
certeza de que há raízes. Na Terra da Voz há também uma floresta negra e
também aldeias, cidades, rios, nuvens, exatamente como na Terra. Mas na
Terra da Voz não se pode vê-los, apenas ouvir. E, assim, não se vê nada,
apenas ouve-se o que se passa na Terra da Voz. Mas mal vocês entrem
nela, vão saber se virar tão bem quanto aqui.

\versal{Locutor} --- E se alguma coisa faltar, eu, o locutor, estou às ordens. Nós
do Rádio conhecemos a Terra da Voz como a palma da própria mão.

\versal{Miguel o Holandês}, \emph{grosseiro} --- Deixe a gente entrar, então, Sr.
Locutor.

\versal{Locutor} --- Devagar, seu grosseiro Miguel Holandês, não é tão simples
assim. É verdade que na Terra da Voz vocês vão poder falar e ainda para
milhares de crianças, mas eu sou o guarda-fronteiras desse país e
precisarei dar a condição que têm que cumprir antes disso.

\versal{Lisbete} --- Uma condição?

\versal{Locutor} --- Sim, Sr.\textsuperscript{a} Lisbete, uma condição que lhes
causará muito esforço para ser cumprida.

\versal{Anão de Vidro} --- Pois bem, diga-nos sua condição, pois estou acostumado
a condições; eu mesmo costumo impô-las.

\versal{Locutor} --- Então escute bem, Anão de Vidro, e vocês outros também: quem
quiser entrar na Terra da Voz, tem que ser muito modesto, se livrar de
qualquer adorno e qualquer beleza exterior para que lhe reste apenas a
voz. Mas esta, em compensação, será ouvida por milhares de crianças ao
mesmo tempo, do jeito que vocês querem.

\emph{Pausa.}

Então, esta é a condição da qual, infelizmente, não posso abrir mão.
Vocês têm ainda um tempinho para pensar nisso.

\versal{Peter Munk do Carvão}, \emph{sussurrando} --- O que vocês acham? Lisbete,
você está disposta a deixar aqui sua bela indumentária de domingo?

\versal{Lisbete}, \emph{sussurrando} --- Mas é claro, Peter, não faço questão
nenhuma dela! Se podemos falar para milhares de crianças!

\versal{Ezequiel}, \emph{sussurrando} --- Olha lá! Também não é tão simples assim!
\emph{Tilintando as moedas}: E o que faço com esses ducados de ouro?

\versal{Anão de Vidro}, \emph{sussurrando} --- Fique feliz de poder livrar-se
delas dessa maneira, seu malandro! \emph{Em voz alta:} Então, Sr.
Locutor, estamos de acordo com sua condição.

\versal{Locutor} --- Muito bem, Anão de Vidro, vamos então.

\versal{Peter Munk do Carvão} --- Só mais um pedido.

\versal{Locutor} --- Qual pedido, Peter?

\versal{Peter Munk do Carvão} --- Sabe, Sr. Locutor, nós nunca estivemos na Terra
da Voz!

\versal{Locutor} --- Claro, claro, e que mais?

\versal{Peter Munk do Carvão} --- Como é que vamos encontrar o caminho?

\versal{Locutor} --- Você tem razão, Peter.

\versal{Peter Munk do Carvão} --- Já que o senhor é o guarda-fronteiras da Terra
da Voz, acho que poderia nos acompanhar como guia.

\versal{Rei do Tablado} --- Também acho, preso junto, enforcado junto.\footnote{Expressão
  alemã (\emph{Mitgefangen, mitgehangen}) para dizer que, quem se
  envolveu em alguma ação com outras pessoas, tem que arcar igualmente
  com as consequências. [\versal{N}. \versal{T}.]}

\versal{Lisbete} --- Ninguém aqui está falando de enforcar, seu bobo! Mas, se o
Sr. Locutor fizer a gentileza\ldots{}

\versal{Locutor}, \emph{lisonjeado} --- Combinado então, vou guiar vocês. Mas não
se incomodem com o barulho que os meus papéis farão.

\emph{Barulho de papel,} pois sem o meu mapa também não sei andar na
Terra da Voz.

\emph{Pausa}.

Então, se não tiverem nada contra, peço que me acompanhem até o
guarda-roupa! Sr.\textsuperscript{a} Lisbete, o chapéu a senhora tem que
deixar aqui, também o espartilho dourado e os sapatos de fivela, aqui
tem a vestimenta de voz em troca. Sr. Peter Munk, tire o colete de honra
e as meias vermelhas.

\versal{Peter Munk do Carvão} --- Aqui estão.

\versal{Locutor} --- Você também, Anão de Vidro, vai ter que tirar o chapéu, o
colete e as calças turcas.

\versal{Anão de Vidro} --- Já está feito.

\versal{Locutor} --- E você, Miguel o Holandês? Não, não, o metro e as botas de
couro também têm que ficar.

\versal{Miguel o Holandês} --- Que seja, em nome do diabo!

\versal{Locutor} --- O Rei do Tablado também já está pronto, e você, pobre
mendigo, não deve ter muita coisa para deixar! Mas o que estou vendo
aqui? Ezequiel o Gordo pendurou seu saquinho de ducados no pescoço! Não,
bom amigo, assim não dá! Para onde estamos indo agora, seus ducados
também não servem de nada. Apenas precisamos de uma voz bonita e clara,
uma que não tenha sofrido a fumaça da taberna como a sua.

\versal{Ezequiel}, \emph{amaldiçoando} --- Não, não vou com vocês! Meu dinheiro
vale mais do que toda essa Terra da Voz de vocês!

\versal{Miguel o Holandês} --- Caramba! Acho que também mando um pouco aqui. Dê-me
o dinheiro, sua pulga humana miserável, ou vou lhe esmagar!

\versal{Locutor} --- Mantenham a calma, meus amigos! Sr. Miguel o Holandês,
controle sua ira, e Sr. Ezequiel, posso lhe garantir que, depois de sua
visita à Terra da Voz, receberá seu dinheiro de volta, centavo por
centavo.

\versal{Ezequiel} --- Tudo bem, Sr. Locutor, se puder me dar uma garantia por
escrito!

\versal{Locutor} --- Todo mundo para a Terra da Voz!

\emph{Gongo.}

\emph{Música: Peter}

\versal{Locutor} --- Olá, Peter Munk do Carvão, olá!

\emph{Várias vozes gritando} --- Olá!

\versal{Peter Munk do Carvão} --- Locutor, você enxerga alguma coisa? Quem é que
está gritando ``Olá''? Onde é que estamos?

\versal{Locutor} --- Não, Peter Munk do Carvão, na Terra da Voz não há nada a ver,
apenas a ouvir.

\emph{Música}: \emph{Moinho.}

\versal{Filho do Moendeiro} --- Você está vendo alguma coisa, pai?

\versal{Moendeiro} --- Tem tanta névoa que a gente não vê um palmo na frente do
nariz. Estou quase tropeçando no meu próprio moinho. -- O que diz,
mulher?

\versal{Moendeira} --- Mas agora ouço vozes se aproximando.

\emph{Música.}

\versal{Peter Munk do Carvão} --- Locutor? Há um ruído aqui como se tivesse um
rio. A minha vida inteira nunca vi nem um riachinho por aqui.

\versal{Locutor} --- Por aqui, Peter? Você fala como se soubesse? Mas espero que
não se assuste quando lhe digo que a gente se perdeu.

\versal{Peter Munk do Carvão} --- A gente se perdeu? Não acredito. Não tinha
vozes?

\versal{Locutor} --- Vozes de outras pessoas.

\emph{Ouve-se novamente o}: ``Olá, olá!

\versal{Moendeira} --- Jesus, de onde vocês surgiram tão tarde da noite?

\versal{Locutor} --- Olá, minha senhora, então já é tarde?

\versal{Moendeiro} --- Quase dez horas da noite.

\versal{Peter Munk do Carvão} --- Sim, boa noite, minha boa gente, é que nós nos
perdemos.

\versal{Moendeiro} --- Então vocês já caminharam muito.

\versal{Peter Munk do Carvão} --- Nem tanto assim. Mas agora sinto a caminhada nas
minhas pernas.

\versal{Locutor} --- Eu então, Peter. Mas não adianta, vou ter que voltar e
procurar meus amigos na Terra da Voz.

\emph{Ouve-se pessoas dizendo} --- Boa noite, Locutor! Cuide-se. Boa
noite! Até logo!

\versal{Moendeira} --- Entrem e fiquem à vontade, Sr. Peter, pois parece que é
esse seu nome. O senhor tem que prestar atenção para não pegar muita
poeira. Nos moinhos sempre há poeira. Vamos, Hanni, ofereça ao senhor as
panquecas que sobraram do jantar, e um licor de cereja ele também não
vai recusar.

\emph{Pausa. Ouve-se barulho de pratos.}

\versal{Filho do Moendeiro}, \emph{sussurrando} --- Esse Sr. Peter tem uma
aparência estranha, mãe.

\versal{Moendeira} --- Sei lá. O que você quer dizer com isso?

\versal{Filho do Moendeiro}, \emph{sussurrando} --- Estranho, mãe, parece que
aconteceu alguma coisa com ele.

\versal{Moendeira} --- Deixe de ser bobo, menino! Vá para a cama, depressa. E o
senhor certamente também não vai demorar para se deitar. O senhor deve
saber que, nos moinhos, o barulho começa cedo. Não é um abrigo para
dorminhocos.

\versal{Peter Munk do Carvão} --- Certo, Sra. Moendeira. Mas permita-me
agradecer-lhe muito pelas panquecas.

\versal{Moendeira} --- Ah, não é nada. Mas agora venha comigo. Vou lhe mostrar a
cama.

\versal{Peter Munk do Carvão} --- Não se preocupe, posso dormir aqui. Com tantas
almofadas! Quase até o teto!

\versal{Moendeira} --- É isso aí, não temos janelas duplas aqui na Floresta Negra.
Por isso tem que ter cobertores grossos quando começa a gear no inverno.

\emph{Ouve-se novamente algumas vozes} --- Bom descanso! Boa noite! Não
se esqueçam de apagar as velas!

\versal{Peter Munk do Carvão}, \emph{bocejando} --- Que coisa, não sabia que uma
pessoa pode ficar com tanto sono. Mesmo se o diabo entrar agora, acho
que eu ficaria deitado e só viraria para o outro lado.

\emph{Pequena pausa. Batidas na porta.}

\versal{Peter Munk do Carvão} --- Estão batendo na porta? Não é possível, todos já
estão dormindo.

\emph{Mais batidas na porta.}

\versal{Peter Munk do Carvão} --- Deve ter alguém na porta. Entre!

\versal{Filho do Moendeiro} --- Meu caro Sr. Peter, por favor, não me entregue.
Deixe-me ficar mais um pouco com o senhor. Estou com tanto medo.

\versal{Peter Munk do Carvão} --- Que é isso, o que você tem? Por que está com
tanto medo?

\versal{Filho do Moendeiro} --- Sr. Peter, o senhor também ficaria com medo se
tivesse visto o que vi hoje. -- Talvez o senhor tenha visto, quando
entrou, o livro coberto com um veludo vermelho, que estava na mesa.

\versal{Peter Munk do Carvão} --- Ah, o álbum, certo. Deve ter retratos nele, não
é?

\versal{Filho do Moendeiro} --- Há retratos nele sim, Sr. Peter, mas numa das
folhas há três que não saem mais da minha cabeça de maneira alguma; eles
me perseguem o tempo todo com seus olhares. O gordo do Ezequiel e o
altão do Schlurker e o Rei do Tablado, pois esses são os nomes que
estavam escritos debaixo dos retratos.

\versal{Peter Munk do Carvão} --- O que você diz? Ezequiel o Gordo, Schlurker o
Altão\ldots{} mas esses nomes eu também já ouvi, e o Rei do Tablado era o
pobre diabo que começou como servo do dono da madeira e depois ficou
riquíssimo de repente. Alguns dizem que teria encontrado um pote cheio
de dinheiro debaixo de um pinheiro. Outros afirmam que, não muito longe
da cidade de Bingen, ele teria pescado um pacote com peças de ouro com a
haste que os madeireiros-jangadeiros usam para espetar peixes no Reno, e
que o pacote faria parte do grande tesouro dos Nibelungos que lá está
enterrado. Resumindo: ele ficou rico de repente e os jovens e os velhos
passaram a respeitá-lo como um príncipe.

\versal{Filho do Moendeiro} --- Mas o senhor deveria ter visto os olhos, aqueles
olhos!

\versal{Peter Munk do Carvão} --- Sim, isso existe, sabe? Pessoas que viram uma
coisa especialmente terrível, às vezes mantêm um olhar estranho durante
toda sua vida.

\versal{Filho do Moendeiro} --- Mas o que o senhor acha que ele poderia ter visto
de tão terrível?

\versal{Peter Munk do Carvão} --- Saber, eu não sei, mas lá, do outro lado da
Floresta Negra, sabe, onde moram os proprietários de madeira e os
madeireiros-jangadeiros, dizem que acontecem coisas não muito normais.

\versal{Filho do Moendeiro} --- Já sei, agora o senhor está falando de Miguel o
Holandês. Meu pai também já me contou algumas coisas sobre ele. É o
gigante da floresta, o rapaz selvagem e de ombros largos, sobre o qual
aqueles que afirmam tê-lo visto garantem que não quer pagar do próprio
bolso o couro dos bezerros necessário para fabricar seus sapatos.

\versal{Peter Munk do Carvão} --- Sim, é nele que acabei de pensar.

\versal{Filho do Moendeiro} --- De repente, o senhor também sabe algo sobre ele.

\versal{Peter Munk do Carvão} --- Menino, você não tem vergonha de dizer algo
assim? Como é que vou saber alguma coisa sobre Miguel o Holandês? Às
vezes, quando ouço as pessoas falarem, me pergunto: será que não é
simplesmente inveja? Será que eles não têm inveja dos donos da madeira
porque eles sempre andam por aí orgulhosos nos seus coletes com os
botões, as fivelas e as correntes, nas quais têm pendurados vários
quilos de prata? Não são poucos os que ficam com inveja quando veem algo
assim.

\versal{Filho do Moendeiro} --- Mas o senhor também já ficou com inveja dele?

\versal{Peter Munk do Carvão} --- Com inveja? De forma alguma; eu seria o último a
sentir inveja.

\versal{Filho do Moendeiro} --- Então o senhor é tão rico quanto ele? Ou até mais
rico?

\versal{Peter Munk do Carvão} --- Olha, menino, você deve ter percebido que sou um
rapaz pobre e que não tenho prata pendurada no corpo, nem em casa. Pois
tenho algo melhor que isso, mas não posso lhe revelar o que é.

\versal{Filho do Moendeiro} --- Agora o senhor me deixou bastante curioso. Não vou
querer sair do seu quarto até me dizer.

\versal{Peter Munk do Carvão} --- Mas você sabe mesmo guardar segredo?

\versal{Filho do Moendeiro} --- Com certeza, Sr. Peter, prometo que ninguém vai
ficar sabendo de nada.

\versal{Peter Munk do Carvão} --- Então quero lhe perguntar uma coisa: alguma vez
você já ouviu falar do Anão de Vidro? Que nunca aparece de outra maneira
a não ser com um chapéu pontudo e largo, com calças turcas e meias
vermelhas? E que é amigo dos sopradores de vidro, dos carvoeiros e dos
pobres em geral que moram deste lado da Floresta?

\versal{Filho do Moendeiro} --- Do Anão de Vidro? Não, Sr. Peter, nunca ouvi falar
dele.

\versal{Peter Munk do Carvão} --- Mas já ouviu falar da criança de domingo?

\versal{Filho do Moendeiro} --- Sim, claro, aquelas que nascem aos domingos ao
meio-dia.

\versal{Peter Munk do Carvão} --- Então, sou uma delas. Entendeu? -- Mas essa é só
a metade do segredo. A outra é o meu verso.

\versal{Filho do Moendeiro} --- Agora já não entendo mais nada, Sr. Peter.

\versal{Peter Munk do Carvão} --- O Anão de Vidro se mostra às crianças de
domingo, mas apenas se elas, ao pé do Morro dos Pinheiros, onde as
árvores encontram-se tão perto umas das outras e são tão altas que fica
escuro em pleno dia, quando não se ouve um machado, nem um pássaro, se
elas souberem o verso correto. E esse verso aprendi com a minha mãe.

\versal{Filho do Moendeiro} --- Então é para ficar com inveja do senhor.

\versal{Peter Munk do Carvão} --- Sim, seria mesmo para ficar com inveja se
tivesse guardado o verso, mas, quando estive na frente do pinheiro agora
mesmo, querendo dizê-lo, vi que tinha me esquecido da última rima, e o
Anão de Vidro desapareceu no mesmo momento em que havia se mostrado.
``Sr. Vidro'', gritei depois de alguma hesitação, ``faça o favor de não
zombar de mim. Sr. Vidro, se o senhor acha que não o vi, está muito
enganado. Vi o senhor muito bem quando apareceu por trás da árvore.''
Mas ele não respondeu e só de vez em quando ouvi uma risadinha baixinha
e rouca vindo por detrás da árvore. Aí pensei: basta uma investida para
pegar esse rapazinho. Mas na hora que dei um salto até o pinheiro, não
havia mais nenhum Anão de Vidro na floresta verde de pinheiros; apenas
um esquilo pequeno e gracioso fugiu para cima da árvore.

\versal{Filho do Moendeiro} --- Quer dizer que o senhor acabou de encontrar com o
Anão de Vidro?

\versal{Peter Munk do Carvão} --- Exatamente.

\versal{Filho do Moendeiro} --- Mas agora o senhor tem que me falar o verso, se o
souber ainda.

\versal{Peter Munk do Carvão} --- Não, rapaz. Agora já é tarde, vamos dormir.
Nesse meio tempo você se esqueceu dos seus três homens maus, e amanhã,
quando acordarmos, todos nós vamos querer estar bem dispostos.

\versal{Filho do Moendeiro} --- Boa noite, Sr. Peter. Mas não estou bem disposto
porque não me falou o verso.

\emph{Ouve-se como os dois se despedem.}

\versal{Peter Munk do Carvão} --- Finalmente estou sozinho e agora quero dormir.
Mas o verso não quero dizer a outra pessoa que não seja o Anão de Vidro;
se ao menos me lembrasse dele!

\emph{Toca uma musiquinha que Peter Munk do Carvão acompanha com uma voz
sonolenta}:

\begin{quote}
\quad \, Dono do tesouro na verde floresta de pinheiros,

Já viveste séculos inteiros,

A ti pertence toda terra que pinheiros tem\ldots{}
\end{quote}

\versal{Peter Munk do Carvão}, \emph{com voz sonolenta} --- Que pinheiros
tem\ldots{}, que pinheiros tem\ldots{} -- Se soubesse a continuação!

\emph{A musiquinha acaba. Depois de uma pausa, ouve-se seis batidas.}

\versal{Locutor} --- Cá estou eu de novo no moinho da Floresta Negra, junto com o
Peter Munk do Carvão. São seis horas e aposto que o Peter dormiu sem
parar; não vai ser fácil acordá-lo.

\emph{Ouve-se o ronco alto do Peter. Uma música baixa vai aumentando
cada vez mais. Alguém canta uma ou duas estrofes.}

\versal{Peter Munk do Carvão}, \emph{sonolento} --- O que é isso? Parece que eles
usam uma caixinha de música como despertador. Quero acordar assim todas
as manhãs, com uma música só minha, como um príncipe. Não, está vindo de
fora. Ah, devem ser aprendizes! Sim, eles têm que levantar cedo!

\emph{Ouve-se o canto}:

\begin{quote}
\quad \, Em cima do morro um lugar tem,

De onde para o vale posso olhar,

De lá meus olhos veem,

Pela última vez ela passar.
\end{quote}

\versal{Peter Munk do Carvão} --- Olá, pessoal, mais uma vez, mais uma vez, cantem
isso mais uma vez!

\emph{Ouve-se como a música vai diminuindo aos poucos e como o canto
fica cada vez mais incompreensível.}

É, eles não estão ligando nada para mim. Já estão lá longe. \emph{Mais
baixo e pensativo}: Mas, como é que foi? \emph{Canta baixinho a mesma
melodia}: ``De lá meus olhos veem'', ``de lá meus olhos veem'' --
``veem'', então esta é a rima, ``tem'' com ``veem''. Agora, Anão de
Vidro, teremos uma palavrinha de novo. \emph{Ele assobia um pouco para
si mesmo.}

\versal{Locutor} --- Para onde vai com tanta pressa, Peter? Há pouco ainda pensei
aflito em como fazer você levantar e mostrar o caminho de casa, e agora
você passa aí na maior rapidez.

\versal{Peter Munk do Carvão}, \emph{apressado} --- Deixe-me, deixe-me, Sr.
Locutor. Lembrei do meu verso\ldots{}

\versal{Locutor} --- Verso? Que verso é esse?

\versal{Peter Munk do Carvão}, \emph{dando sinal para falar baixo} --
Pss\ldots{}, tenho um plano especial, mas não posso falar. O senhor vai
ver depois. Adeus, Sr. Locutor!

\versal{Locutor} --- Olha só esse pândego. \emph{Gritando atrás dele:} Preste
atenção para não esbarrar com o maldoso Miguel o Holandês! Adeus, Peter!

\emph{Pausa. Peter assobia sua cantiga. Pausa. Pigarreia longamente.}

\versal{Peter Munk do Carvão} --- Pronto, aqui temos o pinheiro grande. Atenção,
Peter, vamos lá:

\begin{quote}
\quad \, Dono do tesouro na verde floresta de pinheiros,

Já viveste séculos inteiros,

Tua é toda terra que pinheiros tem,

Só crianças de domingo é que te veem.
\end{quote}

\versal{Anão de Vidro} --- Você não acertou direito, mas, no seu caso, Peter Munk
do Carvão, vou abrir uma exceção. Você encontrou com o malcriado do
Miguel o Holandês?

\versal{Peter Munk do Carvão} --- Sim, Sr. Dono do Tesouro, fiquei com bastante
medo. Apenas procurei o senhor para me dar um conselho; não estou muito
bem e tenho muitos problemas. Um carvoeiro não vai muito longe e, como
ainda sou jovem, estive pensando que poderia fazer de mim algo melhor.
Quando vejo os outros e o progresso que fizeram em tão pouco tempo --
basta olhar o Ezequiel e o Rei do Tablado, eles estão nadando em
dinheiro.

\versal{Anão de Vidro} --- Peter, não fale dessa gente para mim! O que eles ganham
se passam alguns anos aparentemente felizes para depois se tornarem
tanto mais infelizes? Não despreze sua ocupação; seu avô e seu pai eram
pessoas honestas e tinham essa mesma ocupação, Peter! Espero que não
seja amor pelo ócio que o traz até mim.

\versal{Peter Munk do Carvão} --- Não, Sr. Dono do Tesouro do Pinheiral, o ócio é
o começo de todos os males. Mas o senhor não poderia me levar a mal, se
uma outra profissão me agrada mais que a minha atual. Não tem como negar
que um carvoeiro é pouco respeitado no mundo, em comparação aos
sopradores de vidro, aos madeireiros dos rios e aos relojeiros.

\versal{Anão de Vidro} --- A soberba vem amiúde antes da queda.\footnote{A
  passagem ecoa o seguinte versículo bíblico do livro dos
  \emph{Provérbios}: ``A soberba precede a ruína, e a altivez do
  espírito precede a queda.'' (Cf. Pr 16, 18) [\versal{N}. \versal{T}.]} Vocês
homens são um gênero estranho! Raramente alguém está contente com a
profissão em que nasceu e cresceu. Para que isso? Se você fosse um
soprador de vidro, iria querer ser dono de madeireira, e se fosse dono
de madeireira, poderia ter a seu serviço um administrador florestal ou
ter o direito a uma residência de magistrado. Mas que seja! Se você
prometer que vai trabalhar direito, quero ajudá-lo a ter uma vida
melhor, Peter. Costumo realizar três desejos para cada criança de
domingo que sabe me encontrar. E preste atenção! A cada desejo vou bater
o meu cachimbo de vidro no pinheiro. Os dois primeiros são livres, o
terceiro posso recusar se for tolo. Então faça seus desejos, mas, Peter,
que seja algo de bom e útil!

\versal{Peter Munk do Carvão} --- Muito bem! O senhor é um excelente Anão de
Vidro! E com razão o chamam de Dono do Tesouro, pois, estando na sua
casa, os tesouros estão no lugar certo. Então posso desejar o que mais
quero -- para começar quero saber dançar melhor do que o Rei do Tablado
e, a cada vez, trazer o dobro do dinheiro que ele levava para a taberna.

\emph{Batidas do cachimbo.}

\versal{Anão de Vidro} --- Tolo que você é! Que mísero desejo é esse -- saber
dançar bem e ter dinheiro para o jogo! Você não tem vergonha na cara,
seu bobo, de enganar-se a si mesmo no que diz respeito à sua felicidade?
O que adianta para você e sua mãe se sabe dançar bem? Para que serve o
dinheiro, que, pelo seu desejo, só vai para a taberna, da mesma maneira
que o do Rei do Tablado miserável para eles vai? Depois você não tem
mais nada durante uma semana e passa necessidade como antes. Vou lhe
conceder mais um desejo; mas preste atenção para fazer um desejo mais
sensato!

\versal{Peter Munk do Carvão}, \emph{depois de hesitar um tempo} --- Então quero
ser o dono da fábrica de vidro mais bonita e mais rica em toda a
Floresta Negra, com todos os acessórios e o dinheiro.

\versal{Anão de Vidro} --- Só isso? Peter, só isso?

\versal{Peter Munk do Carvão} --- Bem, o senhor pode acrescentar ainda um cavalo e
uma carruagem.

\versal{Anão de Vidro} --- Que tolice, Peter Munk do Carvão! \emph{O cachimbo
espatifa-se.} Cavalos? Carruagens? Inteligência, digo, inteligência e
bom senso você deveria ter desejado, mas não cavalos e carruagens.
Agora, não fique tão triste assim, vamos ver se alguma coisa foi para o
seu bem, pois o segundo desejo não foi inteiramente tolo. Uma boa
fábrica de vidro alimenta seu dono e seu mestre; faltou apenas desejar
inteligência e bom senso -- as carruagens e os cavalos teriam se juntado
de qualquer maneira.

\versal{Peter Munk do Carvão} --- Mas, Sr. Dono do Tesouro, tenho ainda um desejo
livre. Com ele poderia escolher a inteligência, se ela me for tão
necessária quanto o senhor acredita.

\versal{Anão de Vidro} --- Nada disso! Você ainda vai ter que passar por alguns
apertos para ficar contente por ter ainda um desejo livre. E agora vá
para casa! Aqui tem 2.000 florins para você e chega, e não volte mais
aqui para pedir qualquer dinheiro, pois, nesse caso, eu estaria obrigado
a lhe enforcar no pinheiro mais alto! É isso que costumo fazer desde que
moro na floresta. Há três dias morreu o velho Winkfritz, que era o dono
da grande fábrica de vidro na Floresta de Baixo. Vá para lá amanhã cedo
e faça uma oferta boa para comprá-la! Fique bem, seja bom trabalhador e
vou lhe visitar de vez em quando para lhe ajudar com alguns conselhos,
já que ainda não desejou o bom senso. Mas lhe digo com toda seriedade:
seu primeiro desejo foi mau. Evite frequentar as tabernas, Peter! Isso
nunca fez bem a ninguém.

\versal{Peter Munk do Carvão} --- Lá se vai ele. Incrível o quanto fuma, o Dono do
Tesouro. Nem consigo enxergá-lo mais de tanta fumaça. \emph{Farejando}:
Mas é um fumo bem agradável.

\emph{Gongo.}

\versal{Locutor} --- Então, onde é que nós paramos? Vocês, crianças, acabaram de
ouvir a conversa entre o Peter Munk do Carvão e o Sr. Dono do Tesouro.
Vocês ouviram os desejos tolos que o Peter fez e ouviram como o Anão de
Vidro desapareceu numa nuvem de fumaça do melhor fumo holandês. Vamos
ver como vai ser o próximo capítulo. \emph{Ele faz barulho com o papel.}
Mas onde está a continuação? Hum, hum! \emph{O barulho aumenta.}

\versal{Anão do Vidro}, \emph{sussurrando} --- O que está acontecendo? Por que a
gente não continua?

\versal{Locutor}, \emph{sussurrando} --- Hum, também não sei o que fazer. Imagine
só, Sr. Dono do Tesouro, àquela hora na floresta o vento deve ter levado
algumas folhas da história; agora estamos em maus lençóis. Não faço a
mínima ideia de como sair disso.

\versal{Rei do Tablado}, \emph{sussurrando} --- Isso é fatal, fatal! Mas o quê a
gente vai fazer?

\versal{Miguel o Holandês}, \emph{sussurrando} --- Seu tolo Rei do Tablado, você
também não vai encontrar a solução! Nesses casos, só algum grandão para
resolver! Deixe-me pensar!

\versal{Rei do Tablado}, \emph{sussurrando} --- Estou morrendo de rir, Sr. Miguel,
morrendo de rir.

\versal{Miguel o Holandês}, \emph{sussurrando} --- Cale a boca, Rei do Tablado, e
vá cantar a ``Guarda do Reno''.\footnote{\emph{Die Wacht am Rhein} é uma
  canção nacionalista alemã, composta em 1854 por Karl Wilhelm
  (1815-1873) a partir do poema de Max Schneckenburger (1819-1849),
  escrito em 1840, a propósito de um Reno, fronteira sagrada da nação,
  sob a guarda do alemão ``leal, piedoso e forte'' pronto a repelir as
  tentativas de invasão e de domínio francesas. \emph{Du Rhein bleibst
  deutsch wie meine Brust!} (Tu, Reno, permaneces alemão como o meu
  peito!). A canção celebrizou-se na Guerra Franco-Prussiana, tendo-se
  tornado desde então um hino patriótico alemão cuja popularidade fez-se
  notar de forma ainda expressiva quer na Primeira quer na Segunda
  Guerra Mundial. [\versal{N}. \versal{E}.]} Então, Peter Munk do Carvão, você não
ganhou todo aquele dinheiro do Anão de Vidro e conseguiu comprar uma
fábrica de vidro?

\versal{Peter Munk do Carvão} --- Certo, Sr. Miguel o Holandês, era uma bela
fábrica de vidro a que eu tive.

\versal{Rei do Tablado} --- Sim, decerto, Peter Munk do Carvão, você teve, mas
perdeu tudo num átimo, em um jogo com o gordo do Ezequiel na mesa da
taberna. Certo, Ezequiel, ou não?

\versal{Ezequiel} --- Ah, me deixe em paz com essa história, Rei do Tablado, nunca
mais quero ser lembrado dessa história!

\versal{Locutor} --- É verdade, Peter Munk do Carvão! Também me lembro disso. Você
perdeu toda a fábrica de vidro no jogo. Mas vocês têm de admitir -- não
foi uma grande tolice do Peter, esse desejo de ter sempre o mesmo tanto
de dinheiro no bolso quanto Ezequiel o Gordo? Assim foi inevitável que,
numa noite, o dinheiro dele acabasse e que, no dia seguinte, ele tivesse
vendido sua fábrica de vidro. Espere aí: ``tivesse vendido'' --
``tivesse vendido''? Está escrito aqui, na página 16! Graças a Deus,
encontrei de novo o fio da meada! Vamos, gente, podemos continuar!
Então, enquanto o oficial de justiça e o magistrado andavam pela oficina
de vidro para verificar e avaliar tudo para a venda, o nosso Peter Munk
do Carvão pensou: o Morro dos Pinheiros não está muito longe; se o anão
não me ajudou, quero fazer uma tentativa com o gigante. Ele correu até o
Morro dos Pinheiros com tanta rapidez como se os oficiais de justiça
estivessem no seu encalço. Quando passou pelo lugar onde havia
conversado pela primeira vez com o Anão de Vidro, sentiu-se como se uma
mão invisível o segurasse, mas se arrancou e continuou correndo até a
fronteira que havia guardado na memória. Sim, Peter, agora você vai ter
que se virar sozinho, pois não lhe invejo pelas coisas que vêm pela
frente.

\versal{Peter Munk do Carvão}, \emph{sem fôlego} --- Miguel o Holandês, Sr. Miguel
o Holandês!

\versal{Miguel o Holandês}, \emph{rindo} --- É você, Peter Munk do Carvão? Eles
estavam querendo lhe esfolar e vender sua pele aos credores? Ora, fique
tranquilo, toda sua miséria foi causada pelo Anão de Vidro, esse
separatista e hipócrita! Quando se dá um presente, tem que presentear
direito, e não como esse pão-duro! Vem, siga-me até a minha casa para
ver se a gente chega a um acordo!

\versal{Peter Munk do Carvão} --- Acordo, Miguel o Holandês? O que eu poderia
negociar com o senhor? Não vai querer que seja seu servo? Que mais
poderia querer? E como o senhor quer que eu desça para esse abismo?

\versal{Miguel o Holandês}, \emph{como se falasse num megafone} -- Sente-se na
minha mão e segure-se nos meus dedos para não cair.

\emph{Música com diversos ritmos do tique-taque de relógios, começando
baixo e aumentando cada vez mais}.

Pronto, chegamos! Sente-se na bancada do fogão e vamos tomar um bom gole
de vinho. Saúde, brinde comigo, pobre oficial-ajudante, parece que nunca
conseguiu sair dessa tristeza de Floresta Negra!

\versal{Peter Munk do Carvão} --- Claro que não, Miguel o Holandês, como é que eu
faria isso?

\versal{Miguel o Holandês} --- Então, nós madeireiros-jangadeiros somos
oficiais-ajudantes de outro tipo! Cada ano viajamos conduzindo as toras
rio abaixo, flutuando no belo Reno em direção à Holanda, e a isso se
juntam viagens a países distantes, como as que fiz no meu tempo livre.

\versal{Peter Munk do Carvão} --- Quem dera se eu pudesse me dar a esse luxo!

\versal{Miguel o Holandês} --- Depende só de você, melhorar de vida ou não, mas, é
claro, até hoje seu coração impediu que isso acontecesse.

\versal{Peter Munk do Carvão} --- Meu coração?

\versal{Miguel o Holandês} --- Quando você tinha coragem e força em todo o seu
corpo para tomar qualquer iniciativa, algumas batidas desse coração bobo
lhe deixavam trêmulo; e as ofensas e a desgraça, para que um rapaz
sensato deveria se preocupar com esse tipo de coisas? Você sentiu alguma
coisa na cabeça quando lhe chamaram de impostor ou de mau caráter? Seu
estômago doeu quando o oficial de justiça quis lhe expulsar de casa?
Diga-me, onde você sentiu a dor?

\versal{Peter Munk do Carvão} --- No coração.

\versal{Miguel o Holandês} --- Não me leve a mal, mas você desperdiçou centenas de
florins com mendigos miseráveis e outra gentalha. O que isso lhe trouxe?
Eles lhe desejaram a bênção de Deus e muita saúde; sua saúde melhorou
com isso? Você teria contratado um médico pela metade do dinheiro. A
bênção -- que bênção é essa quando tudo é penhorado e o expulsam! E o
que lhe levava a enfiar a mão no bolso sempre quando um mendigo estendia
seu chapéu esfarrapado? -- Seu coração, sempre seu coração, nunca seus
olhos, nem sua língua, nem seus braços ou suas pernas, mas sempre seu
coração; você só deu ouvidos ao seu coração.

\versal{Peter Munk do Carvão} --- Mas, como a gente pode se acostumar a não dar
ouvidos a ele? Estou fazendo muito esforço para abafá-lo, mas meu
coração continua batendo e doendo.

\versal{Miguel o Holandês}, \emph{rindo com sarcasmo} -- Claro, você, meu pobre
coitado, não pode fazer nada contra ele; mas basta me dar essa coisa que
bate e vai ver o quanto ganha com isso.

\versal{Peter Munk do Carvão}, \emph{assustado} --- O que? Dar meu coração ao
senhor? Seria a minha morte imediata! Nunca!

\versal{Miguel o Holandês} --- Claro, se um dos senhores cirurgiões quisesse tirar
seu coração numa operação, você certamente teria que morrer. No meu
caso, as coisas são um pouco diferentes. Mas venha comigo a esse quarto
e veja com seus próprios olhos!

\emph{Música: a fuga das batidas de coração.}

\versal{Peter Munk do Carvão} --- Meu Deus! O que é isso?

\versal{Miguel o Holandês} --- Olhe direito essas coisas nos vidros de formol!
Gastei um dinheirão com isso. Vá lá e leia os nomes nas etiquetas.

\emph{Depois de cada menção de nome, ouve-se a música correspondente.}

Lá nós temos o oficial de justiça e aqui Ezequiel o Gordo. Esse é o
coração do Rei do Tablado e aquele é do guarda-florestal. E aqui temos
toda uma coleção de corações que são de usuários e oficiais de
recrutamento. Olhe, todos eles se livraram dos temores e das
preocupações da vida. Nenhum desses corações bate mais com medo e
angustiado, e seus antigos donos estão muito contentes de ter expulsado
esse hóspede inquieto.

\versal{Peter Munk do Carvão}, \emph{angustiado} --- Mas o que eles têm agora no
lugar do coração?

\versal{Miguel o Holandês} --- Um coração de pedra muito bem trabalhado como este.

\versal{Peter Munk do Carvão}, \emph{horrorizado} --- É verdade? Um coração de
mármore? Mas ouça, Sr. Miguel, esse coração deve estar frio no peito.

\versal{Miguel o Holandês} --- Claro, mas é frio de uma forma agradável. Por que
um coração deveria estar quente? No inverno, o calor não tem utilidade
nenhuma; uma boa aguardente faz mais efeito do que um coração quente. E
no verão você não acredita o quanto um coração desses esfria o corpo. E,
como já disse, não há angústia, nem temor, não há compaixão tola, nem
outro lamento que atinja um coração desses.

\versal{Peter Munk do Carvão}, \emph{relutante} --- E isso é tudo que o senhor
pode me dar? Eu estava esperando dinheiro e o senhor quer me dar uma
pedra!

\versal{Miguel o Holandês} --- Então, penso que 100.000 florins seriam suficientes
para o primeiro momento. Se fizer bom uso desse dinheiro você logo vai
se tornar um milionário.

\versal{Peter Munk do Carvão}, \emph{alegre} --- Ora, pare de bater com tanta
força no meu peito! Logo estaremos resolvidos. Bem, Miguel, dê-me a
pedra e o dinheiro e tire o tique-taque da sua casa!

\versal{Miguel o Holandês}, \emph{alegre} --- Eu sabia que você era um rapaz
sensato. Vem cá, vamos tomar outro gole e depois vou desembolsar o
dinheiro.

\emph{A música dos corações passa para uma fuga de corneta.}

\versal{Peter Munk do Carvão}, \emph{acordando e espreguiçando} --- Uah! Desta vez
dormi demais. Não foi uma corneta de postilhão que me acordou? Estou
acordado ou ainda estou sonhando? Parece que estou viajando; não é um
postilhão e não são cavalos lá na frente? Não estou sentado numa
carruagem? E as montanhas que estou vendo lá atrás, não é a Floresta
Negra? A minha roupa também não é mais a mesma. Por que não estou
ficando melancólico, já que estou saindo pela primeira vez das florestas
onde passei tanto tempo da minha vida? O que será que a minha mãe está
fazendo? Estranho, ela deve estar sem ninguém para ajudá-la e passando
necessidade; mesmo assim, esse pensamento não é capaz de tirar nem uma
lágrima do meu olho. Tudo ficou indiferente para mim. Como isso é
possível? Ah, claro, lágrimas e suspiros, saudade e melancolia são
coisas do coração e, graças a Miguel o Holandês, o meu é frio e de
pedra. Se ele cumprir a palavra com os cem mil tão bem quanto com o
coração, vou achar muito bom. De fato, tem aqui uma bolsa com milhares
de táleres e cartas de crédito de casas de comércio em todas as grandes
cidades.

\emph{Melodia de corneta}.

\versal{Confusão de vozes} --- Frankfurt sobre Meno! Salsichas de Frankfurt! Casa
de Goethe! A Rádio de Frankfurt! O vinho de maçã! O Jornal de Frankfurt!
Bolos e biscoitos frankfurtianos! Frankfurt está cheia de curiosidades!

\versal{Peter Munk do Carvão} --- O que há para comer e para beber aqui? Embrulhe
para mim algumas dúzias de salsichas, algumas canecas de vinho de maçã e
alguns quilos de bolos e biscoitos.

\emph{Melodia de corneta}.

\versal{Confusão de vozes} --- Paris! Le Matin! Paris Midi! Paris Soir! Des
caiqouettes\footnote{Termo pseudo-francês, criado pelos autores da peça.
  [\versal{N}. \versal{T}.]}, des caiqouettes et des caiqouettes! Louvre! Torre
Eiffel! Esquimaux, Pochettes! Surprises!

\versal{Peter Munk do Carvão}, \emph{sonolento} -- Onde é que estamos? Ah, em
Paris! Então empacotem para mim uma boa quantidade de champanhe,
lagostas e ostras para eu não passar fome, nem sede!

\versal{Uma voz} --- Quem é que é esse senhor sonolento, Sr. Postilhão?

\versal{Postilhão} --- Ah, esse é o Sr. Peter Munk do Carvão da Floresta Negra.
Ele já comeu e bebeu tanta coisa em Frankfurt que não consegue mais se
mexer.

\emph{Melodia de corneta}.

\versal{Confusão de vozes} --- London! Britannia rules the waves! Ginger Ale!
Scotch Whisky! Toffies! Muffins! Morning Post! Daily News! The Times!
Turkey and Plumcake!

\versal{Peter Munk do Carvão} \emph{ronca}.

\versal{Uma voz} --- Quem é esse senhor que ronca, Sr. Postilhão?

\versal{Postilhão} --- É o Sr. Peter Munk do Carvão da Floresta Negra, ele já
comeu e bebeu tanta coisa em Paris que não consegue mais manter os olhos
abertos.

\emph{Melodia de corneta}.

\versal{Confusão de vozes} --- Constantinopla! Visitem o Bósforo e o Corno de
Ouro! Tapetes! Que tal um narguilé? Aprendiz de fabricante de gaitas de
fole constantinopolitano! Manjar turco! Visitem os dervixes uivantes em
Galípoli nos minaretes da Hagia Sophia!

\versal{Peter Munk do Carvão} \emph{ronca}.

\versal{Uma voz} --- Quem é esse senhor que ronca, Sr. Postilhão?

\versal{Postilhão} --- É o Sr. Peter Munk do Carvão da Floresta Negra, ele já
comeu e bebeu tanta coisa nas paradas anteriores que de modo algum
consegue mais manter os olhos abertos.

\emph{Melodia de corneta}.

\versal{Confusão de vozes} --- Roma! La Stampa di Roma! Il Corriere della Sera! Il
Foro romano! Il Coliseo! Giovinezza! Vino bianco e vino rosso!
Spaghetti! Polenta! Risotto! Frutti del Mare! Antiguidades! Visitem o
Papa e o Duce!

\versal{Peter Munk do Carvão} \emph{ronca}.

\versal{Uma voz} --- Quem é esse senhor que ronca, Sr. Postilhão?

\versal{Postilhão} --- É o Sr. Peter Munk do Carvão da Floresta Negra, ele já
comeu e bebeu tanta coisa nas paradas anteriores que de modo algum
consegue mais manter os olhos abertos.

\emph{Melodia de corneta}.

\versal{Postilhão} --- Hum, hum -- Cidade na Floresta Negra! Todo mundo descendo!

\emph{Gongo.}

\versal{Locutor} --- Eis o Peter Munk do Carvão em casa de novo. Vocês escutaram a
corneta anunciando a chegada do Postilhão. Mas enquanto entenderam bem,
espero, os nomes de todas as paradas que o Postilhão apregoou, vocês não
entenderam o último nome, o que não é por acaso. Nós não sabemos onde
mora o Peter Munk do Carvão. Não está escrito no livro do qual você,
Peter Munk, e você, Ezequiel o Gordo, e você, Schlurker o Altão, e você,
Miguel o Holandês, e você, Anão de Vidro, saíram. E não queremos ser
curiosos. Basta que esteja de novo em casa, na Floresta Negra de Baden.
Ele não deixa de perceber essas coisas, mas as percebe apenas na cabeça,
não no coração. Entende que voltou para casa, mas não o sente. Ele
também não tem mais nada a fazer. Seu carvoeiro não está mais aceso, ele
vendeu a fábrica de vidro e tem tanto dinheiro que seria uma burrice
aceitar qualquer trabalho. Aí está ele, para passar o tempo sai à
procura de uma mulher. Ele continua sendo um rapaz bonito. De fora não
dá para ver que tem um coração de pedra. Antigamente, quando ainda tinha
um coração de verdade, todos o amavam, e é disso que todo mundo se
lembra e quem se lembra disso em particular é Lisbete, a filha de um
pobre madeireiro. Ela vivia sossegadamente e sozinha, cuidava da casa do
pai com habilidade e zelo e nunca foi vista num baile, nem em
Pentecostes ou na Quermesse. Quando Peter ouve falar dessa maravilha da
Floresta Negra, resolve pedi-la em casamento e vai até a cabana que lhe
indicaram. O pai da bela Lisbete recebe esse senhor bem vestido com
espanto e fica mais espantado ainda quando fica sabendo que se trata do
rico Sr. Peter e que este quer ser seu genro. O pai não hesitou muito
tempo, pois achou que todas as suas preocupações e pobreza chegariam ao
fim e deu seu consentimento. E Lisbete, a boa filha, foi tão obediente
que se tornou a sr. Peter Munk sem qualquer resistência. Lisbete não
tem dinheiro, mas leva um presente milagroso para a casa de Peter.
Trata-se de um relógio cuco, que pertence à família desde muitas
gerações. Esse relógio tem uma característica muito particular; não por
acaso as pessoas contam que o Anão de Vidro o deu uma vez a uma pessoa
querida. A particularidade do relógio é a seguinte: funciona como um
verdadeiro relógio cuco da Floresta Negra, batendo de hora em hora. Ao
meio-dia, porém, só dá doze batidas se não tiver uma pessoa malvada na
sala onde está pendurado. Mas se tiver uma pessoa malvada, ele bate
treze vezes. Estamos agora na sala onde está o relógio. O Peter Munk do
Carvão está sentado à mesa com a Lisbete.

\emph{O relógio dá onze batidas.}

\versal{Lisbete} --- Onze horas? Vou ter que correr e colocar as cenouras no fogo.

\versal{Peter} --- Cenouras de novo? Que nojo, diabos.

\versal{Lisbete} --- Mas Peter, é seu prato preferido.

\versal{Peter} --- Prato preferido! Prato preferido, toda essa comida não tem
graça nenhuma. Agora, se me trouxer um copo grande de conhaque\ldots{}

\versal{Lisbete} --- Você não sabe o que o Sr. Padre disse domingo passado, quando
falava dos beberrões?

\versal{Peter}, \emph{batendo o pé} --- Vamos? Você vai trazer o conhaque ou não?
\emph{Ameaçando}: Ou\ldots{}

\versal{Lisbete}, \emph{choramingando} --- Aqui, para fazer sua vontade. Mas isso
não vai acabar bem.

\versal{Peter} --- Basta que comece bem. A minha vida já é suficientemente triste.
Fico tão irritado quando ouço as pessoas falarem coisas sobre o domingo
ou sobre o bom tempo ou da primavera; estou achando-os totalmente
loucos.

\versal{Lisbete} --- Você está sentindo dores?

\versal{Peter} --- Não, mas esse é o problema, não sinto dores, nem alegria. Outro
dia até cortei o dedo e quase não senti nada. Foi quando serrei em
pedaços aquele baú que você ganhou da sua avó como presente de madrinha,
sabe?

\emph{Batidas na porta.}

\versal{Peter} --- Não se mexa, não responda nada.

\emph{Batem pela segunda vez.}

\versal{Peter} --- Que ele não ouse entrar sem eu mandar. E não vou mandar.

\versal{Lisbete} --- Mas por que? Você nem sabe quem é.

\versal{Peter} --- Um envio de dinheiro não vai ser. Mendigos miseráveis, nada
mais.

\emph{Batem.}

\versal{Lisbete} --- Entre!

\versal{Peter} --- Não falei, sua atrevida? Claro que é um mendigo.

\versal{Mendigo} --- Por favor, peço que me deem alguma coisa.

\versal{Peter} --- Vai pedir ao diabo para que você fique com ele!

\versal{Mendigo} --- Tenha misericórdia, Senhora, me dê apenas um gole de água.

\versal{Peter} --- Prefiro esvaziar toda a minha garrafa de conhaque na cabeça
dele em vez de dar um copo d'água.

\versal{Lisbete} --- Deixe para lá, quero ir buscar para ele um gole de vinho, um
pão e uma pratinha para o caminho.

\versal{Peter} --- Isso é bem típico seu, sua besta. Você não é capaz de seguir o
raciocínio do seu marido? De repente, até me acha cruel ou sem coração?
Não entende que ponderei tudo com muito critério? Será que você não sabe
o que acontece quando a gente deixa esse tipo de pessoa entrar na casa
da gente? É uma gentalha de mendigos. Uma pessoa conta para a outra.
Eles fazem uma marca na porta. São sinais de bandidos. Depois esperam a
oportunidade e levam tudo o que não está fixo e pregado. Se receber dois
ou três desses rapazes, um ano depois você vai dormir entre suas quatro
paredes peladas.

\versal{Mendigo} --- Pessoas ricas como vocês não sabem o quanto a pobreza dói e o
quanto um gole de água fresca faz bem nesse calor.

\versal{Peter} --- Estou ficando cansado com esse falatório.

\emph{O relógio cuco começa a bater.}

\versal{Lisbete} --- Céus! Me esqueci das cenouras! E o senhor pode levar tudo que
trago comigo e vai embora.

\emph{As batidas do relógio têm que soar alto e suceder-se bem devagar
para que as palavras da mulher possam ser ouvidas entre a primeira e a
segunda batida.}

\versal{Peter}, \emph{pensativo, acompanha com uma voz sem ressonância, as
batidas} -- Um, dois, três, quatro, cinco, seis, sete, oito, nove, dez,
onze, doze.

\emph{Silêncio completo, ressoa a décima terceira batida. Ouve-se uma
queda abafada.}

\versal{Lisbete} --- Meu Deus do céu, o Peter perdeu a consciência. Peter, Peter,
o que você tem? Acorde! \emph{Gemidos, suspiros e choro.}

\emph{Gongo.}

\versal{Locutor} --- O Peter não apenas perdeu a consciência, mas por pouco não
perdeu também sua vida por causa de sua arrogância e falta de fé. Agora,
depois do relógio dar a décima terceira batida, ele volta a si, cai em
si e resolve fazer seu terceiro e último pedido com o Dono do Tesouro,
desejando seu coração de volta. Vamos ver como isso acaba!

\emph{Gongo.}

\versal{Peter}:

\begin{quote}
\quad \, Dono do tesouro na verde floresta de pinheiros,

Já viveste séculos inteiros,

Tua é toda terra que pinheiros tem,

Só crianças de domingo é que te veem.
\end{quote}

\versal{Anão de Vidro}, \emph{com uma voz abafada:} O que você quer de mim, Peter
Munk?

\versal{Peter} --- Ainda tenho um desejo livre, Sr. Dono do Tesouro.

\versal{Anão de Vidro} --- Corações de pedra ainda são capazes de desejar alguma
coisa? Você está com tudo que sua maldade exige; dificilmente vou
realizar seu desejo.

\versal{Peter} --- Mas o senhor me concedeu três desejos; resta ainda um.

\versal{Anão de Vidro} --- Se o desejo for tolo, posso negá-lo. Mas, tudo bem,
deixe-me ouvir o que deseja.

\versal{Peter} --- Tire a pedra morta do meu peito e devolva-me meu coração vivo.

\versal{Anão de Vidro} --- Fui eu quem fez o negócio com você? Sou eu Miguel o
Holandês, que presenteia as pessoas com riquezas e corações frios? Lá,
com ele, é que você tem que procurar seu coração.

\versal{Peter} --- Ele nunca vai me devolver meu coração.

\versal{Anão de Vidro}, \emph{depois de uma pausa.} Estou com pena de você, por
mais que você seja mau. Mas, como seu desejo não é tolo, pelo menos não
posso recusar minha ajuda. Consegue guardar um verso?

\versal{Peter} --- Acredito que sim, Sr. Dono do Tesouro, mesmo tendo esquecido o
seu uma vez.

\versal{Anão de Vidro} --- Então repita. Se esquecer, tudo estará perdido: ``Você
não é enviado da Holanda\ldots{}'' Repita.

\versal{Peter} --- ``Você não é enviado da Holanda.''

\versal{Anão de Vidro} --- ``\ldots{} Seu Miguel, mas é o inferno que lhe manda.''
Repita.

\versal{Peter} --- ``Seu Miguel, mas é o inferno que lhe manda.'' Agora consegui,
Sr. Dono do Tesouro, muito bem, certamente se trata de uma fórmula
mágica. Quando Miguel o Holandês ouvir isso, não vai poder me fazer mal
algum.

\versal{Anão de Vidro} --- Tudo bem, mas como vai ser?

\versal{Peter} --- Como vai ser? Nada vai ser. Vou entrar na casa dele e gritar:

\begin{quote}
\quad \, Você não é enviado da Holanda,

Seu Miguel, mas é o inferno que lhe manda.\footnote{No original alemão,
  há aqui um jogo de palavras entre \emph{Holland} (Holanda) e
  \emph{Höllenland} (inferno). [\versal{N}. \versal{E}.]}
\end{quote}

Aí ele não vai poder me fazer mal algum.

\versal{Anão de Vidro} --- É bem sua cara. Está certo, ele não vai poder lhe fazer
mal algum. Mas logo depois de falar essas palavras, Miguel o Holandês
vai desaparecer. O Diabo vai saber para onde. Mas você vai olhar para
todos aqueles corações e não vai poder resgatar o seu.

\versal{Peter} --- Ai meu Deus, como vou fazer então?

\versal{Anão de Vidro} --- Isso eu não sei lhe dizer. Até hoje você refletiu muito
pouco na vida. Está mais do que na hora de começar. E agora vou ter que
cuidar dos meus pica-paus nos pinheiros; eles não causam tanta
preocupação quanto as crianças de domingo.

\emph{Gongo.}

\versal{Locutor} --- Ora, vou ter que dizer uma coisa a vocês: Se é para aguardar,
prefiro aguardar na terra dos homens do que na Terra da Voz. Aqui só tem
neblina. A gente não enxerga nada, só fica aguçando o ouvido, e isso já
venho fazendo durante horas. Porém, na floresta onde mora o Dono do
Tesouro, não há um só galho mexendo, nenhum pica-pau batendo, nenhum
ninho sussurrando. Mas tudo bem, que história é essa, de tanto tédio
acabo fazendo poesia. Agora estou ouvindo um estalo, ou será que é um
sussurro? É a voz do Dono do Tesouro ou a voz do Peter Munk do Carvão?

\versal{Peter Munk do Carvão}, \emph{totalmente abatido e triste} -- Peter Munk
do Carvão.

\versal{Locutor} --- Mas isso não soa muito alegre.

\versal{Peter Munk do Carvão}, \emph{totalmente abatido e triste} -- Parece que
você está fazendo o papel do eco aqui na floresta?

\versal{Peter Munk do Carvão}, \emph{totalmente abatido e triste} -- Ô!

\versal{Locutor} --- Mas você não é uma companhia alegre na floresta. E o que
estou ouvindo lá longe? Parece que é a música assombrosa de vidro do
Miguel o Holandês. Mas por que você não responde? Por que você fica
mudo?

\versal{Peter Munk do Carvão}, \emph{como acima:} Hum!

\versal{Locutor} --- Agora as coisas estão ficando confusas demais para mim.
Confusas e inseguras. Não me leve a mal, Sr. Peter, mas agora vou
procurar um novo caminho.

\versal{Peter Munk do Carvão}, \emph{como acima:} Adeus! \emph{Ele bate na porta
e grita:} Miguel o Holandês!

\emph{Repete isso três vezes.}

\versal{Miguel o Holandês} --- Que bom que você veio. Eu também não aguentaria
ficar com a Lisbete, esse muro de lamentações, que desperdiça todo esse
dinheiro com os mendigos. Quer saber de uma coisa? No seu lugar eu faria
outras viagens. Você fica fora por alguns anos e, quem sabe, quando
voltar para casa, a Lisbete já não estará mais.

\versal{Peter Munk do Carvão} --- Você adivinhou, Miguel o Holandês, quero ir à
América. Mas para isso vou precisar de dinheiro; é muito longe daqui.

\versal{Miguel o Holandês} --- Nada mais fácil, Peterzinho, você vai ter o que
precisa. \emph{Ouve-se o barulho das moedas e alguém contando:} 100,
200, 500, 800, 1.000, 1.200. Não são marcos, Peterzinho, só táleres.

\versal{Peter Munk do Carvão} --- Miguel, você é mesmo um rapaz porreta, mas, no
fundo, um verdadeiro bandido -- mentindo desse jeito para mim, dizendo
que eu teria uma pedra no peito e você teria o meu coração.

\versal{Miguel o Holandês} --- Mas não é assim? Você sente mesmo seu coração? Não
está frio como gelo? Você sente medo ou aborrecimento, alguma coisa é
capaz de lhe deixar arrependido?

\versal{Peter Munk do Carvão} --- Você apenas fez com que meu coração ficasse
parado, mas eu tenho o mesmo coração de sempre no meu peito; e o
Ezequiel também -- ele me disse que você mentiu para nós. Você não é
homem o suficiente para arrancar o coração do peito sem a gente perceber
e sem risco; você teria que ser um mago.

\versal{Miguel o Holandês} --- Mas eu lhe garanto que você e o Ezequiel e todas as
pessoas ricas que me procuraram têm um coração frio como você e que
tenho os verdadeiros corações aqui nesse cômodo.

\versal{Peter Munk do Carvão} --- Mas com que facilidade você sabe mentir para as
pessoas! Conta isso para outro! Você não acha que, nas minhas viagens,
eu vi dúzias desses truques? Esses corações no seu cômodo são imitações
de cera. Concordo que você é um rapaz muito rico, mas não sabe fazer
magia.

\versal{Miguel o Holandês} --- Entre e leia todas as etiquetas; aquele lá é o
coração do Peter Munk. Olha como está batendo! Alguém sabe fazer um
coração assim de cera?

\versal{Peter Munk do Carvão} --- Ele é de cera sim; um coração de verdade não
bate assim. Tenho o meu ainda no meu peito. Não, fazer mágica você não
sabe.

\versal{Miguel o Holandês} --- Vou provar que é verdade! Você vai sentir que esse
é seu coração. Aqui, vou colocar seu coração no seu peito! Como está se
sentindo agora?

\versal{Peter Munk do Carvão} --- É verdade, você estava certo. Nunca teria
acreditado que isso fosse possível!

\versal{Miguel o Holandês} --- Não é? E sei fazer magia sim; mas agora quero
colocar a pedra de volta.

\versal{Peter Munk do Carvão} --- Devagar, Sr. Miguel! É com toucinho que se pega
os ratos, e desta vez, você é o enganado. Ouça o que tenho a lhe dizer.

\emph{Ele começa balbuciando para depois gritar com cada vez mais
coragem, frequência e velocidade sua fórmula mágica}:

\begin{quote}
\quad \, Você não é enviado da Holanda,

Seu Miguel, mas é o inferno que lhe manda.
\end{quote}

\emph{Os corações ressoam alto. Gemidos de Miguel o Holandês.
Tempestade.}

\versal{Peter Munk do Carvão} --- Agora esse malvado do Miguel o Holandês está se
contorcendo. Mas que tempestade horrível! Estou ficando com medo.
Rápido, para casa, para encontrar com a minha Lisbete.

\emph{Gongo.}

\versal{Locutor} --- Ah não, até a gente achar alguma coisa nesta Terra da Voz, é
um verdadeiro jogo de cabra-cega. Mas agora estou sentindo claramente,
isso deve ser a fábrica de vidro do Peter Munk do Carvão, e sua mulher
também não deve estar muito longe, pois de quem mais poderia ser essa
voz, senão da querida Lisbete!

\versal{Lisbete}, \emph{cantando}:

\begin{quote}
\quad \, Vidros de grave e agudo som,

Por que sozinha estou?

Por que foge Peter, meu amor,

Clandestinamente como um traidor?

Mas sei o que tenho a fazer,

Fraldas finas e sapatinhos tecer,

Para o filho de Peter, e tricotar,

Assim o tempo irá passar.

Vidros de som grave e agudo,

Primeiro a camisa, a meia em segundo,

Quando o bebê ao mundo chegar,

Tudo bem preparado vai encontrar.

\end{quote}

\versal{Locutor} --- Que coisa, parece pois que o Peter vai ser pai. Assim fica
duplamente injusto que ele passe tanto tempo fora de casa. Mas, para
mim, é uma boa oportunidade. Há quanto tempo que já queria conversar com
a Sr.\textsuperscript{a} Lisbete. Por que só conversaria o tempo todo
com o Peter na Terra da Voz? Mas como faço para chamar a atenção dela?
Não quero simplesmente chamá-la. A minha voz de urso a assustaria, mesmo
porque ela ainda está com a própria voz no ouvido, que soa tão delicada.

\emph{Pequena pausa.}

Já sei o que vou fazer. Apenas vou bater nos vidros.

\emph{Pequena música de vidro}

\versal{Peter Munk do Carvão} --- Cá estou!

\versal{Lisbete e Locutor} --- Quem é?

\versal{Peter Munk do Carvão} --- Tenho o meu coração de volta.

\versal{Lisbete} --- O meu você sempre teve.

\versal{Locutor} --- Agora quero ir embora. Mas vocês têm que prometer uma coisa:
quando o pequeno Peter Munk do Carvão vier ao mundo, vocês vão escolher
o Dono do Tesouro como padrinho.

\emph{Pequena pausa. Nomes de meses são enumerados.}

Como o tempo passa aqui na Terra da Voz. Lá está o Peter no Morro dos
Pinheiros e diz seus versos.

\emph{Gongo.}

\versal{Peter Munk do Carvão}:

\begin{quote}
\quad \, Dono do tesouro na verde floresta de pinheiros,

Já viveste séculos inteiros,

Tua é toda terra que pinheiros tem,

Só crianças de domingo é que te veem.

\end{quote}

Sr. Dono do Tesouro, escute-me; não quero outra coisa senão pedir que
seja o meu compadre quando meu filhinho nascer!

\emph{Vento.}

Então quero levar essas pinhas como lembrança, já que não quer aparecer.

\versal{Locutor} --- Crianças! Em que vocês acham que essas pinhas se
transformaram? Em um monte de novos táleres de Baden, e sem nenhum falso
entre eles. Esse foi o presente de padrinho do Anão do Pinheiral para o
pequeno Peter.

-- Agora vocês podem me agradecer. Não vocês, crianças que nos ouviram,
mas você Peter Munk do Carvão e o Dono do Tesouro e o Miguel o Holandês
e todo esse monte de personagens de Hauff\footnote{Benjamin faz aqui um
  jogo de palavras ``\emph{Haufen} \emph{Leute von Hauff''}. [\versal{N}.
  \versal{E}.]} que levei para a Terra da Voz a pedido deles e que deixei sãos
e salvos aqui na fronteira de novo.

\versal{Ezequiel} --- Sãos e salvos? Até parece. No meu caso não se pode falar em
``são e salvo'' de maneira alguma, enquanto não receber o meu dinheiro
de volta.

\versal{Lisbete} --- Credo, Ezequiel o Gordo, você não muda mesmo. Sou eu quem lhe
diz isso, a Lisbete.

\versal{Locutor} --- Deixe para lá, minha cara Senhora, ele vai receber tudo de
volta, centavo por centavo.

\versal{Lisbete} --- Sim, Sr. Locutor, e gostaria de expressar-lhe a minha
gratidão pela música de vidro que tanto me alegrou. Pois foi o senhor
que tocou essa melodia graciosa nas garrafas, não foi?

\versal{Locutor}, \emph{com voz grave} --- Sim, sim.

\versal{Lisbete} --- Durante um tempo, passei bastante medo quando não havia como
ir para frente e o senhor não sabia mais o caminho na Terra da Voz.

\versal{Locutor} --- Por favor, chegue mais perto, Sr.\textsuperscript{a} Lisbete.
Dê uma olhada aqui, na página\ldots{}, onde o Hauff também faz uma pausa
grande. Que coincidência, imagine só, por acaso a nossa pausa aconteceu
exatamente na mesma passagem.

\versal{Miguel o Holandês} --- Isso que chamo de sorte na má sorte.

\versal{Locutor} --- A pausa, então, foi o próprio autor que a fez. Por que será?
Essa história é como a cadeia montanhosa da própria Floresta Negra: seu
centro é como um pico do qual se olha os dois lados que descem, a saber,
para o lado do final infeliz e para o lado do final feliz.

\versal{Confusão de vozes} --- Até a vista, Sr. Dono do Tesouro, minha cara
senhora, Sr. Peter etc.

\versal{Miguel o Holandês} --- Esperem aí, por favor, fiquem só mais um momento,
meus senhores, por que tanta pressa? Não estou gostando de ter passado
uma impressão tão má. Por isso, queria chamar a atenção para outros
bandidos ainda piores. Leia, por exemplo, ``O navio dos fantasmas'', ``A
mão decepada'' e muitas outras histórias de Hauff, onde rapazes muito
piores que eu contribuem para um final feliz. Mas, tudo bem. Estou vendo
que todo mundo já se foi. Até a próxima!

\versal{Locutor} --- Até a próxima, Sr. Miguel. Que pessoas simpáticas. Mas agora
estou feliz de estar novamente sozinho no meu escritório. É, eu
pretendia realizar um Programa para Juventude. Será que foi mesmo para a
juventude?

\emph{Gongo.}
\end{Parskip}
\end{myparindent}

\chapter{A Berlim demoníaca\,\footnote[*]{``Das dämonische Berlin'',
  in \versal{GS} \versal{VII}-1, pp. 86-92. Tradução de Georg Otte. Palestra radiofônica
  para crianças datada de 25 de fevereiro de 1930. A cena da leitura
  infantil dos contos de \versal{E.T.A.} Hoffmann, descrita neste texto, também
  aparece em \emph{Infância em Berlim por volta de 1900}. [\versal{N}. \versal{E}.]} }

Hoje vou começar com uma lembrança da época dos meus catorze anos.
Naquele tempo, eu era aluno em um internato. Como é usual nesses
institutos, os alunos e os professores se reuniam várias vezes de noite
durante a semana para tocar música, ouvir um discurso ou ler a obra de
um escritor. Numa noite, o professor de música conduzia a ``Capela'' --
esse era o nome dessa reunião noturna. Era um homem pequeno e engraçado,
com uma expressão inesquecível nos seus olhos sérios, com a careca mais
brilhante que já tinha visto, em torno da qual havia uma coroa de cachos
escuros e muito encaracolados. Seu nome é conhecido entre os amadores
alemães de música: ele se chama August Halm. Esse mesmo August Halm
entrou na Capela para ler histórias de \versal{E.T.A.} Hoffmann para nós,
exatamente daquele escritor sobre o qual quero falar com vocês hoje. Não
me lembro mais qual texto ele leu, mas também não importa. Em
compensação, guardei com exatidão uma única frase de sua apresentação,
que introduziu a leitura. Essa frase caracterizava as obras de Hoffmann,
sua predileção pelo bizarro, excêntrico, fantasmagórico, inexplicável.
Acho que tudo que disse servia para despertar o interesse de nós meninos
pelas histórias que iriam se seguir. Mas depois ele concluiu com a frase
que não esqueci até hoje: ``Na próxima vez, vou contar a vocês para que
alguém escreve histórias desse tipo.'' Estou esperando por essa
``próxima vez'', e, como o bom homem faleceu nesse meio tempo, essa
explicação só chegaria até mim se fosse de uma maneira um tanto
amedrontadora, de modo que prefiro me antecipar a ela tentando cumprir
uma promessa que me fizeram 25 anos atrás.

Se eu quisesse trapacear um pouco, eu poderia facilitar as coisas para
mim. Bastava trocar o ``Para quê?'' pelo ``Por quê?'' e a resposta seria
muito simples. Por que o escritor escreve? Por mil motivos. Porque sente
prazer em inventar alguma coisa; ou porque é tomado por ideias ou
imagens tão impressionantes que só consegue se acalmar depois de
colocá-las no papel; ou porque carrega consigo perguntas e dúvidas pelas
quais encontra um tipo de solução no destino de pessoas fictícias, ou,
simplesmente, porque aprendeu a escrever; ou, e infelizmente é um caso
muito comum, porque não aprendeu nada. Não é difícil dizer por que
Hoffmann escrevia. Ele pertence àqueles escritores que são possuídos
pelas suas personagens: sósias, figuras assustadoras de toda espécie.
Quando os colocava no papel, os via de fato à sua volta; não apenas
quando escrevia, mas em meio à conversa mais inocente à mesa do jantar,
ao tomar vinho ou ponche, e mais de uma vez aconteceu de interromper sua
companhia de mesa com as palavras: ``Desculpe-me por interromper, meu
caro, mas está percebendo lá no canto ao lado direito esse menino
enfezado amaldiçoado aparecendo por debaixo do piso? Observe apenas as
cabriolas que o diabinho está fazendo! Olhe, olhe, agora sumiu! Oh, não
faça cerimônia, meu mais amável pequeno polegar, por favor, fique com a
gente\ldots{} tenha a bondade de escutar a nossa conversa tranquila\ldots{} O
senhor nem acredita o prazer que nos daria com sua companhia altamente
agradável\ldots{} Ah, o senhor voltou\ldots{} não gostaria de se aproximar um
pouquinho?\ldots{} como assim?\ldots{} o senhor gostaria de tomar alguma coisa?\ldots{}
o que gostaria de dizer?\ldots{} como?\ldots{} o senhor se despede\ldots{} seu criado
obediente.'' Etc. etc. E mal falava essas coisas desvairadas com olhar
fixo no canto, de onde tinha surgido a visão, ele levantava de repente o
olhar, dirigindo-se aos seus companheiros de mesa e pedia para
continuarem sossegadamente.

É assim que Hoffmann é descrito por seus amigos. E nós também nos
sentimos contaminados por esse modo de ser quando lemos histórias como
``A casa abandonada'', ``O morgado'', ``Os sósias'' ou ``A panela
dourada''. Havendo circunstâncias externas favoráveis, o impacto dessas
histórias de fantasma pode chegar a um grau muito surpreendente. Eu
mesmo passei por isso e a circunstância favorável, nesse caso, foi que
os meus pais me proibiram a leitura. Quando era pequeno, só podia ler os
contos de Hoffmann clandestinamente, de noite, quando saíam. Lembro-me
de uma daquelas noites de leitura, quando, sentado sozinho debaixo da
lâmpada da imensa mesa da copa -- era na Rua Carmer --, o silêncio era
completo. Enquanto lia ``As Minas de Falun'', todos os horrores se
reuniam aos poucos, em volta da minha mesa na escuridão, como peixes com
suas bocas mudas, de maneira que meus olhos se fixavam nas páginas do
livro como se fossem uma ilha que me salvava, das quais, no entanto, era
que saíam todos esses horrores. Ou, numa outra vez, era de dia e ainda
me lembro que estava parado diante do armário de livros entreaberto e
pronto para, ao mínimo barulho, jogar o volume de volta. Eu lia ``O
morgado'', com o cabelo em pé e o medo redobrado tanto por causa dos
horrores do livro, quanto pelo perigo de ser pego de surpresa, de modo
que não entendi nada de toda a história.

``O diabo não consegue escrever coisas tão diabólicas'', dizia Heinrich
Heine sobre os contos de Hoffmann. De fato: com o assombroso,
fantasmagórico e amedrontador desses textos anda de mãos dadas algo
satânico. E se tentarmos ir atrás disso, certamente conseguimos passar
da resposta do ``porquê'' dos contos de Hoffmann ao seu misterioso
``para quê''. Como se sabe, o diabo possui, ao lado de muitas outras
peculiaridades, também aquelas da esperteza e do saber. Ora, quem
conhece um pouco os contos de Hoffmann, logo vai entender quando digo
que, nessas histórias, o narrador sempre é um rapaz extremamente
sensível que detecta os espíritos por trás de seus disfarces mais sutis.
É com uma certa teimosia que esse narrador insiste em mostrar que todos
esses honrosos arquivistas, conselheiros de medicina, estudantes,
vendedoras de maçãs, músicos e filhas de boa família não são aquilo que
aparentam, da mesma maneira que o próprio Hoffmann não era o magistrado
meticuloso e pedante, do qual tirava seu ganha-pão. Em outras palavras:
as figuras assombrosas e fantasmagóricas que aparecem nas histórias de
Hoffmann não foram simplesmente inventadas pelo narrador consigo mesmo
no silêncio de seu quarto. Como no caso de muitos grandes escritores,
ele não via o extraordinário pairando livremente no espaço, mas em
pessoas, coisas, casas, objetos, ruas etc. bem definidos. Talvez vocês
saibam que a gente chama de fisiognomonistas aqueles que veem o caráter,
a profissão, ou até mesmo o destino de outras pessoas no rosto, no
andar, nas mãos ou na forma da cabeça destas. Por isso, Hoffmann era
menos um vidente (\emph{Seher}) do que um observador (\emph{Anseher}),
que é a tradução certa para fisiognomonista (\emph{Physiognomiker}) em
alemão. Um dos seus principais objetos de observação era Berlim, a
cidade e as pessoas que moravam nela. Na introdução à ``Casa
abandonada'', que, na verdade, era uma casa na avenida \emph{Unter den
Linden}\footnote{\emph{Unter den Linden} (Sob as Tílias) é a avenida
  central de Berlim. [\versal{N}. \versal{T}.]}, ele fala, com certo humor amargo,
do sexto sentido que lhe fora dado, isto é: do dom de enxergar em
qualquer fenômeno, seja ele uma pessoa, uma ação ou um acontecimento,
aquela extravagância à qual nós, na nossa vida cotidiana, normalmente
não temos acesso algum. Era sua paixão flanar pelas ruas, contemplar as
figuras que encontrava e, quem sabe, fazer o horóscopo delas. Durante
dias inteiros, ele andava atrás de pessoas que lhe eram desconhecidas,
mas que tinham algo estranho no andar, na sua maneira de se vestir, no
tom de sua voz ou no olhar. Ele se sente em contato permanente com o
suprassensível e, ao invés de perseguir o mundo dos espíritos, é esse
mundo que o persegue. Nessa Berlim tão racional, esse mundo põe-se no
seu caminho ao meio-dia, vai atrás dele em meio ao barulho da Rua König
(\emph{Königstraße}), até chegar ao pouco que restou da Idade Média nas
proximidades da Prefeitura em ruínas, o faz sentir um cheiro misterioso
de rosas e cravos na Rua Grün (\emph{Grünstraße}) e enfeitiça para ele o
ponto de encontro da alta sociedade, \emph{Unter den Linden}. Poderíamos
chamar Hoffmann o pai do romance berlinense, cujos rastros se perderam
mais tarde em generalidades, quando passaram a chamar Berlim de
``capital'', o zoológico de ``parque'', o Spree de ``rio'', até surgir
novamente em nossos dias -- basta pensar em \emph{Berlin
Alexanderplatz}, de Döblin. Uma das suas figuras, entre as quais ele
imagina a si mesmo, diz a outra: ``Você teve um motivo bem definido para
localizar a cena em Berlim e para mencionar ruas e praças. A meu ver,
não há mal nenhum em designar o cenário com exatidão. Além de conferir
ao todo uma aparência de verdade histórica que estimula uma imaginação
preguiçosa, ele também ganha consideravelmente em vivacidade e frescor,
especialmente para alguém que o conheça.''

Certamente eu conseguiria contar para vocês muitas histórias em que
Hoffmann dá provas de sua qualidade de fisiognomonista de Berlim. Eu
poderia nomear as casas que aparecem na sua obra, a começar pelo próprio
apartamento na Rua Charlotte (\emph{Charlottenstraße}), esquina com a
Rua Taube (\emph{Taubenstraße}), até a ``Águia de Ouro'' na Praça
Dönhoff, e Lutter \& Wegner na Rua Charlotte etc. Mas acredito que seja
mais vantajoso investigar a maneira como Hoffmann analisava Berlim e a
impressão que disso restou em seus contos. Ele nunca foi muito amigo da
solidão, nem da natureza livre. O homem, a comunicação com ele, as
observações sobre ele, o mero olhar para as pessoas valiam mais para
Hoffmann do que qualquer outra coisa. Quando dava um passeio no verão,
algo cotidiano na parte da tarde, quando o tempo estava bom, ele sempre
ia para alguma praça pública para encontrar pessoas. Mesmo estando a
caminho, dificilmente havia uma taberna ou uma confeitaria onde não
entrava para ver se tinha pessoas e quem seriam. Mas não se tratava
apenas de procurar nesses lugares por rostos novos que lhe dessem ideias
curiosas: a taberna era antes para ele uma espécie de laboratório
literário, uma sala de experimentos onde testava todas as noites com
seus amigos o enredo emaranhado e os efeitos de seus contos. Pois
Hoffmann não era um autor de romances, mas um contador de histórias e,
até nos livros, muitas delas, talvez a maioria, apresentam um personagem
em cuja boca ele as coloca para serem contadas. No fundo, claro, esse
contador sempre é o próprio Hoffmann, que se senta com seus amigos à
mesa, onde cada um, em sequência, conta o que tem de melhor. Por isso,
um deles nos diz expressamente que Hoffmann nunca foi ocioso na taberna,
como tantas outras pessoas que vemos sentadas, só bicando seu vinho e
bocejando. Pelo contrário: ele ficava olhando à sua volta com seus olhos
de falcão; tudo que observava de ridículo, de diferente e até de
comovente nos clientes da taverna se transformou em um estudo para suas
obras, ou então o colocava -- Hoffmann era um desenhista muito hábil --
com sua pena expressiva no papel. Mas aí quando as pessoas que se
encontravam na taverna não lhe eram agradáveis; quando cabeças limitadas
e mesquinhas na roda o perturbavam! Nesses casos, ele deve ter sido
totalmente insuportável e fazia uso inteiramente temível de sua arte de
fazer caretas, causar constrangimento e assustar pessoas. O cúmulo do
horror, entretanto, eram para ele as chamadas tertúlias estéticas, que
estavam na moda nessa época em Berlim e que eram reuniões de pessoas
espirituosas, porém ignorantes e irrefletidas, muito presunçosas quanto
ao seu interesse por arte e literatura. Uma dessas tertúlias ele
descreveu de forma muito graciosa em suas \emph{Fantasias}.

Chegando agora ao fim, não queremos que alguém nos critique por ter
esquecido a pergunta pelo para quê. Tão pouco a esquecemos que, de forma
imperceptível, até já a respondemos. Para que Hoffmann escreveu essas
histórias? Certamente ele não perseguia conscientemente determinados
fins com essas histórias. Contudo, podemos lê-las como se ele tivesse se
proposto a tais fins. E esses fins não têm como ser outros a não ser
fisiognomônicos, ou seja, para mostrar que essa Berlim rasa, sóbria,
esclarecida e sisuda não reside apenas em seus cantos medievais, ruas
afastadas e casas abandonadas, mas também em seus habitantes
trabalhadores de todas as classes e em todos os bairros cheios de coisas
que atraem um contador de histórias e que basta seguir-lhes os rastros e
observá-las. E, como se Hoffmann realmente quisesse ensinar isso ao seu
leitor com sua obra, uma de suas últimas histórias, que ditou no leito
de morte, na verdade não é outra coisa senão um desses ensinamentos do
olhar fisiognomônico. Essa história se chama \emph{A janela de esquina
do meu primo}.\footnote{Há edição brasileira. Cf. \versal{E.T.A.} Hoffmann,
  \emph{A janela de esquina do meu primo}.Tradução de Maria Aparecida
  Barbosa. São Paulo: Cosac Naify, 2010. [\versal{N}. \versal{E}.]} O
primo é o próprio Hoffmann, a janela é a janela de esquina do seu
apartamento que dava para a feira do Gendarm (\emph{Gendarmenmarkt}). Na
verdade, essa história é um diálogo. Hoffmann, paralítico, sentado em
sua cadeira de braços, olhando para a feira semanal lá embaixo, instrui
seu primo, que estava de visita, sobre como se poderia rastrear na
roupa, na velocidade e nos gestos das vendedoras e de suas freguesas uma
série de coisas, e fantasiar e inventar mais coisas ainda. E depois de
termos dito tanto em homenagem a Hoffmann, gostaríamos de deixar
registrado por fim algo que os berlinenses nem suspeitam: ou seja, que
ele foi o único escritor que tornou Berlim famosa no exterior e que os
franceses o amavam e liam numa época em que na Alemanha e também em
Berlim nem um cachorro aceitaria um pedaço de pão da sua mão. Hoje isso
mudou, há uma grande quantidade de edições acessíveis e também mais pais
que em meu tempo que permitem aos seus filhos a leitura de Hoffmann.

\part{CONTO E CRÍTICA}

\chapterspecial{Conversa assistindo ao corso}{Ecos do carnaval de Nice\,\footnote[*]{``Gespräch über dem
  \emph{Corso}. Nachklänge vom Nizzaer Karneval'', in \versal{GS} \versal{IV}-2, pp.
  763-771. Tradução de Georg Otte. Texto publicado no \emph{Frankfurter
  Zeitung} de 24 de março de 1935 sob o peseudônimo de Detlef Holz.
  [\versal{N}. \versal{E}.]}}{}

Era terça-feira de carnaval em Nice. Em silêncio, eu já havia virado as
costas para o carnaval e dava um passeio até o porto para descansar das
impressões do dia anterior, assistindo ao movimento há muito costumeiro
que sempre acompanha a chegada e a partida dos navios. Meio sonhando,
acompanhava o trabalho dos estivadores que descarregavam o
\emph{Napoleão Bonaparte}, de Ajaccio, quando um tapa no ombro me
surpreendeu.

--- Que coincidência feliz encontrá-lo por aqui, Doutor! Queria mesmo
localizá-lo de qualquer maneira. Quando perguntei pelo senhor no hotel,
já tinha saído.

Era o meu velho amigo Fritjof, que mora em Nice há anos e que toma conta
de mim nas minhas raras aparições nessa cidade, assim como faz com os
forasteiros, aos quais, quando lhe são simpáticos, mostra a cidade velha
e os arredores.

--- Pois, há alguém esperando pelo senhor, ele explicou depois de nos
cumprimentarmos.

--- Mas onde? --- perguntei um tanto desconfiado. Quem?

--- Na cafeteria \versal{M}., no \emph{Casino Municipal}, como o senhor sabe, onde
se tem a melhor vista para o corso.

Como já disse, eu estava pouco interessado na vista para o corso. Mas a
descrição que Fritjof me dera de um amigo dinamarquês, ao qual havia
prometido me apresentar e que foi o motivo por ter me procurado,
deixou-me curioso.

--- Ele é escultor, disse, um velho companheiro de viagem. Encontrei-o em
1924 na ilha de Capri, em 1926 na de Rodes, em 1927 na de Hiddensee e a
última vez na de Formentera. Ele pertence à estranha espécie de pessoas
que passa a maior parte de sua vida em ilhas e nunca se sente em casa
quando está no continente.

--- No caso de um escultor, esse tipo de vida me parece duplamente
surpreendente, disse eu.

--- Escultor, disse meu amigo, é assim que o chamei. Mas certamente não
se trata de um escultor qualquer. Não acredito que, em algum momento,
ele tenha assumido um serviço. Seus recursos lhe permitem uma vida muito
independente. Aliás, nunca vi qualquer obra dele. Mas, onde o encontrei,
todos falavam dele, principalmente os nativos. Circulava o boato de que
esculpia suas obras diretamente no rochedo, ao ar livre e em regiões
montanhosas distantes.

--- Um artista da natureza, por assim dizer?

--- Na ilha de Rodes o chamavam de ``o bruxo''. Não deve ser tão sério
assim. Mas, com certeza, trata-se de um excêntrico. Aliás, o melhor
mesmo é fingir que não sabe a profissão dele, pois não gosta de falar
sobre si mesmo. Desde que o conheci, há dez anos, só me lembro de uma
única conversa em que tocou nesse assunto. Na época, entendi pouca
coisa, mas ficou claro que tudo que fazia tinha que beirar o gigantesco.
Não sei bem o que achar disso, mas parece que as formações rochosas o
inspiram. Mais ou menos como nos tempos antigos, quando inspiravam a
fantasia dos camponeses ou dos pescadores, que viam nelas deuses,
pessoas ou demônios.

Havíamos atravessado a Praça Massena, que, nesse dia do último cortejo,
se mostrava livre de qualquer tráfego profano e que estava pronta para
receber os carros alegóricos que vinham das ruas laterais.

No primeiro andar da cafeteria, o dinamarquês acenou de uma mesa. Ele
era um homem pequeno, magro, porém com um aspecto agradável, com o
cabelo encaracolado e ligeiramente ruivo. A informalidade da
apresentação provavelmente era proposital da parte de Fritjof, e logo
estávamos sentados em poltronas confortáveis, cada um com seu copo de
Whisky. Um vendedor de jornais com um chapéu pontudo de palhaço dava a
volta pela cafeteria.

--- Cada carnaval tem seu mote, Fritjof explicou, \emph{Le cirque et la
foire} é o mote deste ano, ``A Feira e o Circo''.

--- Nada inábil, disse eu, aproximar as diversões do carnaval àquelas
mais populares.

--- Nada inábil, o dinamarquês repetiu, mas, mesmo assim, talvez um tanto
inadequado. ``A feira e o circo'' -- certamente são coisas próximas ao
ambiente carnavalesco. Mas não seriam próximas demais? O carnaval é um
estado de exceção, descendente dos dias das saturnais, quando o mais
inferior se transmudava em superior e o escravo era servido pelo senhor.
O estado de exceção, portanto, só se destaca claramente em oposição ao
estado normal das coisas. Isso certamente não vale para a feira. Eu
teria preferido outro mote.

--- De onde o senhor tiraria esse mote? --- Fritjof perguntou. Para onde
for, o senhor esbarra no extraordinário, que se transformou no nosso
prato de todos os dias. Sem falar das nossas condições sociais e
econômicas. Basta olhar para o mais próximo: veja o tinhoso lá embaixo,
com seu lápis de um metro atrás da orelha, que representa o ``Cronista
da Feira''. Ele não teria mais semelhança com um boneco de propaganda de
uma fábrica de lápis? Boa parte dessas criaturas gigantescas não
passaria a impressão de que saíram do centro iluminado de uma loja de
departamentos para se juntar a um cortejo de carnaval? Olhe apenas para
o grupo de carros que se aproxima vindo da esquerda! O senhor deve
admitir que representaria muito bem um exército na campanha publicitária
de uma fábrica de sapatos.

--- Aliás, não entendo o que esse grupo pretende representar, falei.

Uma série de carrinhos de mão se aproximava, cada um carregando uma
figura superdimensionada; uma como a outra estava deitada de costas,
esticando uma perna para o alto. Era a única perna que tinham, e no seu
topo havia um pé deformado, largo e chato. Não dava para ver se estava
calçado ou descalço, daquela distância.

--- Também não saberia o que achar delas, respondeu o dinamarquês, se não
tivesse tido contato com eles por acaso, no ano passado em Zurique. Eu
estava fazendo estudos na biblioteca e conheci a famosa coleção de
gravuras avulsas, que é um dos tesouros daquela biblioteca. É de lá que
conheço esses seres fabulosos. Pois é disso que se trata: seres
fabulosos que eram chamados de ``sciapodes'': pés-guarda-sóis. Sabia-se
que viviam no deserto, protegendo-se do sol incandescente com seu único
pé gigantesco. Na Idade Média, provavelmente eram apresentados também
nas feiras -- ou melhor: prometiam apresentá-los aos curiosos, junto com
as pessoas desfiguradas e os milagres da natureza.

Aos poucos, os carros passavam embaixo, seguidos por carros maiores,
puxados por cavalos. Havia o ``carro lotérico'' que se movia sobre seis
rodas da sorte; o ``carro do domador de peixes'', que fazia dançar seu
chicote por cima de pequenas baleias e peixes ornamentais gigantescos de
papel machê; o ``carro do tempo'', que, puxado por uma mula esquálida e
guiado por Cronos, mostrava as estações do verão e do inverno, da Europa
central e ocidental, na forma de senhoras farta e simbolicamente
vestidas.

--- Hoje, disse Fritjof virado para o dinamarquês, esses carros são
apenas estrados em rodas. Mas o senhor deveria tê-los visto antes de
ontem, quando se transformavam em baluartes. Entrincheirada atrás dos
bonecos monstruosos, a tripulação travava uma batalha contra o público
--- contra os meros ``espectadores'', que, nessas ocasiões, se tornam
alvo de todo o rancor que, no decorrer dos dias e dos anos, os
eternamente relegados despertam naqueles que se engajam -- mesmo que
seja apenas como figurantes do carnaval.

--- Os carros em si, disse o dinamarquês com um ar pensativo, simbolizam
alguma coisa. Os carros evocam a ideia do longínquo; é ela, ao meu ver,
que lhes confere seu poder, sua magia, e que um charlatão qualquer sabe
explorar tão bem, quando, para vender seu remédio contra a calvície ou
seu elixir da vida, monta sua venda num carro. Pois o carro é algo que
vem de longe. Tudo que vem de longe, tem um certo mistério.

Ao ouvir essas palavras, tive que pensar num pequeno livro curioso que,
pouco tempo antes, havia encontrado num sebo de Munique, debaixo de uma
pilha de outros livros que diziam respeito às técnicas de transporte e
vinham da administração de uma antiga cavalariça. Era intitulado \emph{O
carro e suas formas no passar dos tempos}. Eu tinha comprado esse livro
por causa de suas gravuras e de seu formato atraente, e raramente o
deixava para trás. Nesse momento, ele também estava comigo. Cansado de
assistir ao espetáculo embaixo, recostei-me na minha poltrona e comecei
a folheá-lo.

Havia reproduções de todo tipo de carro e carruagens e, em um anexo, até
o carro naval -- o \emph{carrus navalis}, do qual muitos acreditam
derivar a origem da palavra tão discutida ``carnaval''. Certamente, essa
derivação deve ser levada mais a sério do que a etimologia caseira dos
monges, que via na palavra uma alusão à Quaresma e a lia como
``\emph{carne vale}!'', ``adeus carne!'' Mais tarde, quando se analisava
as coisas mais de perto, lembraram-se do velho costume de rebatizar as
embarcações em cortejos solenes antes de colocá-las de novo na água após
as tempestades de inverno, e é assim que se descobriu o carro naval
latino.

Fritjof, que estava encostado no peitoril da janela junto com o
dinamarquês, gritava de vez em quando algumas palavras na minha direção:
os nomes das máscaras que passavam embaixo e que havia lido no programa
do cortejo. Algumas das figuras fantásticas que passei a imaginar,
sentado na frente do meu copo, provavelmente podiam concorrer com as que
ficavam balançando lá embaixo -- sem falar do fato que não estavam
deformadas por um número costurado nas costas como aqueles. Assim,
entreguei à minha fantasia a tarefa de imaginar alguma coisa a respeito
do ``domador domado'', dos ``cangurus boxeadores'', do ``vendedor de
castanhas'' ou da ``dama do Maxim'', até que uma música estridente
de uma banda de metais me assustou.

Ela anunciava a chegada do carro-chefe com o Príncipe do Carnaval.

Para fazer jus ao mote do ano, haviam vestido o boneco gigantesco com um
uniforme de domador. Em suas costas, um leão havia encostado suas patas
dianteiras -- o que não o impediu de maneira alguma de dar um sorriso
com seus 32 dentes. Era o sorriso familiar do velho Quebra-Nozes. Mas
senti subitamente a tentação de seguir seus rastros até o canibal
quebra-ossos dos meus livros de infância, que também sorria mostrando
sua dentadura inteira enquanto saboreava sua refeição.

--- O sorriso exagerado desse boneco não é repugnante? -- perguntou
Fritjof a mim, apontando para o boneco cuja cabeça simplória acenava
chegando até a altura da minha poltrona.

--- O exagero, respondi, parece-me ser a alma das figuras carnavalescas.

--- O exagero, replicou o dinamarquês, só nos é repugnante em alguns
casos porque não temos força suficiente para apreendê-lo. A rigor, eu
deveria ter dito que ``não temos inocência suficiente''.

Lembrei das excentricidades que o meu conhecido havia me contado sobre o
trabalho do dinamarquês. Não foi, portanto, sem alguma esperança de
despertar nele a discordância quando, do modo mais natural possível,
disse: ``O exagero é necessário; somente ele se faz crível aos bobos e
desperta a atenção dos desatentos.''

--- Não, disse o dinamarquês, e vi que o havia atingido, as coisas não
são tão simples assim. Ou será que deveria dizer ``são mais simples do
que isso''? Pois faz parte da natureza das coisas, mesmo quando não se
trata de coisas cotidianas. Da mesma maneira que existe um mundo de
cores além do espectro visível, há um mundo de criaturas além dos seres
que conhecemos. Qualquer tradição popular as conhece.

Nesse meio tempo, ele havia se aproximado de mim e, sem interromper sua
fala, sentou-se ao meu lado.

--- Pense nos gigantes e nos anões. Se há uma maneira do corpóreo poder
simbolizar algo espiritual, nada mais significativo do que nessas
criaturas das lendas populares. Há duas esferas de inocência completa, e
estas ficam nas duas fronteiras onde a nossa estatura humana normal,
digamos, se altera em direção ao gigantesco ou ao minúsculo. Pois, tudo
o que é humano carrega culpa. Mas as criaturas gigantescas são
inocentes, e a grosseria de um Gargântua e de um Pantagruel, que, aliás,
fazem parte da dinastia dos príncipes carnavalescos, é apenas uma prova
excessiva disso.

--- E a essa inocência corresponderia, disse eu, a inocência da pequenez?
Isso me faz lembrar a ``Nova Melusina'', de Goethe\footnote{Há tradução
  para o português. Cf. Johann Wolfgang Goethe, \emph{A nova Melusina.
  Novela ou a história de uma caçada}. Tradução de Anneliese Mosch.
  Sintra: Colares Editora, 1997. [\versal{N}. \versal{E}.]}, aquela princesa na
caixinha, cujo esconderijo, assim como seu canto mágico e sua natureza
ínfima, sempre me pareceu incorporar da forma mais perfeita o reino da
inocência -- da inocência infantil, quero dizer, que certamente é
diferente daquela do império dos gigantes.

--- Olhem só, Fritjof nos interrompeu da janela, esse grupo inverossímil!

De fato, era um carro estranho que estava passando, ao cair do sol,
diante das arquibancadas. Na frente de uma parede ou de um biombo, nas
quais tinham pendurado algumas pinturas, estavam pintores com paleta e
pincel, e parecia que estavam prestes a terminar sua obra. Mas um grupo
de bombeiros, com a mangueira na mão, os acuou e ameaçou inundar as
obras-primas e seus criadores.

--- Não faço a mínima ideia do que poderia se tratar, confessei.

--- É o ``\emph{Car des pompiers}'', disse Fritjof. Eles chamam um pintor
acadêmico e arrogante de \emph{pompier}, que também é a palavra para
``bombeiro''. Um jogo de palavras em cima de um carro -- pena que é um
caso isolado.

Nesse momento, ainda antes de escurecer, as fachadas em torno da Praça
Massena começaram a se iluminar. A grade, com a qual a haviam cercado e
que estava enfeitada com todo tipo de símbolos da feira e do circo, com
peças de madeira recortadas e pregadas, de repente foi engolida pelo
fogo de lanternas multicores. Onde antes havia um leão, agora tinha uma
jaula com luzes amarelas e em sua silhueta duas lanterninhas
avermelhadas insinuavam os movimentos da boca da fera. E a menina de
madeira, que antes parecia tomar conta de uma barraca, agora tinha se
transformado numa imagem da deusa Astarte. Mais estranho, entretanto, do
que o jogo das luzes na fachada era o que elas tinham a dizer à própria
Praça, pois eram elas que lhe proporcionavam seu verdadeiro destino.
Logo ficou claro que ela fazia parte daquela série grande e nobre de
praças europeias, em forma de sala, que começaram a surgir na Itália e
graças às quais as festas italianas com seus \emph{corsi} e procissões
-- sem falar do carnaval -- se tornaram modelos para a Europa inteira.
Essas praças tinham como função não apenas abrigar, nos dias úteis, as
feiras e reuniões populares, mas de representar, nos dias de festa, a
sala -- uma sala solenemente iluminada debaixo do céu noturno, que não
devia nada ao palácio do duque, com seu revestimento e seu telhado feito
de materiais preciosos. Era uma praça dessas que estávamos olhando
agora, calados.

Depois de um longo intervalo, o dinamarquês dirigiu-se a mim.

--- O senhor falou antes do mundo do delicado e do diminuto que Goethe
trabalhou na ``Nova Melusina'', e achou que esse, ao contrário do mundo
dos gigantes, seria o mundo em que morasse a inocência infantil. Sabe
que tenho lá as minhas dúvidas? A inocência da criança, a meu ver, não
seria uma inocência humana, se não fizesse parte dos dois reinos, tanto
dos gigantes quanto dos anões. Não pense apenas na delicadeza e em quão
comovente é a cena de crianças fazendo uma hortinha na areia ou
brincando com um coelho. Pense também no outro lado -- na grosseria, no
desumano que predomina nos seus livros infantis mais famosos e que não
apenas fez o sucesso de \emph{Max e Maurício} ou do \emph{João
Felpudo},\footnote{\emph{Max und Moritz} (1865), do escritor e
  desenhista Wilhelm Busch, e \emph{Struwwelpeter} (1845), do médico e
  psiquiatra Heinrich Hoffmann, tornaram-se os livros infantis mais
  populares da Alemanha. [\versal{N}. \versal{T}.] Há edições brasileiras. Em se
  tratando do primeiro, cf. Wilhelm Busch, \emph{As travessuras de Juca
  e Chico}. Tradução de Claudia Cavalcanti. São Paulo: Iluminuras, 2012.
  Veja-se, igualmente, a conhecida tradução de Olavo Bilac,
  originalmente publicada em 1901 e muitas vezes reeditada, a exemplo de
  \emph{Juca e Chico. História de dois meninos em sete travessuras}. São
  Paulo: Pulo do Gato, 2012. Quanto ao segundo, cf. Heinrich Hoffmann,
  \emph{João Felpudo, ou histórias divertidas com desenhos cômicos do
  Dr. Heinrich Hoffmann}. Tradução de Claudia Cavalcanti. São Paulo:
  Iluminuras, 2011, para apenas referirmos uma edição recente. [\versal{N}.
  \versal{E}.]} mas também os tornou úteis. Pois essas características se
apresentam nesses livros em sua inocência. Quero chamá-las de
``canibalescas'', que o senhor também pode detectar nos lábios do
Príncipe Carnaval. O maravilhoso das crianças é que elas conseguem
migrar, sem cerimônia, entre os dois reinos extremos do humano,
permanecendo num ou noutro, sem fazer a mínima concessão ao reino
contrário. Provavelmente é esse descomprometimento que perdemos mais
tarde. Somos muito bem capazes de nos inclinar para o mundo minúsculo,
mas não conseguimos mais nos perder nele. E podemos nos divertir com o
mundo gigantesco, mas sempre com um certo embaraço. As crianças podem
ser tímidas no contato com os adultos, mas se movem entre esses gigantes
lá embaixo como se fossem seus pares. Para nós adultos, porém, pelo
menos uma vez ao ano, o Carnaval deveria ser a oportunidade para nos
comportarmos um pouco como os gigantes -- mais desinibidos e ao mesmo
tempo mais honestos do que somos, dia vai, dia vem.

Um foguete subiu ao céu; um tiro de canhão foi disparado: o sinal para a
queima do 57.º Príncipe Carnaval, de cuja fogueira a última centelha
tinha que se extinguir antes de começar a Quarta-Feira de Cinzas.

\chapterspecial{A mão de ouro}{Uma conversa sobre o jogo\,\footnote[*]{``Die glückliche
  Hand. Eine Unterhaltung über das Spiel'', in \versal{GS} \versal{IV}-2, pp. 771-777.
  Tradução de Georg Otte. Texto provavelmente escrito logo depois da
  publicação de ``Conversa assistindo ao corso'', como uma continuação
  para o \emph{Frankfurter Zeitung}. O manuscrito foi assinado com o
  pseudônimo de Detlef Holz. [\versal{N}. \versal{E}.]}}{}

--- Pois é, a gente tem que ter mão de ouro, falou o dinamarquês.

--- Eu poderia contar-lhe uma história\ldots{}

--- Nada de histórias! --- interrompeu o dono do hotel. Quero saber sua
opinião: o senhor acha que, no jogo, tudo depende do acaso ou ainda há
mais alguma outra coisa?

Éramos quatro. Meu velho amigo Fritjof, o romancista; o escultor
dinamarquês, que Fritjof havia me apresentado em Nice; o sabido e
viajado proprietário do hotel, em cujo terraço tomávamos o chá da tarde
-- e eu. Não me lembro mais como o assunto do jogo surgiu. Mal
participava da conversa, pois estava me entregando ao sol da primavera e
ao prazer de ter encontrado aqui, em Saint-Paul, longe de tudo, os meus
amigos de Nice.

A cada dia que passava entendia melhor por que Fritjof havia escolhido
esse lugar para retomar o trabalho no seu romance, que não tinha saído
do lugar enquanto esteve em Nice. Pelo menos foi isso que deduzi do
sorriso indefinido com que respondera naquela cidade semanas atrás a
minha pergunta pelo romance: ``Perdi minha caneta tinteiro''.

Logo depois eu parti; tanto maior foi a minha alegria de rever Fritjof e
seu companheiro dinamarquês aqui -- uma alegria misturada com uma
surpresa: será que Fritjof, um pobre diabo, havia conseguido mesmo se
alojar nesse confortável hotel?

Agora estávamos sentados nesse pequeno refúgio, deixando, enquanto
conversávamos, pairar os nossos olhares sobre os sinais de bandeiras
balançando com o vento em um varal, estendido acima do portal da cidade
ou das árvores escalonadas no vale.

--- Se o senhor quiser ouvir a minha opinião, disse o dinamarquês, nada
depende das coisas que comentamos até agora. Não depende do capital de
jogo, nem dos chamados ``sistemas'', nem do temperamento do jogador --
antes da falta de temperamento.

--- Agora não entendo mesmo o que o senhor quer dizer com isso.

--- Se tivesse acompanhado pessoalmente o que vivenciei no mês passado em
San Remo, o senhor me entenderia imediatamente.

--- Então? -- perguntei intrigado.

--- Eu entrei, começou a contar o dinamarquês, tarde da noite no casino e
me aproximei de uma mesa, onde uma partida de bacará tinha acabado de
começar. Havia um lugar livre, que estava reservado, e os olhares que se
dirigiam a ele davam a entender que alguém estava sendo aguardado. Eu já
queria me informar sobre o cliente que parecia criar uma expectativa tão
grande, quando alguém ao meu redor mencionou seu nome. No mesmo
instante, apoiada pelo atendente do casino de um lado e pelo seu
secretário do outro, a Marchesina Dalpozzo se aproximou da mesa. Pelo
visto, o trajeto do carro até sua poltrona havia custado à velha senhora
muito esforço. Mal chegou e já se afundou na poltrona. Depois de um
tempo, quando o embaralhador de cartas chegou-se até ela e era sua vez
de gerir a banca, abriu sem pressa sua bolsa de mão para tirar uma
pequena matilha de cachorrinhos de porcelana, vidro e jade, seus
mascotes, que distribuiu em torno do si. Mais uma vez, reservando-se
todo o tempo do mundo, enfiou novamente a mão na bolsa e tirou um maço
de notas de mil liras. Deixou ao crupiê o trabalho de contá-las,
distribuiu as cartas, mas, mal passou a última, afundou-se novamente na
poltrona. Ela nem chegou a escutar o pedido de seu parceiro por mais uma
carta para melhorar as chances do seu jogo, pois havia adormecido. Neste
momento, o secretário entrou em cena para, com muito respeito, acordá-la
delicadamente com a mão, que estava visivelmente exercitada nesta
prática. Devagar, a Marchesina virou seus \emph{points} um após o outro.
``\emph{Neuf à la banque}'', disse o crupiê; ela havia ganhado. Mas isso parecia apenas fazê-la adormecer e, por mais que ela tenha aumentado a soma das notas de mil liras administrando a banca, quase não houve uma única vez em que o secretário não tivesse que exortá-la para a sorte.

--- A quem o ama, o Senhor concede o pão enquanto dorme.\footnote{Trata-se
  de uma expressão proverbial em alemão, cuja origem está nos Salmos, e
  que é utilizada ironicamente em referência a pessoas que acumulam
  lucros ``dormindo'', sem trabalhar: ``É inútil madrugar, deitar tarde,
  comendo um pão ganho com suor; a quem ama, o Senhor o concede enquanto
  dorme.'' (Sl 127, 2) [\versal{N}. \versal{T}.]}

--- Não seria melhor dizer: ``Satanás''? -- observou o dono do hotel
sorrindo.

--- Os senhores sabem, disse Fritjof sem dar qualquer resposta, que já me
perguntei algumas vezes por que o jogo é tão mal-afamado? Claro que não
há nenhum segredo nisso. Suicídios, desvios de dinheiro e o que mais
tiver a origem no jogo não faltam. Mas, como já disse, isso é tudo?

--- Há alguma coisa no jogo que é contra a natureza, disse o dinamarquês.

--- Eu já o vejo muito próximo da natureza, observei. Acho-o tão natural
quanto a incansável e inesgotável esperança na nossa felicidade.

--- Esta é a palavra-chave, respondeu o dinamarquês: ``Fé, amor --
esperança.'' E agora o senhor entende o que sobrou disso!

--- O senhor quer dizer que o assunto não é digno da esperança. ``Vão
dinheiro'' ou algo parecido -- se estou entendendo direito.

-- Mas ele não me entende direito, disse o dinamarquês, virando de
repente as costas para mim e dirigindo-se a Fritjof. O senhor já se
encontrou alguma vez, ele prosseguiu, olhando-o fixamente, no metrô ou
num banco do parque, na proximidade imediata de uma mulher atraente? Mas
realmente na proximidade mais imediata?

-- Quero lhe dizer uma coisa, disse Fritjof: se ela estiver sentada
mesmo muito perto, o senhor dificilmente vai poder fazer uma ideia dela.
Por quê? Porque, quando estivesse muito próximo, seria quase impossível
olhá-la. De qualquer forma, se fosse comigo, acharia isso uma falta de
vergonha.

-- Então o senhor vai me entender tanto melhor se eu retomar agora a
nossa questão: conversamos sobre a esperança. E comparei a esperança a
uma mulher desconhecida jovem e bonita, sendo que captá-la com o olhar,
ou mesmo abordá-la com o olhar, seria falta de pudor.

-- Como assim, perguntei, pois estava perdendo o fio da meada.

-- Eu estava falando da proximidade temporal, disse o dinamarquês. Quero
afirmar que faz muita diferença se cultivo um desejo para o futuro
longínquo ou para o momento. ``O que se deseja para a juventude,
ter-se-á em abundância na velhice'', dizia Goethe. Quanto mais cedo na
vida se tem um desejo, tanto maior a perspectiva dele se realizar\ldots{} Mas
estou divagando.

-- Provavelmente, o senhor estava querendo dizer, comentou Fritjof, que
aquele que aposta no jogo também tem um desejo.

-- Sim, mas um desejo que o próximo momento deve realizar. E é isso que
o transforma em mal-afamado.

-- É um contexto estranho, disse o dono do hotel, em que coloca o jogo.
E o contrário da bola de marfim na roleta, que acaba rolando em sua
casa, seria a estrela cadente que cai ao longe para realizar um desejo.

-- Sim, o desejo justo, que se dirige para algo longínquo, disse o
dinamarquês.

Ditas essas palavras, houve um silêncio. Mas essas palavras haviam
jogado uma nova luz no velho provérbio ``Azar no jogo, sorte no amor''.
Como se quisesse intervir nas minhas divagações, Fritjof, com ar
pensativo, comentou:

-- Uma coisa é certa: no jogo há estímulos que vão além do ganho.
Algumas pessoas não procurariam no jogo uma luta contra o destino? Ou
uma oportunidade de cortejá-lo? Acredite, no pano verde são acertadas
muitas contas às quais os outros nunca tiveram acesso.

-- Deve ser mesmo uma tentação muito grande pôr à prova a aceitação do
próprio destino.

-- Isso pode ter um fim bastante ambíguo, disse o dono do hotel.
Lembro-me de uma cena que observei em Montevidéu. Quando era jovem,
passei boa parte da minha vida por lá. Montevidéu possui o maior casino
do Uruguai; as pessoas viajam oito horas de Buenos Aires até lá para
passar o \emph{weekend} jogando. Uma noite, estive no casino para
assistir aos jogos. Por precaução, não levei dinheiro nenhum. Na minha
frente, havia dois jovens que jogavam com fervor. Faziam apostas
pequenas, porém frequentes. Mas não davam sorte, e logo um dos dois
tinha perdido tudo. O outro ainda tinha algumas fichas, que, no entanto,
não queria arriscar. Portanto, deram fim ao jogo deles, mas permaneceram
para observar o jogo dos outros. Assim, como muitos outros perdedores,
ficaram parados em silêncio e numa atitude humilde, quando aquele que já
não tinha mais nada se animou de repente e sussurrou ao seu amigo:
``Trinta e quatro!'' O outro se limitou a encolher os ombros. Mas, de
fato, deu trinta e quatro. O nosso vidente, cujo sofrimento,
naturalmente, era grande, fez outra tentativa. ``Sete ou vinte e
oito!'', murmurou para seu companheiro, que sorriu sem emoção. Mas deu
mesmo sete, e o outro começou a se irritar. Quase implorando, sussurrou
``Vinte e dois'', e o repetiu três vezes. Em vão: quando deu vinte e
dois, a casa estava sem aposta. Parecia inevitável que se fizesse uma
cena entre os dois. Mas, justo no momento em que o nosso homem dos
milagres, trêmulo de excitação, queria novamente se virar para o amigo,
este, para não ser mais uma entrave para a felicidade comum, passou-lhe
o restante das fichas. O amigo apostou no quatro. Deu quinze. Apostou no
vinte e sete. Deu zero. E apostou as últimas duas fichas e as perdeu de
uma vez só. Abatidos e reconciliados, os dois desapareceram.

-- Estranho, disse Fritjof. Podemos achar que o fato de segurar as
fichas na mão o teria privado do seu dom de vidente.

-- Da mesma maneira, explicou o dinamarquês, o senhor pode dizer que foi
o dom de vidente que o privou do lucro.

-- É um paradoxo um tanto volátil, repliquei.

-- De maneira alguma, respondeu. Se existe algo como um jogador feliz,
algo como um mecanismo telepático nos envolvidos com o jogo, isso está
no inconsciente. É o saber inconsciente que se põe em movimento quando o
jogador tem sucesso. Quando, porém, se desloca para a consciência, está
perdido para a inervação. Embora o nosso jogador vá ``pensar'' o certo,
``agirá'' errado. Vai ficar parado como muitos perdedores, arrancar os
cabelos e gritar ``Eu sabia!''

-- Na sua opinião, portanto, um jogador sortudo procede de forma
instintiva? Como uma pessoa no momento do perigo?

-- O jogo, o dinamarquês confirmou, é mesmo um perigo produzido
artificialmente. E jogar é, por assim dizer, uma maneira blasfema de pôr
a nossa presença de espírito à prova. Pois, no momento do perigo, o
corpo comunica-se diretamente com as coisas, sem passar pela cabeça. Só
quando estamos a salvo e respiramos aliviados, chegamos a nos dar conta
do que aconteceu. Enquanto estávamos agindo, estávamos à frente do nosso
saber. E o jogo é mal-afamado porque provoca de uma maneira
inescrupulosa aquilo que o nosso organismo tem de mais refinado e
preciso.

Deu-se um silêncio. ``Tem que ter mão de ouro\ldots{}'', pensei. O
dinamarquês não estava querendo contar uma história a respeito?
Lembrei-o disso.

-- Ah, sim, a história, disse ele sorrindo. Na verdade, já é meio tarde
demais para ela. Aliás, conhecemos o herói dessa história. E todos nós
gostamos dele. Apenas quero revelar que é escritor, pois isso é
relevante, bem que\ldots{} mas aí já estou quase tirando a graça da história.

Resumindo: o homem estava decidido a tentar sua sorte na Riviera. Não
tinha noção nenhuma dos jogos de azar, tentou um ou outro sistema e
acabou perdendo com todos eles. Depois desistiu dos seus sistemas e
continuou perdendo. Logo seus recursos estavam esgotados, seus nervos
muito mais, e um dia, além de tudo, aconteceu-lhe de perder também sua
caneta tinteiro. Como os senhores sabem, os escritores às vezes são
excêntricos, e o nosso amigo pertence aos mais excêntricos de todos. Ele
precisa ter uma iluminação bem definida sobre sua escrivaninha, um papel
bem definido e um formato bem definido para suas folhas de papel; caso
contrário, ele não consegue trabalhar. Agora os senhores podem
facilmente imaginar o que significava para ele a perda de uma caneta.
Depois de desperdiçar o dia inteiro à procura de uma nova, demos um pulo
no casino. Como nunca jogo, contentei-me em acompanhar o jogo do nosso
amigo. Em pouco tempo, não era apenas eu que o acompanhava, mas esse
homem chamou a atenção de muitos frequentadores do casino, pois ganhava
todos os jogos. Depois de uma hora, partimos para garantir, pelo menos
para aquela noite, o nosso patrimônio líquido. E o dia seguinte também
não deu prejuízo, pois tanto mais passamos em vão a manhã em papelarias,
tanto mais a noite resultou lucrativa. Claro que não se falava mais no
romance, desde que a caneta tinha desaparecido. O nosso amigo,
normalmente um homem assíduo, nem olhava mais para o manuscrito e deixou
de escrever até as cartas mais curtas. Quando o lembrava de alguma
correspondência urgente, ele se esquivava. Os seus apertos de mão se
tornaram escassos; ele evitava carregar o mais leve pacote e mal tinha
força para virar as páginas durante suas leituras. Era como se sua mão
estivesse descansando numa bandagem que só era tirada de noite -- no
casino, onde nunca ficamos por muito tempo. Já tínhamos juntado uma bela
soma, quando, um dia, o porteiro do hotel trouxe a caneta. Alguém a
havia encontrado entre as palmeiras do hotel. O nosso amigo deu uma boa
gorjeta ao porteiro, e no mesmo dia partiu para finalmente escrever seu
romance.

-- Uma bela história, disse o dono do hotel, mas ela prova o quê?

Eu nem queria saber o que a história provava ou não provava, mas me
regozijava em ver meu velho amigo Fritjof, para quem a sorte sorria
muito pouco, feliz da vida na muralha de Saint-Paul, tomando seu chá da
tarde.

\chapterspecial{Jakob Job, \emph{Nápoles. Imagens de viagem e esboços\footnote[*]{``Jakob Job, \emph{Neapel. Reisebilder und
  Skizzen}. Zürich: Rascher u. Cie. \versal{A}.-\versal{G}. 1928. 255 \versal{S}., 32 Abb.'', in \versal{GS}
  \versal{III}, pp. 132-135. Tradução de Georg Otte. Resenha crítica publicada no
  periódico \emph{Die Literarische Welt} de 20 de julho de 1928. [\versal{N}.
  \versal{E}.]}}}{\begin{center}\footnotesize{Zurique: Rascher \& Cia. \versal{A}.-\versal{G}. 1928. 255 p. 32 figuras}\end{center}}{}
% \footnote
% {\emph{Zurique: Rascher \& Cia. \versal{A}.-\versal{G}. 1928. 255 p. 32 figuras} {``Jakob
%   Job, \emph{Neapel. Reisebilder und Skizzen}. Zürich: Rascher u. Cie.
%   \versal{A}.-\versal{G}. 1928. 255 \versal{S}., 32 Abb.'', in \versal{GS} \versal{III}, pp. 132-135. Tradução de
%   Georg Otte. [\versal{N}. \versal{E}.]}\footnote{Resenha crítica publicada no
%   periódico \emph{Die Literarische Welt} de 20 de julho de 1928. [\versal{N}.
%   da \versal{O}.]} }{}

Amar Nápoles olhando do mar é fácil. No entanto, basta pôr o pé em
terra, descer do trem na estação labiríntica e incandescente, andar num
carro destrambelhado, atravessando nuvens de poeira num pavimento que
descansa tão pouco quanto o Vesúvio, chegando, em vão, numa casa de
hóspedes superlotada, para que a página daquela primeira impressão vire.
Acrescenta-se a isso as experiências do primeiro dia para ver que poucas
pessoas conseguem encarar, sem disfarce, a imagem dessa vida, dessa
existência sem sossego nem sombra. Quem não se privar, ao pisar nesse
solo, de tudo que for associado ao conforto, enfrenta uma batalha fadada
à derrota. Já para os outros, que encontram nessa cidade a face mais
suja, mas também mais passional e horrorizada, da qual jamais a pobreza
olhou para sua libertação, a lembrança dela se resume a uma Camorra.
Para tudo que se conta sobre carteiras furtadas, moças sequestradas e
camas infestadas de percevejos só lhes resta um sorriso impassível. Caso
estes últimos acolham o autor desse livro entre seus pares -- tanto amor
emparelhado com tão pouca compreensão traz certas simpatias por ele --,
então farão a inclusão dele nesta liga depender de um voto de silêncio
eterno, mesmo que seja pelo fato de seu alemão ser o mais incorreto que
se possa imaginar.

Passamos as páginas e encontramos títulos prometedores: ``Camaldoli'',
``Sorrento'', ``Dias de outono em Seiano'', ``Ravello''. Começamos a ler
e o livro, talvez, continue agradando. Pois o que está escrito é tão
irrelevante, tão comovente, tão seco quanto a folha prensada de uma
videira qualquer do Golfo. Podemos nos entregar a sonhos maravilhosos
enquanto estivermos com o livro em mãos. Leio ``Positano'' e me vejo de
novo na rua que faz suas curvas pela cidade. É de noite. Formamos um
pequeno grupo: Ernst Bloch, o filósofo, Tavolato, bom de copo, Alfred
Sohn-Rethel, um dos membros mais jovens da família do pintor
teuto-romano. A lua brilhava no céu, e era uma daquelas noites
meridionais em que sua luz não parece cair sobre o cenário de nossa
existência diurna, mas sim sobre uma terra oposta ou paralela\label{supra10}.\footnote{No
  original alemão, \emph{Gegen-Erde} e \emph{Neben-Erde}, neologismos
  criados por Benjamin. [\versal{N}. \versal{T}.]} Era outra Positano, que
atravessamos. As partes abandonadas dessa grande cidade se destacavam
mais daquelas onde hoje sobrevive o pequeno número de descendentes de
uma população que, nos velhos tempos, já tinha chegado a quarenta mil,
pois, na Idade Média, a cidade era enorme. Eu bem sabia das histórias
que circulavam por aqui, mas não apreciava as histórias penetrantes de
fantasmas, que sempre surgem quando um proletariado de intelectuais
vagantes encontra com nativos primitivos, seja aqui, em Ascona, seja em
Dachau. Portanto, certamente não foi a vontade de aprender a lidar com o
medo,\footnote{Alusão ao título do conto de fada dos irmãos Grimm,
  ``Jemand, der auszog, das Fürchten zu lernen'', ``Alguém que saiu de
  casa para aprender a ter medo''. [\versal{N}. \versal{T}.]} nem qualquer
interesse sério que tomou conta de mim, quando, de repente, pedi aos
meus companheiros esperarem por mim na rua para que eu desse alguns
passos morro acima e olhasse uma das casas abandonadas que, nesse
momento, se erguiam ao meu lado. Os degraus de pedra eram gigantescos;
sem pressa, subi um após o outro. Devo ter dado trinta passos grandes
dessa maneira; o silêncio era geral e, da rua, ouvia as vozes dos que
estavam à minha espera. Continuei sentindo vontade de prosseguir na
minha subida. Mas logo ficou mais difícil. Sentia como estava me
afastando dos amigos embaixo, apesar de permanecer bastante próximo a
eles, podendo ser ouvido e visto. Fui cercado por um silêncio e uma
solidão preenchidos de acontecimentos. Com cada passo, eu estava
avançando fisicamente para um acontecimento que não conseguia imaginar,
nem conceber e que não queria me tolerar. De repente, parei entre muros
e vãos de janelas, num mato espinhoso de sombras bem delineadas,
projetadas pela lua. Por nada nesse mundo eu teria dado um passo a mais.
Aqui, na presença dos meus companheiros totalmente despersonalizados,
vivi uma experiência do que significa aproximar-se de um círculo mágico.
Retornei.

Essa experiência não é uma simples curiosidade. Qualquer pessoa pode
fazê-la nesse lugar. Por isso, é duplamente necessário preservá-la de
clichês como: ``De noite, andamos numa escuridão assombradora. Parece
que, dos buracos estreitos, dos nichos apertados, das abóbadas
ressoantes, seres fabulosos nos assaltam.'' Essa é a Positano ``tal como
está no livro''. Evidentemente, a aldeiazinha de papel, que o autor
constrói para nós, também não sabe nada das energias que contribuíam
para a construção da famosa torre de Clavel.\footnote{Gilbert Clavel
  (1883-1927), artista plástico e escritor suíço, adepto do movimento
  futurista. [\versal{N}. \versal{T}.]} No próximo outono, fará um ano que esse
inesquecível excêntrico da Basileia morreu. Um homem que construiu sua
vida no interior da terra, que vivia criativamente nas fundações de sua
torre e que, no grande \emph{carrefour} dos tempos, dos povos e das
classes, que é o Golfo de Sorrento, sabia dar informações como poucos e
que diz mais, em sua breve carta,\footnote{Gilbert Clavel, Carta a Carl
  Albrecht Bernoulli, de 27/08/1927. \emph{Die Annalen}. \emph{Eine
  schweizerische Monatsschrift}, Horgen-Zürich, 1927, pp. 953-955. [\versal{N}.
  de \versal{W.B.}]} da paisagem à sua volta do que o livro inteiro tratado
aqui.

E mais, se o acervo de vivências e saberes é a condição de todos os
relatos de viagem, onde se encontraria, na Europa, um objeto melhor que
Nápoles, que de hora em hora transforma tanto o turista quanto o nativo
numa testemunha que assiste à união da superstição mais antiga com a
mais nova impostura para formar procedimentos úteis, dos quais é o
usufruidor ou a vítima. De que maneira incomparável eles se fundem nas
festas que essa cidade possui em número dez vezes maior, porque cada
bairro comemora seu próprio santo, convidando os outros bairros em seu
dia onomástico. Como seria fácil integrar na descrição dessas festas
conhecimentos sólidos e enriquecedores sobre as localidades e os
costumes da cidade -- um aspecto ao qual o leitor alemão, quase
cinquenta anos depois de Gregorovius e Hehn, teve que aprender a
renunciar. Até o livro em questão retira suas melhores páginas da
descrição das procissões solenes. Mas um único costume entre todos os
dessa festa não teria expresso mais do que a descrição meticulosa,
demasiadamente incolor e imparcial do milagre do sangue de São Januário?
Quando, ao chegar o dia, a multidão fica ajoelhada hora após hora,
rezando fervorosamente dentro e fora da catedral à espera do milagre,
então aqueles entre os napolitanos cuja árvore genealógica remonta à
família do santo, têm o direito de cobrar seus favores aos seus
apadrinhados, com maldições ruidosas e vociferações imperiosas, até o
acenar de um lenço do altar anunciar que o milagre aconteceu, que o
sangue se tornou líquido. Por que não ficamos sabendo da
\emph{Piedigrotta}, do barulhento culto orgiástico da noite do oito de
setembro e das festanças gigantescas para as quais os napolitanos pagam
semanalmente alguns \emph{soldi} em suas mercearias como os nórdicos
pagam o seguro de vida para, ao chegar a época, poder beber e comer além
da medida e além de suas condições. Tradicionalmente, o festim é
encerrado por um frasco de óleo de mamona. Mas o barulho pagão da noite
de \emph{Piedigrotta} se perpetua nas festas cotidianas que o napolitano
comemora com a técnica. Quando chega perto da realização do seu desejo
de adquirir uma motocicleta, experimenta todas que estão ao seu alcance
para ficar com a mais barulhenta. Nunca vou esquecer a inauguração do
metrô, que, durante dias, não podia ser utilizado porque todos os
guichês estavam ocupados pela rapaziada da cidade que superava o barulho
retumbante do trem se aproximando com uma gritaria mais retumbante ainda
e que, durante as viagens, fazia ecoarem os túneis com um uivo
estridente. E até o ``passeio no campo'', a viagem de carro em caravana
para Sto. Elmo ou subindo o Vomero tem que ser banhada em poeira e
estrépitos para proporcionar a devida alegria.

Para tudo isso, o autor do livro não abre as portas. Mesmo assim, aquele
leitor lhe agradecerá, pois, conduzido para o tema, é abduzido de si
mesmo como nesta resenha. E pensando por um momento nas suas fotos
excelentes, podemos nos juntar sem ironia a esse leitor.

\chapter{Anedotas desconhecidas de Kant\footnote[*]{``Unbekannte Anekdoten von
  Kant'', in \versal{GS} \versal{IV}-2, pp. 808-815. Tradução de Georg Otte. Em uma carta
  à Scholem, datada de 28 de outubro de 1931, Benjamin afirma trabalhar
  numa tentativa fisiognomônica de expor a relação entre o
  enfraquecimento mental de Kant na velhice e sua filosofia. Essas
  anedotas, que foram publicadas anonimamente no periódico \emph{Die
  Literarische Welt}, constituem provavelmente um resultado de tal
  projeto. [\versal{N}. \versal{E}.]}}

As seguintes anedotas são de textos que, como tais, não possuem nenhuma
relação com Kant, isto é, são de almanaques desaparecidos, revistas etc.
Apenas o sexto texto é exceção; ele foi tirado da coletânea
\emph{Immanuel Kant. Sua vida em testemunhos de
contemporâneos}\footnote{Cf. Ludwig Ernst von Borowski, Reinhold
  Bernhard Jachmann e Ehregott Andreas \versal{C}. Wasianski, \emph{Immanuel
  Kant. Sein Leben in Darstellungen von Zeitgenossen.} Organização de
  Felix Groß. Berlin: Deutsche Bibliothek, s. d. [1912]. [\versal{N}. \versal{T}.]}, que representa um verdadeiro tesouro não apenas para o
leitor de Kant, mas sobretudo para o fisiognomonista. Algumas dessas
histórias sinalizam a postura graças à qual os ensinamentos de Kant,
ainda antes de chegarem à sua completa penetração e apropriação
filosófica, foram percebidos como uma nova potência vital, da qual
ninguém tinha como escapar.

Walter Benjamin

\begin{enumerate}
\def\labelenumi{\Roman{enumi}.}
\item
  \emph{Anedotas desconhecidas}
\end{enumerate}

Uma história na qual Kant é breve

\emph{Seu fâmulo, um teólogo que não conseguia conciliar a filosofia com
a teologia, pediu o conselho de Kant para saber o que deveria ler.}

\emph{Kant: Leia relatos de viagem.}

\emph{O fâmulo: Há coisas na dogmática que não entendo.}

\emph{Kant: Leia relatos de viagem.}\bigskip

Uma história na qual Kant recorre a uma comparação

\emph{Numa conversa com um filósofo famoso de Königsberg, Kant acabou
falando do belo sexo.}

\emph{``Uma mulher'', disse Kant, ``tem que ser como o relógio de igreja
para fazer tudo pontualmente e no minuto certo, mas também não deve ser
como um relógio de igreja, para não espalhar todos os segredos em voz
alta. Ela tem que ser como um caracol, doméstica, mas também não deve
ser como um caracol, para não carregar tudo sobre o próprio corpo.''}\bigskip

Um relato que mostra que Kant, já que não conseguia ver nada de positivo
no casamento, pelo menos tirou dele uma bela expressão

\emph{``Vale a regra: não se deve casar, exceto um casal muito digno.''
É o final de um poema de casamento de Michael Richey, Hamburgo, 1741.
Kant gostava de citá-lo diversas vezes quando se tratava de falar de uma
exceção a ser considerada como raridade, independentemente da questão se
o assunto era o casamento ou o celibato ou ainda outra coisa.}\bigskip

Uma história em que Kant não é elegante

\emph{``Mulheres eruditas'', Kant observou, ``usam seus livros como seu
relógio; elas o carregam para todo mundo ver que têm um relógio, apesar
de normalmente estar parado ou não estar ajustado de acordo com a
posição do sol.''}\bigskip

Uma história que mostra existirem dois tipos de citação: aquelas que
possuem aspas e aquelas que as receberão mais tarde

\emph{Numa roda de eruditos, a conversa acabou se voltando para a
maioria dos filósofos alemães, sendo que, evidentemente, também se falou
em Kant e suas obras.}

\emph{``Meu Deus'', disse o Conselheiro Oelrich, ``como se pode
vangloriar tanto as obras de Kant? Qualquer um pode escrevê-las numa
ilha deserta -- não contém nem dez citações.''}\bigskip

Uma história sem a qual ninguém entenderá a \emph{Crítica da faculdade
de julgar}

\emph{Durante um verão mais fresco, quando havia poucos insetos, Kant
tinha percebido uma abundante quantidade de ninhos de andorinhas no
grande armazém de farinha perto do distrito de Lizent}\footnote{Distrito
  mercantil e alfandegário no centro de Königsberg. [\versal{N}. \versal{E}.]}
\emph{e encontrado alguns filhotes espatifados no chão. Admirado com
isso, repetiu sua investigação com atenção redobrada e fez uma
descoberta que custou a acreditar: as próprias andorinhas jogavam seus
filhotes dos ninhos. Muito surpreso com esse impulso natural, que, como
se fosse fruto do entendimento, ensinava as andorinhas a sacrificar
alguns para preservar os outros, na falta de alimentação suficiente para
todos, Kant disse: ``Nesse momento, meu pensamento parou e nada mais
pude fazer a não ser cair de joelho e orar.'' Disse isso de uma maneira
indescritível e inimitável. A profunda devoção que ardia no seu rosto, o
tom da voz, suas mãos em posição de oração, o entusiasmo que acompanhava
essas palavras -- tudo isso era único.}\bigskip

Uma história com algumas palavras novas

\emph{Quando, em uma roda de conversa, falava-se sobre a diversidade dos
caracteres nacionais, Kant descreveu as nações europeias mais
importantes com as seguintes palavras:}

\emph{``Os franceses são polidos, vivos, levianos, volúveis,
libertários. Os ingleses são perseverantes, ativos, avarentos,
orgulhosos e antissociais. Os espanhóis são moderados, orgulhosos,
religiosos, majestosos, ignorantes, cruéis e preguiçosos. Os italianos
são alegres, firmes, afetuosos e assassinos. Os alemães, por fim, são
caseiros, honestos, constantes, fleumáticos, trabalhadores, modestos,
resistentes, hospitaleiros, eruditos, gostam de imitar os outros e
cobiçam títulos.}

\emph{Disso deduz-se'', acrescentou de maneira lacônica, ``que a França
é o país da moda, a Inglaterra o país do humor, a Espanha o país dos
ancestrais, a Itália o país da suntuosidade e a Alemanha o país dos
títulos.''}\bigskip

Uma história em que Kant dá uma lição a alguns oficiais

\emph{Embora Kant não tivesse o mau hábito de muitos outros eruditos que
consistia em direcionar a conversa sempre para a sua ciência, ele
gostava de falar sobre assuntos que diziam respeito à Filosofia. Alguns
oficiais do quartel sabiam disso. Uma vez estava jantando com o chefe do
batalhão, o conde Henkel. Então alguns senhores resolveram fazer troça
da sua ciência. Iniciaram, portanto, a conversa nesse sentido, mas de
uma forma tão inábil que o filósofo logo percebeu a intenção. Não
notaram que agora era ele quem puxava os fios da conversa quando passou
a falar de cavalos e cachorros, isto é, do assunto preferido desses
senhores. Estes começaram a brigar a respeito e a discussão ficou
acalorada. ``Os senhores pegaram fogo'', disse Kant, ``isso me
surpreende, pois o assunto não é da Filosofia.''}\bigskip

Um silogismo que não é de Kant, mas sobre Kant

\emph{Uma vez, quando estava lecionando Lógica, Simmel queria mostrar
aos seus ouvintes que duas premissas falsas podem gerar uma conclusão
verdadeira:}

\emph{Premissa maior: Todos os índios usam tranças.}

\emph{Premissa menor: Kant era índio.}

\emph{Conclusão: Logo, Kant usava uma trança.}

\begin{enumerate}
\def\labelenumi{\Roman{enumi}.}
\setcounter{enumi}{1}
\item
  \emph{Kant como conselheiro do amor}\footnote{Segundo os editores
    alemães dos \emph{Gesammelte Schriften}, a segunda parte das
    anedotas, ``\versal{II}. Kant como conselheiro do amor'', não pode ser
    atribuída com certeza a Benjamin, mas foi certamente elaborada a
    partir de sugestão dele. [\versal{N}. \versal{E}.]}
\end{enumerate}

Diferentemente da ``Senhora Cristine'', que, no Jornal da Tarde da
Editora Ullstein, resolve, a cada sábado, os problemas eróticos mais
complicados que as assinantes magoadas e os assinantes desesperados
apresentam, Kant, provavelmente, teve apenas uma vez a oportunidade de
fazer o papel de um conselheiro do amor de uma mulher e seu amor
infeliz. Tratava-se de Maria von Herbert, a irmã de um discípulo
talentoso de Kant, que lhe escreveu em agosto de 1791 uma carta
dilacerante. O primitivismo selvagem e a ortografia totalmente confusa
dessa carta, provavelmente nada de muito surpreendente para o leitor da
época, causa ao leitor de hoje a impressão quase insuportável de uma
derradeira agonia desesperada.

Diante dessa carta, Kant reservou-se um tempo, até a primavera de 1792,
para então responder. O arroubo inesperado das paixões parece tê-lo
impressionado bastante. A resposta que escreveu então certamente pode
ser chamada de a carta mais comovente de um filósofo. Comovente pela
clareza monumental de conteúdo e forma, mas, mais ainda, pela total
ingenuidade na análise das relações entre os sexos e pela total
ignorância quanto às reações do erotismo, que hoje em dia causaria
risadas em qualquer adolescente de quatorze anos: como se a erradicação
dos ``vestígios daquela resistência conforme as regras e baseados em
conceitos da virtude'' pudessem reacender no homem amado um amor uma vez
extinto! A Senhora Cristine entende muito mais do assunto. -- Mas é
exatamente esse muro pétreo e impenetrável, que se ergue entre o mundo
do espírito e o imoralismo da natureza, que parece ser o comentário mais
sublime sobre a figura humana de Kant.

Reproduzimos em seguida a carta completa da mulher e o fragmento
essencial da resposta de Kant.

Sobre essa mulher e sua história de amor, Kant escreve em 17 de janeiro
de 1793 a Ehrhard: ``Ela fracassou no recife do qual eu escapei, talvez
mais por sorte do que por mérito, ou seja, do amor romântico. -- Para
viver um amor ideal, ela se entregou primeiro a um homem que abusou de
sua confiança, e em favor de outro amor ideal ela confessou isso a um
segundo amante.'' Em 11 de fevereiro, Kant envia a carta dessa mulher a
Elisabeth Motherby, finalizando o bilhete que a acompanha com as
seguintes palavras: ``A sorte da educação da senhora dispensa a intenção
de oferecer essa carta como exemplo de alerta para esses desvios de uma
fantasia sublimada, mas ela pode servir assim mesmo para a senhora
sentir essa sorte de uma maneira ainda mais viva.''

Em agosto do ano 1791, Kant recebeu a seguinte carta:

\emph{Grande Kant,}

\emph{Chamo-lhe como um crente clama pela ajuda do seu Deus, clamo por
consolo ou por uma solução na morte, sempre suas razões em suas obras me
satisfaziam para a vida futura, daí refugio-me em você, mas para essa
vida não encontrei nada, nada mesmo, que pudesse substituir meu bem
perdido, pois eu amava uma coisa que, na minha visão, reunia tudo em si,
de maneira que só vivia por ela que era o oposto a todo o restu}
(übrüg)\emph{, pois todas as outras coisas para mim eram futilidades e
todas as pessoas realmente eram para mim como um caldo aguado, acontece
que ofendi essa coisa com uma mentira duradora} (langwirig) \emph{que
lhe revelei agora, mas, para o meu caráter, não vi nada de prejudicial
nisso, pois nunca tive que esconder qualquer erro na minha vida, mas
essa mentira apenas foi motivo suficiente para ele, e o amor dele
desapareceu, ele é um homem honesto, por isso não me nega sua amisade}
(Freindschaft) \emph{e sua fidelidade, mas aquele sentimento intenso que
nos uniu sem fazer qualquer esforço não existe mais, oh meu coração que
explode em mil pedaço} (Stük)\emph{, se já não tivesse lido tanta coisa
do senhor, certamente teria cometido uma violência contra a minha vida,
mas assim a conclusão que tive que tirar de sua tiuria} (Tehorie)
\emph{me segurou, que não deveria morrer por causa dos tormentos da
minha vida, mas que deveria viver por causa da minha existência, agora
coloqe-se} (sich sezen) \emph{no meu lugar e me dê consolo ou me
condene, li a metafísica dos costumes, inclusive o imperativo
categórico, mas não adianta, a minha razão me abandona quando mais
preciso dela, uma resposta lhe imploro ou você mesmo não concegue}
(kanst) \emph{agir conforme seu imperativo estabelesido}
(aufgeseten).\footnote{Com os erros ortográficos deste trecho,
  procurou-se recriar de forma aproximada em português símiles das
  confusões e incorreções ortográficas em que, conforme explicitado
  previamente no texto, o alemão da referida carta de Maria von Herbert
  incorre. [\versal{N}. \versal{E}.]}

Kant respondeu na primavera de 1792:

\emph{Sua carta afetuosa, nascida de um coração que deve ter sido feito
para a virtude e retidão, uma vez que é tão receptivo para uma teoria
das mesmas, que não tem nada de envolvente, leva-me para o lugar onde a
senhora me quer ver, isto é, no seu lugar, para assim refletir sobre o
meio de acalmá-la por vias puramente morais, que são as únicas
verdadeiras\ldots{}}

\emph{Em primeiro lugar, aconselho que reflita sobre a questão de se as
repreensões que dirige a si mesma por causa de uma mentira, que, aliás,
não usou para esconder qualquer vício, são resultado de uma mera falta
de raciocínio, ou representam uma acusação interior por causa da
imoralidade que reside na própria mentira. No primeiro caso, a senhora
apenas repreende a si mesma pela sinceridade de tê-la revelado, ou seja,
se arrepende de ter cumprido com seu dever (pois é disso que se trata
quando se engana alguém propositalmente -- porém sem prejudicá-lo,
mantendo-o nesse engano durante um tempo -- e quando se livra essa
pessoa desse engano). E por que a senhora se arrepende com essa
revelação? Porque dela resultou a desvantagem, certamente considerável,
de perder a confiança do seu namorado. Ora, esse arrependimento não
contém nada de moral em seu motivo porque não foi gerado pela
consciência do erro, mas pelas suas consequências. Mas se a repreensão
que preocupa a senhora se basear mesmo num julgamento exclusivamente
moral de seu comportamento, seria um mau médico aquela pessoa que lhe
aconselhasse, uma vez que o acontecido não pode mais ser anulado, a
extinguir essa repreensão de sua mente e a munir-se, desde já, de alma
inteira, de uma sinceridade correta. A consciência moral é obrigada a
guardar todas as transgressões como um juiz que não extingue o processo
por causa de delitos já julgados, mas o guarda no arquivo para, em caso
de uma nova acusação por causa de delitos semelhantes, ou ainda
diferentes, reforçar ainda mais o julgamento, de acordo com os critérios
de justiça. Ora, ficar cismando sobre um arrependimento e, depois de ter
decidido a mudar a maneira de pensar, ficar continuamente se
repreendendo por causa de erros antigos irreparáveis torna-se inútil
para a vida. Seria (desde que se tenha certeza de ter melhorado) a
posição fantasiosa de uma autoflagelação merecida que como tantos outros
recursos supostamente religiosos, que consistem na súplica de favores
junto a poderes superiores, sem que se tenha nenhuma necessidade de se
tornar um ser humano melhor, não podem nem ser considerados como parte
da responsabilidade moral.}

\emph{Mas, se essa mudança na maneira de pensar se tornou evidente para
seu namorado -- tendo em vista que a sinceridade fala uma língua
inconfundível --, basta que se tenha tempo suficiente para aniquilar aos
poucos os vestígios daquela resistência do mesmo, que, por sua vez, se
baseia em conceitos morais, e para transformar a frieza em afetos ainda
mais sólidos. Todavia, se este último caso não acontecer, isto significa
que o calor afetivo anterior foi mais físico do que moral e, devido à
natureza passageira do mesmo, teria desaparecido de qualquer maneira com
o passar do tempo. Tal infelicidade ocorre várias vezes na nossa vida e
temos que nos resignar a ela com desprendimento, mesmo porque o valor da
vida, enquanto aquilo que podemos desfrutar de bom, é demasiadamente
valorizado pelas pessoas; mas enquanto aquilo que é apreciado de acordo
com o que podemos fazer de bem, é digno do maior respeito e esmero para
ser conservado e utilizado de forma serena para os melhores fins. --
Nesta resposta, querida amiga, a senhora encontra, portanto, como
costuma acontecer nos sermões, ensinamento, punição e consolo, e peço
que se detenha um pouco mais nos primeiros do que no último, pois quando
os primeiros surtirem efeito, o último e a paz perdida da vida, com
certeza, se encontrarão por si mesmos.}

\part{Posfácio}

\chapterspecial{O Crítico e o contador}{}{Patrícia Lavelle}

\section{A crítica: entre literatura e filosofia}

A geração de intelectuais alemães que ainda fora à escola num bonde
puxado por cavalos e se formara no rigor sistemático do neokantismo,
acedeu à maioridade intelectual depois da catástrofe representada pela
Primeira Guerra e pelas transformações sociais e econômicas que dela
decorreram. Marcada pela guerra e pela inflação, essa geração se
confrontou com o fracasso dos grandes sistemas do século~\versal{XIX} e com a
concepção, de cunho hegeliano, da modernidade como ponto culminante do
progresso da razão na História. O novo interesse pelo tema da vida e da
experiência vivida e o retorno a ideias e formas pré-modernas, evocados
pelo próprio Benjamin em ``Experiência e pobreza'', acompanha assim a
descrença na exposição demonstrativa da reflexão filosófica e o
interesse por novas formas de apresentação do pensamento. Neste sentido,
a obra de Benjamin é emblemática de uma atitude intelectual
compartilhada por outros pensadores de sua geração, como Ernst Bloch ou
Theodor \versal{W}. Adorno, embora tenha também afinidades com projetos
literários contemporâneos que enfatizam, inversamente, a reflexividade
contida na obra de arte.

Em ``Experiência e pobreza'', ele se refere à aparição de temas e formas
pré-modernas nas ruas das grandes cidades pintadas por Ensor,
qualificando tal efeito como fantasmagoria:

\begin{quote}
Uma nova miséria surgiu com esse monstruoso desenvolvimento da técnica,
sobrepondo-se ao homem. A angustiante riqueza de ideias que se difundiu
entre, ou melhor, sobre as pessoas, com a renovação da astrologia e da
ioga, da \emph{Christian Science} e da quiromancia, do vegetarismo e da
gnose, da escolástica e do espiritualismo, é o reverso dessa miséria.
Porque não é uma renovação autêntica que está em jogo, e sim uma
galvanização. Pensemos nos esplêndidos quadros de Ensor, nos quais uma
grande fantasmagoria enche as ruas das metrópoles: pequenos burgueses
com fantasias carnavalescas, máscaras disformes brancas de farinha,
coroas de folhas de estanho, rodopiam imprevisivelmente ao longo das
ruas. Esses quadros são talvez a cópia da Renascença terrível e caótica
na qual tantos depositaram suas esperanças.\footnote{Walter Benjamin,
  ``Experiência e Pobreza'', in \_\_\_\_\_\_. \emph{Magia e técnica,
  arte e politica}. Tradução de Sergio Paulo Rouanet. São Paulo:
  Brasiliense, 1993, p. 115~(Obras Escolhidas, v. \versal{I}); Walter Benjamin,
  ``Erfahrung und Armut'', in \_\_\_\_\_\_. \emph{Gesammelte Schriften}
  [daqui em diante: \versal{GS}], v. \versal{II}-1. Frankfurt a. \versal{M}.: Suhrkamp, 1991,
  p. 214.}
\end{quote}

Podemos comparar as fantasmagorias de Ensor, tal como Benjamin as
apresenta aqui, à sua própria atitude diante do contador de histórias,
compreendido como uma figura arcaica, pré-moderna -- espectro que faz
aparição e deixa traços na modernidade de sua obra. Ora, se em
``Experiência e pobreza'' ele valoriza positivamente a barbárie
moderna, propondo um construtivismo vanguardista capaz de partir do
ponto zero de experiência, o ensaio sobre Leskov, que abre esse volume,
assume um tom inegavelmente nostálgico que parece contradizer a primeira
posição. Entretanto, o novo bárbaro disposto a construir com pouco e o
contador de histórias que encontra na riqueza da experiência
transmissível a matéria de sua arte, são as duas faces de um
mesmo.\footnote{Para uma confrontação entre as teses, apenas
  aparentemente antagônicas, apresentadas nestes dois textos, ver o
  ensaio de Jeanne Marie Gagnebin, ``Não contar mais?'', in
  \_\_\_\_\_\_. \emph{História e narração em \versal{W}. Benjamin}. São Paulo:
  Perspectiva: Editora da Unicamp, 1994.} Ao evocar o arcaísmo da
narrativa que se inscreve na tradição oral, Benjamin tem o projeto de
construir uma nova forma, profundamente moderna. De fato, ele não
tematiza a narração tradicional apenas teoricamente, no gênero do ensaio
crítico, mas também numa produção ficcional na qual as estratégias
tradicionais da arte de contar histórias são mobilizadas, discutidas e
ironizadas. Como mostra Marc de Launay no estudo que acompanha uma
coletânea de contos de Benjamin por ele traduzidos para o
francês\footnote{Ver ``Préface'', in Walter Benjamin. \emph{N'oublie pas
  le meilleur: et autres histoires et récits}. Tradução, apresentação e
  notas por Marc de Launay. Paris: L'Herne, 2012.}, tais textos
ficcionais se caracterizam pelo efeito de choque causado pela evocação
nostálgica e pela brusca denúncia da nostalgia, pelo uso de formas
narrativas tradicionais e por sua ironização pelo contista moderno, que
não pode aderir substancialmente a estas.

Em sua reflexão teórico-literária, esta forma arcaica aparece como uma
fantasmagoria capaz de reatualizar algo que o sistema filosófico do
século~\versal{XIX} negligencia, apontando para um projeto alternativo de
modernidade também ao nível do pensamento. Assim, a interrogação sobre a
arte de contar, que Benjamin ao mesmo tempo teoriza e pratica, pressupõe
a compreensão prévia da fecundidade das passagens entre literatura e
filosofia exploradas por seus trabalhos.

É na perspectiva da relação entre literatura e filosofia que, numa carta
de 1920, endereçada a Ernst Schoen, antigo camarada de estudos, Benjamin
circunscreve o vasto campo da crítica:

\begin{quote}
Muito me interessa, efetivamente, o princípio do grande trabalho
crítico-literário: o campo compreendido entre arte e filosofia
propriamente dita, que compreendo apenas como o pensamento ao menos
virtualmente sistemático. É preciso conceber um princípio perfeitamente
originário de uma forma literária a qual pertencem grandes obras como o
diálogo de Petrarca sobre o desprezo do mundo ou os aforismas de
Nietzsche ou as obras de Péguy. Nestas últimas, por um lado, e por outro
no devir e nas relações de uma jovem pessoa minha conhecida tal questão
se colocou sob meus olhos. Além disso, tornei-me consciente do
fundamento originário e do valor da crítica também em meu próprio
trabalho. Neste sentido, a crítica de arte, cujas fundamentações me
ocuparam, é apenas uma parcela deste vasto domínio.\footnote{Walter
  Benjamin, \emph{Gesammelte Briefe} [daqui em diante: \versal{GB}], vol. \versal{II}.
  Edição de Christoph Gödde e Henri Lonitz. Frankfurt a. \versal{M}.: Suhrkamp,
  2000, p. 71 (252).}
\end{quote}

Ao apresentar a crítica como o princípio de um gênero ou um campo de
pesquisa, Benjamin nele situa seus próprios trabalhos. Deste campo
limítrofe entre literatura e filosofia, a crítica de arte -- que
procurara conceitualizar em sua tese de doutorado sobre o primeiro
romantismo, defendida em 1919, e no ensaio sobre \emph{As afinidades
eletivas} de Goethe, publicado em 1922, -- seria apenas uma parte.
Entretanto, esta parte indica o princípio da zona limítrofe entre
literatura e filosofia que constitui o gênero crítico, pois pressupõe a
correlação entre a dimensão teórica inerente à arte e o elemento
estético no discurso filosófico. O conceito benjaminiano de crítica, que
se inspira diretamente nos românticos, seria fundamentalmente tributário
da estética kantiana, isto é, da correlação descoberta por Kant entre a
produção simbólica do gênio e a apresentação das ideias da razão, entre
criação artística e reflexão teórica.

Na \emph{Critica da razão pura}, Kant estabelece uma rigorosa distinção
entre os conceitos do entendimento, que podem ser conhecidos
objetivamente e determinados através de uma exposição direta,
esquemática, e os conceitos da razão, isto é, as ideias, cuja realidade
não pode ser demonstrada, mas apenas pensada. Ora, se as ideias
racionais não podem ser conhecidas objetivamente, elas também não são
meras ilusões, mas correspondem à esfera do questionamento metafísico
que confere sentido à experiência e ao conhecimento possíveis. É no
entanto na \emph{Crítica da faculdade de julgar} que Kant tematiza o
modo de apresentação próprio às ideias. Este implica uma construção
simbólica na qual a ideia é pensada por analogia com um objeto empírico
qualquer, processo no qual se produz uma afinidade entre as regras de
reflexão sobre esse objeto e a reflexão sobre o conceito racional do
qual ele é apenas o símbolo. Assim, a exposição da esfera do
questionamento filosófico requer, ao menos parcialmente, a intervenção
do gênio artístico.

Definido no § 49 da terceira crítica como a faculdade de criar ``ideias
estéticas'', o gênio corresponde ao talento de articular princípios
racionais a representações da imaginação em configurações sensíveis que,
abrindo perspectivas à perda de vista para o pensamento, constituem o
contrário e o correlato das ideias racionais. Se as ideias estéticas são
representações sensíveis às quais nenhum conceito determinado pode
corresponder, as ideias da razão são conceitos indeterminados aos quais
nenhuma representação sensível pode ser adequada. Assim, ao explicitar a
dimensão estética ou simbólica de toda reflexão filosófica, Kant chama a
atenção também para a dimensão reflexiva da arte, sua abertura ao
pensamento.

A tese de doutorado de Benjamin, que se apoia sobretudo em Friedrich
Schlegel e em Novalis, define o conceito romântico de crítica como
autoconsciência da reflexão que se coloca em forma numa formação
artística. Assim, criticar significa desenvolver a reflexão que já se
encontra na obra, isto é, despertar e completar o pensamento nela
colocado em forma. Segundo a interpretação de Benjamin, para os
românticos a crítica não é um julgamento sobre a obra, mas um método de
seu acabamento, pois deve ultrapassá-la em sua própria reflexão, torná-la
absoluta. Assim, o conceito romântico de crítica de arte se funda sobre
a dimensão reflexiva imanente à formação artística, sua criticabilidade
essencial. Nesta perspectiva, aquilo que os românticos chamavam de
ironia corresponde à exacerbação formal do elemento crítico contido na
própria forma da obra de arte, isto é, à acentuação reflexiva de sua
reflexividade.

O ensaio sobre \emph{As afinidades eletivas} de Goethe funciona como uma
crítica paradigmática na qual Benjamin elabora o seu próprio conceito de
crítica. A terceira parte deste texto é dedicada ao problema da relação
entre arte e filosofia. Segundo Benjamin, as obras de arte são figuras
nas quais aparece o Ideal do problema filosófico definido como o
conceito de uma pergunta inexistente sobre a unidade da filosofia, isto
é, como fundamento unitário de todo questionamento filosófico. De acordo
com ele, o Ideal do problema não se encontra numa multiplicidade de
problemas, mas está encerrado na pluralidade das obras de arte e sua
extração seria tarefa da crítica. Assim, em cada obra de arte verdadeira
pode ser encontrada uma manifestação do Ideal do problema filosófico e a
crítica o apresenta ao se confrontar com o mistério de sua beleza.

\begin{quote}
Diante (\ldots{}) de todo belo, a ideia do desvelamento converte-se naquela
da impossibilidade de desvelamento. Essa é a ideia da crítica de arte. A
tarefa da crítica de arte não é tirar o envoltório, mas antes elevar-se
à contemplação do envoltório enquanto envoltório.\footnote{Walter
  Benjamin, ``Sobre \emph{As afinidades eletivas} de Goethe'', in
  \_\_\_\_\_\_. \emph{Ensaios reunidos: escritos sobre Goethe}. Tradução
  do ensaio de Mônica Krausz Bornebusch. São Paulo: Editora 34: Livraria
  Duas Cidades, 2009, p. 112; \versal{GS} \versal{I}-1, p. 195.}
\end{quote}

A crítica não desvela um conteúdo qualquer que estaria escondido na
forma artística, mas ela revela precisamente a dissolução desta
dicotomia na contemplação do belo.

\begin{quote}
A doutrina kantiana de que o fundamento da beleza é um caráter
relacional impõe portanto, com pleno êxito, as suas tendências
metodológicas numa esfera muito mais elevada do que a psicológica. Toda
beleza, assim como a revelação, conserva em si regras
histórico-filosóficas. Pois a beleza não torna a ideia visível, mas sim
o seu segredo.\footnote{Idem, ibidem, p. 113; \versal{GS} \versal{I}-1, p. 195.}
\end{quote}

Tais formulações, que se referem ao conceito do Belo em Kant, se
inscrevem no horizonte inaugurado pela correlação que a \emph{Crítica da
faculdade de julgar} estabelece entre ideia estética e ideia racional,
representação artística e questionamento filosófico. Entretanto,
Benjamin a considera do ponto de vista de uma teoria da linguagem que
compreende a ideia como nome, isto é, como algo que diz respeito à
dimensão simbólica da linguagem. Pois, de acordo com o ``Prefácio
epistemo-crítico'' de seu livro sobre o drama barroco, a cristalização
da reflexão sobre aquelas questões fundamentais, embora sem resposta,
que a filosofia coloca sempre de novo, se apresenta em configurações
discursivas historicamente condicionadas nas quais o elemento conceitual
se articula necessariamente a construções poético-literárias. Nesta
perspectiva, seu segredo, que se enraíza na vida do pensamento, remete à
unidade de forma e conteúdo que também caracteriza a beleza da obra de
arte. A crítica de arte introduz assim o problema da apresentação do
pensamento em sua relação com a capacidade da linguagem de se articular
poeticamente; problema que, embora tenha sido indicado por Kant, é
negligenciado pela exaustividade demonstrativa do sistema filosófico do
século~\versal{XIX}.

Neste sentido, o conceito de crítica de arte leva-nos a pensar a
fecundidade da interseção entre literatura e filosofia que constitui o
princípio do trabalho crítico-literário evocado por Benjamin. A arte de
contar histórias, abordada tanto no ensaio crítico sobre Leskov quanto
em sua produção ficcional, se inscreve neste horizonte de
problematização da apresentação do pensamento que busca em elementos
arcaicos uma alternativa ao projeto de modernidade baseado na ideia de
progresso racional e na instrumentalização da linguagem no sistema.

Ora, esta forma pré-moderna, já evocada no ensaio sobre \emph{As
afinidades eletivas} através da história contada aos personagens de
Goethe -- conto dentro do romance ao qual Benjamin atribui uma grande
importância, considerando-o como uma chave de leitura para a compreensão
da obra --, nos coloca diante da questão moral. Nesta forma mais arcaica
representada pela narrativa, Benjamin identifica o ``núcleo luminoso''
que se abre à esfera da liberdade da qual os personagens do romance,
cativos das convenções como de uma segunda natureza mítica, estão
excluídos. Neste sentido, a conclusão do conto é significativa: o jovem
enamorado não hesita em mergulhar em águas agitadas para salvar a moça
que, num ato desesperado, joga-se do barco. Ao interpretar esse ato
heroico -- e o episódio no qual o rapaz desnuda o corpo da amada, não
para contemplar sua beleza, mas no intuito de salvar-lhe a vida --,
Benjamin chama a atenção para o abandono da esfera do belo em função de
uma ordem mais alta. Como veremos, tanto sua reflexão teórica sobre a
narração tradicional quanto o recurso ficcional a estratégias próprias à
arte de contar se inscrevem na perspectiva do enraizamento de todo pensar
na esfera da liberdade e na interrogação sobre o sentido da ação humana.

\section{Nostalgia e modernidade: o ensaio sobre Leskov}

Contemporâneo de Dostoiévski e de Tolstói, Nikolai Leskov viveu e
escreveu na época de apogeu do romance russo do século~\versal{XIX}. Entretanto,
ao caracterizá-lo, logo nas primeiras linhas do ensaio de 1936, Benjamin
o assimila à figura arcaica do contador de histórias, cujas origens
pré-modernas se encontram na tradição oral, na transmissibilidade da
experiência da vida, irremediavelmente perdida no mundo moderno. Assim,
ele situa Leskov num passado anterior ao da modernidade que, no século~\versal{XIX}, 
encontra sua expressão literária no romance burguês e sua forma
filosófica no sistema. O anacronismo não é arbitrário -- o contista
russo se serve efetivamente de formas narrativas tradicionais que se
alimentam de um farto material popular e da tipologia do conto de fadas
--, mas indica, entretanto, a orientação mais geral do estudo como
caracterização da figura arcaica do contador de histórias em sua relação
com a modernidade.

O contador é um espectro. Aparição do passado no presente, essa figura
construída em torno de Leskov no ensaio de 1936 não é uma simples
metáfora, mas possui uma certa materialidade que se encarna não apenas
no escritor russo, mas também na obra de Hebel, de Poe ou de Kipling,
entre outros autores citados. Sobre o projeto do ensaio acerca do
contador, que pode ser compreendido como a produção de uma
fantasmagoria, Benjamin se exprime em duas cartas de 1936.

``Tenho me ocupado sobretudo com um estudo sobre Nikolai Leskov no qual
falo menos sobre este grande contista russo do que sobre o tipo do
contador de histórias em geral, sua relação com o romancista e com o
jornalista e seu lento desaparecimento da face da Terra'', afirma numa
carta de Paris, endereçada a Werner Kraft.\footnote{\versal{GB} \versal{V}, p. 289 (1041).}
Uma semana mais tarde, torna a escrever sobre o assunto, desta vez
dirigindo-se a Adorno: ``Escrevi nos últimos tempos um trabalho sobre
Nikolai Leskov o qual, sem se referir, nem mesmo de longe, à teoria da
arte, contém alguns paralelos com a `perda da aura' no que esta diz
respeito à arte do contador de histórias''.\footnote{\versal{GB} \versal{V}, p. 307
  (1047).}

Apresentada como uma forma fundamentalmente aurática cujo declínio no
mundo moderno estaria relacionado à emergência do romance, e sobretudo
da informação, o conto tem um caráter artesanal. Isto significa que se
alimenta da experiência de vida (\emph{Erfahrung}) do contador, cuja
marca se imprime na história contada ``como o oleiro deixa a impressão
de sua mão na argila do vaso''.\footnote{\emph{Supra}, p.\,\pageref{supra} (\versal{GS} \versal{II}-2, p.
  447).} A experiência transmissível da tradição é, segundo Benjamin, a
fonte a que recorreram todos os contadores. O senso prático é uma
característica desta figura cuja autoridade se funda na sabedoria, ``o
lado épico da verdade''.\footnote{\emph{Supra}, p.\,\pageref{supra2} (\versal{GS} \versal{II}-2, p. 442).}
Assim, a arte de contar pressupõe a capacidade de aconselhar,
compreendida como o talento para sugerir uma continuação para uma
história que está se desenvolvendo. O declínio da arte de contar estaria
relacionado ao empobrecimento desta experiência de vida que outrora
garantia o valor dos conselhos e se exprimia em provérbios e
ensinamentos morais.

Benjamin caracteriza tal crise através do contraste com o romance e com
a informação. Se os contos podem ser compartilhados e transmitidos
oralmente, a recepção do romance implica a leitura silenciosa pelo
indivíduo isolado. É também a perplexidade diante da singularidade
individual que orienta o romance; este se interroga fundamentalmente
sobre a inefabilidade do sentido de uma vida e sua conclusão corresponde
simbolicamente à morte do personagem. O conto, ao contrário, coloca a
questão moral de tal modo que a história não se acaba com o seu fim, mas
suscita a interrogação sobre o que aconteceu depois, abrindo-se assim a
outras histórias. Segundo Benjamin, o declínio da arte de contar que se
enraíza na tradição oral corresponde à emergência do romance, mas se
acentua com a própria crise deste, associada à importância crescente da
informação que encontra seu espaço nos jornais. Ora, esta se caracteriza
pela proximidade, pela plausibilidade e pela explicação dos fatos. O
saber que vem de longe para se transmitir misteriosamente nos contos não
tem nenhum valor informativo. O interesse da informação está na
proximidade, no aqui e agora de eventos que nunca vêm sem numerosas
explicações. Nela, nada é deixado ao julgamento do leitor. Por outro
lado, segundo Benjamin, a ``metade da arte de contar está em despojar de
explicações a história contada''.\footnote{\emph{Supra}, p.\,\pageref{supra3} (\versal{GS} \versal{II}-2,
  p. 445).}

A ausência de explicações psicológicas é uma característica fundamental
do conto e remete às suas fontes antigas. Benjamin menciona a propósito
disto a história, contada por Heródoto e comentada por Montaigne, do
faraó egípcio Psamético. Segundo Benjamin, tal relato conserva suas
forças depois de tanto tempo justamente porque renuncia a nos dizer o
que se passa no coração e no pensamento do rei deposto. É sua concisão,
aliada à ausência completa de explicações psicológicas, que nos deixa
pensativos. Abrindo-se a uma pluralidade de interpretações e comentários
que procuram, em vão, extrair da história uma moral ou um sentido, o
conto não se explica numa frase, mas nos leva à multiplicidade difusa de
representações que constitui aquilo que Hans Blumenberg chama de
``pensatividade''.

Em um pequeno texto intitulado ``Pensatividade''
(\emph{Nachdenklichkeit})\footnote{Hans Blumenberg, ``Pensivité'',
  tradução de Denis Trierweiler, \emph{Cahiers philosophiques} (Dossier
  Blumenberg), n.° 122, 3.° trimestre de 2010, pp. 83-87.}, Blumenberg
evoca uma fábula de Esopo para refletir sobre este estado de espírito ao
qual nos leva o conto. Ele o caracteriza como um momento de hesitação no
qual nos confrontamos confusamente com aquelas questões que não podemos
responder, mas às quais também não devemos renunciar. Entretanto, a
pensatividade se distingue do pensamento porque não conclui, não resolve
nenhum problema, é apenas uma disposição, um espaço de jogo. Neste
sentido, a pensatividade seria, segundo Blumenberg, um adiamento, um
prazo dado contra os resultados banais e decepcionantes que o pensamento
ordenado pode nos dar quando se interroga sobre a vida e a morte, o
sentido e a ausência de sentido, o ser e o nada. Ora, se a filosofia
passa por saber disciplinar com método tais questões -- e por vezes as
proíbe em razão do caráter inatingível de suas respostas --, a
pensatividade suscitada pelo conto estaria em sua origem. Assim, a
conclusão de Blumenberg nos parece extremamente significativa para
compreendermos o recurso à narrativa na perspectiva deste gênero crítico
que Benjamin situa entre literatura e filosofia.

Blumenberg afirma que a filosofia deve conservar, senão renovar, algo da
origem, vinda do mundo da vida, que encontra na pensatividade. Pois,
segundo ele, a filosofia representa uma constatação mais geral de toda
cultura, isto é, que devemos respeitar as questões às quais não podemos
atribuir uma resposta. Ora, ao evitar explicações, o conto nos ensina
tal respeito -- e a palavra respeito é aqui significativa pois nos
remete aos objetos da razão prática. Diante do conto moderno que se
alimenta de fontes pré-modernas, como a fábula ou o conto de fadas, se
descortina a esfera da liberdade na qual esses questionamentos aos quais
não podemos dar uma resposta adequada encontram sua fonte.

No ensaio sobre o contador, Benjamin afirma que o conto de fadas revela
as primeiras medidas tomadas pela humanidade para libertar-se do
pesadelo mítico. Dirigindo-se às origens do homem, o conto de fadas
apresenta a figura do justo que não se identifica ao místico asceta, mas
ao homem simples e ativo, capaz de enfrentar as adversidades com bondade
e astúcia. Benjamin identifica tais elementos na produção de Leskov, o
que lhe permite delinear os contornos da figura do contador,
fantasmagoria na qual ``o justo se encontra consigo mesmo''.\footnote{\emph{Supra},
  p.\,\pageref{supra4} (\versal{GS} \versal{II}-2, p. 465).} Segundo ele, o primeiro contador verdadeiro é
e continua sendo o do conto de fadas, que sabia dar um bom conselho e
oferecer sua ajuda na necessidade provocada pelo mito.

\begin{quote}
O conto de fadas ensinou há muito tempo à humanidade e ainda hoje ensina
às crianças que o mais aconselhável é enfrentar o mundo do mito com
astúcia e ousadia. (\ldots{}) A magia liberadora do conto de fadas não coloca
em cena a natureza de um modo mítico, mas indica a sua cumplicidade com
o ser humano liberado.\footnote{\emph{Supra}, p.\,\pageref{supra5} (\versal{GS} \versal{II}-2, p. 458).}
\end{quote}

Levando-nos de volta à infância e aos primeiros esforços da humanidade
para libertar-se do mito, o contador de histórias nos conduz à origem da
filosofia na pensatividade -- este espaço de jogo, essa vivência de
liberdade que se abre à esfera da razão prática na qual o questionamento
metafísico se enraíza. Assim, a nostalgia inerente ao ensaio sobre o
contador não diz respeito apenas ao declínio da arte de contar e ao
empobrecimento da experiência que constitui sua fonte, mas concerne
também à pensatividade que se instala no sonho acordado, no devaneio --
esta face interior do tédio que Benjamin associa ao ritmo do trabalho
artesanal. ``O tédio é o pássaro de sonho que choca o ovo da
experiência''\footnote{\emph{Supra}, p.\,\pageref{supra6} (\versal{GS} \versal{II}-2, p. 446).}, diz ele ao
afirmar que as atividades artesanais que o propiciavam, prestando-se ao
dom de ouvir e à arte de contar, estão em vias de extinção.

\section{O conto como gênero crítico}

\begin{quote}
Vivenciamos o surgimento da \emph{short story}, que se emancipou da
tradição oral e não mais permite essa lenta sobreposição de camadas
finas e transparentes que oferece a melhor imagem da maneira pela qual o
conto perfeito vem à luz do dia a partir das camadas acumuladas por suas
diferentes versões.\footnote{\emph{Supra}, p.\,\pageref{supra7} (\versal{GS} \versal{II}-2, p. 448).}
\end{quote}

Grande parte da produção ficcional de Benjamin se inscreve neste novo
gênero literário que surge, por assim dizer, nas ruínas da arte de
contar. Estes textos curtos, escritos ao longo dos anos vinte e trinta,
utilizam elementos que caracterizam a tradição narrativa, tal como é
apresentada no ensaio sobre Leskov. Entretanto, se o uso de tais
recursos provoca em nós a nostalgia da origem do pensamento na
pensatividade, o recurso à ironia marca a distância entre o contista
moderno e as formas tradicionais das quais ele se serve, criando assim
um efeito de choque. Neste sentido, a barbárie construtivista e a
nostalgia da arte de contar histórias correspondem a dois aspectos de um
mesmo projeto filosófico-literário no qual o recurso ao arcaico visa
encontrar uma alternativa para a imanência radical do mundo moderno.

Encontramos um exemplo disso em ``A sebe de cactos'', onde o personagem
principal, o irlandês O'Brien, encarna a figura do bárbaro moderno,
disposto a começar do zero. O conto, escrito na primeira pessoa, narra o
encontro, em Ibiza, com este excêntrico solitário cujas ocupações -- a
pesca, a caça e a arte de fazer e desfazer nós -- remetem a um mundo
arcaico. Da África, onde convivera com uma tribo primitiva, trouxera uma
bela coleção de máscaras, que fora entretanto furtada, há muitos anos,
por um amigo seu. Assim, o narrador do conto se admira ao contemplar um
impressionante conjunto de máscaras africanas reunidas na casa de
O'Brien, que lhe conta como vieram parar ali numa dessas noites em que o
luar e o tédio estimulam a faculdade de produzir e de perceber
semelhanças. Ele vira pela janela a sebe de cactos ganhar vida e, como
um grupo de guerreiros africanos, avançar usando as máscaras
desaparecidas. Resolve então esculpir suas visões oníricas na madeira.

Mas a história não termina com essa invocação nostálgica, na qual o
arcaico perdido reaparece no arcaísmo do sonho. A ironia fica reservada
a um reencontro posterior com três dessas máscaras numa galeria de arte
de Paris, onde especialistas garantem sua antiguidade e autenticidade.
Como sugere o narrador do conto, elas ``inspirariam nossos jovens artistas a
fazer suas próprias tentativas interessantes''.\footnote{\emph{Supra},
  p.\,\pageref{supra8} (\versal{GS} \versal{IV}-2, p. 754).} Os contos de Benjamin são como essas máscaras
primitivas esculpidas pelo excêntrico O'Brien.

Retorno do passado no presente, a fantasmagoria do contador constitui o
próprio tema de ``O Lenço''. Neste conto, um marinheiro, o Capitão O\ldots{},
possui todos os traços que compõem a figura do contador de histórias,
exceto um que aparece como fundamental: a faculdade de contar sua
própria vida. Tal faculdade, que implica a ligação intrínseca entre a
vida do contador e a matéria de suas histórias, corresponde à
transmissão da experiência tradicional, essa \emph{Erfahrung} que não se
confunde com as vivências (\emph{Erlebnisse}) radicalmente individuais e
interiorizadas do homem moderno. Ora, o desenrolar do conto contradirá e
confirmará essa constatação inicial.

Encontrado por acaso durante uma escala, o Capitão~O\ldots{} conta ao
narrador que, há muitos anos, teve como passageira uma moça tão linda
quanto discreta e silenciosa. Um dia, ela deixou cair no convés um lenço
e, quando ele o apanhou, agradeceu seu gesto como se tivesse salvado sua
vida. O marinheiro-contador descreve minuciosamente o lenço, que era
ornamentado com um escudo bordado de estrelas, mas nada diz sobre seus
próprios sentimentos e impressões. Conta apenas que, quando o barco
estava para atracar, a bela passageira precipitou-se sem uma palavra no
mar, justamente no pequeno espaço que restava entre a quilha e o cais. O
perigo era grande e a moça teria sido rapidamente esmagada se, num átimo
de segundo, alguém não tivesse se prontificado a salvá-la. O episódio
faz pensar na pronta decisão do rapaz que salva sua amada das águas
geladas na pequena novela que constitui, segundo Benjamin, o núcleo
luminoso de \emph{As afinidades eletivas} de Goethe. O relato do
marinheiro, no entanto, todo na terceira pessoa, não inclui nenhuma
alusão aos sentimentos, pois não dá lugar a nenhuma explicação
psicológica. Neste ponto, a história do Capitão O\ldots{} é tão reservada e
discreta quanto a moça, e sua beleza delicada vem justamente dessa
extrema concisão. É a figura arcaica do justo que vemos surgir em sua
simplicidade proverbial.

Na contramão do romance psicológico do início do século~\versal{XX}, a
\emph{short story} de Benjamin nada diz sobre a interioridade dos
personagens. Sua modernidade radical está na evocação irônica e
nostálgica do relato arcaico que apenas expõe, sem nada explicar. Se no
final ficamos sabendo que o herói da história é o próprio contador, isso
se deve a uma única frase em primeira pessoa: ``Quando a segurei assim
(\ldots{}), ela sussurrou `obrigada' como se eu tivesse lhe estendido um
lenço que caíra ao chão''.\footnote{\emph{Supra}, p.\,\pageref{supra9} (\versal{GS} \versal{IV}-2, p. 744).}
E a última palavra do conto é confiada justamente a este objeto cuja
presença dispensa maiores explicações: o narrador reconhece o pequeno
escudo bordado no lenço que o capitão agita ao longe, ao despedir-se.

Ao tematizar o próprio contador de histórias, evocando o seu espectro,
Benjamin ironiza a forma narrativa. Penso aqui no conceito romântico de
ironia como um recurso formal, um distanciamento crítico que se inscreve
na própria forma da obra, explicitando a reflexão nela contida e
relativizando o seu caráter condicionado.

Não é por acaso que este conto acerca do contador se abre com a
interrogação sobre o declínio da própria arte de contar. O narrador da
pequena ficção começa por relacionar a morte da narração tradicional com
o desaparecimento das atividades manuais e repetitivas que outrora
deixavam tempo para o tédio. Assim, ao incluir no conto a própria teoria
de seu declínio, distancia-se da simplicidade ``ingênua'' (no sentido de
Schiller) que nos toca no relato de Heródoto, ou mesmo nos contos de
Leskov ou de Hebel. A \emph{short story} do início do século~\versal{XX}, tal
como Benjamin a inventa, é um gênero ``sentimental''. Isto quer dizer
que, nele, a ironização dos procedimentos tradicionais da arte de contar
histórias implica não apenas a consciência de seu fim, e portanto uma
certa nostalgia, como também a reflexão sobre sua reflexividade.
