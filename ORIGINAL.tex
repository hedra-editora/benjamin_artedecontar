%!TEX root= 

%\textbf{Coleção Walter Benjamin}

%\textbf{Coordenação científica e editorial}

%Amon Pinho

%\begin{quote}
%\textbf{Supervisão técnica e de tradução}
%\end{quote}

%Francisco De Ambrosis Pinheiro Machado

%\begin{quote}
%\textbf{\textsc{A arte de contar histórias}}

%Walter Benjamin

%\textbf{Organização e posfácio}

%Patrícia Lavelle

%\textbf{Tradução }

%Georg Otte

%Marcelo Backes

%Patrícia Lavelle

%\textbf{Revisão de tradução}

%Francisco De A. P. Machado

%Amon Pinho

%Patrícia Lavelle

%Blima Carvalho Otte
%\end{quote}

%\textbf{Preparação de originais}

%\begin{quote}
%Amon Pinho

%\textbf{ÍNDICE}
%\end{quote}

%\textbf{\textsc{Apresentação}}

%Amon Pinho

%Francisco De Ambrosis Pinheiro Machado

%\textbf{ENSAIO}

%\textsc{O contador de histórias. Considerações sobre a obra de Nikolai
%Leskov}

%\textbf{CONTOS}

%\textsc{O lenço }

%\textsc{A viagem do ``Mascote''}

%\textsc{O anoitecer da viagem}

%\textsc{A sebe de cactos}

%\textsc{Histórias da solidão}

%\textsc{Quatro histórias}

%\textsc{A morte do pai. Novela}

%\textsc{Palácio D... Y}

%\textsc{``Inscrito na poeira movediça''. Novela}

%\textsc{O segundo eu. Uma história de final de ano para refletir}

%\textsc{Rastelli conta}

%\textsc{Por que o elefante se chama ``elefante''}

%\textsc{Como o barco foi inventado e por que ele se chama barco}

%\textsc{Uma história estranha, de quando ainda não havia humanos}

%\textsc{Myslowitz -- Braunschweig -- Marselha. História de uma
%embriaguez com haxixe}

%\textbf{O CONTADOR DE HISTÓRIAS NO RÁDIO}

%\textsc{No minuto exato}

%\textsc{Caspar Hauser}

%\textsc{O coração gelado. Peça de rádio baseada em Wilhelm Hauff}

%\textsc{A Berlim demoníaca}

%\textbf{CONTO E CRÍTICA}

%\textsc{Conversa assistindo ao corso. Ecos do carnaval de Nice}

%\textsc{A mão de ouro. Uma conversa sobre o jogo}

%\textsc{Jakob Job, \emph{Nápoles. Imagens de viagem e esboços}}

%\textsc{Anedotas desconhecidas de Kant}

%\textbf{\textsc{Posfácio: O crítico e o contador}}

%Patrícia Lavelle

%\textsc{Apresentação}

%Amon Pinho

%Francisco De Ambrosis Pinheiro Machado

{ENSAIO}

\textsc{O contador de histórias. Considerações sobre a obra de Nikolai
Leskov}\footnote{Tradução de Patricia Lavelle a partir do original
  alemão ``Der Erzähler. Betrachtungen zum Werk Nikolai Lesskows'' em
  cotejamento com a versão francesa deste mesmo ensaio, realizada pelo
  próprio Benjamin e intitulada ``Le Narrateur. Réflexions à propos de
  l'oeuvre de Nicolas Lesskov'', in Walter Benjamin, \emph{Gesammelte
  Schriften} {[}a partir daqui, GS{]}, vol. II, tomos 2 e 3. Edição de
  Rolf Tiedemann e Hermann Schweppenhäuser. Frankfurt a. M.: Suhrkamp,
  1991, pp. 438-465 e 1290-1309, respectivamente. {[}N. dos E.{]}\footnote{Escrito
  entre fins de março/inícios de abril e julho de 1936, o ensaio sobre
  Leskov destinava-se à revista \emph{Orient und Okzident}, editada pelo
  teólogo Fritz Lieb, onde foi publicado em outubro do mesmo ano.
  Provavelmente entre julho e outubro de 1936, ou entre este período e
  1939, Benjamin trabalhou numa versão francesa, porventura para
  \emph{Europe}, mas o texto não pôde ser publicado pois a revista parou
  de circular. Esta versão francesa, que foi confiada à Adrienne
  Monnier, personagem influente no meio literário parisiense, seria
  publicada somente em 1952, no \emph{Mercure de France}. {[}N. da O.{]}}

I

Embora seu nome soe familiar, o contador de histórias não está mais
presente entre nós em sua eficácia viva. Ele é para nós algo já
longínquo, e fica cada vez mais longe. Apresentar um Leskov\footnote{Nikolai
  Leskov nasceu em 1831 na província de Oriol e morreu em 1895 em São
  Petersburgo. Ele tinha em comum com Tolstói seus interesses e
  simpatias pelos camponeses e, com Dostoiévski, a sua orientação
  religiosa. Mas precisamente aqueles escritos cuja expressão é
  fundamental e doutrinária, os romances de sua juventude, podem ser
  considerados como a parte mais efêmera de sua obra. A importância de
  Leskov repousa sobre os contos que pertencem a uma fase mais tardia de
  sua produção. Desde o final da guerra, muitas tentativas foram
  empreendidas no sentido de tornar estes contos conhecidos no círculo
  linguístico alemão. Ao lado das pequenas antologias da Editora
  Musarion e da Editora Georg Müller, encontra-se, principalmente, a
  coletânea em nove volumes da Editora C. H. Beck. {[}N. de W.B.{]}}
como contador não significa trazê-lo para mais perto de nós, mas
aumentar a distância que nos separa dele. Se o consideramos com um certo
distanciamento, os traços grandes e simples que caracterizam o contador
nele ganham relevo. Melhor dizendo, aparecem tal como um rosto humano ou
um corpo de animal podem aparecer num rochedo para alguém que o examina
a uma boa distância e do ponto de vista adequado. Esta distância e este
ponto de vista são impostos por uma experiência que fazemos quase
diariamente. Ela nos diz que a arte de contar histórias se aproxima de
seu fim. Torna-se cada vez mais raro encontrarmos pessoas que ainda
sabem contar alguma coisa. Cada vez mais frequentemente generaliza-se o
embaraço quando, num grupo, alguém pede uma história. É como se
tivéssemos sido privados de uma faculdade que nos parecia inalienável, e
era a mais segura entre todas: a faculdade de trocar experiências.

Uma causa deste fenômeno é clara: a cotação da experiência caiu. E
parece que continuará a cair infinitamente. Basta olharmos os jornais
para constatar que ela atingiu um novo nível ainda mais baixo, e que não
apenas a imagem do mundo exterior mas também a do mundo moral sofreram
da noite para o dia alterações que nunca antes nos teriam parecido
possíveis. Com a guerra mundial, começou a tornar-se manifesto um
processo que desde então não encontrou repouso. Não reparamos que,
quando a guerra acabou, os soldados voltaram mudos dos campos de
batalha? Não mais ricos, mas mais pobres em experiências comunicáveis. O
que se difundiu dez anos depois, com a enxurrada de livros sobre a
guerra, não tinha nada a ver com uma experiência passada de boca em
boca. E não havia nada de estranho nisso. Pois nunca experiências foram
tão fundamentalmente desmentidas quanto a experiência estratégica, pela
guerra de trincheira, a experiência econômica, pela inflação, a
experiência corporal, pela batalha com armamentos pesados e com aviões,
e a experiência ética, pelos detentores do poder. Uma geração que ainda
fora à escola num bonde puxado por cavalos se achava a céu aberto numa
paisagem em que nada permanecera inalterado, a não ser as nuvens, e
abaixo delas, num campo de forças cheio de tensões e explosões
destrutivas, o minúsculo, frágil corpo humano.

II

A experiência que se transmite oralmente é a fonte na qual beberam todos
os contadores de histórias. E entre os que as escreveram, os melhores
são aqueles cujos escritos menos se afastam da fala dos muitos
contadores anônimos. Entre estes últimos, além disso, há dois grupos que
se interpenetram de diversas maneiras. A figura do contador só adquire
sua plena corporeidade se o apresentamos sob os traços de ambos. ``Quem
viaja muito, tem sempre muito o que contar'', diz a voz do povo, que
representa o contador como aquele que vem de longe. Mas também é com
prazer que ouvimos os casos daquele que ficou na sua terra, ganhando
honestamente sua vida, e conhece suas histórias e tradições. Se
quisermos apresentar esses dois grupos através de seus representantes
arcaicos, um se encarna no camponês sedentário e o outro no marinheiro
mercador. De fato, estas duas formas de vida constituíram, de certo
modo, sua própria linhagem de contadores. Cada uma delas manteve algumas
de suas características durante séculos. Assim, entre os contadores
alemães modernos, Hebel e Gotthelf pertencem à primeira enquanto
Sealsfield e Gerstäcker vêm da segunda.\footnote{Johann Peter Hebel
  (1760-1826), filho de camponeses, foi teólogo e professor. Poeta de
  expressão alemã e dialetal, escreveu também contos em prosa. Jeremias
  Gotthelf (1797-1854), pseudônimo do escritor suiço de expressão alemã
  Albert Bitzius. Filho de pastor, foi também pastor além de romancista.
  Procurou descrever o impacto da modernização na sociedade camponesa.
  Charles Sealsfield (1793-1864), pseudônimo de Carl Anton Postl,
  jornalista e escritor de origem austríaca, naturalizado americano.
  Friedrich Gerstäcker (1816-1872), viajante e novelista alemão,
  tornou-se conhecido por seus livros sobre o continente americano.
  {[}N. da T.{]}} No entanto, tais linhagens, como foi dito,
correspondem apenas a tipos fundamentais. A real extensão do reino das
narrativas, em toda sua dimensão histórica, não é pensável se não
levamos em conta a íntima interpenetração destes dois tipos arcaicos.
Uma tal interpenetração foi particularmente favorecida pelas corporações
de artesãos da Idade Média. O mestre sedentário e o aprendiz viajante
trabalhavam juntos no mesmo ateliê; e cada mestre fora aprendiz viajante
antes de se estabelecer em sua terra ou no estrangeiro. Se camponeses e
marinheiros foram os antigos mestres da narrativa, o artesanato foi sua
melhor escola. Nele, se associava o saber que vem de longe, trazido para
casa por aquele que viajou muito, com o saber do passado, tal como este
é confiado, de preferência, ao sedentário.\footnote{Na versão em
  francês, a seguinte passagem foi acrescida ao final do parágrafo: ``É
  assim que se constitui esse personagem do contador que, como bem disse
  Jean Cassou, `dá o tom do relato e de sua realidade, aquele perto do
  qual o leitor \ldots{} gosta de se refugiar fraternalmente e de
  encontrar a medida, a escala dos sentimentos e dos fatos humanos
  normais.''' {[}N. da T.{]}}

III

Leskov está em casa no longínquo, seja este espacial ou temporal. Ele
pertenceu à Igreja Ortodoxa e foi um homem de sinceros interesses
religiosos. Mas ele foi também um sincero opositor da burocracia
eclesiástica. Como suas relações com o funcionalismo secular não eram
melhores, as posições oficiais que obteve não duraram muito. Para sua
produção, o posto de agente russo de uma empresa inglesa, que exerceu
durante muito tempo, foi provavelmente o mais útil. Viajou pela Rússia
inteira a serviço desta empresa, e tais viagens formaram tanto sua
inteligência do mundo\footnote{``sua experiência dos homens'', na
  variante da versão francesa. {[}N. da T.{]}} quanto seu conhecimento
da situação russa. Deste modo, teve a possibilidade de entrar em contato
com as seitas implantadas no campo, que deixaram traços em seus contos.
Nas lendas russas, Leskov\footnote{``Leskov, que não escondia suas
  simpatias sectárias,'', conforme acréscimo feito por Benjamin na
  versão francesa. {[}N. da T.{]}} encontrou aliados em sua luta contra
a burocracia ortodoxa. Escreveu uma série de contos lendários em cujo
centro está exposto o justo, raramente um asceta, com mais frequência um
homem simples e ativo, que torna-se um santo da maneira mais natural do
mundo. A exaltação mística não é assunto para Leskov. Por mais que às
vezes se entregasse com prazer ao maravilhoso, mantinha-o, de
preferência, mesmo em questões de piedade, no quadro de uma sólida
naturalidade. Ele via seu modelo no homem acostumado a tratar de coisas
terrenas sem se prender demasiadamente a elas, e adotava, no domínio
secular, uma atitude semelhante a esta. Combina com esse seu modo de ser
o fato de ter começado a escrever tarde, aos 29 anos. Isso foi durante
suas viagens comerciais. Seu primeiro trabalho publicado se intitulava:
``Por que em Kiev os livros são tão caros?'' Uma longa série de escritos
sobre a classe trabalhadora, sobre alcoolismo, sobre os médicos da
polícia, sobre comerciários desempregados precede seus contos.

IV

O interesse prático é um traço característico de muitos contadores
natos. Mais duradouro do que em Leskov, podemos reconhecê-lo por exemplo
em Gotthelf, que dava conselhos de agronomia aos seus camponeses, também
se encontra em Nodier, que se preocupava com os perigos da iluminação a
gás; e Hebel, que transmitia aos seus leitores pequenas informações de
ciência natural na \emph{Caixinha de tesouros},\footnote{Referência à
  coletânea de contos de Johann Peter Hebel (1760-1826), intitulada
  \emph{Caixinha de tesouros do amigo renano das famílias}
  (\emph{Schatzkästlein des Rheinischen Hausfreundes}). {[}N. da T.{]}}
pertence igualmente a este grupo. Tudo isso esclarece sobre as
características da verdadeira narrativa. Ela traz em si, abertamente ou
de modo secreto, sua utilidade. Tal utilidade pode aparecer aqui numa
moral, ali numa recomendação prática, ou ainda num provérbio ou numa
regra de vida -- em cada um desses casos, o contador é um homem que sabe
dar conselhos aos seus ouvintes. Mas se hoje em dia ``saber aconselhar''
começa a soar antiquado aos ouvidos, isto se deve à perda progressiva da
comunicabilidade da experiência. Por isso não sabemos mais dar
conselhos, nem a nós mesmos nem aos outros. Conselho é menos a resposta
a uma pergunta do que uma sugestão de continuação para uma história (que
está se desenrolando). Para poder obtê-lo, é preciso primeiro ser capaz
de contá-la. (Sem considerar que um ser humano só se abre a um conselho
quando deixa que sua situação se expresse em palavras.) Conselho,
entretecido na matéria da vida vivida, é sabedoria. A arte de contar se
aproxima de seu fim pois o lado épico da verdade, a sabedoria, está se
extiguindo. Mas esse processo vem de longe. E nada seria mais insensato
do que considerá-lo como um ``sinal de decadência'' e ainda menos de
``decadência moderna''. Ao contrário, esse processo é antes um fenômeno
ligado às forças produtivas, seculares e históricas, que expulsa aos
poucos o conto do domínio da palavra viva, ao mesmo tempo que confere
uma nova beleza ao que está desaparecendo.\footnote{Na variante da
  versão francesa: ``Trata-se mais propriamente de um fenômeno
  consistindo em forças seculares que pouco a pouco separaram o contador
  do domínio da palavra viva para confiná-lo na literatura. Esse
  fenômeno nos tornou, ao mesmo tempo, mais sensíveis à beleza de um
  gênero que se vai.'' {[}N. da T.{]}}

V

O primeiro indício de um processo que culmina com o declínio da
narrativa é o aparecimento do romance no início da época moderna. O que
distingue o romance do conto (e da epopeia num sentido estrito) é sua
ligação essencial com o livro. A difusão do romance só se tornou
possível com a invenção da imprensa. A transmissão oral, patrimônio da
épica, é de natureza diferente daquela que caracteriza o romance. O que
distingue o romance em comparação com todas as outras formas de prosa --
contos de fada, lendas, e mesmo novelas -- é que nem provém da tradição
oral nem a ela conduz. Isto o distingue em primeiro lugar do conto. O
contador de histórias tira o que ele conta da sua própria experiência ou
da que lhe foi relatada por outros. E ele, por sua vez, o transforma em
experiência para aqueles que escutam sua história. O romancista
isola-se. O lugar de nascimento do romance é o indivíduo em sua solidão:
aquele que não é capaz de expor suas preocupações mais altas de modo
exemplar, é ele mesmo carente de conselhos e não sabe dá-los. Escrever
um romance significa exacerbar o incomensurável na apresentação de uma
vida humana. De dentro da plenitude da vida, e através da apresentação
de tal plenitude, o romance aprofunda a ausência de conselho dos
viventes. O primeiro grande livro deste gênero, \emph{Dom Quixote},
ensina como a grandeza de alma, a coragem e a generosidade de um dos
mais nobres heróis -- o próprio \emph{Dom Quixote} -- foram
completamente abandonadas pelo bom conselho e não contém mais a menor
centelha de sabedoria.\footnote{Na versão francesa, o capítulo termina
  em ``la moindre étincelle de sagesse'', ``a menor centelha de
  sabedoria''. A sequência final do capítulo, constante no original
  alemão, foi eliminada pelo autor. {[}N. da T.{]}} Quando ao longo dos
séculos, aqui e ali, procurou-se incutir ensinamentos no romance -- e
talvez da maneira mais consistente n'\emph{Os Anos de peregrinação de
Wilhelm Meister} --, tais tentativas sempre acabaram por transformar a
própria forma do romance. Por outro lado, o romance de formação não
contraria de modo algum a estrutura fundamental do romance. Integrando o
processo de vida da sociedade no desenvolvimento de uma pessoa, ele
justifica as ordens que condicionam o primeiro do modo mais fraco. Sua
legitimação encontra-se em desaprumo com sua realidade. Precisamente no
romance de formação, o que é insuficiente torna-se evento.

VI

Devemos pensar a mutação das formas épicas em ritmos comparáveis aos das
transformações que se operaram na superfície terrestre ao longo de
milhares de séculos. Dificilmente outras formas de comunicação humana se
desenvolveram mais lentamente e se perderam mais lentamente. O romance,
cujos primórdios remontam à Antiguidade, levou centenas de anos até
encontrar na burguesia nascente os elementos de que precisava para
florescer. Quando esses elementos entraram em cena, o conto começou
lentamente a se refugiar no domínio do arcaico; se apropriou de diversas
maneiras do novo conteúdo, embora não fosse propriamente condicionado
por ele. Por outro lado, reconhecemos que com o apogeu da burguesia --
da qual a imprensa constitui um dos mais importantes instrumentos no
capitalismo avançado -- apareceu uma forma de comunicação que, embora
suas origens sejam muito antigas, nunca antes influenciara a forma
épica. Agora o faz. E torna-se visível que ela se opõe ao conto de modo
não menos estranho, mas muito mais ameaçador que o romance, além de
provocar uma crise no próprio romance. Essa nova forma de comunicação é
a informação.

Villemessant, o fundador do \emph{Figaro}, descreveu a essência da
informação com uma célebre fórmula: ``Para os meus leitores'', dizia,
``o incêndio num sótão do \emph{Quartier Latin} é mais importante do que
uma revolução em Madri''. Essa expressão mostra de modo claro e sucinto
que a informação sobre os acontecimentos próximos encontra hoje uma
audiência muito maior do que a mensagem que vem de longe. A mensagem que
vinha de longe -- seja espacialmente, de terras estrangeiras, seja
temporalmente, da tradição -- dispunha de uma autoridade que lhe
fornecia validade mesmo onde não fora submetida ao controle. Mas a
informação reclama verificação imediata. Em primeiro lugar, ela precisa
ser ``compreensível em si e para si''. Frequentemente não é mais exata
do que relatos dos séculos anteriores. Entretanto, enquanto estes podiam
lançar mão do maravilhoso, a informação deve soar plausível. Por isso
ela é irreconciliável com o espírito do conto. Se a arte de contar
tornou-se hoje rara, a difusão da informação desempenhou um papel
decisivo em tal situação.

Cada manhã nos informa acerca das novidades do globo terrestre. E mesmo
assim somos pobres em histórias dignas de nota. A razão é que nenhum
fato mais nos atinge sem estar cercado de explicações. Em outras
palavras: com isso, quase nada mais do que acontece vem a favor do
conto, quase tudo vira informação. A metade da arte de contar está em
despojar de explicações a história contada. Leskov é mestre nisso (basta
pensar em textos como ``A fraude'' ou ``A águia branca''\footnote{Ambos
  se encontram traduzidos para o português, tendo sido editados na
  coletânea intitulada \emph{A Fraude e outras histórias}. Tradução de
  Denise Sales. São Paulo: Editora 34, 2012. Também pela Editora 34
  foram publicados, de Nikolai Leskov, o volume de contos \emph{Homens
  interessantes e outras histórias} (tradução de Noé Oliveira Policarpo
  Polli, 2012) e a novela \emph{Lady Macbeth do Distrito de Mtzensk}
  (tradução de Paulo Bezerra, 2009). {[}N. da T. e dos E.{]}}). Ele
conta o extraordinário, o maravilhoso com a maior exatidão, mas o
encadeamento psicológico dos acontecimentos não é imposto ao leitor.
Este fica livre para interpretar a coisa como a entende, e com isso o
que é contado atinge uma amplitude que a informação não tem.

VII

Leskov frequentou a escola dos antigos. O primeiro contador de histórias
grego foi Heródoto. No décimo quarto capítulo do terceiro livro de suas
\emph{Histórias}, encontra-se um relato que muito nos ensina. É sobre
Psaménito. Quando o rei egípcio Psaménito foi vencido e aprisionado pelo
rei da Pérsia Cambises, este decidiu humilhar seu prisioneiro. Ele
ordenou que Psaménito fosse levado até a rua na qual passaria o cortejo
triunfal dos persas. E, assim, fez com que o prisioneiro ainda visse sua
filha, reduzida à condição de serva, indo buscar água no poço com um
jarro. Enquanto todos os egípcios bradavam e se lamentavam com esse
espetáculo, só Psaménito ficou silencioso e imóvel, com os olhos baixos;
e quando, logo depois, assistiu seu filho ser conduzido em cortejo para
execução pública, permaneceu igualmente imóvel. Mas ao reconhecer na
fila dos prisioneiros um de seus antigos servidores, um homem velho e
depauperado, então bateu na cabeça com os punhos e deu sinais da mais
profunda tristeza.

Esta história mostra como é o verdadeiro conto. A informação tira seu
valor do instante em que é nova. Ela vive apenas desse instante, deve
render-se a ele e explicar-se nele sem perda de tempo. O conto funciona
de outro modo: não se desgasta. Ele guarda em si mesmo suas forças
reunidas e longo tempo depois ainda é capaz de se desenvolver. Assim,
Montaigne retornou à história do rei egípcio e se perguntou: Por que ele
se lamenta apenas quando vê seu servidor? Montaigne respondeu: ``Como já
estava repleto de tristeza, a menor sobrecarga bastou para arrebentar
todos os diques''.\footnote{Cf. Michel de Montaigne, \emph{Essais},
  Livro I, capítulo II. Paris: Gallimard, 1962. (col. Bibliothèque de la
  Pléiade) {[}N. da T.{]}} Isso disse Montaigne. Mas poderíamos também
dizer: ``O rei não se comove com o destino da realeza pois é o seu
próprio destino''. Ou: ``muito do que nos comove na cena teatral não nos
comove na vida: este servidor é para o rei apenas um ator''. Ou ainda:
``A maior dor fica represada e só vem à tona quando ocorre uma
distensão. A visão desse servidor foi a distensão.'' -- Heródoto não
explica nada. Seu relato é dos mais secos. É por isso que essa história
do antigo Egito ainda desperta espanto e reflexão depois de milhares de
anos. Ela se assemelha àquelas sementes que, durante milênios, ficaram
fechadas hermeticamente nas câmaras das pirâmides e ainda hoje conservam
o seu poder de germinar.\footnote{Por volta de 1928-1929, Benjamin
  escreveu um texto curto intitulado ``Arte de contar histórias''
  (``Kunst zu erzählen'', em \emph{Imagens de Pensamento},
  \emph{Denkbilder}, GS IV-1, pp. 436-438), que já abordava a história
  de Psaménito e explorava a plurivocidade de sua extrema concisão.
  Entre os manuscritos do início dos anos trinta, encontramos também
  notas sobre Psaménito, contendo diversas interpretações propostas por
  seus amigos e conhecidos:

  ``Interpretação de {[}Franz{]} Hessel~: `o rei não se emociona com a
  sorte da família real, pois é a sua própria.'

  Interpretação de Benjamin: `a dor se desencadeia mais facilmente sob
  um pretexto menos importante do que sua causa. É uma grande tampa
  sobre uma pequena panela. Ou então ela evita até o pretexto e
  privilegia o choque. É um tal choque que desencadeia as lágrimas em
  Proust depois da morte de sua avó querida: o gesto de se abaixar para
  abotoar as botinas.'

  Interpretação de Asja {[}Lacis{]}: `Muitas coisas que nos emocionam no
  teatro nos deixam indiferentes na vida; esse velho é apenas um ator
  para o rei.'

  Interpretação de Montaigne: `Ocorre que estando além disso já pleno e
  cheio de tristeza, a menor sobrecarga arrebentou as barreiras da
  paciência.'

  Observação de Martin-Gueillot: `Se Psaménito tivesse vivido em nossos
  dias, todos os jornais nos teriam ensinado que ele preferia seu
  empregado a seus filhos.' (\emph{Nouvelle Revue Française}, 1928, p.
  696)'' Cf. GS IV-2, p. 1011. {[}N. da T.{]}}

VIII

Não há nada que imprima mais duradouramente as histórias na memória do
que essa recatada concisão que as afasta da análise psicológica. E
quanto mais naturalmente o contador renuncia aos detalhes psicológicos,
mais facilmente sua história encontra lugar na memória do ouvinte, mais
perfeitamente junta-se à sua própria experiência, e assim mais prazer
ele terá um dia, próximo ou longínquo, em recontá-la. Esse processo de
assimilação, que ocorre em camadas profundas, exige um estado de
distensão que se torna cada vez mais raro. Se o sono é o ponto mais alto
da distensão física, o tédio corresponde ao ponto mais alto da distensão
espiritual. O tédio é o pássaro de sonho que choca o ovo da experiência.
O menor farfalhar de folhas o afasta. Seus ninhos, as atividades
intrinsecamente ligadas ao tédio, não encontram mais lugar nas cidades e
estão se tornando raras também no campo. Com isto, perde-se o dom de
escutar e desaparece a comunidade dos que escutam. Contar histórias é
sempre a arte de contá-las novamente, arte que se perde quando as
histórias não são mais conservadas. Ela se perde pois ninguém mais tece
nem fia enquanto ouve histórias. Quanto mais o ouvinte se esquece de si
mesmo, mais profundamente aquilo que escuta se imprime nele. Onde o
ritmo do trabalho o toma inteiramente, ele ouve as histórias de tal
maneira que o dom de contá-las lhe vem espontaneamente. Assim foi tecida
a rede na qual está contido o dom de contar. Assim essa rede se desfaz
hoje por todos os lados, depois de ter sido tecida há milhares de anos,
em torno das mais antigas formas de artesanato.

IX

O conto, tal como prosperou longo tempo na esfera do artesanato --
artesanato camponês, marítimo e depois urbano --, é ele mesmo uma forma
artesanal de comunicação. Ele não visa transmitir o puro ``em si'' da
coisa como uma informação ou um relatório. Ele mergulha a coisa na vida
daquele que a relata para em seguida daí retirá-la. O contador deixa sua
marca no conto, assim como o oleiro deixa a impressão de sua mão na
argila do vaso. Os contadores têm sempre a tendência a começar suas
histórias com uma apresentação das circunstâncias em que tomaram
conhecimento do que irão em seguida contar, isso quando não preferem
dizer que o vivenciaram pessoalmente. Leskov inicia ``A
fraude''\footnote{Cf. Nikolai Leskov, \emph{A Fraude e outras
  histórias}, ed. cit. {[}N. da T.{]}} com a descrição de uma viagem de
trem na qual teria ouvido de um outro viajante os acontecimentos que
passa em seguida a contar; ou pensa no enterro de Dostoiévski, no qual
travou conhecimento com a heroína de seu conto ``A propósito de \emph{A
Sonata a Kreutzer}''\footnote{Idem, ibidem. {[}N. da T.{]}}; ou faz
alusão à reunião de um grupo de leitura no qual falou-se dos fatos que
relata em ``Homens interessantes''.\footnote{Cf. Nikolai Leskov,
  \emph{Homens interessantes e outras histórias}, ed. cit. {[}N. da
  T.{]}} Assim, deixa sua marca naquilo que conta de diversos modos,
seja como aquele que o viveu, seja como aquele que o relata.

Esta arte artesanal de contar, o próprio Leskov a considerava como um
ofício. ``A literatura'', diz em suas cartas, ``não é para mim uma arte
livre, mas um ofício artesanal''. Não é de se estranhar que ele tenha se
sentido ligado ao artesanato e hostil à técnica industrial. Tolstói, que
neste ponto de vista devia compreendê-lo, indica este aspecto do talento
narrativo de Leskov quando o aponta como o primeiro a ``denunciar a
deficiência do progresso industrial (...). É estranho que tanta gente
leia Dostoiévski (...). Por outro lado, simplesmente não entendo porque
Leskov não é lido. É um escritor verídico''.\footnote{Citado por Erich
  Müller, ``Nikolai Semjonowitsch Lesskow. Sein Leben und Wirken'',
  in~Nikolai Lesskow, \emph{Am Ende der Welt}. München: C.H. Beck, 1927,
  p. 313. (\emph{Gesammelte Werke}, vol. IX) {[}N. da T.{]}} Numa
história maliciosa e petulante, ``A pulga de aço'', a meio caminho entre
a lenda e a farsa, Leskov louva, através dos ourives de Tula, o
artesanato de sua terra. Sua obra prima, a pulga de aço, chega aos olhos
de Pedro, o Grande, e o convence de que os russos não precisam se
envergonhar diante dos ingleses.

A imagem\footnote{Na variante da versão francesa: ``lei''. {[}N. da
  T.{]}} espiritual desta esfera artesanal da qual provém o contador
nunca foi descrita de modo mais significativo do que por Paul Valéry. Ao
falar de coisas perfeitas na natureza, pérolas imaculadas, vinhos
maduros e profundos, de criaturas verdadeiramente completas, ele
reconhece nelas ``a valiosa obra de uma longa cadeia de causas
sucessivas e semelhantes''.\footnote{Paul Valéry, ``Les broderies de
  Marie Monnier'', \emph{Pièces sur l'art}, in~\emph{Oeuvres}, tomo II.
  Edição de Jean Hytier. Paris: Gallimard, 1960, p. 1244. (col.
  Bibliothèque de la Pléiade) {[}N. da T.{]}} Mas a acumulação de tais
causas encontraria seu limite de tempo apenas na perfeição. ``Outrora'',
continua Paul Valéry, ``o homem imitava esta paciência da natureza.
Iluminuras, marfins inteiramente entalhados à perfeição, pedras duras
perfeitamente polidas e distintamente gravadas; lacas ou pinturas
obtidas através da sobreposição de uma série de camadas finas e
translúcidas (...) -- todas essas produções de um esforço tenaz e pleno
de autorrenúncia estão desaparecendo, e passou o tempo em que o tempo
não contava. O homem de hoje não cultiva mais o que não pode ser
abreviado''.\footnote{Idem, ibidem. {[}N. da T.{]}} De fato, conseguiu
até abreviar o conto. Vivenciamos o surgimento da \emph{short story},
que se emancipou da tradição oral e não mais permite essa lenta
sobreposição de camadas finas e transparentes que oferece a melhor
imagem da maneira pela qual o conto perfeito vem à luz do dia a partir
das camadas acumuladas por suas diferentes versões.\footnote{``Fomos
  testemunhas do nascimento da \emph{short story}. Ela abala o prestígio
  do conto, o qual consiste em religar as gerações de contadores entre
  si.'', conforme variante da versão francesa. {[}N. da T.{]}}

X

Valéry conclui suas considerações com a seguinte frase: ``Pode-se dizer
que o enfraquecimento nos espíritos da ideia de eternidade coincide com
a aversão crescente por tarefas demoradas''.\footnote{Paul Valéry, ``Les
  broderies de Marie Monnier'', \emph{op. cit.}, p. 1244. {[}N. da T.{]}}
A ideia de eternidade teve sempre na morte sua fonte mais poderosa. Se
essa ideia se enfraqueceu, isto significa que o rosto da morte também se
modificou. Essa modificação demonstra ser a mesma que reduziu a
comunicabilidade da experiência na medida em que a arte de contar
chegava ao fim.

Se seguimos o decorrer dos séculos precedentes, percebemos a que ponto a
ideia da morte perde a onipresença e a força plástica que encontrava na
consciência coletiva. Em suas últimas etapas, esse processo se acelerou.
No decurso do século XIX, a sociedade burguesa, com suas instituições
econômicas e sociais, públicas ou privadas, realizou um feito acessório
que, inconscientemente, foi talvez seu objetivo principal: dar às
pessoas a possibilidade de se esquivar à visão dos moribundos. O ato de
morrer, outrora o mais público e o mais exemplar da vida individual
(lembremo-nos aqui das imagens da Idade Média nas quais a cama do
moribundo vira um trono diante do qual se aglomera o povo que entra
pelas portas abertas de sua casa), subtraiu-se aos poucos da atenção dos
vivos no decorrer da época moderna. Outrora não havia casa, por vezes
nem mesmo quarto, onde ninguém tivesse morrido. (A Idade Média tinha
também a intuição espacial do sentimento temporal evocado por aquela
inscrição num relógio solar de Ibiza: \emph{Ultima multis}.) Hoje em dia
os burgueses vivem em espaços depurados de qualquer morte, primeiros
moradores da eternidade, e, quando chegam perto do fim, são depositados
por seus herdeiros em sanatórios e hospitais. Ora, não apenas o
conhecimento e a sabedoria de um ser humano, mas sobretudo sua vida
vivida -- e esta é a matéria da qual as histórias são feitas -- assumem
uma primeira forma transmissível no leito do moribundo. Assim como no
interior da pessoa, no momento final de sua vida, uma sequência de
imagens se põe em movimento -- constituídas das visões de si, sob as
quais, sem se dar conta, encontrou-se consigo mesma --, também se revela
de repente, em seus gestos e olhares, o inesquecível que atribui a tudo
o que lhe diz respeito essa autoridade que mesmo o mais miserável dos
moribundos possui aos olhos dos vivos à sua volta. É essa autoridade que
está na origem do que foi contado.

XI

A morte é a sanção de tudo o que o contador de histórias pode contar. É
a morte que lhe confere sua autoridade. Em outras palavras: suas
histórias remetem à história natural. Isto se exprime de forma exemplar
numa das mais belas passagens do incomparável Johann Peter Hebel, que se
encontra em ``Reencontro inesperado''\footnote{Na versão francesa,
  ``\emph{Reencontro inesperado dos Amantes}''. {[}N. da T.{]}}, conto
incluído n'\emph{A caixa de tesouros do amigo renano das famílias}. Este
relato começa com o noivado de um jovem operário que trabalha nas minas
de Falun.\footnote{``nas minas de Falun na Suécia'', consoante o
  acréscimo pontual da versão francesa. {[}N. da T.{]}} Na véspera do
casamento, no fundo de sua galeria subterrânea, a morte dos mineiros o
surpreendeu. Sua noiva permanece fiel depois de sua morte e vive ainda
um longo tempo. Um dia, quando já é uma velhinha de idade avançada, um
cadáver é trazido à luz do fundo da galeria abandonada. Impregnado de
vitríolo de ferro, escapou da decomposição e ela pôde reconhecer o seu
noivo. Depois deste reencontro, também ela foi chamada pela morte.
Quando Hebel, no curso de seu conto, sentiu a necessidade de tornar
palpável a longa série de anos que separa o começo do fim, ele o fez com
as seguintes frases: ``Entrementes a cidade de Lisboa foi destruída por
um terremoto, e a guerra dos sete anos passou, e o Imperador Francisco I
morreu, e a ordem dos jesuítas foi dissolvida, e a Polônia foi
repartida, e a Imperatriz Maria Teresa morreu, e Struensee foi
executado, a América tornou-se livre e as forças reunidas da França e da
Espanha não puderam conquistar Gibraltar. Os turcos aprisionaram o
General Stein na gruta dos veteranos, na Hungria, e o Imperador José
morreu também. O rei Gustavo da Suécia conquistou a Finlândia russa, e a
Revolução Francesa e a longa guerra começaram, e o Imperador Leopoldo II
também foi para a cova. Napoleão conquistou a Prússia, e os Ingleses
bombardearam Copenhague, e os camponeses semearam e ceifaram. O moleiro
moeu, e os ferreiros forjaram, e os mineiros cavaram a terra em busca de
filões metálicos em seu canteiro subterrâneo. Ora, quando os mineiros de
Falun, no ano de 1809...''\footnote{Johan Peter Hebel, \emph{Sämtliche
  Werke}, vol. III. Karlsruhe: Müller, 1834, p. 188. {[}N. da T.{]}}
Nunca um contador introduzira seu relato na história natural de forma
mais profunda do que Hebel nesta cronologia. Leia-se atentamente: a
morte nela reaparece de modo tão regular quanto o homem com a foice nos
cortejos das procissões que desfilam ao meio-dia em torno do relógio da
catedral.

XII

Cada estudo de uma certa forma épica deve levar em consideração a
relação desta forma para com a historiografia. Mais ainda, podemos até
nos perguntar se a historiografia não apresenta o ponto de indiferença
criadora entre todas as formas épicas. Neste caso, a história escrita
estaria para as formas épicas assim como a luz branca para as cores do
espectro. Seja como for, entre todas as formas épicas não há nenhuma que
tenha sido tão indiscutivelmente incorporada à luz pura e incolor da
história quanto a crônica. E na larga gama colorida da crônica, as
diferentes maneiras de contar se escalonam como as nuances de uma e
mesma cor. O cronista é o contador da história. Lembremo-nos ainda da
passagem de Hebel, que tem o tom da crônica, e notaremos sem dificuldade
a diferença entre aquele que escreve a história, o historiador, e aquele
que a conta, o cronista. O historiador deve de uma maneira ou de outra
explicar os fatos que o ocupam; ele não poderia de modo algum se
contentar em expô-los como amostras do curso do mundo. É justamente o
que faz o cronista, e especialmente os seus representantes clássicos, os
cronistas medievais, que foram os precursores dos historiógrafos
modernos. Na base de sua narrativa da história encontra-se o plano
divino da salvação, que é insondável, e com isso se desembaraçaram de
antemão do ônus de uma explicação demonstrável. Esta cede lugar à
interpretação, a qual não se preocupa de modo algum em encadear com
precisão fatos determinados, mas limita sua tarefa a descrever como
estes se inserem na trama insondável do curso do mundo.

Se o curso do mundo é condicionado por uma história sagrada ou por uma
história natural não faz nenhuma diferença. No contador de histórias, a
figura do cronista conservou-se transformada, e por assim dizer
secularizada. Leskov está entre aqueles cuja obra dá testemunho de modo
particularmente claro deste estado de coisas. O cronista, com sua
orientação para a história sagrada, e o contador, com sua orientação
para a história profana, fundam-se ambos em sua obra, de tal modo que em
muitos contos é difícil decidir se o fundo sobre o qual estes se
destacam é a trama dourada de uma concepção religiosa ou a trama
colorida de uma visão secularizada do curso das coisas. Pensemos no
conto ``A Alexandrita'', que leva os leitores a ``priscas eras, quando
tanto as pedras nas entranhas da terra quanto os planetas nas alturas
celestes, todos eles se preocupavam com o destino do homem, e não
atualmente, quando até nos céus há desgosto e sob a terra restou a
indiferença fria pelo destino dos filhos dos homens e de lá não chegam
vozes nem obediência. Todos os planetas, de novo descobertos, já não
recebiam mais nenhuma atribuição nos horóscopos; há também muitas pedras
novas, e todas são medidas, pesadas, comparadas em termos de peso
específico e densidade, mas depois nada profetizam, não são úteis em
nada. O seu tempo de falar ao homem já virou passado''.\footnote{Nikolai
  Leskov, ``Alexandrita'', in \emph{A Fraude e outras histórias}, ed.
  cit., p. 160. {[}N. da T.{]}}

Como se vê, não é possível caracterizar o curso do mundo de modo
unívoco, tal como o ilustra esta história de Leskov. É ele determinado
pela história sagrada ou pela história natural? É certo apenas que,
enquanto tal, o curso do mundo é estranho a toda categoria propriamente
histórica. Foi-se a época, diz Leskov, em que o homem podia acreditar
viver em uníssono com a natureza. Schiller chamou essa era de o tempo da
poesia ingênua. O contador permanece fiel a ela e seu olhar não se
desvia do relógio diante do qual avança a procissão das criaturas, onde
a morte encontra seu lugar, seja na dianteira do cortejo ou como uma
pobre retardatária.

XIII\footnote{Neste capítulo, Benjamin utiliza termos que cobrem o campo
  semântico da memória (\emph{Erinnerung}, \emph{Gedächtnis},
  \emph{Eingedenken}) e não têm equivalente imediato em português. O
  próprio autor encontra dificuldade em transpô-los para o francês,
  preferindo traduzir tanto \emph{Erinnerung} quanto \emph{Gedächtnis}
  por ``memória'' e introduzir uma distinção entre a \emph{souvenance},
  que caracteriza o romance, e o \emph{souvenir}, que concerne ao conto,
  distinção que não aparece no original alemão. {[}N. da T.{]}}

Raramente nos damos conta de que a relação ingênua do ouvinte com o
contador é determinada pelo interesse em guardar o que foi contado. O
que importa ao ouvinte isento é assegurar-se da possibilidade da
repetição. A memória é a faculdade épica por excelência. Somente graças
a uma memória abrangente, a épica pode apropriar-se do curso das coisas,
por um lado, e por outro resignar-se com sua perda irreparável, com o
poder da morte. Não é de se espantar que para um homem simples do povo,
tal como Leskov o imaginou um dia, o tsar, enquanto soberano do cosmos
no qual suas histórias acontecem, possui a memória mais vasta. ``Nosso
imperador e toda a sua família, diz ele, tem efetivamente uma memória
prodigiosa.''

Mnemosyne, aquela que recorda, era entre os gregos a musa do gênero
épico. Esse nome nos reconduz a uma bifurcação na história do mundo. Com
efeito, se aquilo que é registrado pela recordação -- a historiografia
-- expõe a indistinção criadora entre os diversos gêneros épicos (assim
como a grande prosa expõe a ausência de diferenciação criadora entre as
diversas métricas poéticas), sua forma mais antiga, a epopeia
propriamente dita, contém, por força de um tipo de indistinção, o conto
e o romance. Quando, no decorrer dos séculos, o romance começou a surgir
no seio da epopeia, viu-se claramente que o elemento inspirador da
poesia épica, a recordação, nele aparecia de um modo diferente daquele
do conto.

A \emph{recordação}\footnote{\emph{Erinnerung}, no original alemão,
  \emph{mémoire}, na versão francesa. {[}N. da T.{]}} estabelece a
cadeia da tradição que transmite os acontecimentos de geração em
geração. Ela é a musa da épica em geral e preside todas as variedades do
gênero épico. Entre estas, encontramos em primeiro lugar aquela
encarnada pelo contador. Ela tece a rede formada por todas as histórias.
Uma está ligada à outra, como mostraram todos os grandes contadores, e
principalmente os orientais. Em cada um deles vive uma Sherazade, que em
cada passagem de sua história lembra-se de outra. Esta é uma
\emph{memória}\footnote{\emph{Gedächtnis}, no original alemão,
  \emph{mémoire}, na versão francesa. {[}N. da T.{]}} épica e a musa
inspiradora do conto. É preciso opor a ela um outro princípio
inspirador, o do romance, o qual, ainda indiferenciado daquele que é
próprio ao conto, também habita o gênero épico em sentido
estrito.\footnote{Na variante da versão francesa: ``Encontra-se um
  elemento análogo, mas intrinsecamente diferente, na base do romance. E
  como no que diz respeito ao conto, podemos avançar, para o romance,
  que primitivamente, isto é, na epopeia, formava apenas um germe na
  unidade indivisa do gênero épico.'' {[}N. da T.{]}} Às vezes, já
podemos pressenti-lo nas epopeias. E isto ocorre antes de tudo nas
invocações solenes das musas que abrem os poemas homéricos. O que se
anuncia em tais passagens é a memória perpetuadora\footnote{\emph{Verewigende
  Gedächtnis}, no original alemão, \emph{souvenance éternisante}, na
  versão francesa. {[}N. da T.{]}} do romancista em oposição à memória
divertida\footnote{\emph{Kurzweilige Gedächtnis}, no original alemão,
  \emph{souvenir passe-temps}, na versão francesa. {[}N. da T.{]}} do
contador. A primeira diz respeito a \emph{um} herói, a \emph{uma}
odisseia ou a \emph{uma} guerra; a segunda concerne a \emph{múltiplos}
fatos dispersos. Em outras palavras: a \emph{rememoração}\footnote{\emph{Eingedenken},
  no original alemão, \emph{souvenance}, na versão francesa. Tanto o
  termo alemão quanto o francês conotam o desenrolar do processo em
  oposição aos seus efeitos pontuais, conotação que procuramos preservar
  na tradução. {[}N. da T.{]}}, musa do romance, surge ao lado da
memória\footnote{\emph{Gedächtnis}, no original alemão, \emph{souvenir},
  na versão francesa. {[}N. da T.{]}}, musa do conto, depois que a
unidade de sua origem na recordação\footnote{\emph{Erinnerung}, no
  original alemão, \emph{mémoire}, na versão francesa. {[}N. da T.{]}}
se rompeu com o declínio da epopeia.

XIV

``Ninguém'', disse Pascal, ``morre tão pobre que não deixe alguma
coisa''. Com certeza, deixa ao menos recordações -- só que estas nem
sempre encontram herdeiros. O romancista recolhe esse legado, e
raramente sem uma profunda melancolia. Pois com a soma por ele recebida,
ocorre o mesmo que se pode dizer de uma morta num romance de Arnold
Bennett: ``ela não tinha efetivamente vivido''. Devemos o esclarecimento
mais importante sobre esse aspecto das coisas a Georg Lukács, que viu no
romance ``a forma do desenraizamento transcendental''. Do mesmo modo, de
acordo com Lukács, o romance é a única forma de arte que inclui o tempo
entre seus princípios constitutivos. ``O tempo'', diz ele na
\emph{Teoria do Romance}, ``só pode tornar-se constitutivo quando o
homem deixou de estar ligado a uma pátria transcendental (...). No
romance, separam-se sentido e vida, e com isso também o essencial e o
temporal; pode-se quase dizer que toda a ação interna do romance não é
nada mais do que uma luta contra o poder do tempo (...). E desse
(...)\footnote{A passagem suprimida neste ponto da citação, feita por
  Benjamin, da obra referida de Lukács, e sem a qual a inteligibilidade
  fica parcialmente comprometida, é: ``sentimento maduro e resignado''
  (\emph{resigniert-mannbaren Gefühl}). Cf. Georg Lukács, \emph{Die
  Theorie des Romans. Ein geschichtsphilosophischer Versuch über die
  Formen der großen Epik}. 9.ª ed. Darmstadt; Neuwied: Luchterhand,
  1984, p. 110 {[}1.ª ed., Berlim, 1920{]}; e também Georg Lukács,
  \emph{A teoria do romance. Um ensaio histórico-filosófico sobre as
  formas da grande épica}. Tradução de José Marcos Mariani de Macedo.
  São Paulo: Editora 34: Duas Cidades, 2009, p. 130. {[}N. dos E.{]}}
emergem as vivências temporais autenticamente épicas (\ldots{}): a
esperança e a recordação. (...) Apenas no romance (...) aparece uma
recordação criadora que atinge o fundamento do seu objeto e o transforma
(...). A dualidade da interioridade e do mundo exterior'' só ``pode ser
superada pelo sujeito quando este percebe a unidade (...) de toda a sua
vida (...) resumida na recordação da corrente vital do passado (...). A
percepção que apreende essa unidade (...) torna-se compreensão
divinatoriamente intuitiva do sentido da vida inatingido e, portanto,
inexprimível''.\footnote{Cf. Georg Lukács, \emph{Die Theorie des
  Romans}, ed. cit., pp. 107-115; e Georg Lukács, \emph{A teoria do
  romance}, ed. cit., pp. 128-136\emph{.} {[}N. da T. e dos E.{]}}

``O sentido da vida'' é de fato o eixo em torno do qual gira o romance.
O questionamento deste sentido não é, entretanto, nada mais do que a
expressão simples do desaconselhamento do leitor quando este se vê
inserido no âmago desta vida escrita. De um lado, ``o sentido da vida'';
de outro, ``a moral da história'': com essas fórmulas, romance e conto
se confrontam; elas permitem distinguir completamente o índice histórico
do estatuto das duas formas artísticas. Se o primeiro modelo perfeito de
romance é o \emph{Dom Quixote}, \emph{Educação Sentimental} pode ser
considerado como o último. Nas palavras que concluem este romance, o
sentido, que caracterizava o início do declínio da época burguesa em seu
modo de agir e omitir, se deposita como o resíduo\footnote{``resíduo do
  vinho'', na variante da versão francesa. {[}N. da T.{]}} no fundo do
copo da vida. Amigos de juventude, Frédéric e Deslauriers rememoram sua
amizade de juventude. Uma pequena história lhes vem à cabeça, eles se
lembram do dia em que, cheios de constrangimento e clandestinamente,
estiveram na casa de tolerância de sua cidade natal apenas para oferecer
à patroa um buquê de flores colhidas no jardim. ``Isso deu numa história
que três anos depois ainda não tinha sido esquecida. Eles a contaram a
si mesmos prolixamente, cada um completando as recordações do outro, e
quando finalmente acabaram: `É o que tivemos de melhor!', disse
Frédéric. -- `Sim, talvez seja mesmo! Isso aí é o que tivemos de
melhor!', disse Deslauriers''. Com tal reconhecimento, o romance chega
ao seu fim; fim que lhe é próprio no sentido mais estrito do que a
qualquer conto. Não há, de fato, nenhum conto em que a questão ``o que
aconteceu depois?'' perca seus direitos. O romance, ao contrário, não
pode esperar dar o menor passo além daquela fronteira onde, ao escrever
a palavra ``fim'' na parte inferior da página, convida o leitor a ter em
mente, por pressentimento, o sentido da vida.

XV

Quem ouve uma história se encontra na companhia do contador; mesmo quem
a lê participa desta companhia. Por outro lado, mais do que qualquer
outro, o leitor de um romance é solitário. (Pois até mesmo aquele que lê
um poema tende a emprestar sua voz às palavras, visando ouvintes
possíveis.) E nessa solidão que lhe é própria, o leitor do romance se
apodera da matéria lida de modo mais possessivo do que todos os demais.
Ele está disposto a se apropriar dela inteiramente e, de certo modo,
devorá-la. Sim, ele a destrói, consumindo-a como faz o fogo com a lenha
na lareira. A tensão que atravessa o romance se assemelha muito à
corrente de ar que alimenta a chama e reanima o seu jogo.

É um material seco que alimenta o interesse ardente do leitor. -- O que
isto significa? ``Um homem que morre aos trinta e cinco anos é'', como
disse uma vez Moritz Heimann, ``a cada instante de sua vida, um homem
que morre aos trinta e cinco anos''. Nada é mais duvidoso do que esta
sentença. Mas somente porque o autor se engana quanto ao tempo verbal. A
verdade é que um homem que morreu aos trinta e cinco anos aparecerá na
\emph{rememoração}, em cada instante de sua vida, como um homem que
morre aos trinta e cinco anos. Em outras palavras: a sentença, que não
tem sentido para a vida real, torna-se irrefutável para a vida
recordada. Nada apresenta melhor a essência do personagem de romance. O
``sentido'' de sua vida -- é o que a frase nos diz -- só se mostra a
partir de sua morte. Ora, o leitor de romances procura efetivamente
seres humanos nos quais pode ler o ``sentido da vida''. Ele precisa de
antemão ter certeza de que poderá, de um jeito ou de outro, assistir à
morte deles. Se necessário, a morte no sentido figurado: o fim do
romance. Ainda melhor quando isso ocorre no sentido próprio. Como o
protagonista indica que a morte já o espera, isto é, uma morte bem
determinada, em um ponto bem determinado? Eis a questão que alimenta o
interesse devorador do leitor por aquilo que ocorre no romance.

O romance, portanto, não é significativo graças a um ensinamento
qualquer que um destino alheio nos apresentaria, mas porque, através da
chama que o devora, este destino alheio nos transmite um calor que não
podemos tirar da nossa própria vida. O que prende o leitor ao romance é
a esperança de aquecer sua vida gelada em uma morte sobre a qual ele lê.

XVI

``Leskov'', escreve Gorki, ``é o escritor mais profundamente (...)
enraizado no povo e o mais isento de toda influência estrangeira''. O
grande contador terá sempre suas raízes no povo, e em primeiro lugar nas
camadas artesanais. Entretanto, assim como estas englobam, nos múltiplos
estágios de seu desenvolvimento econômico e técnico, os elementos
camponês, marítimo e urbano, também os conceitos nos quais a soma de
suas experiências se cristaliza para nós possuem múltiplas gradações.
(Sem falar na considerável contribuição dos mercadores para a arte de
contar; eles não precisaram tanto aumentar o conteúdo instrutivo dos
contos\footnote{``através de seus relatos de terras longínquas'',
  consoante acréscimo efetuado por Benjamin na versão francesa. {[}N. da
  T.{]}}, quanto refinar as estratégias destinadas a captar a atenção
dos ouvintes.\footnote{``De fato, não vemos entre os contistas árabes o
  ouvinte tornar-se cliente de um contador de histórias?'', segundo
  acréscimo da versão francesa. {[}N. da T.{]}} Deixaram marcas
profundas no ciclo de histórias das \emph{Mil e uma noites}.) Em suma,
sem desconsiderar o papel elementar que a narrativa desempenha na
economia doméstica da humanidade, os conceitos através dos quais podemos
colher os seus frutos são de uma grande diversidade. O que, em Leskov,
deve ser compreendido de modo mais tangível em um sentido religioso,
parece se ajustar por si mesmo, em Hebel, às perspectivas pedagógicas do
Iluminismo, surge em Poe como tradição hermética e encontra um último
asilo em Kipling, no campo de ação de marinheiros e soldados coloniais
britânicos. Por outro lado, todos os grandes contadores de história têm
em comum a facilidade com a qual descem e sobem os degraus de sua
experiência, como numa escada. Uma escada cuja base desce até as
profundezas da terra e cujo topo se perde nas nuvens é a imagem de uma
experiência coletiva para a qual o mais profundo choque de cada
experiência individual, a morte, não representa nem um escândalo nem uma
barreira.

``E se não morreram, vivem até hoje'', diz o conto de fadas. O conto de
fadas, que ainda hoje é o primeiro conselheiro das crianças, pois foi
outrora o primeiro da humanidade, sobrevive secretamente no conto. O
primeiro contador verdadeiro é e continua sendo o dos contos de fadas.
Este conto sabia trazer um bom conselho, onde nada era mais difícil de
se encontrar, e onde a necessidade era a mais urgente, a \emph{sua}
ajuda era a que estava mais próxima. Essa necessidade era a do mito. O
conto de fadas nos informa sobre as primeiras tentativas da humanidade
em sacudir para fora o pesadelo que o mito depositou em seu peito. A
figura do bobo nos mostra como a humanidade ``faz-se de boba'' contra o
mito; a do irmão caçula nos mostra como suas chances aumentam com o
distanciamento em relação à época primeva mítica; a figura daquele que
sai de casa para aprender o medo nos mostra que podemos tornar
transparentes as coisas que tememos; a figura do esperto nos mostra que
as questões do mito são tão simples quanto as da esfinge; as figuras dos
bichos que vêm ajudar a criança do conto nos mostram que a natureza
prefere muito mais associar-se ao homem do que comprometer-se com o
mito. O conto de fadas ensinou há muito tempo à humanidade e ainda hoje
ensina às crianças que o mais aconselhável é enfrentar o mundo do mito
com astúcia e ousadia. (Deste modo, o conto de fadas polariza a coragem,
a saber, dialeticamente, em astúcia e em ousadia.\footnote{Benjamin faz,
  nesta passagem, um jogo de palavras com \emph{Mut} (``coragem''),
  \emph{Übermut}, que quer dizer ``alegria exuberante'', ``animação
  excessiva que chega às raias da insolência ou da ousadia'', e um termo
  insólito, \emph{Untermut}, que aparece aqui como equivalente
  a~\emph{List}, ``astúcia''. {[}N. da T.{]}}) A magia liberadora do
conto de fadas não coloca em cena a natureza de um modo mítico, mas
indica a sua cumplicidade com o ser humano liberado. O homem maduro
concebe esta cumplicidade apenas ocasionalmente, isto é, quando está
feliz; para a criança, ela aparece pela primeira vez nos contos de fadas
e provoca sua felicidade.

XVII

Poucos contadores tiveram uma afinidade tão profunda com o espírito dos
contos de fadas quanto Leskov. Trata-se de uma tendência reforçada pelos
dogmas da Igreja Ortodoxa grega. Nesta dogmática, a especulação de
Orígenes sobre a apocatástase -- a admissão de todas as almas no paraíso
--, que foi descartada pelo catolicismo romano, desempenha um papel
significativo. Leskov foi muito influenciado por Orígenes. Planejou
traduzir sua obra \emph{Sobre os Princípios}.\footnote{Há tradução para
  o português. Cf. Orígenes. \emph{Tratado sobre os Princípios}.
  Tradução de João Eduardo Pinto Basto Lupi. São Paulo: Paulus, 2012.
  (Coleção Patrística, vol. 30) {[}N. dos E.{]}} De acordo com as
crenças populares russas, interpretou a ressurreição menos como uma
transfiguração do que como um desencantamento (num sentido parecido com
o dos contos de fadas). Essa interpretação de Orígenes constitui a base
do conto ``O peregrino encantado''. Aqui, como em muitas outras
histórias de Leskov, trata-se de um misto de conto de fadas e lenda, não
muito diferente do misto de conto de fadas e saga de que fala Ernst
Bloch numa passagem em que retoma a seu modo a nossa distinção entre
mito e conto de fadas. Segundo Bloch, um ``misto de conto de fadas e
saga inclui, de modo figurado, um elemento mítico que atua de modo
estático e encantatório, embora não fora do ser humano. Assim, na saga,
personagens de tipo taoista são `míticos', em particular os muito
antigos, como o casal Filemon e Baucis: redimidos, como nos contos de
fadas, apesar de repousarem como natureza. E certamente esse tipo de
relação existe também no Tao menos acentuado de Gotthelf; às vezes, ele
priva a saga do local encantado, salva a luz da vida, a luz propriamente
humana da vida que arde tranquilamente, por dentro e por
fora''.\footnote{Citação de Ernst Bloch, \emph{Erbschaft dieser Zeit}.
  Frankfurt a. M.: Suhrkamp, 1962. (\emph{Gesammtausgabe}, vol. 4) {[}N.
  da T.{]}} Redimidas, ``como nos contos de fadas'', são aquelas
criaturas que abrem o cortejo da criação de Leskov: os justos. Pavlin,
Figura, o artista cabeleireiro, o domador de ursos, a sentinela
prestativa -- todos aqueles que personificam a sabedoria, a bondade, a
consolação do mundo amontoam-se em torno do contador. Estão todos
incontestavelmente impregnados da imago de sua mãe. De acordo com a
descrição de Leskov, ``ela tinha a alma tão boa que era incapaz de fazer
mal a qualquer ser humano, ou mesmo aos animais. Não comia carne nem
peixe tal sua compaixão por todas as criaturas vivas. Meu pai tinha o
costume de reprovar-lhe, de vez em quando, tal atitude. Mas ela
respondia: `Eu mesma criei esses bichinhos, são como meus filhos. Não
posso comer meus próprios filhos!' Mesmo na casa dos vizinhos não comia
carne. `Eu os vi vivos', dizia, `são meus conhecidos. Não posso comer
meus conhecidos'''.

O justo é ao mesmo tempo o porta-voz da criatura e sua mais alta
personificação. Em Leskov, ele possui um traço maternal que se eleva às
vezes até o mítico (colocando assim, é verdade, em risco a pureza do
conto de fadas). Exemplar neste sentido é o personagem principal de seu
conto ``Kótin, o provedor, e Platonida''. Este personagem, o camponês
Pizónski, é bissexuado. Durante doze anos, foi criado pela mãe como
menina. Seu lado feminino amadureceu ao mesmo tempo que sua
masculinidade, e sua dupla sexualidade tornou-se um símbolo do
Homem-Deus.\footnote{Cf. Nikolai Leskov, ``Kótin, o provedor, e
  Platonida'', in \emph{A Fraude e outras histórias}, ed. cit., pp.
  10-11. {[}N. da T.{]}}

Neste símbolo, Leskov acredita poder atingir o apogeu da criatura e, ao
mesmo tempo, estabelecer uma ponte entre os mundos terrestre e
supraterrestre. Pois estes homens cuja potência vem da terra, estas
figuras masculinas maternais, que reiteradamente tomam posse da arte
fabuladora de Leskov, foram subtraídas à escravidão das pulsões sexuais
no pleno florescimento de suas forças. Mas nem por isso personificam um
ideal propriamente ascético; ao contrário, a temperança destes justos
tem um caráter tão pouco privado que torna-se, na ordem das paixões, o
polo oposto ao do furor sexual que o contador pintou em \emph{Lady
Macbeth do Distrito de Mtzensk}.\footnote{Cf. Nikolai Leskov, \emph{Lady
  Macbeth do Distrito de Mtzensk}, ed. cit. {[}N. da T.{]}} Se a
envergadura entre Pavlin e esta esposa de negociante permite medir a
extensão do mundo das criaturas, então Leskov também sondou, na
hierarquia de suas criaturas, a sua profundeza.

XVIII

A hierarquia do mundo das criaturas, que atinge com o justo o seu ponto
mais alto, desce em múltiplos graus até as profundezas do inanimado. A
propósito disto, uma circunstância particular deve ser levada em conta.
Esta totalidade do mundo das criaturas não se expressa tanto na voz
humana quanto naquela que pode ser nomeada de acordo com o título de um
de seus contos mais significativos: ``A voz da natureza''. Este conto é
sobre o pequeno funcionário Filip Filipovitch que se esforça por todos
os meios para hospedar em sua casa um marechal de campo de passagem em
sua cidadezinha. Ele consegue o que queria. O hóspede, que se espantara
a princípio com o convite insistente do funcionário, com o tempo pensa
reconhecer nele alguém que já havia encontrado antes. Mas quem? Não
consegue lembrar-se. E o estranho é que o dono da casa recusa-se a
facilitar o reconhecimento. Em vez disso, consola a alta personalidade
um dia depois do outro, dizendo que ``a voz da natureza'' acabará por
lhe falar claramente. Isso dura até que um dia, pouco antes de
prosseguir sua viagem, quando, atendendo ao pedido público do anfitrião
durante um jantar, o hóspede deve dar sua permissão para que este possa
fazer soar ``a voz da natureza''. Então a dona da casa se afasta. Ela
``voltou com uma grande trompa de cobre, reluzentemente polida, e
entregou-a ao marido. Ele pegou a trompa, encostou o bocal aos lábios e
transformou-se inteiro num minuto. Foi só ele inflar as bochechas e sair
um ribombo vibrante para o marechal de campo gritar: -- Estou
reconhecendo, irmão, agora estou reconhecendo, você é aquele músico do
regimento de caçadores, que, por sua honestidade, enviei para vigiar um
intendente trapaceiro. -- Exatamente, meu príncipe -- respondeu o
anfitrião. -- Não queria eu lembrar-lhe disso, então a própria natureza
o fez.''\footnote{Nikolai Leskov, ``A voz da natureza'', in \emph{A
  Fraude e outras histórias}, ed. cit., p. 100. {[}N. da T.{]}} A
maneira pela qual o sentido profundo desta história se esconde atrás de
sua aparente tolice nos dá uma ideia do humor magnífico de Leskov.

Este humor confirma-se na mesma história de um modo ainda mais discreto.
Ouvimos que o pequeno funcionário foi enviado ``por sua honestidade''
para ``vigiar um intendente trapaceiro''. Isso é dito no final, na cena
do reconhecimento. Logo no início da história, porém, ouvimos o seguinte
sobre o anfitrião: ``Os moradores locais, todos eles, conheciam esse
homem e sabiam que o seu título não era alto, já que não era funcionário
civil nem militar, mas só encarregado do pequeno depósito local da
intendência ou do comissariado e, junto com as ratazanas, roía torradas
do erário e lambia botas, tendo conseguido, com a roedeira e a lambição,
uma casa bonitinha, de madeira (...)''.\footnote{Idem, ibidem, p. 90.
  {[}N. da T.{]}} Como se vê, esta história mostra a tradicional
simpatia do contador pelos trapaceiros e malandros. Toda literatura
burlesca é testemunha desta simpatia que não se desmente nem na mais
alta arte: Hebel é acompanhado do modo mais fiel, entre todas as suas
personagens, por \emph{Zundelfrieden}, \emph{Zundelheiner} e o ruivo
\emph{Dieter}.\footnote{Alusão a uma série de contos de Hebel nos quais
  os dois irmãos Zundel -- Frieden e Heiner -- se associam ao ruivo
  Dieter, seu antigo camarada de escola, para cometerem diversos tipos
  de furtos, malandragens e farsas. Embora não passem de pequenos
  bandidos, os três comparsas são apresentados com simpatia pelo autor.
  {[}N. da T.{]}} E, claro, também para Hebel o justo desempenha o papel
principal no \emph{theatrum mundi}. Mas como ninguém está propriamente à
altura deste papel, ele passa de um para outro. Ora é o vagabundo, ora o
judeu avarento, ora é o imbecil que surge para desempenhá-lo. Trata-se
sempre apenas de um estágio provisório, que varia de um caso a outro,
uma improvisação moral. Hebel é um casuísta. Não se solidariza de
nenhuma maneira com algum princípio, mas também não recusa nenhum deles,
pois todos podem um dia servir de instrumento ao justo. Podemos comparar
essa atitude com a de Leskov. ``Reconheço'', diz ele em ``A propósito de
\emph{A Sonata a Kreutzer}'', ``que, nas minhas reflexões, entra muito
mais senso prático do que filosofia abstrata e moral elevada; contudo
sou inclinado a pensar como penso.''\footnote{Nikolai Leskov, ``A
  propósito de \emph{A Sonata a Kreutzer}'', in \emph{A Fraude e outras
  histórias}, ed. cit., pp. 176-177. {[}N. da T.{]}} É verdade, por
outro lado, que as grandes catástrofes morais que ocorrem em seu mundo
estão para os incidentes morais de Hebel como o grande curso silencioso
do rio Volga para a tagarelice de um pequeno riacho que faz girar a roda
do moinho. Entre os contos históricos de Leskov, há vários em que as
paixões agem de modo tão avassalador quanto a cólera de Aquiles ou o
ódio de Hagen.\footnote{Personagem da mitologia escandinava e, depois,
  da mitologia germânica, responsável pelo assassinato do herói
  Siegfried. {[}N. da T.{]}} É surpreendente como, neste autor, o mundo
às vezes pode tornar-se sombrio e com qual majestade o mal é capaz de
empunhar seu cetro. Leskov claramente conheceu estados de espírito
próximos de uma ética antinomista -- e este deve ser um dos poucos
pontos de contato com Dostoiévski. As naturezas elementares de suas
\emph{Histórias dos velhos tempos}\footnote{``Erzählungen aus der alten
  Zeit'', conforme os termos de Benjamin no original alemão. Na edição
  em nove volumes das \emph{Gesammelte Werke} (Obras Reunidas) de
  Nikolai Leskov, que a editora C.H. Beck fez publicar entre 1924 e
  1927, utilizada por Benjamin e por ele referida em nota inicial ao
  presente ensaio, o título que corresponde ao supracitado é o do quarto
  volume, \emph{Geschichten aus alter Zeit}. Daí a nossa opção final por
  \emph{Histórias} (\emph{Geschichten}) \emph{dos velhos tempos}, em
  lugar de \emph{Contos} (\emph{Erzählungen}) \emph{dos velhos tempos}.
  {[}N. dos E.{]}} vão até o fim em sua paixão sem escrúpulos. E este
fim era justamente o ponto no qual para os místicos a perversão
consumada transforma-se bruscamente em seu contrário, tornando-se
santidade.

XIX

Quanto mais baixo Leskov desce na escala das criaturas, mais abertamente
sua concepção de mundo se aproxima da mística. Aliás, como mostraremos,
há boas razões para se dizer que tal característica pertence à própria
natureza do contador de histórias. Certamente, são raros os que se
aventuraram nas profundezas da natureza inanimada e, na literatura
narrativa moderna, não há muitos casos em que a voz do contador anônimo,
anterior a toda escrita, ressoe tão claramente como na história de
Leskov ``A Alexandrita''. Trata-se de uma pedra, o piropo. A pedra
corresponde à camada mais baixa da criação. Mas para o contador, ela se
liga diretamente à mais alta. Ele tem o dom de vislumbrar, nesta pedra
semipreciosa, o piropo, uma profecia da natureza petrificada, inanimada,
sobre o mundo histórico, no qual ele mesmo vive. Este mundo é o de
Alexandre II. O contador -- ou melhor, o homem a quem ele atribui seu
próprio saber -- é um lapidador, de nome Wenzel, que atingiu em seu
ofício a mais alta perfeição imaginável. Podemos compará-lo aos ourives
de Tula e dizer, de acordo com Leskov, que o artesão perfeito tem acesso
às câmaras mais internas do reino das criaturas. É uma encarnação do
homem piedoso. Vejamos o que é dito sobre ele: ``de repente, agarrou-me
pelo anel com a alexandrita, que agora, sob a luz, estava
vermelha\footnote{Na tradução do russo para o alemão de que fez uso
  Benjamin, há uma variação relativamente à tradução do russo para o
  português, aqui adotada, que importa destacar. De modo que onde nesta
  figura ``(\ldots{}) alexandrita, que agora, sob a luz, estava
  vermelha'', naquela se lê ``(\ldots{}) alexandrita, que, como se sabe,
  com iluminação artificial irradia um brilho vermelho''. {[}N. dos
  E.{]}}, e pôs-se a gritar: -- (...) Vejam só, eis aqui aquela pedra
russa profética (...)! Siberiana astuta! O tempo todo estava verde como
a esperança, mas agora, com a aproximação do anoitecer, banhou-se de
sangue. Desde priscas eras ela é assim, mas escondeu-se o tempo todo, no
seio da terra, e permitiu que a encontrassem apenas no dia da maioridade
do tsar Aleksandr, quando um grande feiticeiro, mago, bruxo, foi à
Sibéria procurar por ela. -- O senhor está falando asneiras --
interrompi. -- Essa pedra não foi encontrada por um feiticeiro, foi por
um cientista: Nordenskiöld! -- Feiticeiro! Estou dizendo: feiticeiro! --
pôs-se a gritar Wenzel bem alto. -- Veja só que pedra! Nela a manhã é
verde e a noite sangrenta... É o destino, é o destino do nobre tsar
Aleksandr! E o velho Wenzel voltou-se para a parede, apoiou a cabeça no
braço e... pôs-se a chorar.''\footnote{Nikolai Leskov, ``Alexandrita'',
  in \emph{A Fraude e outras histórias}, ed. cit., pp. 164-165. {[}N. da
  T.{]}}

Não podemos apreender mais diretamente o sentido deste importante conto
do que com o auxílio de algumas palavras escritas por Paul Valéry sobre
outro assunto, bem diferente.

``A observação do artista -- disse ele em suas considerações sobre um
artista -- pode atingir uma profundidade quase mística. Os objetos
iluminados por ela perdem seus nomes: sombras e claridades formam
sistemas e problemas bem particulares, que não dizem respeito a nenhuma
ciência, que não se relacionam com nenhuma prática, mas que recebem toda
sua existência e seu valor de certos acordos singulares entre a alma, o
olho e a mão de alguém, nascido para, dentro de si, apreendê-los e
evocá-los''.\footnote{Paul Valéry, ``Autour de Corot'', \emph{Pièces sur
  l'art}, in~\emph{Oeuvres}, tomo II. Edição de Jean Hytier. Paris:
  Gallimard, 1960, p. 1318. (col. Bibliothèque de la Pléiade) {[}N. da
  T.{]}}

Estas palavras estabelecem uma estreita relação entre alma, olho e
mão.\footnote{``Relação de colaboração que determina todo trabalho
  artesanal'', conforme acréscimo efetuado na versão francesa. {[}N. da
  T.{]}} Interagindo, determinam uma prática com a qual não estamos mais
acostumados. O papel da mão na produção tornou-se mais restrito e o
lugar que ela ocupava no contar histórias foi deixado de lado. (Pois
contar histórias não é de modo algum, do ponto de vista sensível, apenas
um trabalho da voz. No autêntico contar, a mão atua decisivamente,
apoiando o que é dito de diversos modos, com seus gestos aprendidos por
experiência no trabalho.) A antiga coordenação de alma, olho e mão, que
aparece nas palavras de Valéry, é artesanal, e é ela que encontramos
onde quer que a arte de contar esteja em seu domínio. Sim, podemos ir
além e nos perguntar se a relação do contador com sua matéria, a vida
humana\footnote{``a experiência humana'', segundo variação da versão
  francesa. {[}N. da T.{]}}, não seria ela própria uma relação
artesanal? Se sua tarefa não consiste em elaborar, de um modo sólido,
útil e único, a matéria crua da experiência, seja da sua própria ou da
alheia? Trata-se aqui de uma reelaboração da qual o provérbio nos dê
talvez mais facilmente uma ideia, se o concebemos como o ideograma de
uma narração. Provérbios são, por assim dizer, ruínas que ocupam o lugar
de antigas histórias nas quais uma moral cresce em torno de um gesto,
como a hera numa muralha.

O contador de histórias pode assim ser considerado como um mestre ou
como um sábio. Ele sabe aconselhar -- não em alguns casos, como o
provérbio, mas em muitos\footnote{Na variante da versão francesa: ``em
  todos os casos''. {[}N. da T.{]}}, como o sábio. Pois lhe é dado
recorrer a toda uma vida. (Uma vida que não inclui apenas sua própria
experiência, mas também uma boa parte da experiência alheia. O contador
assimila ao que tem de mais intimamente seu aquilo que aprendeu por
ouvir dizer.) Seu talento é poder contar sua vida; sua dignidade é poder
contá-la \emph{por inteiro}. O contador é o homem que poderia deixar a
mecha de sua vida consumir-se completamente na doce chama de sua
narração. Daí vem essa atmosfera\footnote{``esse halo'', na variante da
  versão francesa. {[}N. da T.{]}} incomparável que, em Leskov como em
Hauff, em Poe como em Stevenson, cerca o contador. O contador de
histórias é a figura na qual o justo se encontra consigo mesmo.

\textbf{CONTOS}

O LENÇO\footnote{``Das Taschentuch'', in Walter Benjamin,
  \emph{Gesammelte Schriften} {[}a partir daqui, GS{]}, vol. IV-2.
  Edição de Rolf Tiedemann e Hermann Schweppenhäuser. Frankfurt a. M.:
  Suhrkamp, 1991, pp. 741-745. Tradução de Marcelo Backes. {[}N. dos
  E.{]}}\footnote{Escrito provavelmente entre abril e junho de 1932,
  durante uma estadia em Ibiza, o conto foi publicado na
  \emph{Frankfurter Zeitung} em novembro de 1933. O ensaio sobre Leskov,
  de 1936, retoma vários temas e formulações contidas neste texto
  ficcional. {[}N. da O.{]}}

Por que a arte de contar histórias está chegando ao fim? -- eu já me
fizera esta pergunta muitas vezes enquanto me entediava, sentado com
outros convidados durante uma noite inteira em torno de uma mesa.
Naquela tarde, porém, quando estava em pé no convés de passeio do
``Bellver'', ao lado da cabine do timoneiro, e buscava com meu excelente
binóculo todos os aspectos da imagem inigualável que Barcelona oferecia
do alto do navio, acreditei ter encontrado a resposta para ela. O sol
descia sobre a cidade e parecia derretê-la. Tudo que era vida havia se
recolhido nas gradações cinza claro entre a copa das árvores, o cimento
das construções e a rocha das montanhas distantes. O ``Bellver'' é um
navio a motor, belo e espaçoso, ao qual se gostaria de creditar um
encargo maior do que o de abastecer o pequeno trânsito insular para as
Baleares. E, de fato, sua imagem pareceu se encolher quando, no dia
seguinte, o vi esperando pela viagem de volta no molhe de Ibiza, pois eu
imaginara que de lá ele tomaria seu curso até as Ilhas Canárias. Assim,
eu estava parado, e voltava meus pensamentos para o capitão O..., do
qual eu me despedira há algumas horas, o primeiro e talvez último
contador de histórias que encontrei em minha vida. Pois, conforme disse,
a arte de contar está chegando ao fim. E, quando me lembrava das várias
horas em que o capitão O... havia passeado de um lado a outro no convés
traseiro, olhando de quando em vez para a distância, ocioso, então eu já
sabia: quem jamais se entedia não é capaz de contar. O tédio, porém, não
tem mais espaço em nosso agir. As atividades, que se uniram a ele
secreta e intimamente, estão se extinguindo. E também por isso o dom de
contar histórias está chegando ao fim: não se tece e não se fia mais,
não se constrói nem se aplaina objetos enquanto se ouve histórias. Em
resumo: histórias precisam de trabalho, ordem e disciplina para vingar.

Contar histórias, na verdade, não é apenas uma arte, é muito mais uma
dignidade, se é que não é, como no oriente, um ofício. Contar termina em
uma sabedoria, assim como por outro lado a sabedoria muitas vezes se
revela numa narrativa. O contador de histórias é, portanto, alguém que
sempre sabe dar conselhos. E, para recebê-los, é preciso que também se
conte algo a ele. Mas nós sabemos apenas suspirar a respeito de nossas
preocupações, nos lamentar, e não contar. E, lembrando um terceiro
aspecto, pensei no cachimbo do capitão: o cachimbo que ele batia quando
começava a contar algo e batia quando se calava, mas, entre um e outro,
quando se fazia necessário, deixava se apagar calmamente. O cachimbo
tinha uma piteira de âmbar, mas sua cabeça era de chifre e adornada com
pesados trabalhos em prata. Ele o herdara do avô, e acredito que esse
cachimbo era o talismã do contador de histórias. Pois também por causa
disso não há mais nada interessante a ouvir, porque as coisas não duram
mais do modo correto. Quem um dia usou um cinto de couro por tanto tempo
até ele se desintegrar em pedaços, sempre haverá de encontrar algo
interessante: em algum momento, no decorrer do tempo, uma história se
prendeu a ele. O cachimbo do capitão já devia conhecer muitas delas.

Assim eu sonhava quando, lá embaixo, bem no fundo, nas docas, apareceu
um homem atarracado com o rosto mais maciço, que já esteve enfiado
debaixo de um quepe de capitão: o capitão O..., em cujo navio de carga
eu havia chegado pela manhã. Quem está acostumado a partidas solitárias
de cidades estranhas, sabe ou então é capaz de avaliar o que significa o
aparecimento de um rosto conhecido, ainda que não seja um dos mais
familiares, em tais momentos, quando a partida que em breve ocorrerá
tira do caminho todas as ponderações de uma conversa mais longa, mas ao
mesmo tempo também coloca à sua disposição um chapéu qualquer, uma mão,
um lenço, nos quais o olhar desabrigado pode encontrar seu ninho antes
de se perder na superfície imensa do mar. Eis, pois, que ali estava o
capitão, como se eu o tivesse chamado com meus pensamentos. Ele havia
saído de casa aos quinze anos, cruzado por aí durante três anos em um
navio-escola o Pacífico e o Atlântico, e mais tarde chegado a um vapor
americano do Lloyd, que ele no entanto -- por que motivo era
desconhecido -- abandonara logo em seguida. Mais do que isso eu não
consegui descobrir. Sobre sua vida parecia pairar uma sombra, e ele
também não falava com prazer a respeito. E, com tudo isso, no fundo lhe
parecia faltar o que é a coisa mais maravilhosa no contador: o fato de
poder contar sua vida, deixando que essa mecha se consuma nas chamas
suaves da narrativa. Como quer que seja, sua vida parecia ser pobre se
comparada com a do navio que ele sabia fazer viver em todas as suas
ripas e vigas. E assim estava ele, ali à minha frente, quando pela manhã
deixei o navio. Eu conhecia tão bem tanto seu ano de construção e suas
tarifas, seu espaço de carga e sua tonelagem, quanto os salários dos
grumetes e as preocupações dos oficiais. Sim, que tempos eram aqueles em
que o transporte de cargas ainda era feito por veleiros, onde o capitão,
ele mesmo, contratava os fretes nos portos! Nessa época ainda imperava a
velha sentença irônica: ``Abandonar as viagens marítimas e entrar em um
navio a vapor.'' Hoje, porém... e então se seguiam, na maior parte das
vezes, algumas frases a partir das quais se podia deduzir como também
aqui as necessidades econômicas haviam mudado as coisas.

Em tais oportunidades o capitão O..., de quando em vez, dizia também uma
palavra sobre a política. Mas jamais o vi com um jornal. Tornou-se
inesquecível para mim sua resposta quando, certo dia, eu conduzi a
conversa ao assunto. ``Dos jornais'', disse ele, ``não se pode saber
absolutamente nada; é que as pessoas querem explicar tudo para a
gente.'' E, de fato: já não é metade da arte de relatar, manter o relato
livre de explicações? E os antigos por acaso não são exemplares nisso,
eles que por assim dizer apresentavam o acontecido a seco, deixando
antes que escorresse dele tudo que era fundamentação psicológica e
opinião? Suas próprias histórias, de qualquer modo, deve-se admitir, se
mantinham livres de explicações supérfluas, sem que, conforme me
parecia, perdessem algo por causa disso. Até havia algumas mais
estranhas entre elas, mas nenhuma que confirmasse aquela peculiaridade
tanto quanto a seguinte, sobre a qual ainda recairia, nesta tarde junto
ao molhe de Barcelona, o mais surpreendente reflexo.

``Tudo aconteceu'', assim me contou o capitão na altura de Cádiz, ``há
muitos anos, em uma das minhas primeiras viagens aos Estados Unidos,
quando eu ainda era um oficial bem jovem. Estávamos sete dias em
alto-mar e, ao meio-dia do dia seguinte, chegaríamos a Bremerhaven. À
hora costumeira, eu fazia minha ronda no convés de passeio, trocava
algumas palavras aqui e ali com os passageiros. Foi quando fiquei de
repente estupefato; a sexta espreguiçadeira da fila estava vazia. Um
sentimento de angústia se manifestou dentro de mim, mesmo que eu tenha,
acho, passado por ela ainda mais angustiado nos dias anteriores, quando
me voltava com uma saudação muda para a jovem mulher que costumava
contemplar o vazio a sua frente sobre ela, imóvel, com as mãos cruzadas
sob a nuca. A jovem mulher era muito bonita, mas tão chamativa quanto
sua beleza era também sua discrição. Esta ia tão longe que só raras
vezes tinha-se oportunidade de ouvir sua voz -- a voz mais maravilhosa
da qual consigo me lembrar --, frágil e rouca, sombria e metálica. Certa
vez, quando lhe entreguei um lenço que havia caído ao chão -- ainda hoje
sei como seu emblema me impressionou: um brasão tripartido com três
estrelas em cada um dos campos --, ouvi-a dizer seu `obrigada' com uma
expressão que era como se eu tivesse salvo sua vida. Desta vez, pois,
terminei minha ronda e já estava a ponto de procurar pelo médico do
navio, a fim de saber se a dama por acaso não estava doente, quando de
repente um redemoinho de franjas brancas me envolveu. Levantei os olhos
e vi como aquela que eu dava por perdida, apoiada sobre o parapeito do
convés solar, seguia com os olhos, ausente, um bando de bilhetes e
papéis com os quais o vento e as ondas brincavam. No meio-dia seguinte
-- eu assumira meu posto no convés e inspecionava as manobras de aportar
--, meu olhar cruzou mais uma vez com o da estranha de passagem. O navio
estava a ponto de ancorar e, vagarosamente, a quilha se aproximava do
cais no qual havíamos amarrado a popa. Reconhecia-se com nitidez a
feição dos que esperavam; febril, a estranha os media. O baixar das
âncoras havia ocupado minha atenção quando, de uma hora para outra, se
ergueu um grito em várias vozes. Eu me virei e no mesmo instante vi que
a estranha havia desaparecido; no movimento da multidão se podia
perceber que ela havia se precipitado abaixo. Qualquer tentativa de
salvação era inútil. Mesmo que se conseguisse desligar as máquinas
naquele mesmo instante -- o casco do navio não estava mais do que três
metros distantes do cais, e seu movimento era impossível de ser detido.
Quem ficasse no meio, estava perdido. Então aconteceu o improvável:
houve alguém que fez a tentativa formidável. Era possível vê-lo, cada um
dos músculos distendidos, as sobrancelhas virando uma só, como se
quisesse fazer mira, saltar da grade e, enquanto o vapor foi se
aproximando em todo seu comprimento a estibordo -- para horror de todos
os que acompanhavam o espetáculo --, a bombordo apareceu, sem que a
princípio se o notasse, pois ninguém olhava para aquele lado, o homem,
salvo, e em seu braço a moça, na superfície da água. Ele de fato fizera
mira -- exatamente, aproveitando todo seu peso e se precipitando sobre a
outra, arrancando-a consigo para as profundezas, para em seguida
reaparecer por baixo da quilha do navio --, trazendo-a para a
superfície. `Quando a segurei assim', disse ele mais tarde a mim, `ela
sussurrou `obrigada' como se eu tivesse lhe estendido um lenço que caíra
ao chão'.''

Eu ainda tinha no ouvido a voz com a qual o contador havia pronunciado
estas últimas palavras. Se eu quisesse lhe dar a mão ainda uma vez, não
havia tempo a perder. Quando eu já fazia menção de descer correndo as
escadas que levavam até ele, percebi como os armazéns, barracas e gruas
recuavam lentamente. Estávamos zarpando. O binóculo diante dos olhos,
deixei Barcelona passar à minha frente pela última vez. Depois o baixei
devagar até o cais. Ali estava o capitão, em meio à multidão; ele devia
ter acabado de me notar. Saudando, levantou a mão, eu acenei com a
minha. Quando pus o binóculo outra vez diante dos olhos, ele havia
desdobrado um lenço e o abanava. Nítido, vislumbrei o emblema em um dos
cantos: um brasão tripartido com três estrelas em cada um dos campos.

A VIAGEM DO MASCOTE\footnote{``Die Fahrt der Mascotte'', in GS IV-2, pp.
  738-740. Tradução de Marcelo Backes. {[}N. dos E.{]}}\footnote{Finalizado
  e corrigido juntamente com ``O Lenço'' e ``O Anoitecer da viagem'',
  este conto data provavelmente de 1932, quando Benjamin realizou uma
  viagem marítima às Ilhas Baleares, na Espanha. {[}N. da O.{]}}

Esta é uma dessas histórias que se costuma ouvir no mar, para a qual o
casco do navio é a caixa de ressonância perfeita e o socar das máquinas
o melhor acompanhamento, e diante das quais não se deve insistir para
saber de onde elas vêm.

Conforme contou meu amigo, o sinaleiro de bordo, tudo se deu depois do
final da guerra, quando alguns donos de navios passaram a querer trazer
de volta à pátria veleiros, navios de salitre, que haviam sido
surpreendidos por uma catástrofe no Chile. A questão jurídica era
simples; os navios continuavam sendo propriedade alemã, e agora se
tratava apenas de preparar a tripulação e os homens necessários para
assumir o controle sobre eles em Valparaíso ou Antofagasta. Havia
marinheiros suficientes esperando para ser engajados nos portos. Mas
havia um detalhe problemático na questão toda. Pois como se poderia
transportar as equipes até o lugar? Uma coisa era certa: eles só
poderiam subir à bordo como passageiros e entrariam em serviço apenas no
lugar determinado. Por outro lado, também estava claro que se tratava de
pessoas das quais dificilmente se lograria conseguir alguma coisa com os
poderes que um capitão tem sobre seus passageiros, e com certeza não em
tempos nos quais o clima da Revolta de Kiel ainda se fazia sentir nos
ossos dos marinheiros.

Ninguém sabia disso melhor do que os hamburgueses e, portanto, o mesmo
acontecia com a equipe de comando do navio de quatro mastros
``Mascote'', que consistia em uma elite de oficiais decididos e com
experiência marítima. Eles viam naquela viagem uma aventura na qual
poderiam colocar sua pele em risco. E, uma vez que o homem inteligente
age de modo precavido, não confiaram apenas na própria coragem. Muito
antes, avaliaram cada um dos homens que seriam engajados com a maior
exatidão. Porém, se entre os escolhidos mesmo assim havia um cara alto
cujos documentos não estavam completamente regulares e cuja constituição
física também deixava um pouco a desejar, seria precipitado creditar
isso ao desleixo dos comandantes. Por que era assim, logo ficará claro.

Ainda não haviam se afastado cinquenta milhas de Cuxhaven quando se
fizeram notar coisas que prenunciavam perspectivas bem ruins para a
viagem. No convés e nas cabines, e até mesmo nas escadarias, se
instalavam desde bem cedo até bem tarde as mais diferentes reuniões e
círculos, e, antes de Helgoland, já havia três clubes de jogo em pleno
funcionamento, um ringue de boxe permanente e um palco para amantes,
cuja visita não era recomendável a pessoas escrupulosas. Na sala dos
oficiais, cujas paredes da noite para o dia foram ornadas com desenhos
drásticos, os homens dançavam \emph{jimmy} uns com os outros à tarde, e
no salão de carga havia se estabelecido uma bolsa de valores de bordo,
cujos membros negociavam com notas de dólares, binóculos, nus
fotográficos, facas e passaportes ao clarão de lâmpadas de bolso. Em
resumo, o navio era uma Magic City navegante, e até seria possível
acreditar que todas as maravilhas da vida no porto poderiam ser
arrancadas da terra -- ou, antes, das vigas do navio --, até mesmo sem
mulheres.

O capitão, um daqueles homens do mar que unem o pouco saber escolar a
muita esperteza de vida, permaneceu senhor de seus nervos mesmo sob
circunstâncias tão desconfortáveis, e não os perdia nunca, nem mesmo
quando certa tarde -- deve ter sido na altura de Dover --, Frieda, uma
moça bem crescidinha, mas de muito má fama, do bairro de Sankt Pauli, em
Hamburgo, apareceu no deque traseiro, com um cigarro na boca. Sem dúvida
alguma, havia pessoas a bordo que sabiam onde ela estivera enfiada até
então, e as mesmas também sabiam das medidas que deveriam ser tomadas
caso fossem feitos preparativos na parte de cima para afastar os
passageiros em excesso.

O movimento à noite se tornou ainda mais chamativo a partir de então.
Mas não se estaria no ano de 1919 se, além de todos os outros
divertimentos, também a política não se juntasse a eles. Fizeram-se
ouvir vozes que queriam ver naquela expedição o princípio de uma nova
vida em um novo mundo; outros viam se aproximar o instante, há muito
almejado, em que seriam acertadas as contas com os dominantes.
Inequivocamente, soprava um vento mais agudo. E em breve também se
descobriria de onde ele vinha: havia ali um certo Schwinning, um homem
alto de postura frouxa, que trazia o cabelo ruivo dividido por uma
risca, e do qual apenas se sabia que havia trabalhado como acompanhante
de passageiros em várias linhas marítimas, além de saber tudo, até mesmo
sobre os segredos profissionais de contrabandistas finlandeses de
combustível.

No princípio, ele se mantivera afastado, mas agora podia ser encontrado
a cada passo, onde quer que fosse. Quem ouvia o que ele dizia era
obrigado a admitir que se tratava de um agitador de peso. E quem seria
capaz de não ouvi-lo quando ele envolvia um ou outro em uma conversa
briguenta em voz alta no ``bar'', na qual sua voz sobrepujava o disco
tocando, ou quando distribuía informações precisas junto ao ``ringue'',
mesmo sem ser perguntado, sobre as opções partidárias dos lutadores. E
assim ele trabalhava, enquanto a massa se entregava a suas distrações,
incansável em sua politização, até que por fim uma assembleia noturna
recompensou seus esforços ao nomeá-lo diretor de um conselho de
marinheiros.

Com a entrada no Canal do Panamá as eleições adquiriram ímpeto. E não
havia pouca coisa a votar: uma comissão de alimento, uma coluna de
controle, um secretariado de bordo, um tribunal político -- em resumo,
um aparato grandioso foi posto em pé, sem que houvesse o menor
desentendimento com o comando do navio. Tanto maior era a frequência, no
entanto, com que aconteciam discórdias no interior da direção
revolucionária, e elas eram tanto mais aborrecidas na medida em que, se
vistas as coisas com mais exatidão, no fundo todos pertenciam à referida
direção. Quem não tinha posto, por certo poderia esperá-lo da seguinte
assembleia da comissão, e assim não passava um dia sem que houvesse
dificuldades a esclarecer, votações a examinar, resoluções a tomar.
Quando enfim o comitê de ação havia definido em todos os seus detalhes o
plano para um ataque surpresa -- em duas noites, às onze horas, o
comando seria dominado e em seguida tomado o curso oeste em direção a
Galápagos --, o ``Mascote'' já tinha, sem o saber, Callao às costas.
Mais tarde, os balizamentos provaram estar falsificados. Mais tarde
ainda, mais exatamente na manhã seguinte, quarenta e oito horas antes do
motim planejado e cuidadosamente preparado, o navio de quatro mastros
adentrou o molhe de Antofagasta como se nada houvesse acontecido.

Foi isso que disse meu amigo. A segunda guarda chegou ao fim. Nós
entramos na casa de mapas e instrumentos onde o cacau já esperava por
nós nas fundas xícaras de pedra... Eu fiquei em silêncio e me ocupava em
fazer um verso acerca do que ouvira. Mas o sinaleiro, no exato momento
em que daria seu primeiro gole, estacou de repente e olhou para mim por
sobre a borda de sua xícara. ``Deixa estar!'', disse ele. ``Na época nós
também não estávamos em cena. Mas quando, três meses mais tarde, no
prédio da administração, em Hamburgo, dei de cara com Schwinning que --
com um grosso Virginia entre os lábios -- acabava de sair do escritório
do chefe... então eu compreendi com exatidão o que havia sido a viagem
do `Mascote'.''

O ANOITECER DA VIAGEM\footnote{``Der Reiseabend'', in GS IV-2, pp.
  745-748. Tradução de Marcelo Backes. {[}N. dos E.{]}}\footnote{Um
  primeiro esboço deste conto encontra-se no conjunto de notas
  \emph{Espanha 1932} sob o seguinte titulo: ``Sobre a honestidade dos
  nativos e seu contrário. Duas histórias''. A outra história planejada
  não chegou a ser escrita ou se perdeu. O presente texto foi redigido,
  provavelmente, no mesmo ano. {[}N. da O.{]}}\footnote{Quanto ao
  conjunto referido de notas (cf. GS VI, pp. 446-464), Benjamin o
  escreveu em sua viagem de abril a junho de 1932, quando foi de
  Hamburgo a Barcelona em um navio cargueiro, percorrendo um trajeto de
  11 dias e ficando amigo do capitão. Depois, em um navio a vapor do
  correio, seguiu para Ibiza, onde ficou por três meses. {[}N. dos E.{]}}

A economia na ilha é arcaica. Eles não colhem os cereais com máquinas,
mas com foices. Em algumas regiões, as mulheres os recolhem à mão e aí
não resta um colmo sequer. Depois de colhido, eles são levados à eira,
onde um cavalo, arreado e tangido pelo camponês, que fica parado no meio
do lugar, debulha os grãos das vagens, pisando com seus cascos. Há
sessenta anos ainda não se conhecia pão ali; o principal alimento era o
milho. E ainda hoje se irriga os campos à maneira antiga, com rodas
d'água que são movidas por mulas. Vacas, há apenas um punhado delas na
ilha. Alguns dizem que é devido ao pasto; no entanto, dom Rosello,
deputado e comerciante de vinhos, que representa o progresso ali, diz:
por causa do atraso dos moradores. Não faz nem tanto tempo assim que
alguém, quando chegava a Ibiza, podia ficar sabendo pelo primeiro que
encontrava em seu caminho: agora temos tantos estrangeiros na ilha. E é
dessa época que vem a seguinte história, que se contou à mesa de dom
Rosello:

Um estrangeiro, que depois de ficar vários meses na ilha conseguiu
angariar amizades e confiança, vê o último dia de sua permanência
chegando. Acontece que é um dia abrasivo de tão quente e, quando ele
termina seus preparativos de viagem, decide se livrar o mais rápido
possível da preocupação com suas coisas, a fim de desfrutar ainda duas
horas do entardecer à sombra fresca, no terraço de um comerciante de
vinhos ibicense. No navio lhe prometem cuidar de sua bagagem, incluindo
seu casaco, e, visivelmente aliviado, o estrangeiro vai até o dono da
\emph{tienda}, para o qual ele é assaz bem-vindo, mesmo estando em
mangas de camisa. Sem esforço, o dono da \emph{tienda} chega logo com as
primeiras \emph{copitas} de um Alicante corriqueiro. Mas quanto mais o
tempo avança para ele em meio à bebida, tanto mais difícil lhe parece se
tornar a despedida, sobretudo uma despedida tão prosaica e fortuita.
Ocorrem-lhe perguntas sobre a história dos belos galgos, descendentes
dos cães do faraó, que perambulam sem dono pela ilha, sobre os velhos
costumes de seduzir e raptar mulheres, a respeito dos quais jamais
conseguiu saber algo mais detalhado, sobre a origem daqueles nomes
estranhos que os pescadores usam para designar as montanhas, e que são
bem diferentes dos nomes que as mesmas montanhas têm na língua dos
camponeses. Na hora certa, ele se lembra de, às vezes, ter ouvido o
proprietário daquela pequena \emph{tienda} falar como se fosse uma
autoridade em todas as questões que dizem respeito à crônica do lugar.
No último momento, ele ainda gostaria de garantir informações sobre isso
e aquilo, e de superar assim a solidão da noite que entra. Ele pede uma
garrafa do melhor vinho e, enquanto o dono saca a rolha diante de seus
olhos, uma conversa já se desenrolou entre os dois. Eis que o
estrangeiro nas últimas semanas conheceu a hospitalidade fanática dos
moradores da ilha, de modo suficiente para saber que é preciso estipular
de antemão e com boa antecipação a hora de lhes oferecer alguma coisa.
Assim, a primeira coisa que ele faz é convidar o dono a beber com ele, e
nesse ponto ele se mantém firme também à segunda e à terceira garrafa,
tanto mais que enquanto isso consegue, com boas maneiras, anotar em seu
caderno, na forma de palavras-chave, uma ou outra dessas informações. E
enquanto vai folheando o caderno ao clarão da vela, acaba dando de cara
-- ele também é um pouco desenhista -- com esboços feitos nos primeiros
dias, logo após sua chegada. Ali está o cego com a coxa crua de uma
cabra ou de um carneiro, que anda sempre pelas ruas, guiado por um
garoto; em outra folha, os perfis vivazes dos muros, que em nenhum
momento tocam qualquer medida de referência mais exata; e, em seguida, a
escadaria de azulejos com as cifras misteriosas, com a qual ele deu de
cara logo no princípio, ao procurar por moradia. O dono da taverna
olhava por sobre o seu ombro com interesse. É claro que ele conhece a
história da coxa de carneiro: ele mesmo se engajou na municipalidade, a
fim de conceder ao cego a permissão para fazer uma loteria precária e
distribuir números cujo único prêmio é aquela coxa. E os azulejos
misteriosamente cifrados, ele mesmo ainda vira em uma rua na qual eles
anunciavam os números das casas. Mais do que isso: ele também sabe o que
significam as cruzes brancas aos pés de algumas casas, que
proporcionaram ao estrangeiro um quebra-cabeças e tanto. Elas são uma
espécie de altar de descanso. Por toda parte onde aparecem, representam
um dos pontos nos quais as procissões estacam de repente ao seguirem
pelas ruas. E agora o estrangeiro se lembra vagamente de ter visto algo
semelhante em aldeias da Vestfália. Entrementes esfriou; o dono da
taverna faz questão de dar ao hóspede um de seus próprios casacos, e a
última garrafa é aberta. Mas voltemos às anotações do estrangeiro; em
que passagem há, nas novelas italianas de Stendhal, um tema comparável a
esse tema típico de Ibiza: o da moça casadoira, cercada de pretendentes
em um dia de feriado, mas com o pai estipulando rigorosamente a duração
das conversas com todos os candidatos; uma hora, uma hora e meia no
máximo, ainda que sejam trinta rapazes ou mais, de modo que cada um
deles é obrigado a resumir tudo o que quer dizer em poucos minutos...
Uma boa metade da garrafa ainda está à espera quando a sirene ecoa até o
lugar em que eles estão. É o Ciudad de Mahon, há dez minutos de
distância, chamando para a partida do porto onde se encontra, com a
bagagem do estrangeiro já a bordo. Sobre os telhados, a luz de seu
mastro aparece na escuridão do céu. Que não resta mais muito tempo para
saudações, o próprio dono da taverna admite, de modo que estende, sem
grandes relutâncias e conforme o combinado, a conta ao estrangeiro.
Este, no entanto, se assusta, antes mesmo de lançar os olhos sobre ela.
Seu dinheiro se foi. Rápido como um raio, ele lança um olhar ao dono da
taverna. O rosto singelo deste expressa consternação. Impossível que ele
esteja com o envelope e as cédulas que este contém. Com as mesuras mais
distintas, ele pede que o estrangeiro não dê importância ao fato. De
resto, já lhe parecera pouco adequado ser convidado do homem em sua
própria casa. E, quanto ao dinheiro, estaria por certo na jaqueta que se
encontrava a bordo. Para o estrangeiro, no entanto, isso representa
apenas meio consolo. As cédulas das quais sente falta não são de pequeno
valor, e também não são poucas. A bordo, o pior se confirma. O casaco
está vazio, e ele agora sabe o que pensar da louvada honradez dos
moradores da ilha. Diante da alternativa de suspeitar do dono da taverna
ou do marinheiro que cuidou de suas coisas, ele se decide, durante a
noite insone em sua cabine, pela última alternativa. Mas se enganou. Foi
o dono da taverna que pegou o dinheiro. Mal chegou em casa, recebe a
prova disso na feição do seguinte telegrama: ``Dinheiro na jaqueta que o
senhor usou quando esteve aqui. Instruções seguem.''

``No que diz respeito ao telegrama'', disse dom Rosello, que ouvira com
um sorriso de concessão no rosto, ``com certeza foi o primeiro que ele
mandou em sua vida toda.''

``Sim, mas e o que isso tem a ver?''

``Sei muito bem'', replicou ele, ``onde o senhor quer chegar. Na
intocabilidade dos nativos. Na idade de ouro. Os lugares-comuns de
Rousseau. Há sete anos foram abertas as portas da prisão que se
localizava em um castelo mouro, e de fato não se precisava mais dela.
Mas o senhor sabe por quê? Vou dizê-lo com as palavras do velho guarda,
que na época tivemos de demitir: `A nossa gente\ldots{}, ela agora já
andou tanto pelo vasto mundo. E acabou aprendendo a diferenciar entre o
bem e o mal.' O contato com o mundo incentiva a moralidade. Isso é
tudo.''

A SEBE DE CACTOS\footnote{``Die Kaktushecke'', in GS IV-2, pp. 748-754.
  Tradução de Marcelo Backes. {[}N. dos E.{]}}\footnote{Este conto foi
  publicado em 8 de janeiro de 1933 no suplemento literário da
  \emph{Vossischen Zeitung.} Ele corresponde à versão resumida de um
  texto mais longo que infelizmente se perdeu. {[}N. da O.{]}}

O primeiro estrangeiro que veio até nós, em Ibiza, foi um irlandês,
O'Brien. Isso agora faz mais ou menos vinte anos, e o homem na época já
estava na casa dos quarenta. Antes de se aposentar aqui conosco, ele
viajara pelo mundo inteiro. Na juventude, vivera por muito tempo como
fazendeiro no leste da África, foi um grande caçador e laçador, mas era
sobretudo um tipo esquisito como nunca conheci outro igual. Ele se
mantinha longe dos círculos instruídos, dos padres, dos funcionários da
magistratura, e até mesmo com os nativos mantinha apenas alguns contatos
esparsos. Mesmo assim sua memória continua viva entre os pescadores
ainda hoje, e sobretudo por causa da sua maestria com os nós. De resto,
sua timidez parecia apenas em parte a consequência de sua natureza;
experiências adversas com pessoas próximas terminaram por completar a
dose.

Na época, eu não consegui descobrir muita coisa além do fato de que um
amigo, ao qual ele havia confiado sua única e valiosa propriedade,
simplesmente desaparecera com ela. Tratava-se de uma coleção de máscaras
negras, que ele havia adquirido com os próprios nativos em seus anos de
África. E de resto elas não trouxeram sorte àquele que se apropriara
delas. Ele morrera em um incêndio de navio, levando consigo a coleção de
máscaras que o acompanhara a bordo.

O'Brien se estabeleceu em sua finca, bem alto acima da baía. Quando
tinha algum trabalho em vista, no entanto, seu caminho sempre o levava
ao mar. Ali ele se ocupava da pesca, fazia as armadilhas de cana
baixarem cem metros ou mais na água, onde as lagostas passeiam sobre o
fundo rochoso do mar, ou saía às tardes tranquilas para lançar suas
redes, que doze horas depois eram retiradas. Além disso, ele continuava
gostando de capturar animais terrestres, e tinha relações suficientes
com amadores e cientistas na Inglaterra para apenas raramente ficar sem
encargos de arrumar algum pássaro empalhado, espécies raras de besouros,
gecos ou borboletas. Na maior parte do tempo, contudo, ele se ocupava
das lagartixas. Lembre-se dos terrários que na época, primeiro na
Inglaterra, acabaram por se estabelecer nos cantos de cacto dos
\emph{boudoirs} ou nos jardins de inverno. Lagartixas começaram a se
tornar artigos da moda, e nossas Baleares em pouco tempo ficaram tão
conhecidas entre os comerciantes de animais quanto o foram no passado
entre os chefes de legiões romanas por suas catapultas. Pois ``balea''
significa catapulta.

O'Brien, eu já disse, era um tipo esquisito. Acho que desde o fato de
caçar lagartixas até o modo de cozinhar e inclusive o de dormir e
pensar, ele não fazia nada do jeito que os outros costumam fazer. No que
dizia respeito a alimentos, ele dava pouco valor a vitaminas, calorias e
coisas do tipo. Tudo que era de comer, ele costumava dizer, era cura ou
envenenamento, e entre ambas as coisas não havia nada de intermediário.
Aquele que comia, portanto, deveria ver sempre a si mesmo como uma
espécie de convalescente, caso quisesse se alimentar de modo correto. E
logo se podia ouvir dele uma lista inteira de alimentos, dos quais
alguns eram mais adequados ao comportamento sanguíneo, os outros ao
colérico, outros ainda ao fleumático, e por fim outros ao melancólico,
se mostrando curativos para eles por incorporarem as devidas substâncias
complementares e atenuantes.

Coisa bem semelhante sucedia com o sono; ele tinha a respeito do assunto
sua própria teoria dos sonhos, e afirmava ter conhecido entre os
pangwes, uma tribo negra do interior africano, o meio infalível de
manter distante de si os pesadelos e os rostos torturantes que retornam
durante o sono. Seria preciso apenas, antes de ir dormir à noite,
invocar -- como os pangwes fazem em suas cerimônias -- a imagem
assustadora, pois assim se ficaria livre dela durante a noite inteira.
Ele chamava a isso de vacina do sonho.

E, por fim, o pensamento -- como ele agia em relação ao pensamento eu
ficaria sabendo certa tarde, quando estávamos em um barco, em meio à
água, para retirar as redes que haviam sido lançadas no dia anterior. A
pesca foi miserável. Nós havíamos recolhido quase toda a rede
praticamente vazia, quando algumas das malhas ficaram presas a um recife
e, apesar de todo o cuidado, acabaram se rasgando ao ser retiradas.

Eu enrolei minha capa de chuva, coloquei-a no meu barco e me estendi no
chão. O tempo estava encoberto, o ar tranquilo. Em pouco tempo caíram
alguns pingos de chuva, e a luz, que nesta terra solicita todas as
coisas tão violentamente do alto do céu, se afastou para devolvê-las à
terra.

Quando me levantei, meu olhar caiu sobre ele. Ele ainda segurava sua
rede nas mãos, mas elas já descansavam; o homem estava como que ausente.
Estranhando, eu o contemplei mais detidamente; seu rosto não tinha
expressão e não mostrava idade; em torno da boca cerrada, brincava um
sorriso. Eu peguei meu par de remos; alguns golpes nos levaram sobre a
água tranquila.

O'Brien levantou os olhos.

``Agora ela vai pegar de novo'', disse ele, e examinou, forçando
bastante, os novos nós na rede. ``Mas também se trata de um flamengo
duplo.''

Sem compreender, eu olhei para ele.

``Um flamengo duplo'', repetiu ele. ``Olhe bem, ele pode servir também
para a pesca com linha.''

E, ao dizê-lo, tomou um pedaço de barbante, dobrou uma de suas pontas e
o envolveu em si mesmo três, quatro vezes, até que ele se tornasse o
eixo de uma espiral cujos verticilos de um só movimento se juntaram em
nó.

``Na verdade'', prosseguiu ele, ``esse nó é apenas uma variação do nó
duplo das galés e, em todo caso, enlaçado ou não, preferível ao nó
carpinteiro.'' Tudo isso ele acompanhava de volteios rápidos e laçadas.
Eu sentia vertigens.

``Quem dá de primeira esse nó'', concluiu ele, ``conseguiu chegar bem
longe, e pode enfim descansar. E digo isso literalmente: descansar
mesmo, pois dar nós é uma arte da yoga; talvez o mais maravilhoso dos
meios para se descontrair. E só podemos aprendê-lo através do exercício
e da repetição -- não quando já estamos na água, mas em casa, com toda a
calma e paciência, no inverno, sobretudo se estiver chovendo. Melhor
ainda nos momentos em que se está angustiado ou preocupado. O senhor não
acredita quantas vezes encontrei soluções para perguntas que me
importunavam fazendo esse exercício.''

Por fim, ele prometeu me dar aulas nessa matéria, introduzindo-me em
todos os seus mistérios: dos nós cruzados aos de tecelão e inclusive os
de amortecimento e de Hércules.

Mas isso acabou não acontecendo; pois logo depois passou a ser visto
cada vez mais raramente no mar. Primeiro, ele ficava três, quatro dias
longe dele, depois semanas inteiras. Ninguém tinha a menor ideia do que
O'Brien fazia enquanto isso. Murmuravam acerca de uma ocupação secreta.
Sem dúvida alguma ele havia descoberto uma nova paixão.

Passaram-se alguns meses até estarmos outra vez juntos no barco. Nessa
ocasião, a pesca foi mais abundante e, quando por fim encontramos uma
grande truta do mar em seu anzol, O'Brien me convidou para ir visitá-lo
na noite seguinte para um pequeno jantar.

Depois de concluída a refeição, O'Brien disse, abrindo uma porta:
``Minha coleção, da qual o senhor com certeza já ouviu falar.''

Por certo eu já ouvira falar da coleção de máscaras de negros,
entretanto sabia apenas que elas haviam naufragado.

Mas eis que ali estavam penduradas, vinte ou trinta peças, no quarto
vazio, sobre paredes brancas. Eram máscaras de expressões grotescas, que
revelavam sobretudo uma severidade levada ao cômico, uma recusa
completamente inexorável de tudo que era desmedido. Os lábios superiores
abertos, as estrias abobadadas que haviam se tornado a fenda das
pálpebras e sobrancelhas pareciam expressar algo como asco infinito
contra aquele que se aproximava, até mesmo contra tudo o que se
aproxima, enquanto os cimos empilhados dos adornos na testa e os
reforços das mechas de cabelos entrançadas se destacavam como marcas que
anunciavam os direitos de um poder estranho sobre aquelas feições. Para
qualquer dessas máscaras que se olhasse, em lugar nenhum sua boca
parecia destinada, como quer que fosse, a emitir sons; os lábios grossos
e entreabertos, ou então bem cerrados, eram cancelas instaladas antes ou
depois da vida, como os lábios dos embriões ou os dos mortos.

O'Brien havia ficado para trás.

``Esta aqui'', disse ele de repente atrás de mim e como se falasse
consigo mesmo, ``foi a primeira que reencontrei.''

Quando me virei, ele estava parado diante de uma cabeça alongada, lisa,
de ébano negro, que mostrava um sorriso. Era um sorriso por assim dizer
do princípio ao fim, que no fundo parecia um ruminar do sorriso por trás
dos lábios cerrados. De resto, aquela boca jazia bem profunda, como
também o semblante inteiro não parecia mais do que o rebento monstruoso
da testa formidavelmente abaulada, que descia abaixo em arcos
inexoráveis, interrompidos apenas pelos círculos redondos e solenes dos
olhos, que se destacavam como que de um escafandro.

``Essa foi a primeira que reencontrei. E eu também poderia dizer como
foi ao senhor.''

Eu apenas olhei para ele. Com as costas, ele se apoiou contra a janela
baixa, e em seguida principiou:

``Se o senhor olhar para fora, verá a sebe de cactos diante de seus
olhos. É a maior de toda a região. Observe o tronco, como está
amadeirado até bem no alto. Nele o senhor pode reconhecer a idade da
sebe; pelo menos cento e cinquenta anos. Era uma noite como a de hoje,
só que a lua brilhava. Lua cheia. Não sei se o senhor já se deu conta do
efeito da lua nessa região, pois sua luz não parece cair sobre o cenário
de nossa existência diurna, mas sim sobre uma terra oposta ou
paralela.\footnote{Esta passagem figura igualmente numa resenha de 1928,
  constante do presente volume, em que Benjamin trata do livro de Jakob
  Job, intitulado \emph{Nápoles. Imagens de viagem e esboços}.
  \emph{Vide} página \textbf{\emph{XX}}, adiante. {[}N. dos E.{]}} Eu
havia passado aquele entardecer inteiro diante dos meus mapas marítimos.
O senhor precisa saber que meu cavalo de batalha é melhorar os mapas do
Ministério da Marinha britânico, o que é ao mesmo tempo uma fama
conquistada de modo bem barato, pois onde ocupo um novo lugar com minhas
nassas também acabo fazendo sondagens. Eu havia, pois, identificado o
lugar correto de algumas colininhas no mar e pensado como seria bonito
se me eternizassem lá embaixo, nas profundezas, dando a uma delas o meu
nome. E em seguida fui para a cama. O senhor por certo há de ter visto
que minhas janelas estão cobertas por cortinas; na época elas ainda não
estavam, e a lua avançava sobre a minha cama, enquanto eu estava
deitado, insone. Eu começara outra vez minha brincadeira predileta de
dar nós. Acho que já falei disso ao senhor uma vez. Isso vai acontecendo
assim que dou um nó complicado em pensamentos, logo o ponho comigo mesmo
de lado e acabo conseguindo dar um segundo, outra vez em pensamentos.
Então o primeiro volta a me ocupar a mente. Só que dessa vez não preciso
atá-lo, mas sim desatá-lo. É claro que tudo depende de manter, com toda
a precisão, a forma do nó na memória, sobretudo o primeiro não pode se
confundir com o segundo. Volto a fazer esses exercícios, nos quais
realmente consegui adquirir um bocado de conhecimento, sempre que tenho
ideias na cabeça e não encontro nenhuma solução, ou cansaço nos membros
e não encontro sono. Em ambos os casos o resultado é o mesmo:
descontração.

Dessa vez, porém, minha maestria não me ajudou em nada, pois quanto mais
eu me aproximava da solução, tanto mais próximo ficava também o clarão
ofuscante da lua em minha cama. Então resolvi fugir para um outro
método. Passei em revista todas as sentenças, os enigmas, as canções e
ditados que eu havia aprendido aos poucos na ilha. Isso já estava dando
mais certo. Sentia minhas contrações internas cedendo, quando meu olhar
caiu sobre a sebe de cactos. Um antigo versinho de zombaria me veio à
memória: `Buenas tardes chlumbas figas.' O jovem campônio diz `Boa
tarde' aos figos do cacto, arranca sua faca e, como se diz, dá-lhe um
talho da espinha até o traseiro.

Mas a época dos frutos do cacto já havia passado há tempo. A sebe estava
pelada; suas folhas ora espetavam o vazio, inclinadas, ora se
amontoavam, cascas grossas que esperavam em vão pela chuva.

`Nenhuma cerca, mas muitos espectadores olhando por cima dela', foi o
que me passou pela cabeça.

Pois, entrementes, parecia ter ocorrido uma transformação com aquela
sebe. Era como se aqueles lá fora olhassem todos para o clarão que agora
envolvia toda a minha cama; como se ali houvesse um bando dependurado,
prendendo a respiração, em meus olhares. Uma confusão de escudos
erguidos, clavas e machados de guerra. E, ao adormecer, eu reconheci de
repente o método através do qual aquelas figuras ali fora mantinham-me
em xeque. Eram máscaras que se erguiam em minha direção!

Assim o sono acabou tomando conta de mim. Na manhã seguinte, porém, eu
não me permiti ficar em paz. Peguei uma faca e em seguida me tranquei
durante oito dias com o bloco do qual acabou surgindo a máscara que está
pendurada aqui. As outras surgiram uma após a outra, e sem que eu algum
dia tenha perdido mais um olhar sequer para a sebe de cactos. Não quero
dizer que são todas parecidas com as minhas máscaras de antes; mas
poderia jurar que nenhum conhecedor seria capaz de diferenciá-las
daquelas que há anos estiveram em seu lugar.''

Foi isso que me contou O'Brien. Nós ainda papeamos por um momento,
depois eu fui embora.

Algumas semanas mais tarde, ouvi que O'Brien havia se trancado outra vez
com um trabalho misterioso e se tornara inacessível para todo mundo.
Jamais voltei a vê-lo, pois logo em seguida ele morreu.

Por muito tempo não pensara mais nele quando, para minha surpresa,
descobri certo dia três máscaras de negros numa caixa de vidro de um
comerciante de arte parisiense na Rue La Boétie.

``Posso'', disse eu, voltando-me para o diretor da casa, ``parabenizá-lo
de coração por essa aquisição incrivelmente bela?''

``Vejo com prazer'', foi a resposta, ``que o senhor sabe honrar a
qualidade! Vejo ademais que o senhor é um conhecedor! As máscaras que o
senhor com razão admira não são mais do que uma pequena amostra da
grande coleção cuja exposição estamos preparando no momento!''

``E eu poderia pensar, meu senhor, que essas máscaras certamente
inspirariam nossos jovens artistas a fazer suas próprias tentativas
interessantes.''

``É o que eu espero, inclusive!... Aliás, se o senhor se interessar mais
de perto pelo assunto, posso fazer com que cheguem até o senhor, do meu
escritório, os pareceres de nossos maiores conhecedores de Haia e de
Londres. O senhor haverá de ver que se trata de objetos de centenas de
anos. De dois deles eu diria até que de milhares de anos.''

``Ler esses pareceres de fato me interessaria muito! Eu poderia
perguntar a quem pertence essa coleção?''

``Ela pertence ao espólio de um irlandês. O'Brien. O senhor com certeza
jamais ouviu seu nome. Ele viveu e morreu nas Ilhas Baleares.''

HISTÓRIAS DA SOLIDÃO\footnote{``Geschichten aus der Einsamkeit '', in GS
  IV-2, pp. 755-757. Tradução de Marcelo Backes. {[}N. dos E.{]}}\footnote{De
  acordo com o testemunho de Gershom Scholem, essas histórias teriam
  sido escritas entre 1932 e 1933. Não há certeza quanto à ordem da
  série. {[}N. da O.{]}}

\emph{O muro}

Eu vivia há alguns meses em um ninho nas rochas, na Espanha. Muitas
vezes eu me propusera a sair para explorar os arredores, envolvidos por
uma coroa de cumes perigosos e formações florestais de pinheiros bem
escuras. No meio, havia algumas aldeias escondidas; a maior parte
recebera nomes de santos, que poderiam muito bem habitar aquela região
paradisíaca. Mas era verão; o calor fazia com que eu adiasse a cada dia
meu propósito, e mesmo ao passeio preferido até a colina dos moinhos de
vento, que eu via da minha janela, acabei renunciando. De modo que
restou-me o perambular costumeiro pelas ruelas estreitas e sombreadas,
em cuja rede jamais se encontra o mesmo nó do mesmo jeito. Certa tarde,
acabei encontrando, em minhas errâncias, uma loja de bugigangas na qual
podiam ser comprados cartões postais. De qualquer modo havia alguns na
vitrine, um deles com a foto da muralha de um dos lugarejos, uma das
muitas que encontramos neste canto do mundo. Eu jamais vira uma muralha
semelhante, no entanto. O fotógrafo havia captado toda a sua magia, e
ela serpenteava através da paisagem como uma voz, como um hino por todos
os séculos de sua duração. Eu prometi não comprar aquele cartão antes de
ter visto pessoalmente o muro que era apresentado nele. Não falei a
ninguém do meu propósito, e podia fazê-lo facilmente pois o cartão me
conduzia ao lugar certo com sua assinatura: ``S. Vinez''. Por certo, eu
não sabia nada sobre um santo Vinez. Mas saberia mais acerca de um são
Fabiano, de um são Romano ou de um são Sinfório, como eram chamados os
outros lugares da região? Ainda que meu guia de viagem não apontasse o
nome, isso não queria dizer nada, a princípio. A região era habitada por
camponeses, e marinheiros faziam suas marcações segundo ela: ambos,
porém, tinham nomes diferentes para os mesmos lugares. De modo que
busquei o auxílio de mapas mais antigos, e como isso também não me levou
mais adiante, consegui um mapa de navegação. Em pouco tempo essa
pesquisa já me deixava fascinado, e teria sido contra a minha honra
buscar a ajuda ou o conselho de um terceiro em um estágio tão avançado
das minhas pesquisas. Eu acabara de passar mais uma vez uma hora sobre
meus mapas quando um conhecido, nativo do lugar, convidou-me para um
passeio ao entardecer. Ele queria me levar para a colina fora da cidade,
da qual os moinhos de vento há muito desativados me cumprimentaram
tantas vezes por sobre a copa dos pinheiros. Quando chegamos lá em cima,
começou a escurecer, e nós descansamos para esperar a lua, a cujo
primeiro raio nós nos pusemos a caminho de casa. Saímos de um bosque de
pinheiros. Ali estava, à luz da lua, próximo e inconfundível, o muro
cuja imagem me acompanhava há dias, e sob sua proteção a cidade para a
qual estávamos voltando. Eu não disse palavra, mas logo me separei de
meu amigo... Na tarde seguinte, dei de cara sem querer com minha loja de
bugigangas. O cartão postal ainda estava pendurado na vitrine. Sobre a
porta, no entanto, li em uma placa que antes não havia percebido,
escrito em letras vermelhas, ``Sebastiano Vinez''. O pintor havia
acrescentado um pão-de-açúcar e uma fatia de pão.

\emph{O cachimbo}

Durante um passeio na companhia de um casal com o qual fizera amizade,
passei pelas proximidades da casa que eu habitava na ilha. Tive vontade
de acender meu cachimbo. Como não o encontrei ao tocar no bolso, num
gesto habitual, pareceu-me que era a oportunidade adequada para ir
buscá-lo no quarto, onde haveria de estar sobre a mesa. Com uma breve
explicação, pedi ao amigo que se adiantasse com sua mulher, enquanto eu
procurava o desaparecido. Dei meia volta; mas eu mal havia me afastado
dez passos quando senti, vasculhando de novo, que o cachimbo estava em
meu bolso. Foi assim que os outros me viram retornar para junto deles em
menos de um minuto, soltando nuvens de fumaça pelo cachimbo. ``Não é que
ele estava realmente sobre a mesa'', expliquei eu, seguindo um capricho
incompreensível. No olhar do homem apareceu algo que o fez se parecer
com alguém que despertava e, depois de um sono profundo, ainda não
conseguira descobrir muito bem onde estava. Nós seguimos adiante, e a
conversa retomou seu curso normal. Um pouco mais tarde, eu a reconduzi
ao \emph{intermezzo}. ``Como é possível, perguntei eu, que o senhor não
tenha percebido nada? O que eu afirmei era completamente impossível.''

``Isso com certeza, respondeu o homem depois de uma breve pausa. Eu até
quis dizer algo. Mas então pensei comigo: no fundo deve estar certo. Por
que ele haveria de mentir pra mim?''

\emph{A luz }

Eu estava pela primeira vez sozinho com minha amada e em um povoado
desconhecido. Esperava diante do meu alojamento, que não era o dela. Nós
ainda queríamos fazer um passeio noturno. À espera, andei para cima e
para baixo na rua do povoado. Então vi ao longe, entre as árvores, uma
luz. ``Essa luz'', foi o que pensei comigo, ``nada diz àqueles que todas
as noites a têm diante do olhos. Ela deve pertencer a um farol ou a uma
fazenda. Para mim, que não sou daqui, no entanto, ela diz muito.'' E com
isso dei meia volta para seguir novamente pela rua do povoado. Assim
continuei por algum tempo e, sempre que eu me voltava, depois de alguns
instantes, a luz entre as árvores atraía meu olhar. Mas então aconteceu
que ela ordenou que eu parasse. Isso foi pouco antes da minha amada me
reencontrar. Eu me desviara outra vez, e reconheci: a luz que eu
vislumbrara junto à terra era a da lua, que aos poucos havia subido
acima das copas longínquas das árvores.

QUATRO HISTÓRIAS\footnote{``Vier Geschichten'', in GS IV-2, pp. 757-761.
  Tradução de Marcelo Backes. {[}N. dos E.{]}}\footnote{Esse conjunto de
  quatro histórias foi publicado em 5 de agosto de 1934 no \emph{Prager
  Tagblatt}. Outras publicações, parciais, apareceram na
  \emph{Kölnischen Zeitung}, em 12/07/1933; na \emph{Frankfurter
  Zeitung}, em 05/09/1934; e na \emph{Basler Nachrichten}, em
  26/09/1935. Uma tradução dinamarquesa de ``A assinatura'' foi
  publicada em 16 de setembro de 1934 num periódico de Copenhagen. {[}N.
  da O.{]}}\footnote{``A assinatura'' e ``O desejo'' aparecem também no
  ensaio de Benjamin sobre Franz Kafka, de 1934. E histórias semelhantes
  a estas duas são narradas por Ernst Bloch, respectivamente, em
  \emph{Rastros} (\emph{Spuren}, 1930) e \emph{Através do deserto}
  (\emph{Durch die Wüste}, 1923). {[}N. dos E.{]}}

\emph{O alerta }

Junto a um lugar de excursão, não muito distante de Tsingtau, havia uma
formação rochosa que se distinguia por sua localização romântica e pelas
paredes íngremes com que se precipitava nas profundezas. Essa formação
rochosa era visitada por muitos amantes em seus momentos felizes, os
quais, depois de terem admirado a paisagem nos braços de suas namoradas,
voltavam na companhia das mesmas até uma hospedaria próxima. Essa
hospedaria ia muito bem. Ela pertencia ao senhor Ming.

Então certo dia um amante, que havia sido abandonado, teve a ideia de
pôr um fim em sua vida justamente ali onde ele a desfrutara com mais
entusiasmo e, não muito longe da hospedaria, precipitou-se da rocha para
as profundezas. Esse amante inventivo encontrou imitadores, não demorou
muito e essas formações rochosas ficaram tão famigeradas com seu
cemitério de crânios quanto famosas como mirante. O estabelecimento do
senhor Ming, no entanto, sofria com a nova fama; nenhum cavalheiro
poderia ousar levar sua dama a um lugar onde a todo instante poderia
chegar uma ambulância. Os negócios do senhor Ming iam de mal a pior, e
não lhe restou outra coisa a fazer senão refletir.

Ele se trancou certo dia em seu quarto. Quando voltou a aparecer, foi
até a estação elétrica que se localizava nas proximidades. Depois de
poucos dias, um arame contornava a extremidade da romântica formação
rochosa. Sobre uma tabuleta pendurada a ele, era possível ler:
``Atenção! Alta tensão! Perigo de morte!'' Desde então, os candidatos a
suicida evitaram aquela região, e os negócios do senhor Ming voltaram a
florescer como antes.

\emph{A assinatura }

Potemkin sofria de pesadas depressões, que sempre voltavam mais ou menos
regularmente, durante as quais ninguém podia se aproximar dele e o
acesso a seu quarto era estritamente proibido. Na corte, esse sofrimento
não era mencionado, e se sabia, sobretudo, que qualquer alusão ao fato
suscitaria o desagrado fatal da czarina Catarina. Uma dessas depressões
do chanceler se mostrou extraordinariamente longa. Problemas sérios
foram a consequência; nos registros, se amontoavam autos cujas demandas
a czarina exigia que fossem solucionadas ainda que isso se mostrasse
impossível sem a assinatura de Potemkin. Os altos funcionários não
sabiam mais o que fazer.

Por essa época, e em razão de um acaso, o pequeno e insignificante
funcionário de chancelaria Schuwalkin acabou adentrando a antessala do
palácio do chanceler, onde os conselheiros de Estado, como de costume,
haviam se reunido, lamentando e se queixando. ``O que há, excelências?
Como posso servir a vossas excelências?'', observou o zeloso Schuwalkin.
Explicaram-lhe o caso, e lamentaram não poder fazer uso absolutamente
nenhum de seus serviços. ``Se não for nada além disso, meus senhores'',
respondeu Schuwalkin, ``podem deixar os autos comigo. Eu peço que seja
assim.'' Os conselheiros de Estado, que nada tinham a perder, se
deixaram convencer, e Schuwalkin tomou, com a pilha de pastas debaixo do
braço, o caminho até o quarto de Potemkin, atravessando galerias e
corredores. Sem bater, e até mesmo sem parar um momento sequer, ele
baixou a maçaneta. O quarto não estava trancado. No lusco-fusco,
Potemkin estava sentado em sua cama, roendo as unhas, vestia um pijama
puído. Schuwalkin foi até a escrivaninha, mergulhou a pena na tinta e,
sem perder tempo em dizer uma palavra sequer, empurrou-a até a mão de
Potemkin, colocando o primeiro documento sobre seus joelhos. Lançando um
olhar ausente para o intruso, como se estivesse dormindo, Potemkin
assinou; depois uma segunda vez; e assim com todas as pastas. Quando a
última estava devidamente assinada, Schuwalkin deixou, sem quaisquer
circunstâncias e do mesmo jeito que viera, os aposentos, com seu dossiê
debaixo do braço.

Triunfante e sacudindo os documentos, Schuwalkin adentrou a antessala.
Os conselheiros de Estado se precipitaram ao encontro dele e arrancaram
os papéis de suas mãos. Todos se curvaram esbaforidos sobre eles.
Ninguém disse uma palavra; o grupo ficou pasmo. O pequeno funcionário da
chancelaria se aproximou mais uma vez, e mais uma vez perguntou, zeloso,
pelo motivo do espanto dos senhores. Então também seu olhar caiu sobre a
assinatura. Tanto aquele quanto todos os outros documentos estava
assinado com: Schuwalkin, Schuwalkin, Schuwalkin...

\emph{O desejo }

Certa noite, em uma aldeia hassídica, ao fim do \emph{shabat}, estavam
sentados os judeus em uma taverna pobre. Eram todos do lugar, exceto um,
a quem ninguém conhecia, um bem pobre e esfarrapado, encolhido ao fundo,
à sombra da lareira. As conversas haviam ido de um lado a outro. Então
alguém teve a ideia de perguntar o que cada um deles desejaria caso
tivesse direito a um desejo. Um queria dinheiro, outro um genro, o
terceiro um banco de carpinteiro novo, e assim os desejos seguiram a
roda.

Todos haviam tomado a palavra, restava apenas o mendigo no canto da
lareira. Contra a vontade e hesitando, ele cedeu às perguntas: ``Eu
gostaria de ser um rei muito poderoso e reinar sobre um país gigante e
deitar à noite e dormir em meu palácio, e que na fronteira o inimigo
invadisse o reino e, antes que escurecesse, a cavalaria teria avançado
até diante do meu castelo e não haveria resistência, e, acordando
assustado do meu sono, sem tempo nem mesmo de me vestir, em mangas de
camisa, eu teria de me pôr em fuga e seria acossado por montanhas e
vales e pela floresta e pelas colinas, sem paz, dia e noite, até que
enfim chegasse aqui, a esse banco, salvo nesse vosso cantinho. Isso eu
desejaria.''

Sem compreender, os outros se entreolharam. ``E o que você conseguiria
com tudo isso?'', perguntou um deles.

``Uma camisa'', foi a resposta.

\emph{O agradecimento}

Beppo Aquistapace era empregado de um banco nova-iorquino. O humilde
homem vivia apenas para seu trabalho. Em quatro anos de serviços
prestados, ele faltara no máximo três vezes, e jamais sem uma desculpa
plausível. Por isso, chamou obrigatoriamente a atenção ao faltar certo
dia sem nada anunciar. Quando também no dia seguinte não chegaram nem o
homem nem sua desculpa, o senhor McCormik, chefe de pessoal, lançou uma
série de palavras questionadoras no escritório de Aquistapace. Mas
ninguém soube lhe dar informações. O desaparecido tinha poucas relações
com seus colegas; circulava com italianos, que como ele eram oriundos
das baixas classes sociais. E foi justamente esta circunstância que
Aquistapace invocou em uma carta, na qual, depois de uma semana, deu
informações sobre seu destino ao senhor McCormik.

Essa carta veio de uma cela da detenção. Nela, Aquistapace se dirigia a
seu chefe com palavras tão ponderadas quanto urgentes. Um acontecimento
lamentável no bar que costumava frequentar, acontecimento no qual ele
aliás não tivera absolutamente nenhuma participação, havia determinado
sua detenção. Ainda hoje ele não era capaz de mencionar o motivo que
levara a uma briga de faca entre seus conterrâneos. Lamentavelmente,
houvera uma vítima. E eis que ele não conhecia ninguém a não ser o
senhor McCormik para nomear como avalista de sua boa fama... Este não
apenas tinha um certo interesse no trabalho fiel e cumpridor de seu
dever do preso, mas inclusive relações que tornaram fácil para ele dar
uma palavrinha a favor do outro na hora e no lugar adequados.
Aquistapace estava detido há apenas dez dias quando voltou a assumir seu
trabalho no banco.

Depois de fechado o escritório, ele anunciou sua presença junto a
McCormik. Encabulado, Aquistapace se encontrava em pé diante de seu
chefe. ``Senhor McCormik'', principiou ele, ``não sei como posso
agradecer ao senhor. Ao senhor, e apenas ao senhor, devo o fato de ter
sido libertado. Acredite em mim, nada me deixaria mais alegre do que me
mostrar reconhecido ao senhor. Lamentavelmente, sou um homem pobre. E'',
acrescentou ele com um sorriso humilde, ``que eu também não ganho nenhum
tesouro no banco, o senhor sabe melhor do que ninguém. Mas, senhor
McCormik'', concluiu ele em voz firme, ``uma coisa posso lhe garantir:
se algum dia o senhor estiver em uma situação em que a eliminação de um
terceiro poderia ser útil, peço que se lembre de mim. Comigo o senhor
pode contar.''

A MORTE DO PAI\footnote{``Der Tod des Vaters. Novelle'', in GS IV-2, pp.
  723-725. Tradução de Marcelo Backes. {[}N. dos E.{]}}\footnote{Benjamin
  se refere à redação deste conto numa carta datada de 7 de junho de
  1913 e endereçada ao seu colega de estudos Herbert Belmore. Escrito no
  período de militância no Movimento da Juventude
  (\emph{Jugendbewegung}), o texto permaneceu inédito até a publicação
  póstuma. {[}N. da O.{]}}

\textsc{Novela}

Durante a viagem, ele evitou tornar claro o sentido daquele telegrama:
``Venha imediatamente. Mudança para pior.'' Ao anoitecer, havia deixado
o lugar em que estava, na Riviera, em meio ao tempo ruim. As lembranças
o envolviam como as luzes matinais que caem sobre um frequentador de bar
que chega atrasado: doces e vergonhosas. Indignado, ele ouvia os ruídos
da cidade, em cujo meio-dia adentrava. Estar humilhado lhe parecia a
única resposta às perseguições da terra natal. Gorjeando, porém, ele
sentia a volúpia das horas perdidas junto a uma mulher casada.

Ali estava seu irmão. E, como um choque elétrico que descia por seus
quadris, ele odiava aquele homem vestido de preto. Cumprimentou-o às
pressas com um olhar melancólico. Um carro já estava pronto. A viagem
começou matraqueante. Otto balbuciou uma pergunta, mas a lembrança de um
beijo o arrebatou.

De repente, nas escadarias do prédio, estava a criada, e ele desabou
quando ela pegou sua mala pesada. Ele ainda não vira sua mãe, mas o pai
estava vivo. Ali estava ele, sentado junto à janela, inchado, em sua
cadeira de braços... Otto foi até ele e lhe deu a mão. ``Você não vai me
dar um beijo, Otto?'', perguntou o pai em voz baixa. O filho se jogou
sobre ele, correu para fora -- parou na sacada e berrou para a rua.
Ficou cansado de tanto chorar, e se lembrou, sonhando, da época em que
entrou no colégio, dos anos de comerciante, da viagem para os Estados
Unidos.

``Senhor Martin.'' Ele estava tranquilo e agora se sentia envergonhado
por seu pai ainda estar vivo. Assim que soluçou mais uma vez, a criada
botou a mão sobre o seu ombro. Mecanicamente, ele viu: uma pessoa
saudável e loura, a refutação do doente que ele havia tocado. Ele se
sentiu em casa.

A biblioteca que Otto usou nas duas semanas de sua estadia ficava no
bairro mais movimentado da cidade. Todas as manhãs ele trabalhava três
horas num texto que deveria lhe conceder o título de doutor em Economia.
À tarde, ele também ia até lá para estudar as revistas de arte
ilustradas. Ele amava a arte e lhe dedicava muito tempo. Naquelas salas,
não ficava sozinho. Se entendia muito bem com o digno funcionário que
lhe emprestava e depois recebia os livros que ele devolvia. Quando
levantava os olhos da obra que estava lendo, franzindo a testa e perdido
em pensamentos, não eram poucas as vezes em que encontrava uma cabeça
conhecida dos tempos do primário.

A solidão daqueles dias, que jamais era ociosa, lhe fazia bem, depois de
nas últimas semanas na Riviera pôr cada um dos seus nervos a serviço de
uma mulher sensual. À noite, na cama, ele procurava por detalhes do
corpo dela, ou então agradava-lhe enviar, em belas ondas, sua
sensualidade cansada até onde ela estava. Pensava nela raramente. Quando
se encontrava sentado diante de uma mulher no bonde, apenas distendia as
sobrancelhas de modo significativo, com expressão vazia, um gesto com o
qual implorava solidão inacessível em troca da doce inércia.

A azáfama em torno do moribundo era bem regular na casa; e nem sequer
lhe importava. Certa manhã, no entanto, o acordaram mais cedo do que de
costume e o levaram até diante do cadáver de seu pai. O quarto estava às
claras. Diante da cama, a mãe jazia desmaiada. O filho, porém, sentiu
tanta força que a agarrou por baixo dos braços e disse com voz firme:
``Levante-se, mamãe.'' Nesse dia, ele foi para a biblioteca como sempre.
Seu olhar, quando passava pelas mulheres, estava ainda mais vazio e
firme do que de costume. Apertou a pasta, na qual havia dois maços com
as folhas de seu trabalho, junto ao corpo quando subiu à plataforma do
bonde.

De qualquer modo, ele trabalhou mais inseguro desde aquele dia. Percebeu
defeitos, problemas fundamentais que até então simplesmente ignorara
começaram a ocupá-lo. Quando encomendava livros, perdia de repente
qualquer medida e objetivo. Pilhas inteiras de revistas o envolviam, nas
quais buscava com um detalhismo estúpido os dados mais desimportantes.
Se interrompia a leitura, jamais era abandonado pela sensação
equivalente a de uma pessoa que usa roupas largas demais. Quando jogou
os torrões de terra na tumba de seu pai, vislumbrou o nexo entre o
discurso fúnebre, a sequência infinita de conhecidos e a própria falta
de ideias. ``Tudo isso já foi assim tantas vezes. Como isso é típico.''
E, quando saiu das proximidades do túmulo, se misturando à multidão em
luto, a dor de sua alma já se tornara como uma coisa que simplesmente se
carrega consigo por aí, e seu rosto parecia mais largo devido à
indiferença. As conversas em voz baixa entre a mãe e o irmão o
incomodavam quando eles estavam sentados à mesa a três. A criada loura
trazia a sopa. Despreocupadamente, Otto levantava a cabeça e olhava para
seus olhos castanhos, desamparados.

Assim Otto ainda conseguia, com uma certa frequência, embelezar para si
mesmo o medo mesquinho daqueles dias de luto. Certa vez beijou a criada
-- à noite -- no corredor. A mãe recebia sempre palavras calorosas
quando estava sozinha com ele; na maior parte das vezes, contudo, ela
discutia assuntos de negócios com o irmão mais velho.

Quando ele, num desses meios-dias, voltou da biblioteca, teve a ideia de
viajar. Pois o que ele ainda tinha a fazer por ali? Importava era
estudar.

Ele se encontrava sozinho em casa, e assim entrou no escritório de seu
pai como de hábito fazia. Ali, sobre o divã, o falecido passara suas
últimas horas de sofrimento. As cortinas haviam sido baixadas, porque
estava quente, e pelas frestas aparecia o céu. A criada veio e colocou
anêmonas sobre a escrivaninha. Otto estava apoiado junto ao divã e,
quando ela passou, puxou-a para si sem fazer ruído. Uma vez que ela
pressionava o corpo ao dele, eles se deitaram juntos. Depois de algum
tempo, ela o beijou e se levantou sem que ele a segurasse.

Ele viajou dois dias mais tarde. Deixou a casa bem cedo. Ao lado dele
seguia a criada com a mala, e Otto lhe contava da cidade universitária e
da faculdade. Mas, na despedida, ele apenas deu a mão a ela, pois a
estação ferroviária estava cheia. ``O que meu pai haveria de dizer?'',
ele pensou, enquanto se recostava e bocejava, afastando o derradeiro
sono de seu corpo.

PALÁCIO D... \textsc{Y}\footnote{``Palais D\ldots{}y'', in GS IV-2, pp.
  725-728. Tradução de Marcelo Backes. {[}N. dos E.{]}}\footnote{Conto
  publicado no periódico \emph{Die Dame}, em junho de 1929. {[}N. da
  O.{]}}

Nos anos de mil oitocentos e setenta e cinco a mil oitocentos e oitenta
e cinco, o barão X costumava chamar a atenção no Café de Paris, e,
quando se pedia aos estranhos de alguma distinção para atentarem ao
conde de Caylus, ao marechal Fécamts, ao cavaleiro Raymond Grivier e
também a ele, o barão, não era por causa de sua elegância, sua origem,
suas conquistas esportivas, mas simplesmente por reconhecimento, sim,
por admiração à fidelidade que este havia dedicado ao estabelecimento
por tantos anos. Uma fidelidade que ele mais tarde demonstraria a alguém
bem diferente, e ademais bem pouco usual. Mas é disso, justamente, que
trata essa história.

Ela principia, mais precisamente, com a herança que durante trinta anos
sempre deveria ter sido dada, e aliás justamente dada, ao barão, e
finalmente também lhe foi dada em setembro de mil oitocentos e oitenta e
quatro. Na época, o herdeiro não estava muito distante de seu
quinquagésimo aniversário e há tempos já não era mais um \emph{bon}
\emph{vivant}. E por acaso o havia sido algum dia? Às vezes até se fazia
a pergunta. Se então alguém poderia afirmar que jamais dera de cara uma
única vez com o nome do barão na crônica escandalosa de Paris, e mesmo
na boca dos frequentadores de clube mais inescrupulosos e das cocotes
mais famigeradas jamais ouvira qualquer alusão a ele, não se poderia
duvidar de outro que dissesse: o barão em suas calças ajustadas, com a
gravata \emph{lavallière} larga era mais do que uma figura mundana; em
seu semblante havia algumas rugas que revelavam um conhecedor de
mulheres que pagou por sua sabedoria. De modo que, até aquele momento, o
barão permanecera sendo um mistério, e ver em suas mãos aquela herança
vultosa, esperada há tanto tempo, despertou em seus amigos, além de uma
benevolência desprovida de inveja, a curiosidade mais discreta e
maliciosa. O que nenhum papo junto à lareira, nenhuma garrafa de
Borgonha haviam conseguido -- erguer o véu que encobria aquela vida --,
eles acreditavam poder esperar da riqueza repentina.

Depois de dois ou três meses, no entanto, a opinião de todos era
unânime: a decepção não poderia ter sido mais completa. Nada, nem mesmo
uma sombra havia mudado nas vestes, no humor, na distribuição do tempo,
e até mesmo nos gastos e na moradia do barão. Ele continuava sendo o
indolente distinto, para o qual o tempo parecia tomado até a borda como
ao mais mesquinho dos amanuenses, ele continuava, ao sair do clube,
sendo recebido pela \emph{garçonnière} da Avenue Victor Hugo, e jamais
amigos que queriam acompanhá-lo até em casa à noite foram despedidos com
desculpas. Sim, parecia que o dono da casa mantinha a banca aberta até
às cinco horas da manhã, e ainda mais tempo se fosse em seu quarto de
visitas, no espaço onde outrora ficava um magnífico armário Chippendale
que desde sempre por lá estivera, sentado diante de uma mesa verde que
passou a ocupar o lugar. O barão costumava ter sorte no jogo -- isso se
sabia pelas raras vezes em que ele aparecera mais cedo para se sentar à
mesa verde. Mas agora nem mesmo os jogadores mais contumazes podiam
deixar de vivenciar as sequências de sorte que o inverno de mil
oitocentos e oitenta e quatro acabou lhe trazendo. Elas perduraram por
toda a primavera e assim continuaram quando o verão invadiu, com seus
lagos de sombra, os bulevares. Como foi que o barão se tornou um homem
pobre em setembro? Pobre não; mas exatamente tão flutuante, indefinível
entre pobre e rico como já era antes, e apenas mais pobre por já não ter
a expectativa de uma grande herança. Tanto que começou a se impor
limites, visitava o clube apenas para uma xícara de chá ou uma partida
de xadrez. E ninguém ousava fazer alguma pergunta. O que, ademais,
deveria parecer questionável em uma existência que decorria em seu
âmbito estreito e mundano diante dos olhos de todo mundo, da cavalgada
matinal, do exercício de florete e do almoço até a hora em que o sino
batia, às quinze para as seis, quando ele deixava o Café de Paris para
duas horas mais tarde jantar em boa companhia no Delaborde? No
intervalo, ele nem sequer tocava o baralho. E mesmo assim aquelas duas
horas do dia lhe custaram toda a fortuna.

Como isso aconteceu, soube-se em Paris apenas anos mais tarde, quando o
barão já havia se retirado sabe-se lá para onde -- o que o nome de uma
propriedade rural aristocrática localizada em terras lituanas distantes
acrescentaria aqui? --, e, em certa manhã chuvosa, um de seus amigos, em
meio às perambulações esquecidas da vida, estacou assustado, ele mesmo
não sabia por que motivo no primeiro momento: se devido a uma visão ou a
uma ideia. Na verdade, devido a ambas. Pois o monstro que descia
balançando diante dele, sobre os ombros de três transportadores, a
escada do Palácio D...y era aquele valioso móvel Chippendale que certo
dia havia cedido lugar à mesa de jogo que tanta sorte lhe trouxera. O
armário era maravilhoso, e não podia ser confundido com nenhum outro.
Mas o amigo não o reconheceu apenas nisso. Balançando do mesmo jeito e
abalado na estrutura de seus ombros largos na época, à despedida, também
apareceram pela última vez e em seguida desapareceram as costas
formidáveis de seu proprietário diante dos que acenavam na plataforma da
estação. Às pressas, o desconhecido se acotovelou passando pelos
carregadores, e subindo os degraus baixos, entrou pela grande porta
aberta e ficou parado, quase sentindo vertigens, no gigantesco saguão
vazio. Diante dele se erguia, em espirais, uma escada que levava ao
primeiro andar, e sua rampa maciça não era mais do que um único relevo
em mármore sem fim: faunos, ninfas; ninfas, sátiros; sátiros, faunos. O
novato voltou a se controlar e investigou os corredores, as suítes dos
quartos. Por todo o lado, bocejavam paredes vazias ao encontro dele.
Nenhum rastro de moradores até o \emph{boudoir} também abandonado, mas
suntuoso, e tomado de peles e travesseiros, divindades em jade e
recipientes de incenso, vasos luxuosos e \emph{gobelins}. Uma leve
camada de pó encobria tudo. Aquela soleira nada tinha de convidativo, e
o estranho quis recomeçar a busca outra vez quando, por trás dele, uma
bela moça, ainda jovem, vestindo uniforme de aia, fez menção de adentrar
o ambiente. E ela, a única que tinha alguma familiaridade com o que ali
se passara, contou:

Fazia um ano que o barão havia alugado aquele palácio de seu
proprietário, um duque montenegrino, por uma soma inacreditavelmente
alta. Ainda no dia da assinatura do contrato, ela tivera de começar seus
afazeres, que por duas semanas consistiram em vigiar o pessoal de
serviço e receber entregadores. Em seguida, vieram novas instruções,
prescrições bem parcas, mas inflexíveis, cuja maior parte dizia respeito
ao cuidado com as flores que ainda haviam deixado um pouco de seu
perfume no quarto diante do qual os dois agora estavam parados. As
outras coisas diziam respeito apenas a uma das ordens, a última, e era
justamente esta que a moça acreditava estar vinculada ao pagamento
fabuloso, que lhe foi então prometido. ``Dia sim, dia não, nem um minuto
antes, nem um minuto depois das seis, aparecia'', prosseguiu ela, ``o
barão na escadaria, para subir até a grande porta vagarosamente. E
jamais ele vinha sem um buquê enorme nas mãos.'' Mas qual era a
sequência assumida por orquídeas, lírios, azaleias, crisântemos, e em
que relação se encontravam com a época do ano, isso não havia ficado
claro. Ele tocava a campainha. A porta se abria. A aia, justamente ela,
de quem ficamos sabendo tudo isso, abria para receber as flores e a
pergunta que era a palavra-chave para seu serviço mais secreto:

``A honorável senhora se encontra em casa?''

``Lamento'', respondia-lhe a aia, ``a honorável senhora deixou a casa há
pouco.''

Pensativo, o amante principiava o caminho de volta logo depois, para no
dia seguinte continuar sua espera no palácio abandonado.

Assim ficou-se sabendo como a riqueza, que por tantas vezes serve ao
objetivo ordinário de fomentar fulgores amorosos alheios, nessa única
vez levou os de seu proprietário às derradeiras chamas.

``\textsc{INSCRITO NA POEIRA MOVEDIÇA''}\footnote{``Dem Staub, dem
  beweglichen, eingezeichnet. Novelle'', in GS IV-2, pp. 780-787.
  Tradução de Georg Otte. {[}N. dos E.{]}}\footnote{Conto escrito
  provavelmente em 1929. {[}N. da O.{]}}\footnote{Verso do ``Divan
  ocidental-oriental'', de Johann Wolfgang von Goethe. {[}N. do T.{]}}

\textsc{Novela}

Lá estava ele. Sempre estava lá nessa hora. Mas não dessa maneira. Esse
homem imóvel, que costumava manter o olhar fixo ao longe, hoje estava
olhando para baixo. E mesmo assim, não parecia fazer diferença, pois,
nesse último caso, ele também não estava vendo nada. Mas a bengala de
mogno com seu punho de prata não se encontrava, como de costume, ao lado
dele encostada no banco. Ele a segurava, a conduzia, deixando que
deslizasse sobre a areia: ``O'' -- pensei numa fruta; ``L'' -- parei;
``I'' -- fiquei envergonhado como ao fazer algo proibido. Vi que não
escrevia isso como alguém que quisesse ser lido, mas os signos se
entrelaçavam e, como quisessem incorporar-se um ao outro, seguiram,
quase se sobrepondo aos outros, ``MPIA'', sendo que o primeiro começou a
desaparecer quando os últimos surgiram. Aproximei-me; isso tampouco o
fez levantar o olhar -- ou deveria dizer: ``acordar''? --, de tão
acostumado que ele estava comigo.

-- Fazendo cálculos de novo? Perguntei, fazendo-me de desentendido. Pois
eu sabia que seu ócio estava totalmente voltado para os custos
fantásticos de longas viagens que se estendiam de Samarcanda à Islândia,
que ele nunca faria. Será que ele jamais tinha deixado o país -- a não
ser naquela viagem secreta, claro, que tinha feito para fugir à
lembrança de um amor de juventude, um amor selvagem e, como não se
cansavam de assegurar, indigno e vergonhoso por Olímpia, cujo nome
acabou de desenhar em seu devaneio.

-- Estou pensando na minha rua. Ou em você, se prefere, pois acaba dando
na mesma. A rua em que uma palavra sua se tornou tão viva quanto nenhuma
outra que ouvi desde então, ou antes. Trata-se daquilo que você me falou
uma vez em Travemünde, isto é: que cada aventura de viagem, para se
poder mesmo contá-la, deve girar, em última instância, em torno de uma
mulher, ou, pelo menos, em torno do nome de uma mulher. Pois, querendo
ou não, esse seria o ponto de partida do qual careceria o fio condutor
do vivido para poder passar de uma mão à outra. Você estava certo, mas
quando estava subindo aquela rua calorenta, não tive ainda como imaginar
de que maneira estranha e por que, depois de alguns segundos, os meus
próprios passos nessa rua, que ecoava abandono, pareciam me chamar como
uma voz. As casas em volta tinham pouco a ver com aquelas que fizeram a
fama dessa cidadezinha do Sul da Itália. Sem ser suficientemente velha
para ser decaída, nem suficientemente nova para ser convidativa, era um
conjunto dos caprichos do limbo da arquitetura. Venezianas fechadas
reforçavam a mudez das fachadas cinzentas, e parecia que a glória do Sul
havia se retirado totalmente para as sombras, que se acumulavam por
debaixo das escoras anti-terremoto e dos arcos das ruelas laterais. Cada
passo me afastava mais de tudo que tinha sido o motivo da minha visita;
deixei a pinacoteca e a catedral para trás. Dificilmente, eu teria
encontrado força para mudar o rumo, mesmo se não tivesse sido matéria
para novos devaneios a visão de braços vermelhos de madeira, uma espécie
de suportes de candelabro que, como só agora percebo, cresciam dos muros
em ambos os lados a intervalos regulares. Digo ``matéria para
devaneios'' justamente porque não conseguia entender, nem procurava
explicar, como os restos de uma iluminação tão arcaica podiam ter
sobrevivido numa cidade nas montanhas, que, apesar de tudo, tinha
canalização e eletricidade. Por isso, também não me surpreendi quando,
alguns passos à frente, encontrei echarpes, cortinas, xales ou
passadeiras que, pelo visto, tinham acabado de lavar. Algumas lanternas
com papel amassado na frente de vidros turvos completavam a imagem,
nestas casas, de pobreza e administração decadente. Nesse momento,
gostaria de perguntar a alguém como voltar para o centro por outra via,
pois estava cansado dessa rua, também por estar tão vazia de gente. Foi
justamente por isso que tive que desistir da minha intenção e, quase
humilhado e subjugado, tive que retornar pelo mesmo caminho. Decidido a
não ter que arcar com o prejuízo do tempo perdido, mas também para pagar
por aquilo que considerava como uma derrota, abri mão do almoço e, o que
era mais amargo, do repouso, de modo que, depois de uma breve escalada
por escadas íngremes, encontrei-me na praça da catedral.

Se aquilo que, pouco tempo antes, me cercava era a ausência angustiante
de pessoas, agora era uma solidão que fez com que me sentisse livre. E,
com isso, o meu humor mudou por completo. Nessa hora, nada teria sido
pior para mim do que ser abordado ou apenas ser notado. De uma só vez,
fui devolvido ao meu destino de viajante, à aventura solitária, e de
novo vi na minha frente o momento quando, acima da Marina Grande, não
muito longe de Ravello, dei-me conta desse destino pela primeira vez e
de forma dolorosa. Desta vez também havia uma montanha em volta, mas no
lugar dos despenhadeiros rochosos que levam de Ravello ao mar, havia o
flanco de mármore da catedral e, no lugar dos declives cheios de neve,
inúmeros santos de pedra pareciam fazer sua romaria até nós homens.
Quando segui o cortejo com os olhos, vi que a fundação do edifício
estava à vista. Haviam cavado um corredor que, depois de vários degraus,
levava perpendicularmente a uma porta de bronze debaixo da terra, que
não estava trancada. Não sei por que passei clandestinamente por essa
porta lateral subterrânea; talvez fosse apenas o medo que nos acomete
quando visitamos pessoalmente um dos sítios mil vezes reproduzidos e
descritos, e que tentei evitar por esse desvio. Entretanto, se havia
achado que entraria no escuro de uma cripta, fui amplamente castigado
pelo meu esnobismo. Como se não bastasse o fato de ser esse espaço, na
verdade, a sacristia pintada de branco, cuja iluminação pelas janelas
superiores era ofuscante, além de tudo, era ocupado por um grupo de
turistas para o qual o sacristão estava contando, pela centésima ou
milésima vez, uma daquelas histórias em que ressoa o eco das moedas de
cobre que iria receber pela centésima ou milésima vez. Lá estava ele,
imponente e corpulento, ao lado do pedestal no qual se concentrava a
atenção dos ouvintes. Um capitel pelo visto muito antigo, porém muito
bem conservado, no estilo do gótico antigo, estava fixado nele com
grampos de ferro. Enquanto falava, segurava um lenço. Tudo levava a
supor que fosse por causa do calor, sendo que, de fato, o suor escorria
pela sua testa. Mas, longe de se enxugar, o sacristão apenas o passava
distraidamente no bloco de pedra, como uma empregada doméstica que, em
meio a uma conversa constrangedora com seu senhorio, entre uma palavra e
outra, passa, seguindo um velho hábito, o pano de limpeza pela estante
ou pelo console. A constituição autodestrutiva que cada pessoa que viaja
sozinha já sentiu, voltou a tomar conta de mim, de maneira que deixei
chegar aos meus ouvidos as explicações do sacristão.

``Há dois anos'', esse era o teor não exatamente literal das suas
informações, passadas pausadamente, ``havia entre os moradores um homem
que, mediante as formas mais ridículas de blasfêmia e lascívia, fez com
que a cidade fosse assunto de todas as conversas. Pelo resto de sua
vida, ele pagou pelo seu deslize e fazendo penitência ainda quando o
prejudicado, isto é, Deus, talvez o tivesse perdoado há muito tempo. Ele
era canteiro e, depois de trabalhar durante dez anos na manutenção da
catedral, graças ao seu talento, foi promovido a diretor de todo o
projeto de restauração. Era um homem na melhor idade, de natureza
imperiosa, sem família, nem parentes, quando caiu na rede da cocote mais
bonita e despudorada jamais vista na boemia do balneário vizinho. Talvez
a natureza terna e fechada desse homem tenha impressionado a mulher. De
qualquer forma, não se tem notícia de que ela tivesse prestado seus
favores a outra pessoa na região. Na época, ninguém desconfiava do preço
que ele teria que pagar, e toda a história nem teria vindo à tona se o
departamento de fiscalização da construção de Roma não tivesse vindo
inesperadamente para inspecionar a famosa obra de restauração. Entre os
superiores, havia um jovem arqueólogo, petulante, porém com bons
conhecimentos, que havia se especializado no estudo dos capitéis do
século XIV. Ele estava prestes a ampliar seus planos de escrever um
livro monumental a respeito de um estudo sobre `Um capitel do púlpito na
catedral de V...' e havia anunciado sua visita ao nosso chefe da
\emph{opera del duomo}. O mesmo, mais de dez anos depois de suas
melhores noites, levava uma vida na mais completa solidão, sendo que a
época do brilho e da autoafirmação havia acabado há muito tempo. No
entanto, o resultado que o jovem pesquisador levou para casa desse
encontro foi tudo menos um ensinamento da história do estilo, e sim uma
informação que não guardou para si. As autoridades acabaram tendo
conhecimento dos seguintes fatos: O amor que a cocote havia dedicado ao
seu galã não era obstáculo para ela, talvez antes um estímulo, para
exigir um preço satânico em troca dos seus favores. Ela queria ver seu
\emph{nom de guerre}, o nome comercial que essas mulheres usam de acordo
com a mais antiga tradição, gravado numa pedra da catedral, o mais
próximo do Santíssimo. O amante resistia, mas suas forças se esgotaram
e, num belo dia, ele começou, na presença da própria prostituta, o
trabalho naquele capitel do gótico antigo, que ficava por debaixo de um
capitel mais velho e corroído, até acabar como corpo de delito na mesa
dos seus juízes clericais. Todo esse processo, entretanto, levou muitos
anos e, quando todas as formalidades foram cumpridas e todas as atas
reunidas, era tarde demais. Quem estava olhando para sua obra era um
velho frágil e meio demente e ninguém acreditava em dissimulação quando
olhava para aquela cabeça que, em outros tempos, impunha respeito. Agora
ficava inclinada, com a testa enrugada, para o emaranhado de arabescos,
tentando em vão depreender dele o nome que, anos incontáveis antes, nele
havia escondido.''

Com surpresa observei como, eu mesmo não sei por que, aproximei-me do
capitel. Mas antes de poder estender a minha mão em direção à pedra,
senti a do sacristão no meu ombro. Com boa vontade e uma certa surpresa,
ele tentou entender os motivos do meu interesse. Eu, todavia, na minha
insegurança e no meu cansaço, balbuciei a coisa mais sem sentido que
pudesse ter dito: ``Colecionador''. E, nesse estado, voltei para casa.

Se o sono, como dizem, não é apenas uma necessidade física do organismo,
mas também uma força compulsiva que o inconsciente exerce no consciente
para que este saia de cena, cedendo o lugar às pulsões e às imagens, o
esgotamento que me assaltou talvez tivesse significado mais do que
normalmente significaria numa cidade nas montanhas do Sul da Itália ao
meio-dia. Seja como for, eu sonhei, sei que sonhei com o nome. Mas não
da maneira que ele estava gravado na pedra, escondido de mim, mas
conduzido para outro reino, ao mesmo tempo enaltecido, desencantado e
mais nítido, e no emaranhado variado de capins, folhas e flores, as
letras que, na época, causavam as batidas mais dolorosas do meu coração,
abanavam e tremiam na minha direção. Quando acordei, eram oito horas.
Hora de jantar e de me perguntar como iria passar o resto da noite. A
minha sesta de várias horas me proibia de terminar o dia cedo e não
tinha dinheiro, nem disposição, para gastá-lo com qualquer aventura.
Depois de dar alguns passos sem rumo, cheguei numa praça livre, o
\emph{Campo}. Já estava escurecendo. Algumas crianças ainda estavam
brincando em torno de uma fonte. Esse lugar, proibido para todos os
veículos, onde nunca mais haveria reuniões, apenas feiras, desempenhava
seu papel vivo como grande praça de banho e de jogos das crianças. Por
isso, ela era ao mesmo tempo o local preferido para carrinhos com doces,
nozes ou melancias, sendo que dois ou três deles estavam presentes,
acendendo aos poucos suas tochas. Um brilho destacou-se nas proximidades
do último, que havia atraído alguns ociosos e crianças. Ao chegar mais
perto, reconheci instrumentos de sopro. Sou um \emph{flâneur} atento.
Qual vontade ou qual desejo proibido haviam me impedido de reparar
aquilo que não escapava nem ao menos atento. Alguma coisa estava
acontecendo nessa rua, em cujo fim me encontrei novamente, sem
suspeitá-lo. Ao contrário do que achava, as passadeiras de seda que
estavam penduradas nas janelas não eram roupas estendidas para secar, e
por que logo aqui e em nenhum outro lugar do país, a velha iluminação
teria sido mantida? A banda de música começou a se movimentar, entrando
na rua que em pouco tempo se encheu de pessoas. E agora ficou claro que
a riqueza, quando chega perto dos pobres, apenas dificulta a fruição
daquilo que é seu: a luz das velas e o fogo das tochas travavam uma luta
feroz contra os feixes amarelos das lâmpadas elétricas que se projetavam
no pavimento e nas paredes das casas. Fui o último a sair atrás da
banda. Haviam preparado tudo para receber o cortejo na frente de uma
igreja. Aqui os lampiões e as lâmpadas incandescentes estavam mais
próximos do que em qualquer outro lugar, e da multidão em festa
desprendeu-se o fluxo ininterrupto dos devotos para perder-se nas dobras
da cortina que escondia a abertura do portão.

Parei a uma certa distância desse centro iluminado de vermelho e verde.
A multidão que agora preenchia a rua por completo, não era uma massa
incolor. Era a população bem delimitada e estreitamente interligada do
bairro e, uma vez que era um bairro da pequena burguesia, não se via
pessoas das classes mais altas, muito menos estrangeiros. Da maneira em
que fiquei parado junto ao muro, eu, pela roupa e pelo aspecto,
normalmente deveria ter chamado a atenção das pessoas. Mas nessa
multidão, curiosamente, ninguém olhava para mim. Será que ninguém
reparava ou será que este homem totalmente perdido nessa rua tomada pelo
calor e pelo canto, em que havia me transformado cada vez mais, parecia
pertencer a eles? Ao pensar nisso, enchi-me de orgulho; uma grande
felicidade tomou conta de mim. Não entrei na igreja e queria, satisfeito
de ter desfrutado a parte profana da festa, tomar o rumo de casa junto
com os primeiros saciados e muito antes das crianças que iriam cair de
sono, quando meu olhar esbarrou numa das placas de mármore com que as
cidades pobres dessa região envergonham as placas de rua do mundo
restante. A luz das tochas a inundava, parecia pegar fogo. Mas, bem
delineadas e ardentes, as letras saltaram do seu centro, formando
novamente o nome que, da pedra transformada em flor, da flor
transformada em fogo, procurava me pegar de forma cada vez mais quente e
devoradora. Tomando a decisão irrevogável de voltar para casa, iniciei o
meu retorno e fiquei feliz de encontrar uma ruela pequena que prometia
ser um atalho considerável. Em todas as partes, a vida já estava se
retirando e a rua principal, que pouco tempo atrás ainda estava cheia de
vida e onde tinha que estar o meu hotel, não me parecia ser apenas mais
sossegada, mas também mais estreita. Enquanto ainda refletia sobre as
leis que associam imagens acústicas e óticas, uma música de longe e alta
chocou-se contra o meu ouvido e, com os primeiros toques, fui atingido
pelo relâmpago da iluminação: aqui, então, acontece o grande evento. Por
isso havia tão poucas pessoas, cidadãos, naquela rua. Aqui iria
acontecer o grande concerto da tarde de V..., onde todo sábado os
moradores se reúnem. Uma nova cidade ampliada, até com uma história mais
rica e movimentada, estava de repente na frente dos meus olhos. Redobrei
os meus passos, virei uma esquina e, mais uma vez, parei, imobilizado de
estupor, na frente daquela rua que havia me atraído violentamente, como
puxado por um laço, eu, como ouvinte atrasado e solitário, para quem a
banda apresentou sua última música, a mais perdida de todas.''

Aqui o meu amigo interrompeu. De repente, parecia que sua história havia
fugido. E apenas os lábios que, ainda há pouco, estavam falando, me
acenaram com um longo sorriso. Eu, no entanto, olhava para os signos
que, diluídos na poeira, estavam inscritos aos nossos pés. E o indelével
verso passou majestosamente pelo arco dessa história como por um portão.

O SEGUNDO EU

\textsc{Uma história de final de ano para refletir}\footnote{``Das
  zweite ich. Eine Sylvestergeschichte zum Nachdenken'', in GS VII-1,
  pp. 296-298. Tradução de Marcelo Backes. {[}N. dos E.{]}}\footnote{Conto
  escrito entre 1930 e o início de 1933, aproximadamente; publicação
  póstuma. {[}N. da O.{]}}

Krambacher é um funcionário de bem baixo nível e além disso um homem
``sem reboques'', conforme ele garante às locadoras de seus quartos
mobiliados, que troca de cada quatro a seis semanas. Durante semanas,
ele refletiu onde poderia passar a noite de final de ano. Mas todos os
arranjos acabaram se desfazendo; com seu último dinheiro, ele arrumou
duas garrafas de ponche. A partir das 9 horas, principiou um banquete
solitário, sempre na esperança de que a campainha irá tocar, de que
alguém irá procurá-lo e lhe fazer companhia.

A esperança é desiludida. Pouco antes das 11, ele se prepara para sair:
está com medo da solidão em sua biboca. Nós seguimos seu passo um tanto
sinistramente animado pelas ruas noturnas. Percebe-se nele que bebeu.
Talvez ele nem sequer ande, talvez apenas sonhe que esteja andando. Essa
suposição pode surgir fugidiamente no leitor.

Krambacher vem por uma ruela bem fora de mão. Uma lâmpada sombria chama
sua atenção. Um lugar dúbio com movimento na noite de final de ano? Mas
por que tão silencioso? Ele se aproxima, não há rastro de que se trate
de um estabelecimento: com letras de madeira, apagadas, está escrito
sobre uma vitrine branqueada e sem transparência, da qual vem a luz
leitosa: PANORAMA IMPERIAL.

Ele quer passar sem parar, mas um bilhete sujo na vitrine o faz estacar:
Hoje! Apresentação de gala! \emph{Viagem pelo ano velho!} Krambacher
fica parado, abre a porta timidamente, toma coragem, uma vez que não
encontra ninguém, e entra. Ali está o panorama imperial. Agora ele é
descrito com suas 32 cadeiras em roda. Sobre uma dessas cadeiras o
proprietário, um italiano viúvo, Geronimo Cafarotti, dormindo. Quando o
cliente se aproxima, ele se levanta de um salto.

Grande torrente discursiva. De suas palavras, se pode ouvir que noite a
noite a casa ficava lotada e sem mais vagas; que hoje, coincidentemente,
estava pouco visitada, apesar da programação de gala; mas ele sabia que
alguém acabaria por vir: a pessoa certa. Enquanto obriga o visitante a
se sentar sobre uma banqueta diante de dois buracos pelos quais poderia
espiar, ele explica:

Aqui o senhor conhecerá alguém bem estranho, verá um homem que não tem
nenhuma semelhança com o senhor: seu segundo eu... O senhor passou a
noite fazendo autoacusações, tem complexos de inferioridade, se sente
tolhido, faz censuras a si mesmo por não seguir seus impulsos. Pois bem,
o que são esses impulsos? É a pressão do segundo eu no trinco da porta
que conduz para dentro de sua vida. E agora o senhor haverá de
reconhecer por que foi que sempre manteve esta porta tão trancada,
porque se tolheu tanto e não seguiu seus impulsos.

A viagem pelo ano velho começa. Doze imagens, para cada uma delas uma
pequena legenda; além disso as explicações do velho, que escorrega de
uma cadeira para outra. As imagens:

\emph{O caminho que pretendias tomar }

\emph{A carta que pretendias escrever }

\emph{O homem que pretendias salvar }

\emph{O lugar que pretendias ocupar }

\emph{A mulher que pretendias seguir }

\emph{A palavra que pretendias ouvir }

\emph{A porta que pretendias abrir }

\emph{O traje que pretendias usar }

\emph{A pergunta que pretendias fazer }

\emph{O quarto de hotel que pretendias ter }

\emph{O livro que pretendias ler }

\emph{A oportunidade que pretendias aproveitar }

Em algumas das imagens, o segundo eu pode ser visto, em outras apenas as
situações nas quais ele pretendia enredar o primeiro. As imagens são
descritas, como elas se livram do lugar em que estão com um pequeno
tilintar para permitir que a seguinte se aproxime, e como elas, mal
voltaram a se aquietar, tremendo, dão lugar a uma nova. O último tinido
é suplantado pelo reboar dos sinos do ano novo. Krambacher desperta com
o copo de ponche vazio nas mãos, sentado em sua cadeira.

RASTELLI CONTA...\footnote{``Rastelli erzählt\ldots{}'', in GS IV-2, pp.
  777-780. Tradução de Marcelo Backes. {[}N. dos E.{]}}\footnote{Conto
  escrito provavelmente entre setembro e outubro de 1935, publicado na
  \emph{Neue Zürcher Zeitung} em 6 de novembro de 1935. {[}N. da O.{]}}

Ouvi essa história de Rastelli, o malabarista inigualável, inesquecível,
que a contou certa noite em seu camarim.

Era uma vez, principiou ele, nos velhos tempos, um malabarista. Sua fama
se espraiara pelo globo terrestre, levada pelas caravanas e pelos navios
mercantes, e certo dia também Mohammed Ali Bei, que na época imperava
sobre os turcos, ouviu falar dele. E eis que enviou seus mensageiros aos
quatro cantos do mundo com a missão de convidar o mestre a vir a
Constantinopla, a fim de que ele pudesse se convencer em sua própria e
imperial pessoa das habilidades artísticas do homem. Mohammed Ali Bei
teria sido um príncipe imperioso e até mesmo cruel de quando em vez, e
dele se contava, inclusive, que, a seu aceno, um cantor que buscara seus
ouvidos mas não encontrara seus aplausos, havia sido jogado ao mais
profundo dos cárceres. Mas também sua generosidade era conhecida, e um
artista que o satisfazia podia contar com uma grande recompensa.

Depois de alguns meses, o mestre chegou à cidade de Constantinopla. Ele
não chegou sozinho, no entanto, ainda que não anunciasse seu
acompanhante em altos brados. E isso muito embora pudesse alcançar
honras especiais com ele na corte do sultão. Qualquer um sabe que os
déspotas do oriente têm um fraco por anões. O acompanhante do mestre era
justamente um anão, ou, mais exatamente, um criado anão. E um tão
excepcionalmente suave, uma criaturinha tão delicada e rápida, que com
certeza não teria encontrado outra igual na corte do sultão. O mestre
manteve esse anão escondido, e tinha seu bom motivo para tanto. É que
ele trabalhava de um modo um pouco diferente do de seus colegas. Estes,
conforme se sabe, frequentaram a escola chinesa, e lá aprenderam a lidar
com bastões e pratos, com espadas e fogos. Mas nosso mestre não buscava
sua honra no número e na variedade dos requisitos, e sim a mantinha com
um único número, que ainda por cima era o mais simples e que se
destacava pura e exclusivamente por sua grandiosidade incomum. E esse
número era com uma bola, uma única bola. Essa bola lhe trouxera sua fama
mundial, e de fato não havia nada que se comparasse com os milagres que
fazia com ela. Àqueles que acompanhavam a brincadeira do mestre chegava
a parecer que ele estava lidando com um ser vivo, ora dócil ora
renitente, ora suave ora zombeteiro, ora atento ora distraído, mas
jamais com uma coisa morta. Os dois pareciam habituados um ao outro, e
sequer pareciam conseguir viver separados tanto em tempos bons quanto em
tempos difíceis. E ninguém conhecia o segredo da bola. O anão, esse elfo
flexível, ficava sentado dentro dela. Depois de muitos anos de
exercício, ele soubera se adequar a cada um dos impulsos e a cada um dos
movimentos de seu senhor, e agora brincava com as molas localizadas no
interior da bola com tanta desenvoltura como se estivesse tocando as
cordas de uma viola. Para fugir a qualquer suspeita, os dois jamais se
deixavam ver um ao lado do outro, e senhor e ajudante também nunca
moravam sob o mesmo teto em suas viagens.

O dia ordenado pelo sultão chegara. Um estrado envolvido por cortinas
havia sido instalado na sala da meia-lua, lotada pelos dignitários do
soberano. O mestre fez uma mesura diante do trono e levou uma flauta aos
lábios. Depois de algumas melodias de prelúdio, ele passou a um
\emph{stacatto} em cujo ritmo a grande bola se aproximou aos saltos,
vinda dos bastidores. De repente, ela havia tomado lugar sobre os ombros
de seu dono, para em seguida não mais sair de perto dele. Ela brincava
em torno de seu senhor, adulava-o, acariciava-o. Este, porém, deixara
sua flauta de lado e, como se nada soubesse a respeito de seu visitante
peculiar, começara uma dança lenta que teria sido um prazer acompanhar
se a bola não tivesse cativado os olhos de todos. Assim como a Terra
gira em torno do Sol e ao mesmo tempo em torno de si mesma, também a
bola girava em torno do dançarino, sem esquecer nisso de sua própria
dança. Da cabeça aos pés, não havia lugar que a bola não tocasse, e cada
um desses lugares se tornava seu próprio parque de diversões ao passar
voando. Ninguém pensaria em pedir música para aquela ciranda muda. E
isso porque os dois davam as deixas um ao outro da maneira mais
harmônica: o mestre à bola e a bola ao mestre, conforme o pequeno
ajudante escondido já dominava com precisão depois de tantos anos de
exercício.

E assim continuou por muito tempo, até que em dado momento, em um giro
do dançarino, a bola, impulsionada para longe ao mesmo tempo, rolou ao
encontro da rampa na qual bateu e junto à qual ficou saltitando,
enquanto o mestre se compunha. Pois agora se aproximava o grande final.
O mestre voltou a pegar a flauta. Primeiro pareceu que ele quisesse
acompanhar com música baixa e cada vez mais baixa os saltos de sua bola,
que ficavam cada vez mais fracos. Mas então a flauta se fez dona da
situação. A respiração daquele que soprava se fez mais forte e, como se
soprasse nova vida à sua bola com o novo e mais vigoroso modo de tocar,
os saltos da mesma foram ficando aos poucos mais e mais altos, enquanto
o mestre começou a levantar o braço para, depois de alcançar
relaxadamente a altura do ombro, esticar o dedo mindinho sempre a tocar,
ao que a bola, obedecendo a um último e longo trinado, com um único
salto, se pôs imóvel no chão.

Sussurros de admiração percorreram as fileiras do público, e o sultão
convidou, ele mesmo, ao aplauso. O mestre, porém, concedeu uma
derradeira prova de sua arte ao aparar em pleno voo o saco pesado, cheio
de ducados, que lhe foi lançado por ordens vindas de cima.

Pouco depois ele saiu do palácio, para esperar por seu fiel anão em uma
saída distante. Foi então que um mensageiro apareceu diante dele, se
acotovelando entre os guardas. ``Procurei o senhor por toda a parte'',
disse ele, dirigindo-se ao mestre. ``Mas o senhor acabou deixando seu
alojamento antes da hora, e não permitiram que eu entrasse no palácio.''
Com essas palavras, ele apresentou uma carta que trazia a letra do anão.
``Meu caro mestre, peço que não se irrite comigo'', estava escrito nela.
``Mas hoje o senhor não poderá se mostrar diante do sultão. Estou doente
e não conseguirei deixar meu leito.''

Conforme o senhor pode ver, acrescentou Rastelli depois de uma pausa,
nossa casta não é de ontem e nós também temos nossa história -- ou pelo
menos nossas histórias...

POR QUE O ELEFANTE SE CHAMA ``ELEFANTE''\footnote{``Warum der Elefant
  `Elefant' heißt'', in GS VII-1, pp. 298-299. Tradução de Marcelo
  Backes. {[}N. dos E.{]}}\footnote{Conto não publicado durante a vida
  do autor. Provavelmente escrito em setembro de 1933, quando Benjamin
  retornava à Paris depois de uma temporada em Ibiza. {[}N. da O.{]}}

Era uma vez. Vivia por aí um homem que se chamava Elefante; mas na época
ainda nem sequer se conhecia o elefante como ele é hoje, isso foi há
vários milhares de anos. E, de repente -- todas as pessoas se admiraram
muito --, apareceu um animal por aí que nem sequer nome tinha, e o homem
o viu e, uma vez que tinha um nariz curto e era assim meio parecido com
um homem, o levou consigo e o animal ficou com ele.

E o animal estava com ele. Ele pegou um pedaço de madeira, não muito
longo, mas pesado, e o jogou para que o animal fosse buscá-lo. E uma vez
que o animal não tinha mãos com as quais pudesse pegar o pedaço de
madeira, tentou pegá-lo com o nariz.

Mas o nariz era curto demais, e isso deu muito trabalho ao animal. E uma
vez que o tentou por diversas vezes, repetindo o gesto seguidamente -- e
isso demorou um bocado! --, o nariz foi ficando cada vez mais longo e
mais longo com as tentativas.

Isso do nome acontecera já antes, quando o nariz ainda era curto. Pois
uma vez que o animal estava com o homem que se chamava Elefante, as
pessoas também o chamaram de elefante.

E agora o nariz já estava tão comprido que ele podia pegar o pedaço de
madeira com facilidade. E tudo ia bem, e o nariz ficava cada vez maior.
E hoje ele é tão grande e grosso e tem esse nariz-mão comprido -- sim,
este é justamente o nosso elefante. E esta é a história.

COMO O BARCO FOI INVENTADO E POR QUE ELE SE CHAMA BARCO\footnote{``Wie
  das Boot erfunden wurde und warum es Boot heißt'', in GS VII-1, p.
  299. Tradução de Marcelo Backes. {[}N. dos E.{]}}\footnote{Assim como
  ``Por que o elefante se chama `elefante''', este conto foi redigido
  provavelmente em setembro de 1933, quando Benjamin retornava de Ibiza
  à Paris. Não conheceu publicação em tempo de vida do autor. {[}N. da
  O.{]}}

Antes de todos os outros humanos vivia um que se chamava barco. Ele foi
o primeiro humano, pois antes dele existia apenas o anjo, que havia se
rebaixado e se transformado em um humano; e esta é uma outra história.

O homem barco quis, pois, ir para a água -- na época havia bem mais água
do que hoje, disso você precisa saber. Então ele atou tábuas ao redor de
seu corpo usando cordas, uma tábua comprida debaixo da barriga, e isso
era a quilha. E tomou um gorro pontudo de tábuas que ficava, quando ele
estava deitado na água, na parte da frente -- e isso se tornou a proa. E
atrás ele esticou uma perna e guiou com ela.

Assim ele se deitou na água e guiou e remou com os braços e traçou sua
rota com o gorro de tábuas, uma vez que era pontudo, com toda a
facilidade através da água. Sim, foi assim; o homem barco, o primeiro
homem, havia feito um barco de si mesmo, com o qual se podia navegar.

E por isso -- não é verdade? O que me parece até bem claro --, porque
ele próprio era o barco, chamou aquilo que estava fazendo de ``barco''.
E por isso o barco se chama ``barco''.

\textsc{UMA HISTÓRIA ESTRANHA, DE QUANDO AINDA NÃO HAVIA
HUMANOS}\footnote{``Eine komische Geschichte, als es noch keine Menschen
  gab'', in GS VII-1, p. 300. Tradução de Marcelo Backes. {[}N. dos
  E.{]}}\footnote{Como os dois contos imediatamente precedentes, este
  também teve publicação póstuma, e foi escrito provavelmente em 1933,
  no mês de setembro, quando Benjamin regressava à Paris desde Ibiza.
  {[}N. da O.{]}}

Na época a Terra ainda não era firme e tudo era um pântano, como massa
molhada. Existia apenas uma única árvore, que era gigantesca e sabia
correr -- é que as primeiras árvores sabiam correr como animais. A
árvore gigantesca saiu a passear, e de repente começou a correr,
justamente à beira do pântano mais profundo, e caiu com um catrapus
formidável dentro da água.

E no mesmo instante tudo ficou firme, a massa ficou bem dura, e por toda
parte na terra havia pedras em torrões e paus, de modo que o homem --
que ainda não existia -- não teria podido caminhar simplesmente, porque
teria se machucado demais.

Então o anjo se transformou pela primeira vez rebaixando-se, e tinha
asas de ferro e examinou a terra. E então o Deus mais uma vez borrifou
muita coisa molhada sobre a terra, de modo que tudo se tornou pântano e
oceano e mar de novo.

Mas tudo secou ao sol e então passou a estar liso em vários lugares. Mas
agora também havia montanhas -- porque o grande borrifar havia lavado a
areia e aberto sulcos e dobraduras --, ou seja, montanhas. Quando eu
borrifo, o resultado são apenas pequenos sulcos e mares, quando Deus
borrifa, aparecem montanhas.

E o anjo, que então caminhava ali por baixo, deixou suas asas derreterem
e logo elas sumiram, e o anjo era como um humano. Mas continuavam
havendo torrões sobre a terra -- uma gororoba de ovo, tudo colava.

Disso se fizeram os humanos -- e por primeiro o senhor que se chamava
barco. Eles se fizeram -- simplesmente passaram a ser, e o anjo, que
também se tornara humano, precisava apenas ficar olhando. Eles se
fizeram conforme seu aspecto.

Então os homens construíram molhes e botaram muitos monumentos e humanos
de ferro de asas bem abertas sobre eles. Mas isso foi bem mais tarde,
pouco tempo antes de inventarem as lâmpadas.

MYSLOWITZ -- BRAUNSCHWEIG -- MARSELHA

\textsc{história de uma embriaguez com haxixe}\footnote{``Myslowitz -
  Braunschweig -- Marseille. Die Geschichte eines Haschisch-Rausches'',
  in GS IV-2, pp. 729-737. Tradução de Georg Otte. {[}N. dos E.{]}}\footnote{No
  final dos anos 20, Benjamin realizou, com a ajuda de um amigo médico,
  uma série de experiências de absorção de haxixe. As associações e
  fantasias que surgiam sob o efeito da droga eram anotadas em
  ``protocolos'' que foram, por vezes, utilizados como material para a
  redação de textos teóricos ou ficcionais. Este conto está relacionado
  ao protocolo de 29 de setembro de 1928, que também serviu para a
  redação de \emph{Haxixe em Marselha}. Publicado na revista \emph{UHU,}
  n.° 7, caderno 2, em novembro de 1930. {[}N. da O.{]}}

Esta história não é minha. Não quero me deter na questão de saber se o
pintor Eduard Scherlinger, que vi pela primeira e última vez naquela
noite, quando a contou, era um grande contador de histórias ou não,
porque, nesta época dos plágios, sempre há alguns ouvintes que nos
atribuem uma história mesmo quando explicamos que ela apenas foi
relatada de forma fidedigna. Mas eu a escutei numa noite em um dos
poucos lugares que há em Berlim para contar e escutar histórias, na casa
Lutter \& Wegener. Era agradável sentar em torno da mesa no nosso
pequeno grupo, mas as conversas já haviam se dispersado e só ressurgiam
de modo escasso e abafado entre duas ou três pessoas, sem serem notadas
pelos demais.

De repente, numa situação cujas circunstâncias eu nunca mais
vivenciaria, meu amigo, o filósofo Ernst Bloch, saiu com a frase de que
não haveria ninguém que, em algum momento de sua vida, não tivesse
chegado a um fio de cabelo da oportunidade de se transformar em
milionário. Todo mundo riu. Achamos que fosse um dos seus paradoxos. Mas
aí aconteceu uma coisa curiosa: quanto mais tempo dedicamos a essa
afirmação, tanto mais interesse tivemos em debatê-la, para ver como um
depois do outro se tornou pensativo quando se lembrou do momento de sua
vida em que quase tocou nesse milhão. Dentre várias histórias notáveis
que vieram à tona encontra-se, portanto, a de Scherlinger, hoje
desaparecido. Na medida do possível, vou reproduzi-la com suas próprias
palavras.

``Quando, após a morte do meu pai,'' ele começou, ``herdei um patrimônio
nada desprezível, precipitei a minha partida para a França. Fiquei feliz
principalmente pelo fato de conhecer, ainda antes do fim dos anos 20, a
cidade natal de Monticelli, a quem devo tudo na minha arte, sem falar de
outras coisas que Marselha representava para mim. Deixei a minha herança
no pequeno banco particular que, durante décadas, havia atendido meu pai
a contento. O chefe júnior não chegou a ser meu amigo, mas eu tinha um
excelente contato com ele. Assim, o mesmo garantiu que, durante a minha
longa ausência, daria uma atenção especial ao meu patrimônio e que me
informaria imediatamente sobre qualquer possibilidade favorável de
aplicação.

-- Você apenas teria que deixar um código conosco, ele concluiu.

Olhei para ele sem entender.

-- Nós só podemos executar ordens telegráficas se nos protegermos contra
qualquer abuso. Suponha que lhe mandemos uma mensagem telegráfica e o
telegrama caia nas mãos de outra pessoa. Protegemo-nos contra as
consequências combinando com você um nome secreto que sirva de
assinatura de suas ordens telegráficas.

Entendi e fiquei perplexo por um momento. Não é mesmo muito fácil
disfarçar-se com um nome falso como se fosse uma peça de roupa. Há
milhares e milhares de opções; a ideia de o nome ser totalmente
indiferente paralisa a escolha, que se torna ainda mais estática por uma
sensação -- escondida e que mal se tornou um pensamento: quão
incalculável tal escolha, e quão plena de graves consequências. Como um
jogador de xadrez, que se encontra num beco sem saída e que gostaria
muito de deixar tudo como está, mas acaba sendo obrigado a mover uma
peça, acabei dizendo `Braunschweiger'. Não conhecia ninguém com esse
nome -- aliás, nem conhecia a cidade que deu origem a ele.

Depois de um descanso de quatro semanas em Paris, cheguei num dia muito
abafado de julho na Gare Saint Louis em Marselha. Meus amigos haviam me
recomendado o hotel Regina, não muito longe do porto. Gastei apenas o
tempo necessário para me instalar, para testar o abajur da mesa de
cabeceira e as torneiras para depois dar um passeio. Como era o meu
primeiro nesta cidade, esse passeio tinha que obedecer à minha velha
regra de viajante, que consistia em explorar antes de mais nada a
periferia em sua extensão, ao contrário da maioria dos passantes que,
mal chegam, já se comprimem desajeitadamente no centro da cidade
estrangeira. Logo ficou claro para mim o quanto justamente aqui essa
regra era válida. Nunca a primeira hora trouxe-me mais ganhos do que
essa entre os portos internos e as docas, os armazéns, os bairros da
pobreza e os redutos dispersos da miséria. Sabemos que a periferia é o
estado de exceção da cidade, o terreno em que se trava ininterruptamente
a grande batalha decisiva entre a cidade e o campo. Em lugar algum ela é
mais acirrada do que entre Marselha e a paisagem da Provença. É a luta
corpo-a-corpo de postes de telefones contra agaves, arame farpado contra
palmeiras espinhosas, correntes de neblina em vielas mal cheirosas
contra a sombra úmida dos plátanos em campos ensolarados, escadarias de
pouco fôlego contra colinas imponentes. A longa Rue de Lyon é o caminho
de pólvora que Marselha cavou na paisagem para fazê-la explodir em
Saint-Lazare, Saint-Antoine, Arenc, Septèmes e cobrir com estilhaços de
granada de todas as línguas dos povos e das empresas: \emph{Alimentation
Moderne}, \emph{Rue de Jamaïque}, \emph{Comptoir de la Limite},
\emph{Savon Abat-Jour}, \emph{Minoterie de la Campagne}, \emph{Bar du
Gaz}, \emph{Bar Facultatif}. E por cima de tudo a poeira densa, composta
de sal marinho, calcário e minerais brilhantes. Depois caminhei em
direção ao cais mais externo, usado apenas pelos grandíssimos vapores
ultramarinos, sob os raios ardidos de um sol forte que estava se pondo.
Andei entre as fundações muradas da cidade velha do lado esquerdo e os
morros ou as pedreiras sem vegetação do lado direito, ao \emph{Pont
Transbordeur}, esse polígono quadrado que os fenícios isolaram do mar
como uma praça grande e que se ergue no final do porto velho. Se, mesmo
nos subúrbios mais populosos, eu havia seguido meu caminho sozinho, aqui
me sentia integrado no cortejo de marinheiros que se divertiam, de
estivadores voltando para casa e donas de casa passeando. Cortejo este
que, repleto de crianças, se movia ao longo dos bares e dos bazares,
para depois se perder nas ruas laterais, sendo que a grande artéria
principal, a avenida do comércio, da bolsa e dos estrangeiros, \emph{La
Cannebière}, só foi alcançada por alguns marujos ou \emph{flâneurs} como
eu. Através de todos os bazares, de um lado do porto até o outro, se
estende a serra dos `Souvenirs'. Forças sísmicas empilharam essa
elevação de vidros, calcário e esmalte, na qual frascos de tinta, barcos
a vapor, âncoras, colunas de mercúrio e sereias são imbricados. Para
mim, todavia, a pressão de mil atmosferas sob a qual todo esse mundo de
imagens se comprime, ergue-se e se escalona parecia ser a mesma força
que se experimenta nas mãos duras dos marinheiros quando, depois de uma
longa jornada, tocam coxas e peitos de mulheres. E a volúpia que, nas
caixinhas de conchas faz surgir do mundo de pedras um coração de veludo
vermelho ou azul para ser espetado de agulhas e broches, a mesma que,
nos dias do pagamento, sacode as ruelas.

Mergulhado nesses pensamentos eu havia, há muito tempo, deixado para
trás a \emph{Cannebière}. Sem prestar muita atenção, tinha passado por
baixo das árvores da \emph{Allée de Meilhan}, e das grades das janelas
do \emph{Cours Puget}. Isso, até que o acaso, que continuou guiando os
meus primeiros passos na cidade, levou-me à \emph{Passage de Lorette}, a
sala mortuária da cidade, no pátio estreito, onde, na companhia
sonolenta de alguns homens e mulheres, o mundo inteiro parecia ter se
encolhido numa única tarde de domingo. Algo do luto que, até hoje, amo
na luz das pinturas de Monticelli, tomou conta de mim. Creio que, nessas
horas, quando vivenciadas por um estrangeiro, alguma coisa se comunica a
ele que, normalmente, só os nativos sentem. Pois a infância é o detector
das fontes da melancolia e, para conhecer o luto de cidades tão
gloriosas, a pessoa tem que ter sido criança nelas.

Seria um belo arranjo romântico'', disse Scherlinger com um sorriso,
``se agora relatasse como, num bar mal afamado qualquer, consegui o
haxixe através de um árabe, que poderia ter sido fogueiro num cargueiro
ou também carregador. Mas esse arranjo não tem nenhuma utilidade para
mim, pois, talvez, eu me assemelhasse mais a esses árabes do que aos
estrangeiros cujos caminhos conduzem a esses bares, pelo menos no
sentido de que eu também levo comigo haxixe nas minhas viagens. Não
penso que, depois, no meu quarto, tivesse sido o desejo subalterno de
escapar à minha tristeza que fez com que, por volta das sete horas da
noite, eu fumasse o haxixe. Era antes a tentativa de me subjugar por
inteiro à mão mágica da cidade que havia me tomado silenciosamente pelo
pescoço. Como já disse, o entorpecente não era novidade para mim, mas,
seja pelas minhas depressões quase cotidianas em casa, seja a companhia
escassa ou ainda os locais inadequados, nunca havia me sentido acolhido
na comunidade dos iniciados, cujos testemunhos todos eram-me familiares,
dos \emph{Paraísos artificiais} de Baudelaire até o \emph{Lobo da
Estepe} de Hermann Hesse. Deitei na minha cama, comecei a ler e a fumar.
A janela à minha frente dava, logo abaixo, para uma das ruas escuras e
estreitas do bairro portuário, que são como a linha de corte de uma faca
no corpo da cidade. Apreciei a certeza incondicional de permanecer
escondido de centenas de milhares de pessoas dessa cidade, onde ninguém
me conhecia, sem ser incomodado e totalmente entregue aos meus
devaneios.

Mas o haxixe demorou a fazer efeito. Já haviam passado 45 minutos e
comecei a desconfiar da qualidade da droga. Ou será que a havia guardado
tempo demais comigo? De repente, alguém bateu com força na minha porta.
Nada era mais inexplicável. Levei um susto mortal, mas não tive
disposição nenhuma de abrir a porta; apenas perguntei o que era, sem
alterar minimamente a minha posição.

-- Há um senhor que quer lhe falar, disse o empregado do hotel.

-- Mande-o subir, eu disse. Não tive a presença de espírito, ou então, a
coragem de perguntar pelo nome.

Com o coração batendo, fiquei apoiado no encosto da cama, olhando fixo
para a fresta aberta da porta, até um uniforme aparecer nela. `O senhor'
era um mensageiro do telégrafo.

-- Proposta comprar 1000 Royal Dutch sexta-feira primeiro câmbio mande
de acordo.

Olhei para o relógio: eram oito horas. Um telegrama em regime de
urgência teria como chegar na manhã do dia seguinte ao escritório
berlinense do meu banco. Despedi o mensageiro com uma gorjeta. Fiquei
alternando entre a inquietação e o desprazer. Inquietação sobre o fato
de ser molestado logo naquele momento por um negócio bancário e um
caminho a ser feito; e desprazer sobre a demora em sentir qualquer
efeito da droga. Pareceu-me ser o mais indicado pegar logo o caminho do
correio central, que, pelas minhas informações, aceitava telegramas até
meia-noite. Diante da segurança com que meu conselheiro de confiança me
atendia, não havia dúvida de que eu concordaria. Entretanto, fiquei
preocupado com a ideia de que pudesse esquecer o código combinado caso o
haxixe, inesperadamente, começasse a surtir efeito. Sendo assim, era
melhor não perder tempo.

Enquanto descia a escada, lembrava da última vez em que havia fumado
haxixe -- já fazia alguns meses -- e que não tinha como matar a fome
atormentadora que havia me assaltado no quarto. Comprar uma barra de
chocolate era o mais aconselhável. De longe, uma vitrine com vidros de
balas, papel brilhante de alumínio e belos confeitos empilhados me
chamou a atenção. Entrei na loja e fiquei parado. Não havia ninguém. Mas
isso era menos surpreendente do que as poltronas bastante estranhas,
cujo aspecto me deu a entender, pelo bem ou pelo mal, de que, em
Marselha, se toma chocolate em poltronas altas como tronos que se
pareciam mais com cadeiras cirúrgicas. Nesse momento, o dono da loja,
vestido com um avental branco, veio correndo do outro lado da rua e mal
tive tempo, dando altas risadas, para escapar de sua oferta de fazer a
minha barba ou de cortar meu cabelo. Só então ficou claro para mim que o
haxixe já tinha começado a fazer efeito e, se a transformação de
caixinhas de pó de arroz em vidros de balas, estojos de níquel em barras
de chocolate e perucas em rocamboles não me tivessem mostrado isso, as
minhas próprias risadas teriam sido alerta suficiente. Pois o
entorpecimento começa com essas risadas ou então com um riso mais
silencioso e interno, mas ainda mais prazeroso. E agora o reconheci
também pela ternura infinita do vento que movia, no outro lado da rua,
as franjas das marquises.

Logo depois se fizeram sentir as exigências de tempo e espaço do fumante
de haxixe. Como se sabe, tomam dimensões absolutamente reais. Para quem
consome haxixe, o palácio de Versalhes não é grande demais e a
eternidade não é demasiadamente longa. E diante dessas dimensões
gigantescas da vivência interior, da duração absoluta e do espaço
incomensurável, um humor maravilhoso acompanha, junto com aquele sorriso
feliz, tanto mais prazerosamente o caráter questionável de todo ser.
Além disso, senti uma leveza e uma determinação no passo que transformou
o chão de terra irregular e cheio de pedras da grande praça, que estava
atravessando, numa estrada mestra pela qual eu, caminhante valente,
passei durante a noite. No final dessa praça, no entanto, ergueu-se uma
construção feia na forma de um galpão simétrico, com um relógio
iluminado no frontão: o correio.

Só agora chamo essa construção de feia; na época, não teria achado isso.
Não apenas porque nada é feio quando consumimos haxixe, mas sobretudo
porque o correio desperta em mim uma sensação profunda de gratidão --
esse correio escuro que estava aguardando, me aguardando, que, em todos
os seus compartimentos e suas cavidades, estava disposto a acolher e
passar para frente a inestimável concordância que faria de mim um homem
rico. Não conseguia tirar dele meu olhar e até sentia o quanto teria
perdido se tivesse me aproximado demasiadamente, deixando de ver o todo,
sobretudo o relógio-lua brilhante.

Nesse momento, bem no lugar certo, mesas e cadeiras de um pequeno bar e,
nesse caso, com certeza mal afamado, foram colocadas na escuridão.
Embora suficientemente distante do bairro dos apaches, não havia
fregueses burgueses; na melhor das hipóteses havia, ao lado do
proletariado portuário propriamente dito, algumas famílias proprietárias
de boutiques na vizinhança. Foi nesse pequeno bar que tomei lugar.
Naquela direção, era o último que, na minha opinião, ainda era acessível
sem perigo, e que, sob o efeito do haxixe, havia avaliado com a mesma
segurança com que se consegue, em estado de profundo cansaço, encher um
copo de água até a beirada sem derramar uma única gota -- algo que não
se consegue em hipótese alguma com os sentidos despertos.

Mas, mal o haxixe sentiu meu descanso, ele começou a soltar sua magia
com uma agudeza primitiva que não havia sentido antes, nem iria mais
sentir depois disso. Pois ele fez com que me tornasse um
fisiognomonista. Logo eu, que normalmente não estou em condições de
reconhecer conhecidos distantes, de guardar traços faciais na memória,
fiquei obstinadamente atento aos rostos ao meu redor e que teria evitado
normalmente por dois motivos: não teria desejado atrair seus olhares,
nem teria suportado sua brutalidade. Entendi de repente como, para um
pintor -- não havia acontecido isso com Leonardo e muitos outros? -- a
fealdade era o verdadeiro reservatório da beleza, ou melhor, o cofre do
seu tesouro, a montanha escarpada que esconde todo o ouro da beleza que
fulgurava das rugas, dos olhares, dos traços. Lembro-me especialmente do
rosto de um homem infinitamente animalesco e vulgar, do qual me atingiu
repentinamente a ``dobra da renúncia'' de forma assustadora. Eram
principalmente os rostos dos homens que me tocavam. Começou também o
velho jogo de identificar em cada face nova algum conhecido. Muitas
vezes eu lembrava do nome, outras vezes não; a ilusão passou, como
passam as ilusões no sonho, isto é, não de forma envergonhada ou
comprometida, mas pacífica e amável, como um ser que cumpriu sua
obrigação. Meu vizinho, porém, um pequeno burguês pela sua postura,
mudou constantemente a forma, a expressão e a plenitude do seu rosto. O
corte de cabelo e os óculos de armação preta o deixaram ora severo, ora
bondoso. Disse para mim mesmo que não é possível alguém mudar com tanta
rapidez, mas não adiantou. E ele já havia passado por muitas vidas
quando, de repente, transformou-se num aluno de ginásio numa cidade
pequena da Europa oriental. Ele tinha um quarto de estudos bonito e
cultivado. Perguntei para mim mesmo: `de onde esse jovem tirou tanta
cultura? Qual será a profissão do seu pai? Comerciante de têxteis ou de
cereais?' De repente eu soube que se tratava de Myslowitz. Levantei o
olhar e, nesse momento, vi realmente, no final da praça, não, mais longe
ainda, no final da cidade, o ginásio de Myslowitz e o relógio da escola
que -- será que parou? Ele não tinha avançado -- mostrava que era pouco
depois das onze. As aulas já deviam ter começado. Mergulhei inteiramente
nesse quadro, não alcançava mais o fundo. As pessoas que, pouco tempo
atrás -- ou será que foram duas horas atrás -- haviam-me encantado
totalmente, estavam como que apagadas. `De um século para outro, as
coisas se tornam mais estranhas', pensei. Hesitei muito em tomar o
vinho. Era uma meia garrafa de Cassis, um vinho seco que havia pedido.
Um pedaço de gelo boiava no copo. Não sei quanto tempo fiquei com as
imagens que o povoavam. Mas quando olhava para a praça vi que ela tendia
a mudar com cada pessoa que chegava, como se essa pessoa representasse
uma figura que, bem entendido, não tinha nada a ver com aquela que a
olhava, mas antes com o olhar com que os grandes retratistas do século
XVII colocam, de acordo com o caráter das pessoas de nobreza, essas
pessoas numa galeria de colunas ou na frente de uma janela, para
destacá-las dessa galeria ou dessa janela.

De repente, um susto me despertou do mais profundo devaneio. Tudo estava
claro em mim e só pensei numa coisa: o telegrama. Tinha que mandá-lo
imediatamente. Para me manter completamente acordado, pedi um café puro.
Aí começou uma espera de meia eternidade, até o garçom aparecer com a
xícara. Ávido pelo café, peguei-o, o odor subiu pelo meu nariz. No
entanto, quando estava próximo dos meus lábios, a minha mão parou --
para a minha própria surpresa ou de tanta surpresa, quem saberia
dizê-lo? De uma só vez entendi claramente a pressa instintiva do meu
braço e me dei conta do odor envolvente do café, só agora lembrei o que
transforma essa bebida, para qualquer fumante de haxixe no auge da
fruição: nada mais do que o fato de aumentar o efeito da droga. Por isso
quis parar, e acabei parando. A xícara não encostou na boca. Mas também
não tocou a superfície da mesa. Ela ficou flutuando na minha frente,
sustentada pelo meu braço que começou a ficar insensível, segurando-a
como um emblema, uma pedra sagrada ou um osso, rígido e morto. Meu olhar
caiu nas dobras geradas por minha calça de praia, a reconheci, dobras do
albornoz. Caiu na minha mão, a reconheci, uma mão bronzeada, etíope, e
enquanto os meus lábios continuavam rigorosamente colados um ao outro,
recusando-se igualmente à bebida e à palavra, do seu interior surgia um
sorriso, um sorriso arrogante, africano, sardanapalesco, o sorriso do
homem que está prestes a entender tudo que está por trás do curso do
mundo e dos destinos e para quem não há mais nenhum segredo nas coisas e
nos nomes. Vi a mim mesmo sentado lá, bronzeado (\emph{braun}) e em
silêncio (\emph{schweigend}). Braunschweiger. O Sésamo do nome, que
escondia no seu interior todas as riquezas, havia se aberto. Com um
sorriso de compaixão infinita tive que pensar, pela primeira vez, nos
moradores da cidade de Braunschweig, que vegetam humildemente em sua
cidadezinha da Alemanha central, sem saber das forças mágicas que
receberam junto com seu nome. Nesse momento ocorreram-me, como um coral,
solene e enfaticamente com suas badaladas da meia noite, todas as torres
das igrejas de Marselha.

Escureceu e o bar foi fechado. Fiquei vagueando pelo cais, lendo um nome
depois do outro dos barcos que lá estavam atracados. Nisso, uma alegria
incompreensível tomou conta de mim e fiquei sorrindo para o rosto de
todos os nomes de meninas da França. Marguerite, Louise, Renée, Yvonne,
Lucile -- pareceu-me que o amor que havia sido prometido a esses barcos
mediante seus nomes era maravilhoso, bonito e comovente. Ao lado do
último barco havia um banco de pedra -- `banco', disse para mim mesmo e
desaprovei o fato de ele não ter seu nome escrito em letras douradas em
fundo preto. Este foi o último pensamento claro que tive naquela noite.
O próximo foi-me trazido pelos jornais da manhã, quando acordei num
banco do cais ao sol quente do meio-dia:

`Alta sensacional da \emph{Royal Dutch}.'

Nunca'', encerrou o narrador, ``me senti tão ressoante, claro e solene
depois de uma embriaguez.''

\textbf{O CONTADOR DE HISTÓRIAS NO RÁDIO}

NO MINUTO EXATO\footnote{``Auf die Minute'', in GS IV-2, pp. 761-763.
  Tradução de Marcelo Backes. {[}N. dos E.{]}}\footnote{Texto publicado
  no \emph{Frankfurter Zeitung} em 6 de dezembro de 1934 com o
  pseudônimo de Detlef Holz. {[}N. da O.{]}}

Depois de meses tentando, acabei recebendo da direção de programação da
D... o encargo de entreter os ouvintes durante vinte minutos com um
relato sobre a minha especialidade, a atividade livreira. Caso meu papo
encontrasse eco, acenaram com a possibilidade de repetir regularmente os
referidos relatos. O chefe de departamento se mostrou amável o
suficiente para chamar minha atenção ao fato de que eram decisivos, além
da estrutura de tais considerações, o modo como eu as apresentaria.
``Principiantes'', disse ele, ``cometem o erro de acreditar que têm de
fazer sua apresentação diante de um público mais ou menos grande, que
apenas por acaso não é visível. Nada é mais errado do que isso. O
ouvinte de rádio é quase sempre um solitário, e mesmo que o senhor
excepcionalmente alcance alguns milhares, serão apenas milhares de
ouvintes solitários. O senhor tem de se comportar, portanto, como se
falasse com uma única pessoa -- ou com várias pessoas isoladas, se
preferir; mas de modo algum a muitas pessoas reunidas. Essa é a primeira
coisa. E há ainda uma segunda: o senhor deve se manter rigorosamente no
tempo previsto. Se não o fizer, nós precisaremos fazê-lo em seu lugar, e
o faremos simplesmente desligando os microfones sem qualquer piedade.
Qualquer atraso, até mesmo o mais mínimo, tem, como sabemos por
experiência, a tendência de se multiplicar no decorrer da programação.
Caso nossa intervenção não ocorra no momento exato, toda a nossa
programação acaba fugindo ao controle... Portanto, peço ao senhor que
não se esqueça: apresentação descontraída! E terminar no minuto exato!''

Eu dei toda a atenção a essas orientações e as segui com cuidado; para
mim, muita coisa também dependia da gravação desse meu primeiro
programa. Em casa eu havia lido em voz alta o manuscrito, com o qual me
apresentei na estação à hora prevista, controlando com o relógio o tempo
que demorava. O locutor que me anunciaria recebeu-me com distinção, e eu
por certo poderia tomar como um sinal particular de sua confiança, o
fato de ele ter aberto mão de supervisionar minha estreia de uma cabine
incômoda. De seu anúncio até a despedida, eu era senhor de mim mesmo.
Pela primeira vez eu me encontrava em uma sala de transmissões moderna,
onde tudo está disposto para o mais absoluto conforto do locutor e o
desenvolvimento tranquilo de suas capacidades. Ele pode falar em pé, de
um púlpito, ou então sentar em uma das poltronas espaçosas, tem à sua
escolha as mais diferentes fontes de luz, pode até caminhar de um lado a
outro e levar consigo o microfone. Por fim, um relógio de mesa, cujo
mostrador não marca horas, mas apenas minutos, deixa claro ao locutor
quanto vale o instante naquela câmara vedada. Quando o mostrador
estivesse sobre o quarenta, eu tinha de estar pronto.

Eu lera pouco mais da metade do meu manuscrito quando voltei outra vez
os olhos para o relógio, no qual o ponteiro dos segundos descrevia o
mesmo círculo prescrito ao ponteiro dos minutos, mas com uma velocidade
sessenta vezes maior. Por acaso eu cometera um erro de direção quando
treinara em casa? Ou errara no tempo agora? Uma coisa era certa, dois
terços do meu tempo haviam se passado. Enquanto eu continuava lendo
palavra a palavra do meu discurso com o tom compromissado de antes,
procurava, febril e em silêncio, por uma saída. Só uma decisão ousada
poderia ajudar, parágrafos inteiros tinham de ser sacrificados,
considerações que levassem logo ao final precisavam ser improvisadas em
seu lugar. Afastar-me do meu texto não era desprovido de perigo. Mas não
me restava alternativa. Reuni todas as minhas forças, virei várias
páginas do manuscrito enquanto estendia uma frase mais longa e, por fim,
cheguei, feliz como um aviador em seu campo de voo, ao círculo de ideias
reunido no parágrafo final. Suspirando de alívio, reuni logo depois meus
papéis e, no êxtase da conquista, me afastei do púlpito para vestir meu
sobretudo.

Eis que agora o locutor que me anunciara deveria entrar. Mas ele se fez
esperar, e eu me virei para a porta. Nisso, meu olhar caiu outra vez
sobre o relógio. O ponteiro dos minutos mostrava o trinta e seis!...
Ainda faltavam quatro minutos inteiros até o quarenta! O que eu antes
vislumbrara às pressas devia ter sido o \emph{ponteiro dos segundos}!
Agora eu também entendia o fato de o locutor não chegar. No mesmo
instante, porém, o silêncio que ainda há pouco me parecera tão benfazejo
me envolveu como se fosse uma rede. Naquela câmara, cujo funcionamento
era determinado pela técnica e pelo homem que imperava através dela, um
novo horror se estendeu sobre mim, que no entanto era aparentado daquele
mais antigo que nós conhecemos. Eu emprestei a mim mesmo meu ouvido, e
ao encontro dele ecoou de repente nada mais do que o próprio silêncio.
Mas este eu reconheci como sendo o silêncio da morte, que me derrubava
avassaladoramente naquele exato instante em milhares de ouvidos e
milhares de lares ao mesmo tempo.

Um medo indescritível tomou conta de mim, e logo em seguida uma
determinação selvagem. Salvar o que ainda poderia ser salvo, disse
comigo mesmo, e arranquei o manuscrito do bolso do sobretudo, selecionei
a primeira e melhor que encontrei entre as páginas que havia pulado e
recomecei a leitura com uma voz que para mim parecia ser sobrepujada
pelas batidas do coração. Era impossível exigir ideias de mim. E, uma
vez que o trecho do texto que eu havia encontrado era breve, estendi as
sílabas, fiz as vogais vibrarem suas asas, rolei os erres e incluí entre
as frases pausas repletas de reflexão. E assim cheguei ao final mais uma
vez -- desta vez o final correto. O locutor chegou e me dispensou,
diligente, como antes havia me recebido. Minha inquietude continuava, no
entanto. Quando por fim, no dia seguinte, encontrei um amigo do qual eu
sabia que me ouvira, perguntei de passagem sobre a sua impressão. ``Foi
muito bom'', disse ele. ``O problema é que os receptores sempre nos
deixam um pouco na mão. O meu mais uma vez simplesmente ficou fora do ar
durante um minuto.''

\textsc{CASPAR HAUSER}\footnote{``Caspar Hauser'', in GS VII-1, pp.
  174-180. Tradução de Georg Otte. {[}N. dos E.{]}}\footnote{Escrita
  para um público infanto-juvenil, essa narrativa radiofônica foi
  difundida em 22 de novembro e 17 de dezembro de 1930, tendo o próprio
  Benjamin como locutor. {[}N. da O.{]}}

Hoje, para variar, vou contar para vocês simplesmente uma história. Três
coisas vou adiantar logo: primeiro, cada palavra dela é verdadeira;
segundo, ela é intrigante tanto para adultos quanto para crianças, e as
crianças a entendem tão bem quanto os adultos; terceiro, mesmo se o
personagem principal morre ao final dessa história, ela não tem um
verdadeiro fim. Em compensação, ela tem a vantagem de não ter acabado, e
que, talvez, nós todos saberemos seu fim um dia.

Se eu começar a contá-la agora, vocês não devem pensar: mas isso está
começando como qualquer outra história ilustrada para jovens. Quem
começa a contar de forma tão detalhada e tão sossegada, no entanto, não
sou eu, mas o conselheiro secreto do tribunal de segunda instância,
Anselm von Feuerbach, que, com certeza, não a escreveu para a juventude
mais madura, mas dirigiu seu livro sobre Caspar Hauser aos adultos, que
foi lido em toda Europa. E, espero, da mesma maneira que vocês vão
escutá-la durante 20 minutos, a Europa inteira a acompanhou sem fôlego
durante cinco anos, de 1828 a 1833. Ela começa assim:

``Em Nuremberg, o segundo dia de Pentecostes é um dos dias de diversão
por excelência, pois a maior parte dos seus habitantes se dispersa no
campo e nas cidades vizinhas. Nesses dias, a cidade, que por si só já é
muito extensa em relação à sua população escassa, torna-se tão
silenciosa e vazia, especialmente com o tempo bom da primavera, que se
parece muito mais com aquela cidade encantada do Saara do que com uma
cidade movimentada de negócios e de comércio. Nessa situação,
principalmente em algumas partes mais distantes do seu centro, as coisas
secretas podem facilmente ocorrer em público, sem deixarem, por isso, de
ser secretas. -- Assim, no segundo dia de Pentecostes, em 26 de maio de
1828, na parte da tarde entre as 4 e as 5 horas, aconteceu o seguinte:
Um cidadão, domiciliado na praça chamada \emph{Unschlittplatz}, ainda
estava na frente de sua casa para, de lá, ir até o chamado Portão Novo
(\emph{Neues Tor}), quando, não muito longe de si, se deparou com um
jovem vestido como um camponês e que ficava parado numa postura
altamente peculiar, esforçando-se, à maneira de uma pessoa embriagada, a
se movimentar para frente, sem conseguir manter-se adequadamente ereto e
governar seus pés. O mencionado cidadão aproximou-se do forasteiro, que
lhe estendeu uma carta endereçada `Ao Senhor Capitão de cavalaria do 4.º
Esquadrão do 6.º Regimento \emph{Chevaux-Leger} de Nuremberg'.'' Acho
que, nesta altura, devo interromper a história, não apenas para explicar
que um regimento \emph{Chevaux-Leger} é um regimento de cavalaria, mas
também para dizer a vocês que essa palavra francesa estava escrita de
forma completamente errada, apenas imitando seu som. Isso é importante,
pois é assim que vocês devem imaginar a ortografia da carta inteira que
Caspar Hauser levava consigo e que vou ler para vocês depois. Assim que
tiverem escutado essa carta, vão entender facilmente por que o capitão
não ficou com o menino por muito tempo, mas tentou se livrar dele da
maneira mais rápida, ou seja, chamando a polícia. Como vocês sabem, a
primeira coisa que se faz quando alguém se dirige à polícia é um
registro. E naquela época em que o capitão, que não sabia o que fazer
com o Caspar Hauser, o entregou à polícia, formaram-se os primeiros
registros do gigantesco processo ``Caspar Hauser'', que hoje está
guardado em 49 volumes no Arquivo do Estado de Munique. Uma coisa desse
processo está bem clara: Caspar Hauser chegou em Nuremberg como uma
pessoa totalmente embrutecida e tosca, cujo vocabulário não abrangia
mais do que 50 palavras e que não entendia nada que lhe era dito e que
só tinha duas respostas a todas as perguntas dirigidas a ele: ``tinham
cabalero'' e ``num sei''. Mas como ele chegou a ter o nome de Caspar
Hauser? É uma história bastante estranha. Quando o capitão o levou para
o posto de polícia, a maioria dos guardas não chegou a um consenso sobre
se a questão era tratá-lo como uma pessoa demente ou um semi-selvagem.
Um ou outro, entretanto, ponderou a possibilidade de que esse rapaz
poderia ser um refinado impostor. E essa posição ganhou, num primeiro
momento, uma certa probabilidade devido ao seguinte fato: tiveram a
ideia de fazer um teste para ver se ele sabia escrever, deram-lhe uma
pena com tinta, colocaram uma folha de papel para ele e o mandaram
escrever algo. Ele parecia ficar feliz com isso, pegou a pena habilmente
entre seus dedos e, para a surpresa de todos, escreveu com traços firmes
e legíveis o nome \emph{Caspar Hauser}. Depois disso, o mandaram
acrescentar o nome do seu lugar de origem. Mas ele não fez outra coisa a
não ser balbuciar novamente seu ``tinham cabalero'' e ``num sei''.

O que esses bons policiais não conseguiram saber na época, ninguém
conseguiu saber até o presente momento; ninguém ficou sabendo de onde
Caspar Hauser surgiu. Mas a conversa daquele posto de polícia, segundo a
qual esse rapaz poderia ser um impostor muito esperto, também se manteve
como boato ou como convicção até o dia de hoje. Vocês ainda vão ouvir
algumas curiosidades que motivaram essa afirmação. Pelo menos eu, como
narrador da história, não quero esconder que a considero como
equivocada. Não se deve procurar no rapaz, mas em um outro lugar a
impostura que deu início a esta história. Para tal, agora preciso ler
para vocês a carta que estava com Caspar Hauser quando chegou em
Nuremberg.

``Magnífico Senhor Capitão! Estou lhe enviando um menino que gostaria de
ser um fiel servidor do seu Rei, ele pediu. Esse menino foi deixado
comigo -- quer dizer: me foi empurrado clandestinamente -- no dia 7 de
outubro de 1812. Eu mesmo sou um pobre trabalhador diarista, tenho dez
filhos, mal tenho como me sustentar, e não consegui saber nada sobre a
mãe dele. Mas também não falei no tribunal que o menino foi deixado
comigo, pois eu pensei que deveria tratá-lo como um filho; dei-lhe uma
educação cristã e desde o ano de 1812 não deixei que saísse de casa para
que ninguém soubesse onde ele cresceu, e ele mesmo não sabe o nome da
minha casa, e o lugar ele também não conhece; o senhor pode lhe
perguntar, ele não vai poder dizer. Caro Senhor Capitão, o senhor não
deve insistir, pois ele não sabe o lugar onde moro, eu o levei de noite,
ele não sabe mais voltar para casa. E ele está sem um tostão porque eu
mesmo não tenho nada. Se o senhor não ficar com ele, deve matá-lo ou
pendurá-lo na chaminé e defumar.''

Ora, junto com essa carta havia um pequeno bilhete que não estava
escrito em letra gótica, como essa carta, mas em letra latina, e ainda
em outro papel, aparentemente com uma caligrafia totalmente diferente.
Supostamente era a carta que acompanhava a criança deixada pela mãe 16
anos atrás. Ali estava escrito que ela era uma moça pobre e que não
teria como alimentar a criança. O pai pertenceria ao Regimento
\emph{Chevaux-Leger} de Nuremberg e que era para mandar o menino também
para esse Regimento assim que fizesse 17 anos. -- No entanto, e aqui se
evidencia pela primeira vez a impostura que fazia parte desse jogo
esdrúxulo: o exame químico mostrou que as duas cartas, aquela de 1828,
supostamente do diarista, e a outra, de 1812, supostamente da mãe, foram
escritas com a mesma tinta. Como vocês podem imaginar, logo não se
acreditou mais numa, nem na outra carta, nem na existência do pretenso
diarista, nem muito menos da pretensa moça pobre.

Nesse meio tempo, colocaram Caspar Hauser primeiro na cadeia municipal
de Nuremberg, considerando-o menos como um prisioneiro do que como uma
curiosidade que representava um dos pontos de atração para os
forasteiros. Do grande número de pessoas ilustres, que esse caso
extraordinário havia levado a Nuremberg, havia também o Conselheiro
Anselm von Feuerbach, que chegou a conhecer Caspar Hauser na ocasião e
sobre o qual escreveu o livro cujo começo li para vocês. Foi ele quem
deu a essa história uma virada decisiva, pois foi o primeiro que não
enxergou Caspar Hauser superficialmente, mas o estudou com o mais
profundo interesse. Ele percebeu que a inépcia, o idiotismo e a
ignorância do menino encontravam-se no mais gritante contraste com seus
dons extraordinários e seu caráter nobre. Essa natureza e a excelência
de suas dádivas, mas também certas marcas externas como, por exemplo, as
cicatrizes de vacina -- sendo que, naquele tempo, apenas as famílias
mais ilustres mandavam vacinar seus filhos --, tudo isso fez com que
Feuerbach fosse o primeiro a ponderar a possibilidade de que esse
forasteiro misterioso pudesse ser o filho de uma família da alta
aristocracia, e que ele fora escondido criminosamente por parentes para
privá-lo de sua herança. Em suas especulações, Feuerbach pensava
especificamente na família do Grão-duque de Baden. Até os jornais da
época veiculavam essas suspeitas de maneira disfarçada, o que aumentava
ainda mais o interesse nessa pessoa. Pode-se imaginar como isso deve ter
inquietado todos aqueles que supunham que Caspar Hauser tivesse
desaparecido silenciosamente em algum asilo de pobres ou hospício de
Nuremberg. Mas as coisas tomaram outro rumo. Feuerbach, em sua qualidade
de alto funcionário do Estado, tinha uma certa influência e cuidava que
o menino ficasse em um ambiente que satisfizesse sua avidez de aprender,
que havia despertado com grande vivacidade. Em seguida, ele foi acolhido
como um filho na casa do professor Daumer de Nuremberg, um homem bom e
nobre, mas ao mesmo tempo bastante excêntrico. Ele nos deixou não apenas
um livro volumoso sobre Caspar Hauser, mas toda uma biblioteca de obras
extravagantes sobre sabedoria oriental, segredos naturais, curas
milagrosas e ainda sobre magnetismo. O professor fazia algumas
experiências com Caspar Hauser nesse sentido, certamente com muito
cuidado e sensibilidade humana. Segundo as descrições que nos deu disso,
Caspar deve ter mostrado uma sensibilidade muito delicada, clareza de
pensamento, sobriedade e pureza. Seja como for, fazia muito progresso e
logo esteve em condições de tentar descrever sua vida por conta própria.
Nessa ocasião, veio à tona tudo que sabemos até hoje do tempo anterior à
sua aparição em Nuremberg. Parece que passou muitos anos num cárcere
subterrâneo, onde nunca chegou a ver um raio de luz, nem um ser vivo.
Dois cavalinhos de madeira e um cachorro, também de madeira, teriam sido
seus únicos companheiros; água e pão sua única alimentação. Pouco antes
de ter sido tirado de sua prisão, um desconhecido teria feito contato
para visitá-lo e, sempre ficando nas suas costas para não ser visto,
conduzir sua mão e ensinar-lhe, assim, a escrever. É claro que esses
relatos, feitos num alemão tão truncado quanto as anotações, despertaram
muitas dúvidas. Mas causa estranheza que há ao mesmo tempo testemunhos
de que Caspar Hauser, nos seus primeiros meses em Nuremberg, não
tolerava outra coisa a não ser pão e água, nem sequer leite, e que era
capaz de enxergar no escuro. Os jornais não perderam a oportunidade de
noticiar que Caspar Hauser teria começado a trabalhar em sua biografia.
Já naquela época, isso quase significou seu fim, pois, pouco depois de a
notícia se espalhar, encontraram-no, sem consciência e sangrando na
testa, no porão da casa do professor Daumer. Um desconhecido, contou,
teria aplicado um golpe de machado, enquanto estava no abrigo debaixo da
escada. Nunca descobriram esse desconhecido. Mas dizem que, mais ou
menos quatro dias depois disso, um senhor elegante teria abordado uma
mulher fora da cidade para perguntar se o ferido estava vivo ou morto,
para depois acompanhar essa mulher até o portão, onde estava afixado um
aviso da polícia sobre o ferido. Depois de ter lido esse aviso, ele
teria se afastado de uma maneira altamente suspeita, sem retornar para a
cidade.

Ora, se tivéssemos tanto tempo não apenas quanto eu queria, mas vocês
também, espero, queriam ter, poderia apresentar-lhes outra pessoa
notável que apareceu nesse momento na vida de Hauser, um senhor distinto
que o adotou. Não temos como entrar em detalhes sobre a importância
desse senhor, mas apenas quero salientar que, a partir daquele momento,
tratava-se de cuidar melhor da segurança de Hauser, levando-o de
Nuremberg a Ansbach, onde Anselm von Feuerbach passou a ocupar o cargo
de presidente do tribunal. Isso foi em 1831. Caspar Hauser teria mais
dois anos a viver, até ser assassinado em 1833. Como isso se deu, vou
contar para vocês agora, para encerrar.

No decorrer do tempo, ele tinha passado por uma grande mudança. Por mais
que suas capacidades intelectuais tivessem progredido, por mais que seus
dons tivessem se enobrecido, sua evolução mental parou depois de algum
tempo e seu caráter perdeu a pureza. Dizem que no final de sua vida --
sabemos que não passou dos 31 anos --, ele foi um homem um tanto mau e
medíocre, que ganhava discretamente sua vida como escrivão e com
trabalhos em papelão, nos quais mostrava muita habilidade. No mais,
porém, ele não se destacava por uma maior dedicação, nem era
especialmente honesto.

Então, numa manhã de dezembro do ano 1833, aconteceu que um homem o
abordou na rua com as palavras: ``Uma recomendação do senhor jardineiro
da Corte e um convite para visitar, hoje à tarde, o poço artesiano no
parque etc.'' -- Por volta das quatro horas, Caspar Hauser compareceu ao
Jardim da Corte. Não havia ninguém perto do poço artesiano. Ele deu mais
cem passos na mesma direção, quando um homem saiu da mata, estendeu um
saquinho roxo na direção dele, dizendo: ``Dou-lhe esse saquinho de
presente!'' Mal Caspar Hauser tinha tocado o saquinho, sentiu uma
facada. O homem desapareceu, Caspar deixou o saquinho cair e ainda
conseguiu arrastar-se até sua casa. Mas a ferida era mortal. Depois de
três dias, morreu. Ainda haviam feito um interrogatório com ele, mas a
questão se esse desconhecido era o mesmo que tentou matá-lo quatro anos
antes em Nuremberg ficou no escuro, assim como o resto. Por isso, nesse
momento também havia pessoas que afirmavam que Caspar Hauser tinha
esfaqueado a si mesmo. Mas encontraram o saquinho. E este era bastante
enigmático, pois não continha outra coisa a não ser um bilhete dobrado
em que estava escrito, em letra espelhada: ``O Hauser poderá contar a
vocês como sou e de onde venho. Para poupá-lo, vou dizer eu mesmo de
onde venho. Sou da fronteira com a Baviera. Vou até dizer a vocês meu
nome.'' Mas aí seguem-se apenas três letras maiúsculas: MLO.

Já disse a vocês que há 49 volumes de processo no Arquivo do Estado em
Munique. Dizem que o Rei Luís I, que se interessava muito pelo assunto,
havia folheado todos. Depois disso, muitos eruditos ainda os
consultaram. A polêmica em torno da questão se Caspar Hauser era ou não
um príncipe de Baden até hoje não foi esclarecida. Cada ano sai um ou
outro livro no qual se afirma que o enigma finalmente está resolvido.
Podemos fazer uma aposta de 100 contra 1. Quando vocês estiverem
adultos, ainda haverá pessoas que não conseguirão se livrar dessa
história. Quando esbarrarem em um livro dessa natureza, vocês talvez o
lerão para ver se ele tem a solução que a rádio ficou devendo a vocês.

O CORAÇÃO GELADO

\textsc{Peça de rádio baseada em Wilhelm Hauff}\footnote{``Das Kalte
  Herz. Hörspiel nach Wilhelm Hauff'', in GS VII-1, pp. 316-346.
  Tradução de Georg Otte, com a colaboração de Francisco De Ambrosis
  Pinheiro Machado. {[}N. dos E.{]}}\footnote{Peça radiofônica para
  crianças escrita em colaboração entre Benjamin e Ernst Schoen, seu
  antigo colega de escola, que compôs também a música para a emissão.
  Produzido em 1932, o programa foi ao ar em 16 de maio. O texto é uma
  adaptação de um conto de fadas de Wilhelm Hauff. Benjamin, que era
  colecionador de livros infantis, possuía uma edição das obras
  completas do autor desde 1918, como testemunha uma carta na qual ele
  conta à Ernst Schoen ter recebido o volume como presente de
  aniversário. {[}N. da O.{]}}

Walter Benjamin e Ernst Schoen

\emph{Personagens}

O locutor

Peter Munk do Carvão

Anão de Vidro

Miguel o Holandês\footnote{Na época, transportava-se a madeira
  simplesmente juntando os troncos em jangadas, que flutuavam rio
  abaixo. Miguel o Holandês trabalha nessa atividade, sendo que o
  apelido de ``Holandês'' vem do fato de a madeira da Floresta Negra ser
  transportada no rio Reno até a Holanda, onde os troncos serviam como
  escoras para as construções em solo pantanoso. {[}N. do T.{]}}

Ezequiel

Schlurker

O Rei do Tablado

Lisbete

Mendigo

Moendeiro

Moendeira

Filho do Moendeiro

Uma Voz

Postilhão

\emph{Prelúdio}

LOCUTOR -- Prezados ouvintes. Começa mais uma vez o nosso Programa da
Juventude e estou pensando em ler outro conto de fadas para vocês. Qual
conto será que vocês vão querer? Vamos dar uma olhada no nosso grande
dicionário no qual estão os nomes de todos os autores de contos de
fadas, como no catálogo de telefone, onde posso escolher um nome. Então:
A, de Abracadabra, não serve; vamos folhear mais; B, de Bechstein, já é
mais interessante, mas esse já tivemos recentemente.

\emph{Batidas na porta.}

C de Celsius, o contrário de Réaumur, D, E, F, G.

\emph{Som mais forte das batidas.}

H de Hauff, Wilhelm Hauff, este sim, seria o certo para nós.

\emph{Agora estão espancando a porta}.

Que barulho infernal é esse? Aqui na Rádio? Assim não dá para fazer o
nosso Programa da Juventude, ora bolas! Entre! Entre logo!
\emph{Sussurrando:} Vocês estão perturbando o meu Programa da Juventude
-- mas, o que é isso? Que figuras estranhas vocês são! O que vocês
querem?

Peter Munk do Carvão -- Somos as personagens do conto de fadas ``O
coração frio'', de Wilhelm Hauff.

LOCUTOR -- Do ``Coração frio'' de Wilhelm Hauff? Então é como se vocês
tivessem sido chamados, sejam bem-vindos! Mas como conseguiram entrar?
Não sabem que aqui é uma rádio? E que ninguém pode entrar aqui sem
autorização?

Peter Munk do Carvão -- O senhor é o locutor?

LOCUTOR -- Claro que sou o locutor!

Peter Munk do Carvão -- Então estamos no lugar certo. Entrem todos e
fechem a porta. E agora, quem sabe, poderemos primeiramente nos
apresentar.

LOCUTOR -- Sim, mas...

\emph{Cada apresentação é acompanhada de uma melodia tocada por uma
caixinha de música}.

Peter Munk do Carvão -- Sou o Peter Munk, nascido na Floresta Negra, me
chamam de Peter Munk do Carvão, porque herdei do meu pai, junto com o
colete de honra com botões de prata e as meias vermelhas para os dias de
festa, a profissão de carvoeiro.

ANÃO DE VIDRO -- Eu sou o anão de vidro, tenho apenas 3 pés e meio de
altura, mas muito poder sobre o destino dos homens. Se for uma criança
de domingo\footnote{No original \emph{Sonntagskind,} pessoa nascida em
  um domingo que, segundo crença popular alemã, é favorecida pelo
  destino. {[}N. dos E.{]}}, Sr. Locutor, faça um passeio pela Floresta
Negra e, quando vir um anão na sua frente, com um chapéu pontudo e
largo, com colete e calças turcas e meias vermelhas, faça logo um
desejo, pois isso significa que me achou.

MIGUEL O HOLANDÊS -- E eu sou o Miguel Holandês. Meu colete é de linho
escuro, as calças de couro preto estão presas por largos suspensórios
verdes. No bolso carrego um metro de latão e uso botas de madeireiro que
leva as toras rio abaixo, mas tudo isso de tamanho gigantesco, de
maneira que, só para fabricar as botas, seria necessária uma dúzia de
bezerros.

EZEQUIEL -- Sou Ezequiel o Gordo. Chamam-me assim porque a minha cintura
é imensa. Também tenho as riquezas que combinam com a minha barriga. Não
é à toa que me consideram o mais rico do grupo. Todo ano vou duas vezes
a Amsterdã para vender madeira de construção, e, enquanto os outros têm
que voltar a pé, eu subo o Reno com muita pompa.

SCHLURKER -- Sou Schlurker o Altão, o homem mais alto e mais magro da
Floresta Negra inteira, mas também o mais atrevido, porque, mesmo com
todo mundo sentado apertado no bar, preciso de mais espaço do que quatro
gordos.

REI DO TABLADO, \emph{afetado} -- Permita-me que me apresente, Sr.
Locutor, sou o Rei do Tablado.

MIGUEL O HOLANDÊS, \emph{interrompendo --} Já chega, Rei do Tablado, não
precisa de tanta cerimônia. Sei muito bem de onde vem seu dinheiro e que
antes você era um pobre servo lenhador.

LISBETE -- Sou a Sr.\textsuperscript{a} Lisbete, filha de um pobre
sitiante lenhador, mas a mais bela e mais cheia de virtudes da Floresta
Negra inteira e sou casada com o Peter Munk do Carvão.

MENDIGO -- E eu sou o último de todos, pois sou apenas um pobre mendigo
e por isso faço um papel que, embora importante, é pequeno.

LOCUTOR -- Agora já chega de ouvir quem vocês são; a minha cabeça já
está muito confusa. Mas o quê vocês querem aqui na Rádio? Por que me
atrapalham no meu trabalho?

Peter Munk do Carvão -- Para dizer a verdade, Sr. Locutor, a gente
queria muito visitar a Terra da Voz.

LOCUTOR -- A Terra da Voz, Peter Munk do Carvão? Como vou entender isso
agora? Dá para ser mais claro?

Peter Munk do Carvão -- Veja, Sr. Locutor, já faz 100 anos que estamos
no livro de contos de fadas de Hauff. Assim, só podemos falar apenas
para uma criança de cada vez. Mas dizem que agora está na moda que as
personagens dos contos de fadas saiam dos livros para ir para a Terra da
Voz, onde podem se apresentar a milhares de crianças de uma única vez.
Queremos fazer isso também, e nos falaram que o Sr. Locutor seria a
pessoa certa para nos ajudar nisso.

LOCUTOR, \emph{lisonjeado} -- Isso certamente é verdade, se estiver
falando da Terra da Voz do Rádio.

MIGUEL O HOLANDÊS, \emph{grosseiro} -- Claro que estamos falando disso!
Nos deixe entrar em cena então, Sr. Locutor, sem cerimônia.

Ezequiel, \emph{grosseiro} -- Não fale besteira, Miguel. Na Terra da Voz
não dá para ver nada!

Peter Munk do Carvão -- Com certeza dá para ver alguma coisa na Terra da
Voz, mas não dá para ser visto. E percebi que é isso que lhe incomoda.
Claro, você não é feliz quando não consegue ser visto com suas
correntes, suas joias e seus lenços. Mas pense bem no que recebe em
troca: todas as pessoas que você pode ver do pico mais alto da Floresta
Negra e ainda mais que isso vão poder lhe ouvir sem você ter que
levantar a voz nem um pouco.

REI DO TABLADO -- Pensando bem, Peter Munk do Carvão, não concordo muito
com você. Na Floresta Negra, tudo bem, lá conheço tudo -- mas na Terra
da Voz, temo que vou errar o caminho e tropeçar o tempo todo por causa
das raízes.

EZEQUIEL -- Raízes! Não há raízes na Terra da Voz!

PETER MUNK DO CARVÃO -- Não dê crédito a ele, Rei do Tablado. Tenho
certeza de que há raízes. Na Terra da Voz há também uma floresta negra e
também aldeias, cidades, rios, nuvens, exatamente como na Terra. Mas na
Terra da Voz não se pode vê-los, apenas ouvir. E, assim, não se vê nada,
apenas ouve-se o que se passa na Terra da Voz. Mas mal vocês entrem
nela, vão saber se virar tão bem quanto aqui.

LOCUTOR -- E se alguma coisa faltar, eu, o locutor, estou às ordens. Nós
do Rádio conhecemos a Terra da Voz como a palma da própria mão.

MIGUEL O HOLANDÊS, \emph{grosseiro} -- Deixe a gente entrar, então, Sr.
Locutor.

LOCUTOR -- Devagar, seu grosseiro Miguel Holandês, não é tão simples
assim. É verdade que na Terra da Voz vocês vão poder falar e ainda para
milhares de crianças, mas eu sou o guarda-fronteiras desse país e
precisarei dar a condição que têm que cumprir antes disso.

LISBETE -- Uma condição?

LOCUTOR -- Sim, Sr.\textsuperscript{a} Lisbete, uma condição que lhes
causará muito esforço para ser cumprida.

ANÃO DE VIDRO -- Pois bem, diga-nos sua condição, pois estou acostumado
a condições; eu mesmo costumo impô-las.

LOCUTOR -- Então, escute bem, Anão de Vidro, e vocês outros também: quem
quiser entrar na Terra da Voz, tem que ser muito modesto, se livrar de
qualquer adorno e qualquer beleza exterior para que lhe reste apenas a
voz. Mas esta, em compensação, será ouvida por milhares de crianças ao
mesmo tempo, do jeito que vocês querem.

\emph{Pausa.}

Então, esta é a condição da qual, infelizmente, não posso abrir mão.
Vocês têm ainda um tempinho para pensar nisso.

PETER MUNK DO CARVÃO, \emph{sussurrando} -- O que vocês acham? Lisbete,
você está disposta a deixar aqui sua bela indumentária de domingo?

LISBETE, \emph{sussurrando} -- Mas é claro, Peter, não faço questão
nenhuma dela! Se podemos falar para milhares de crianças!

EZEQUIEL, \emph{sussurrando} -- Olha lá! Também não é tão simples assim!
\emph{Tilintando as moedas}: E o que faço com esses ducados de ouro?

ANÃO DE VIDRO, \emph{sussurrando} -- Fique feliz de poder livrar-se
delas dessa maneira, seu malandro! \emph{Em voz alta:} Então, Sr.
Locutor, estamos de acordo com sua condição.

LOCUTOR -- Muito bem, Anão de Vidro, vamos então.

PETER MUNK DO CARVÃO -- Só mais um pedido.

LOCUTOR -- Qual pedido, Peter?

PETER MUNK DO CARVÃO -- Sabe, Sr. Locutor, nós nunca estivemos na Terra
da Voz!

LOCUTOR -- Claro, claro, e que mais?

PETER MUNK DO CARVÃO -- Como é que vamos encontrar o caminho?

LOCUTOR -- Você tem razão, Peter.

PETER MUNK DO CARVÃO -- Já que o senhor é o guarda-fronteiras da Terra
da Voz, acho que poderia nos acompanhar como guia.

REI DO TABLADO -- Também acho, preso junto, enforcado junto.\footnote{Expressão
  alemã (\emph{Mitgefangen, mitgehangen}) para dizer que, quem se
  envolveu em alguma ação com outras pessoas, tem que arcar igualmente
  com as consequências. {[}N. do T.{]}}

LISBETE -- Ninguém aqui está falando de enforcar, seu bobo! Mas, se o
Sr. Locutor fizer a gentileza...

LOCUTOR, \emph{lisonjeado} -- Combinado então, vou guiar vocês. Mas não
se incomodem com o barulho que os meus papéis farão.

\emph{Barulho de papel,} pois sem o meu mapa também não sei andar na
Terra da Voz.

\emph{Pausa}.

Então, se não tiverem nada contra, peço que me acompanhem até o
guarda-roupa! Sr.\textsuperscript{a} Lisbete, o chapéu a senhora tem que
deixar aqui, também o espartilho dourado e os sapatos de fivela, aqui
tem a vestimenta de voz em troca. Sr. Peter Munk, tire o colete de honra
e as meias vermelhas.

PETER MUNK DO CARVÃO -- Aqui estão.

LOCUTOR -- Você também, Anão de Vidro, vai ter que tirar o chapéu, o
colete e as calças turcas.

ANÃO DE VIDRO -- Já está feito.

LOCUTOR -- E você, Miguel o Holandês? Não, não, o metro e as botas de
couro também têm que ficar.

MIGUEL O HOLANDÊS -- Que seja, em nome do diabo!

LOCUTOR -- O Rei do Tablado também já está pronto, e você, pobre
mendigo, não deve ter muita coisa para deixar! Mas o que estou vendo
aqui? Ezequiel o Gordo pendurou seu saquinho de ducados no pescoço! Não,
bom amigo, assim não dá! Para onde estamos indo agora, seus ducados
também não servem de nada. Apenas precisamos de uma voz bonita e clara,
uma que não tenha sofrido a fumaça da taberna como a sua.

EZEQUIEL, \emph{amaldiçoando} -- Não, não vou com vocês! Meu dinheiro
vale mais do que toda essa Terra da Voz de vocês!

MIGUEL O HOLANDÊS -- Caramba! Acho que também mando um pouco aqui. Dê-me
o dinheiro, sua pulga humana miserável, ou vou lhe esmagar!

LOCUTOR -- Mantenham a calma, meus amigos! Sr. Miguel o Holandês,
controle sua ira, e Sr. Ezequiel, posso lhe garantir que, depois de sua
visita à Terra da Voz, receberá seu dinheiro de volta, centavo por
centavo.

EZEQUIEL -- Tudo bem, Sr. Locutor, se puder me dar uma garantia por
escrito!

LOCUTOR -- Todo mundo para a Terra da Voz!

\emph{Gongo.}

\emph{Música: Peter}

LOCUTOR -- Olá, Peter Munk do Carvão, olá!

\emph{Várias vozes gritando} -- Olá!

PETER MUNK DO CARVÃO -- Locutor, você enxerga alguma coisa? Quem é que
está gritando ``Olá''? Onde é que estamos?

LOCUTOR -- Não, Peter Munk do Carvão, na Terra da Voz não há nada a ver,
apenas a ouvir.

\emph{Música}: \emph{Moinho.}

FILHO DO MOENDEIRO -- Você está vendo alguma coisa, pai?

MOENDEIRO -- Tem tanta névoa que a gente não vê um palmo na frente do
nariz. Estou quase tropeçando no meu próprio moinho. -- O que diz,
mulher?

MOENDEIRA -- Mas agora ouço vozes se aproximando.

\emph{Música.}

PETER MUNK DO CARVÃO -- Locutor? Há um ruído aqui como se tivesse um
rio. A minha vida inteira nunca vi nem um riachinho por aqui.

LOCUTOR -- Por aqui, Peter? Você fala como se soubesse? Mas espero que
não se assuste quando lhe digo que a gente se perdeu.

PETER MUNK DO CARVÃO -- A gente se perdeu? Não acredito. Não tinha
vozes?

LOCUTOR -- Vozes de outras pessoas.

\emph{Ouve-se novamente o}: ``Olá, olá!

MOENDEIRA -- Jesus, de onde vocês surgiram tão tarde da noite?

LOCUTOR -- Olá, minha senhora, então já é tarde?

MOENDEIRO -- Quase dez horas da noite.

PETER MUNK DO CARVÃO -- Sim, boa noite, minha boa gente, é que nós nos
perdemos.

MOENDEIRO -- Então vocês já caminharam muito.

PETER MUNK DO CARVÃO -- Nem tanto assim. Mas agora sinto a caminhada nas
minhas pernas.

LOCUTOR -- Eu então, Peter. Mas não adianta, vou ter que voltar e
procurar meus amigos na Terra da Voz.

\emph{Ouve-se pessoas dizendo} -- Boa noite, Locutor! Cuide-se. Boa
noite! Até logo!

MOENDEIRA -- Entrem e fiquem à vontade, Sr. Peter, pois parece que é
esse seu nome. O senhor tem que prestar atenção para não pegar muita
poeira. Nos moinhos sempre há poeira. Vamos, Hanni, ofereça ao senhor as
panquecas que sobraram do jantar, e um licor de cereja ele também não
vai recusar.

\emph{Pausa. Ouve-se barulho de pratos.}

FILHO DO MOENDEIRO, \emph{sussurrando} -- Esse Sr. Peter tem uma
aparência estranha, mãe.

MOENDEIRA -- Sei lá. O que você quer dizer com isso?

FILHO DO MOENDEIRO, \emph{sussurrando} -- Estranho, mãe, parece que
aconteceu alguma coisa com ele.

MOENDEIRA -- Deixe de ser bobo, menino! Vá para a cama, depressa. E o
senhor certamente também não vai demorar para se deitar. O senhor deve
saber que, nos moinhos, o barulho começa cedo. Não é um abrigo para
dorminhocos.

PETER MUNK DO CARVÃO -- Certo, Sr.ª Moendeira. Mas permita-me
agradecer-lhe muito pelas panquecas.

MOENDEIRA -- Ah, não é nada. Mas agora venha comigo. Vou lhe mostrar a
cama.

PETER MUNK DO CARVÃO -- Não se preocupe, posso dormir aqui. Com tantas
almofadas! Quase até o teto!

MOENDEIRA -- É isso aí, não temos janelas duplas aqui na Floresta Negra.
Por isso tem que ter cobertores grossos quando começa a gear no inverno.

\emph{Ouve-se novamente algumas vozes} -- Bom descanso! Boa noite! Não
se esqueçam de apagar as velas!

PETER MUNK DO CARVÃO, \emph{bocejando} -- Que coisa, não sabia que uma
pessoa pode ficar com tanto sono. Mesmo se o diabo entrar agora, acho
que eu ficaria deitado e só viraria para o outro lado.

\emph{Pequena pausa. Batidas na porta.}

PETER MUNK DO CARVÃO -- Estão batendo na porta? Não é possível, todos já
estão dormindo.

\emph{Mais batidas na porta.}

PETER MUNK DO CARVÃO -- Deve ter alguém na porta. Entre!

FILHO DO MOENDEIRO -- Meu caro Sr. Peter, por favor, não me entregue.
Deixe-me ficar mais um pouco com o senhor. Estou com tanto medo.

PETER MUNK DO CARVÃO -- Que é isso, o que você tem? Por que está com
tanto medo?

FILHO DO MOENDEIRO -- Sr. Peter, o senhor também ficaria com medo se
tivesse visto o que vi hoje. -- Talvez o senhor tenha visto, quando
entrou, o livro coberto com um veludo vermelho, que estava na mesa.

PETER MUNK DO CARVÃO -- Ah, o álbum, certo. Deve ter retratos nele, não
é?

FILHO DO MOENDEIRO -- Há retratos nele sim, Sr. Peter, mas numa das
folhas há três que não saem mais da minha cabeça de maneira alguma; eles
me perseguem o tempo todo com seus olhares. O gordo do Ezequiel e o
altão do Schlurker e o Rei do Tablado, pois esses são os nomes que
estavam escritos debaixo dos retratos.

PETER MUNK DO CARVÃO -- O que você diz? Ezequiel o Gordo, Schlurker o
Altão... mas esses nomes eu também já ouvi, e o Rei do Tablado era o
pobre diabo que começou como servo do dono da madeira e depois ficou
riquíssimo de repente. Alguns dizem que teria encontrado um pote cheio
de dinheiro debaixo de um pinheiro. Outros afirmam que, não muito longe
da cidade de Bingen, ele teria pescado um pacote com peças de ouro com a
haste que os madeireiros-jangadeiros usam para espetar peixes no Reno, e
que o pacote faria parte do grande tesouro dos Nibelungos que lá está
enterrado. Resumindo: ele ficou rico de repente e os jovens e os velhos
passaram a respeitá-lo como um príncipe.

FILHO DO MOENDEIRO -- Mas o senhor deveria ter visto os olhos, aqueles
olhos!

PETER MUNK DO CARVÃO -- Sim, isso existe, sabe? Pessoas que viram uma
coisa especialmente terrível, às vezes mantêm um olhar estranho durante
toda sua vida.

FILHO DO MOENDEIRO -- Mas o que o senhor acha que ele poderia ter visto
de tão terrível?

PETER MUNK DO CARVÃO -- Saber, eu não sei, mas lá, do outro lado da
Floresta Negra, sabe, onde moram os proprietários de madeira e os
madeireiros-jangadeiros, dizem que acontecem coisas não muito normais.

FILHO DO MOENDEIRO -- Já sei, agora o senhor está falando de Miguel o
Holandês. Meu pai também já me contou algumas coisas sobre ele. É o
gigante da floresta, o rapaz selvagem e de ombros largos, sobre o qual
aqueles que afirmam tê-lo visto garantem que não quer pagar do próprio
bolso o couro dos bezerros necessário para fabricar seus sapatos.

PETER MUNK DO CARVÃO -- Sim, é nele que acabei de pensar.

FILHO DO MOENDEIRO -- De repente, o senhor também sabe algo sobre ele.

PETER MUNK DO CARVÃO -- Menino, você não tem vergonha de dizer algo
assim? Como é que vou saber alguma coisa sobre Miguel o Holandês? Às
vezes, quando ouço as pessoas falarem, me pergunto: será que não é
simplesmente inveja? Será que eles não têm inveja dos donos da madeira
porque eles sempre andam por aí orgulhosos nos seus coletes com os
botões, as fivelas e as correntes, nas quais têm pendurados vários
quilos de prata. Não são poucos os que ficam com inveja quando veem algo
assim.

FILHO DO MOENDEIRO - Mas o senhor também já ficou com inveja dele?

PETER MUNK DO CARVÃO -- Com inveja? De forma alguma; eu seria o último a
sentir inveja.

FILHO DO MOENDEIRO -- Então o senhor é tão rico quanto ele? Ou até mais
rico?

PETER MUNK DO CARVÃO -- Olha, menino, você deve ter percebido que sou um
rapaz pobre e que não tenho prata pendurada no corpo, nem em casa. Pois
tenho algo melhor que isso, mas não posso lhe revelar o que é.

FILHO DO MOENDEIRO -- Agora o senhor me deixou bastante curioso. Não vou
querer sair do seu quarto até me dizer.

PETER MUNK DO CARVÃO -- Mas você sabe mesmo guardar segredo?

FILHO DO MOENDEIRO -- Com certeza, Sr. Peter, prometo que ninguém vai
ficar sabendo de nada.

PETER MUNK DO CARVÃO -- Então quero lhe perguntar uma coisa: alguma vez
você já ouviu falar do Anão de Vidro? Que nunca aparece de outra maneira
a não ser com um chapéu pontudo e largo, com calças turcas e meias
vermelhas? E que é amigo dos sopradores de vidro, dos carvoeiros e dos
pobres em geral que moram deste lado da Floresta?

FILHO DO MOENDEIRO -- Do Anão de Vidro? Não, Sr. Peter, nunca ouvi falar
dele.

PETER MUNK DO CARVÃO -- Mas já ouviu falar da criança de domingo?

FILHO DO MOENDEIRO -- Sim, claro, aquelas que nascem aos domingos ao
meio-dia.

PETER MUNK DO CARVÃO -- Então, sou uma delas. Entendeu? -- Mas essa é só
a metade do segredo. A outra é o meu verso.

FILHO DO MOENDEIRO -- Agora já não entendo mais nada, Sr. Peter.

PETER MUNK DO CARVÃO -- O Anão de Vidro se mostra às crianças de
domingo, mas apenas se elas, ao pé do Morro dos Pinheiros, onde as
árvores encontram-se tão perto umas das outras e são tão altas que fica
escuro em pleno dia, quando não se ouve um machado, nem um pássaro, se
elas souberem o verso correto. E esse verso aprendi com a minha mãe.

FILHO DO MOENDEIRO -- Então é para ficar com inveja do senhor.

PETER MUNK DO CARVÃO -- Sim, seria mesmo para ficar com inveja se
tivesse guardado o verso, mas, quando estive na frente do pinheiro agora
mesmo, querendo dizê-lo, vi que tinha me esquecido da última rima, e o
Anão de Vidro desapareceu no mesmo momento em que havia se mostrado.
``Sr. Vidro'', gritei depois de alguma hesitação, ``faça o favor de não
zombar de mim. Sr. Vidro, se o senhor acha que não o vi, está muito
enganado. Vi o senhor muito bem quando apareceu por trás da árvore.''
Mas ele não respondeu e só de vez em quando ouvi uma risadinha baixinha
e rouca vindo por detrás da árvore. Aí pensei: basta uma investida para
pegar esse rapazinho. Mas na hora que dei um salto até o pinheiro, não
havia mais nenhum Anão de Vidro na floresta verde de pinheiros; apenas
um esquilo pequeno e gracioso fugiu para cima da árvore.

FILHO DO MOENDEIRO -- Quer dizer que o senhor acabou de encontrar com o
Anão de Vidro?

PETER MUNK DO CARVÃO -- Exatamente.

FILHO DO MOENDEIRO -- Mas agora o senhor tem que me falar o verso, se o
souber ainda.

PETER MUNK DO CARVÃO -- Não, rapaz. Agora já é tarde, vamos dormir.
Nesse meio tempo você se esqueceu dos seus três homens maus, e amanhã,
quando acordarmos, todos nós vamos querer estar bem dispostos.

FILHO DO MOENDEIRO -- Boa noite, Sr. Peter. Mas não estou bem disposto
porque não me falou o verso.

\emph{Ouve-se como os dois se despedem.}

PETER MUNK DO CARVÃO -- Finalmente estou sozinho e agora quero dormir.
Mas o verso não quero dizer a outra pessoa que não seja o Anão de Vidro;
se ao menos me lembrasse dele!

\emph{Toca uma musiquinha que Peter Munk do Carvão acompanha com uma voz
sonolenta}:

\begin{quote}
Dono do tesouro na verde floresta de pinheiros,

Já viveste séculos inteiros,

A ti pertence toda terra que pinheiros tem...
\end{quote}

PETER MUNK DO CARVÃO, \emph{com voz sonolenta} -- Que pinheiros
tem\ldots{}, que pinheiros tem... -- Se soubesse a continuação!

\emph{A musiquinha acaba. Depois de uma pausa, ouve-se seis batidas.}

LOCUTOR -- Cá estou eu de novo no moinho da Floresta Negra, junto com o
Peter Munk do Carvão. São seis horas e aposto que o Peter dormiu sem
parar; não vai ser fácil acordá-lo.

\emph{Ouve-se o ronco alto do Peter. Uma música baixa vai aumentando
cada vez mais. Alguém canta uma ou duas estrofes.}

PETER MUNK DO CARVÃO, \emph{sonolento} -- O que é isso? Parece que eles
usam uma caixinha de música como despertador. Quero acordar assim todas
as manhãs, com uma música só minha, como um príncipe. Não, está vindo de
fora. Ah, devem ser aprendizes! Sim, eles têm que levantar cedo!

\emph{Ouve-se o canto}:

\begin{quote}
Em cima do morro um lugar tem,

De onde para o vale posso olhar,

De lá meus olhos veem,

Pela última vez ela passar.
\end{quote}

PETER MUNK DO CARVÃO -- Olá, pessoal, mais uma vez, mais uma vez, cantem
isso mais uma vez!

\emph{Ouve-se como a música vai diminuindo aos poucos e como o canto
fica cada vez mais incompreensível.}

É, eles não estão ligando nada para mim. Já estão lá longe. \emph{Mais
baixo e pensativo}: Mas, como é que foi? \emph{Canta baixinho a mesma
melodia}: ``De lá meus olhos veem'', ``de lá meus olhos veem'' --
``veem'', então esta é a rima, ``tem'' com ``veem''. Agora, Anão de
Vidro, teremos uma palavrinha de novo. \emph{Ele assobia um pouco para
si mesmo.}

LOCUTOR -- Para onde vai com tanta pressa, Peter? Há pouco ainda pensei
aflito em como fazer você levantar e mostrar o caminho de casa, e agora
você passa aí na maior rapidez.

PETER MUNK DO CARVÃO, \emph{apressado} -- Deixe-me, deixe-me, Sr.
Locutor. Lembrei do meu verso...

LOCUTOR -- Verso? Que verso é esse?

PETER MUNK DO CARVÃO, \emph{dando sinal para falar baixo} --
Pss\ldots{}, tenho um plano especial, mas não posso falar. O senhor vai
ver depois. Adeus, Sr. Locutor!

LOCUTOR -- Olha só esse pândego. \emph{Gritando atrás dele:} Preste
atenção para não esbarrar com o maldoso Miguel o Holandês! Adeus, Peter!

\emph{Pausa. Peter assobia sua cantiga. Pausa. Pigarreia longamente.}

PETER MUNK DO CARVÃO -- Pronto, aqui temos o pinheiro grande. Atenção,
Peter, vamos lá:

\begin{quote}
Dono do tesouro na verde floresta de pinheiros,

Já viveste séculos inteiros,

Tua é toda terra que pinheiros tem,

Só crianças de domingo é que te veem.
\end{quote}

ANÃO DE VIDRO -- Você não acertou direito, mas, no seu caso, Peter Munk
do Carvão, vou abrir uma exceção. Você encontrou com o malcriado do
Miguel o Holandês?

PETER MUNK DO CARVÃO -- Sim, Sr. Dono do Tesouro, fiquei com bastante
medo. Apenas procurei o senhor para me dar um conselho; não estou muito
bem e tenho muitos problemas. Um carvoeiro não vai muito longe e, como
ainda sou jovem, estive pensando que poderia fazer de mim algo melhor.
Quando vejo os outros e o progresso que fizeram em tão pouco tempo --
basta olhar o Ezequiel e o Rei do Tablado, eles estão nadando em
dinheiro.

ANÃO DE VIDRO -- Peter, não fale dessa gente para mim! O que eles ganham
se passam alguns anos aparentemente felizes para depois se tornarem
tanto mais infelizes? Não despreze sua ocupação; seu avô e seu pai eram
pessoas honestas e tinham essa mesma ocupação, Peter! Espero que não
seja amor pelo ócio que o traz até mim.

PETER MUNK DO CARVÃO -- Não, Sr. Dono do Tesouro do Pinheiral, o ócio é
o começo de todos os males. Mas o senhor não poderia me levar a mal, se
uma outra profissão me agrada mais que a minha atual. Não tem como negar
que um carvoeiro é pouco respeitado no mundo, em comparação aos
sopradores de vidro, aos madeireiros dos rios e aos relojeiros.

ANÃO DE VIDRO -- A soberba vem amiúde antes da queda.\footnote{A
  passagem ecoa o seguinte versículo bíblico do livro dos
  \emph{Provérbios}: ``A soberba precede a ruína, e a altivez do
  espírito precede a queda.'' (Cf. Pr 16, 18) {[}N. do T.{]}} Vocês
homens são um gênero estranho! Raramente alguém está contente com a
profissão em que nasceu e cresceu. Para que isso? Se você fosse um
soprador de vidro, iria querer ser dono de madeireira, e se fosse dono
de madeireira, poderia ter a seu serviço um administrador florestal ou
ter o direito a uma residência de magistrado. Mas que seja! Se você
prometer que vai trabalhar direito, quero ajudá-lo a ter uma vida
melhor, Peter. Costumo realizar três desejos para cada criança de
domingo que sabe me encontrar. E preste atenção! A cada desejo vou bater
o meu cachimbo de vidro no pinheiro. Os dois primeiros são livres, o
terceiro posso recusar se for tolo. Então faça seus desejos, mas, Peter,
que seja algo de bom e útil!

PETER MUNK DO CARVÃO -- Muito bem! O senhor é um excelente Anão de
Vidro! E com razão o chamam de Dono do Tesouro, pois, estando na sua
casa, os tesouros estão no lugar certo. Então posso desejar o que mais
quero -- para começar quero saber dançar melhor do que o Rei do Tablado
e, a cada vez, trazer o dobro do dinheiro que ele levava para a taberna.

\emph{Batidas do cachimbo.}

ANÃO DE VIDRO -- Tolo que você é! Que mísero desejo é esse -- saber
dançar bem e ter dinheiro para o jogo! Você não tem vergonha na cara,
seu bobo, de enganar-se a si mesmo no que diz respeito à sua felicidade?
O que adianta para você e sua mãe se sabe dançar bem? Para que serve o
dinheiro, que, pelo seu desejo, só vai para a taberna, da mesma maneira
que o do Rei do Tablado miserável para eles vai? Depois você não tem
mais nada durante uma semana e passa necessidade como antes. Vou lhe
conceder mais um desejo; mas preste atenção para fazer um desejo mais
sensato!

PETER MUNK DO CARVÃO, \emph{depois de hesitar um tempo} -- Então quero
ser o dono da fábrica de vidro mais bonita e mais rica em toda a
Floresta Negra, com todos os acessórios e o dinheiro.

ANÃO DE VIDRO -- Só isso? Peter, só isso?

PETER MUNK DO CARVÃO -- Bem, o senhor pode acrescentar ainda um cavalo e
uma carruagem.

ANÃO DE VIDRO -- Que tolice, Peter Munk do Carvão! \emph{O cachimbo
espatifa-se.} Cavalos? Carruagens? Inteligência, digo, inteligência e
bom senso você deveria ter desejado, mas não cavalos e carruagens.
Agora, não fique tão triste assim, vamos ver se alguma coisa foi para o
seu bem, pois o segundo desejo não foi inteiramente tolo. Uma boa
fábrica de vidro alimenta seu dono e seu mestre; faltou apenas desejar
inteligência e bom senso -- as carruagens e os cavalos teriam se juntado
de qualquer maneira.

PETER MUNK DO CARVÃO -- Mas, Sr. Dono do Tesouro, tenho ainda um desejo
livre. Com ele poderia escolher a inteligência, se ela me for tão
necessária quanto o senhor acredita.

ANÃO DE VIDRO -- Nada disso! Você ainda vai ter que passar por alguns
apertos para ficar contente por ter ainda um desejo livre. E agora vá
para casa! Aqui tem 2.000 florins para você e chega, e não volte mais
aqui para pedir qualquer dinheiro, pois, nesse caso, eu estaria obrigado
a lhe enforcar no pinheiro mais alto! É isso que costumo fazer desde que
moro na floresta. Há três dias morreu o velho Winkfritz, que era o dono
da grande fábrica de vidro na Floresta de Baixo. Vá para lá amanhã cedo
e faça uma oferta boa para comprá-la! Fique bem, seja bom trabalhador e
vou lhe visitar de vez em quando para lhe ajudar com alguns conselhos,
já que ainda não desejou o bom senso. Mas lhe digo com toda seriedade:
seu primeiro desejo foi mau. Evite frequentar as tabernas, Peter! Isso
nunca fez bem a ninguém.

PETER MUNK DO CARVÃO -- Lá se vai ele. Incrível o quanto fuma, o Dono do
Tesouro. Nem consigo enxergá-lo mais de tanta fumaça. \emph{Farejando}:
Mas é um fumo bem agradável.

\emph{Gongo.}

LOCUTOR -- Então, onde é que nós paramos? Vocês, crianças, acabaram de
ouvir a conversa entre o Peter Munk do Carvão e o Sr. Dono do Tesouro.
Vocês ouviram os desejos tolos que o Peter fez e ouviram como o Anão de
Vidro desapareceu numa nuvem de fumaça do melhor fumo holandês. Vamos
ver como vai ser o próximo capítulo. \emph{Ele faz barulho com o papel.}
Mas onde está a continuação? Hum, hum! \emph{O barulho aumenta.}

ANÃO DO VIDRO, \emph{sussurrando} -- O que está acontecendo? Por que a
gente não continua?

LOCUTOR, \emph{sussurrando} -- Hum, também não sei o que fazer. Imagine
só, Sr. Dono do Tesouro, àquela hora na floresta o vento deve ter levado
algumas folhas da história; agora estamos em maus lençóis. Não faço a
mínima ideia de como sair disso.

REI DO TABLADO, \emph{sussurrando} -- Isso é fatal, fatal! Mas o quê a
gente vai fazer?

MIGUEL O HOLANDÊS, \emph{sussurrando} -- Seu tolo Rei do Tablado, você
também não vai encontrar a solução! Nesses casos, só algum grandão para
resolver! Deixe-me pensar!

REI DO TABLADO, \emph{sussurrando} -- Estou morrendo de rir, Sr. Miguel,
morrendo de rir.

MIGUEL O HOLANDÊS, \emph{sussurrando} -- Cale a boca, Rei do Tablado, e
vá cantar a ``Guarda do Reno''.\footnote{\emph{Die Wacht am Rhein} é uma
  canção nacionalista alemã, composta em 1854 por Karl Wilhelm
  (1815-1873) a partir do poema de Max Schneckenburger (1819-1849),
  escrito em 1840, a propósito de um Reno, fronteira sagrada da nação,
  sob a guarda do alemão ``leal, piedoso e forte'' pronto a repelir as
  tentativas de invasão e de domínio francesas. \emph{Du Rhein bleibst
  deutsch wie meine Brust!} (Tu, Reno, permaneces alemão como o meu
  peito!). A canção celebrizou-se na Guerra Franco-Prussiana, tendo-se
  tornado desde então um hino patriótico alemão cuja popularidade fez-se
  notar de forma ainda expressiva quer na Primeira quer na Segunda
  Guerra Mundial. {[}N. dos E.{]}} Então, Peter Munk do Carvão, você não
ganhou todo aquele dinheiro do Anão de Vidro e conseguiu comprar uma
fábrica de vidro?

PETER MUNK DO CARVÃO -- Certo, Sr. Miguel o Holandês, era uma bela
fábrica de vidro a que eu tive.

REI DO TABLADO -- Sim, decerto, Peter Munk do Carvão, você teve, mas
perdeu tudo num átimo, em um jogo com o gordo do Ezequiel na mesa da
taberna. Certo, Ezequiel, ou não?

EZEQUIEL -- Ah, me deixe em paz com essa história, Rei do Tablado, nunca
mais quero ser lembrado dessa história!

LOCUTOR -- É verdade, Peter Munk do Carvão! Também me lembro disso. Você
perdeu toda a fábrica de vidro no jogo. Mas vocês têm de admitir -- não
foi uma grande tolice do Peter, esse desejo de ter sempre o mesmo tanto
de dinheiro no bolso quanto Ezequiel o Gordo? Assim foi inevitável que,
numa noite, o dinheiro dele acabasse e que, no dia seguinte, ele tivesse
vendido sua fábrica de vidro. Espere aí: ``tivesse vendido'' --
``tivesse vendido''? Está escrito aqui, na página 16! Graças a Deus,
encontrei de novo o fio da meada! Vamos, gente, podemos continuar!
Então, enquanto o oficial de justiça e o magistrado andavam pela oficina
de vidro para verificar e avaliar tudo para a venda, o nosso Peter Munk
do Carvão pensou: o Morro dos Pinheiros não está muito longe; se o anão
não me ajudou, quero fazer uma tentativa com o gigante. Ele correu até o
Morro dos Pinheiros com tanta rapidez como se os oficiais de justiça
estivessem no seu encalço. Quando passou pelo lugar onde havia
conversado pela primeira vez com o Anão de Vidro, sentiu-se como se uma
mão invisível o segurasse, mas se arrancou e continuou correndo até a
fronteira que havia guardado na memória. Sim, Peter, agora você vai ter
que se virar sozinho, pois não lhe invejo pelas coisas que vêm pela
frente.

PETER MUNK DO CARVÃO, \emph{sem fôlego} -- Miguel o Holandês, Sr. Miguel
o Holandês!

MIGUEL O HOLANDÊS, \emph{rindo} -- É você, Peter Munk do Carvão? Eles
estavam querendo lhe esfolar e vender sua pele aos credores? Ora, fique
tranquilo, toda sua miséria foi causada pelo Anão de Vidro, esse
separatista e hipócrita! Quando se dá um presente, tem que presentear
direito, e não como esse pão-duro! Vem, siga-me até a minha casa para
ver se a gente chega a um acordo!

PETER MUNK DO CARVÃO -- Acordo, Miguel o Holandês? O que eu poderia
negociar com o senhor? Não vai querer que seja seu servo? Que mais
poderia querer? E como o senhor quer que eu desça para esse abismo?

MIGUEL O HOLANDÊS, \emph{como se falasse num megafone} -- Sente-se na
minha mão e segure-se nos meus dedos para não cair.

\emph{Música com diversos ritmos do tique-taque de relógios, começando
baixo e aumentando cada vez mais}.

Pronto, chegamos! Sente-se na bancada do fogão e vamos tomar um bom gole
de vinho. Saúde, brinde comigo, pobre oficial-ajudante, parece que nunca
conseguiu sair dessa tristeza de Floresta Negra!

PETER MUNK DO CARVÃO -- Claro que não, Miguel o Holandês, como é que eu
faria isso?

MIGUEL O HOLANDÊS -- Então, nós madeireiros-jangadeiros somos
oficiais-ajudantes de outro tipo! Cada ano viajamos conduzindo as toras
rio abaixo, flutuando no belo Reno em direção à Holanda, e a isso se
juntam viagens a países distantes, como as que fiz no meu tempo livre.

PETER MUNK DO CARVÃO -- Quem dera se eu pudesse me dar a esse luxo!

MIGUEL O HOLANDÊS -- Depende só de você, melhorar de vida ou não, mas, é
claro, até hoje seu coração impediu que isso acontecesse.

PETER MUNK DO CARVÃO -- Meu coração?

MIGUEL O HOLANDÊS -- Quando você tinha coragem e força em todo o seu
corpo para tomar qualquer iniciativa, algumas batidas desse coração bobo
lhe deixavam trêmulo; e as ofensas e a desgraça, para que um rapaz
sensato deveria se preocupar com esse tipo de coisas? Você sentiu alguma
coisa na cabeça quando lhe chamaram de impostor ou de mau caráter? Seu
estômago doeu quando o oficial de justiça quis lhe expulsar de casa?
Diga-me, onde você sentiu a dor?

PETER MUNK DO CARVÃO -- No coração.

MIGUEL O HOLANDÊS -- Não me leve a mal, mas você desperdiçou centenas de
florins com mendigos miseráveis e outra gentalha. O que isso lhe trouxe?
Eles lhe desejaram a bênção de Deus e muita saúde; sua saúde melhorou
com isso? Você teria contratado um médico pela metade do dinheiro. A
bênção -- que bênção é essa quando tudo é penhorado e o expulsam! E o
que lhe levava a enfiar a mão no bolso sempre quando um mendigo estendia
seu chapéu esfarrapado? -- Seu coração, sempre seu coração, nunca seus
olhos, nem sua língua, nem seus braços ou suas pernas, mas sempre seu
coração; você só deu ouvidos ao seu coração.

PETER MUNK DO CARVÃO -- Mas, como a gente pode se acostumar a não dar
ouvidos a ele? Estou fazendo muito esforço para abafá-lo, mas meu
coração continua batendo e doendo.

MIGUEL O HOLANDÊS, \emph{rindo com sarcasmo} -- Claro, você, meu pobre
coitado, não pode fazer nada contra ele; mas basta me dar essa coisa que
bate e vai ver o quanto ganha com isso.

PETER MUNK DO CARVÃO, \emph{assustado} -- O que? Dar meu coração ao
senhor? Seria a minha morte imediata! Nunca!

MIGUEL O HOLANDÊS -- Claro, se um dos senhores cirurgiões quisesse tirar
seu coração numa operação, você certamente teria que morrer. No meu
caso, as coisas são um pouco diferentes. Mas venha comigo a esse quarto
e veja com seus próprios olhos!

\emph{Música: a fuga das batidas de coração.}

PETER MUNK DO CARVÃO -- Meu Deus! O que é isso?

MIGUEL O HOLANDÊS -- Olhe direito essas coisas nos vidros de formol!
Gastei um dinheirão com isso. Vá lá e leia os nomes nas etiquetas.

\emph{Depois de cada menção de nome, ouve-se a música correspondente.}

Lá nós temos o oficial de justiça e aqui Ezequiel o Gordo. Esse é o
coração do Rei do Tablado e aquele é do guarda-florestal. E aqui temos
toda uma coleção de corações que são de usuários e oficiais de
recrutamento. Olhe, todos eles se livraram dos temores e das
preocupações da vida. Nenhum desses corações bate mais com medo e
angustiado, e seus antigos donos estão muito contentes de ter expulsado
esse hóspede inquieto.

PETER MUNK DO CARVÃO, \emph{angustiado} -- Mas o que eles têm agora no
lugar do coração?

MIGUEL O HOLANDÊS -- Um coração de pedra muito bem trabalhado como este.

PETER MUNK DO CARVÃO, \emph{horrorizado} -- É verdade? Um coração de
mármore? Mas ouça, Sr. Miguel, esse coração deve estar frio no peito.

MIGUEL O HOLANDÊS -- Claro, mas é frio de uma forma agradável. Por que
um coração deveria estar quente? No inverno, o calor não tem utilidade
nenhuma; uma boa aguardente faz mais efeito do que um coração quente. E
no verão você não acredita o quanto um coração desses esfria o corpo. E,
como já disse, não há angústia, nem temor, não há compaixão tola, nem
outro lamento que atinja um coração desses.

PETER MUNK DO CARVÃO, \emph{relutante} -- E isso é tudo que o senhor
pode me dar? Eu estava esperando dinheiro e o senhor quer me dar uma
pedra!

MIGUEL O HOLANDÊS -- Então, penso que 100.000 florins seriam suficientes
para o primeiro momento. Se fizer bom uso desse dinheiro você logo vai
se tornar um milionário.

PETER MUNK DO CARVÃO, \emph{alegre} -- Ora, pare de bater com tanta
força no meu peito! Logo estaremos resolvidos. Bem, Miguel, dê-me a
pedra e o dinheiro e tire o tique-taque da sua casa!

MIGUEL O HOLANDÊS, \emph{alegre} -- Eu sabia que você era um rapaz
sensato. Vem cá, vamos tomar outro gole e depois vou desembolsar o
dinheiro.

\emph{A música dos corações passa para uma fuga de corneta.}

PETER MUNK DO CARVÃO, \emph{acordando e espreguiçando} -- Uah! Desta vez
dormi demais. Não foi uma corneta de postilhão que me acordou? Estou
acordado ou ainda estou sonhando? Parece que estou viajando; não é um
postilhão e não são cavalos lá na frente? Não estou sentado numa
carruagem? E as montanhas que estou vendo lá atrás, não é a Floresta
Negra? A minha roupa também não é mais a mesma. Por que não estou
ficando melancólico, já que estou saindo pela primeira vez das florestas
onde passei tanto tempo da minha vida? O que será que a minha mãe está
fazendo? Estranho, ela deve estar sem ninguém para ajudá-la e passando
necessidade; mesmo assim, esse pensamento não é capaz de tirar nem uma
lágrima do meu olho. Tudo ficou indiferente para mim. Como isso é
possível? Ah, claro, lágrimas e suspiros, saudade e melancolia são
coisas do coração e, graças a Miguel o Holandês, o meu é frio e de
pedra. Se ele cumprir a palavra com os cem mil tão bem quanto com o
coração, vou achar muito bom. De fato, tem aqui uma bolsa com milhares
de táleres e cartas de crédito de casas de comércio em todas as grandes
cidades.

\emph{Melodia de corneta}.

CONFUSÃO DE VOZES -- Frankfurt sobre Meno! Salsichas de Frankfurt! Casa
de Goethe! A Rádio de Frankfurt! O vinho de maçã! O Jornal de Frankfurt!
Bolos e biscoitos frankfurtianos! Frankfurt está cheia de curiosidades!

PETER MUNK DO CARVÃO -- O que há para comer e para beber aqui? Embrulhe
para mim algumas dúzias de salsichas, algumas canecas de vinho de maçã e
alguns quilos de bolos e biscoitos.

\emph{Melodia de corneta}.

CONFUSÃO DE VOZES -- Paris! Le Matin! Paris Midi! Paris Soir! Des
caiqouettes\footnote{Termo pseudo-francês, criado pelos autores da peça.
  {[}N. do T.{]}}, des caiqouettes et des caiqouettes! Louvre! Torre
Eiffel! Esquimaux, Pochettes! Surprises!

PETER MUNK DO CARVÃO, \emph{sonolento} -- Onde é que estamos? Ah, em
Paris! Então empacotem para mim uma boa quantidade de champanhe,
lagostas e ostras para eu não passar fome, nem sede!

UMA VOZ -- Quem é que é esse senhor sonolento, Sr. Postilhão?

POSTILHÃO -- Ah, esse é o Sr. Peter Munk do Carvão da Floresta Negra.
Ele já comeu e bebeu tanta coisa em Frankfurt que não consegue mais se
mexer.

\emph{Melodia de corneta}.

CONFUSÃO DE VOZES -- London! Britannia rules the waves! Ginger Ale!
Scotch Whisky! Toffies! Muffins! Morning Post! Daily News! The Times!
Turkey and Plumcake!

PETER MUNK DO CARVÃO \emph{ronca}.

UMA VOZ -- Quem é esse senhor que ronca, Sr. Postilhão?

POSTILHÃO -- É o Sr. Peter Munk do Carvão da Floresta Negra, ele já
comeu e bebeu tanta coisa em Paris que não consegue mais manter os olhos
abertos.

\emph{Melodia de corneta}.

CONFUSÃO DE VOZES -- Constantinopla! Visitem o Bósforo e o Corno de
Ouro! Tapetes! Que tal um narguilé? Aprendiz de fabricante de gaitas de
fole constantinopolitano! Manjar turco! Visitem os dervixes uivantes em
Galípoli nos minaretes da Hagia Sophia!

PETER MUNK DO CARVÃO \emph{ronca}.

UMA VOZ -- Quem é esse senhor que ronca, Sr. Postilhão?

POSTILHÃO -- É o Sr. Peter Munk do Carvão da Floresta Negra, ele já
comeu e bebeu tanta coisa nas paradas anteriores que de modo algum
consegue mais manter os olhos abertos.

\emph{Melodia de corneta}.

CONFUSÃO DE VOZES -- Roma! La Stampa di Roma! Il Corriere della Sera! Il
Foro romano! Il Coliseo! Giovinezza! Vino bianco e vino rosso!
Spaghetti! Polenta! Risotto! Frutti del Mare! Antiguidades! Visitem o
Papa e o Duce!

PETER MUNK DO CARVÃO \emph{ronca}.

UMA VOZ -- Quem é esse senhor que ronca, Sr. Postilhão?

POSTILHÃO -- É o Sr. Peter Munk do Carvão da Floresta Negra, ele já
comeu e bebeu tanta coisa nas paradas anteriores que de modo algum
consegue mais manter os olhos abertos.

\emph{Melodia de corneta}.

POSTILHÃO -- Hum, hum -- Cidade na Floresta Negra! Todo mundo descendo!

\emph{Gongo.}

LOCUTOR -- Eis o Peter Munk do Carvão em casa de novo. Vocês escutaram a
corneta anunciando a chegada do Postilhão. Mas enquanto entenderam bem,
espero, os nomes de todas as paradas que o Postilhão apregoou, vocês não
entenderam o último nome, o que não é por acaso. Nós não sabemos onde
mora o Peter Munk do Carvão. Não está escrito no livro do qual você,
Peter Munk, e você, Ezequiel o Gordo, e você, Schlurker o Altão, e você,
Miguel o Holandês, e você, Anão de Vidro, saíram. E não queremos ser
curiosos. Basta que esteja de novo em casa, na Floresta Negra de Baden.
Ele não deixa de perceber essas coisas, mas as percebe apenas na cabeça,
não no coração. Entende que voltou para casa, mas não o sente. Ele
também não tem mais nada a fazer. Seu carvoeiro não está mais aceso, ele
vendeu a fábrica de vidro e tem tanto dinheiro que seria uma burrice
aceitar qualquer trabalho. Aí está ele, para passar o tempo sai à
procura de uma mulher. Ele continua sendo um rapaz bonito. De fora não
dá para ver que tem um coração de pedra. Antigamente, quando ainda tinha
um coração de verdade, todos o amavam, e é disso que todo mundo se
lembra e quem se lembra disso em particular é Lisbete, a filha de um
pobre madeireiro. Ela vivia sossegadamente e sozinha, cuidava da casa do
pai com habilidade e zelo e nunca foi vista num baile, nem em
Pentecostes ou na Quermesse. Quando Peter ouve falar dessa maravilha da
Floresta Negra, resolve pedi-la em casamento e vai até a cabana que lhe
indicaram. O pai da bela Lisbete recebe esse senhor bem vestido com
espanto e fica mais espantado ainda quando fica sabendo que se trata do
rico Sr. Peter e que este quer ser seu genro. O pai não hesitou muito
tempo, pois achou que todas as suas preocupações e pobreza chegariam ao
fim e deu seu consentimento. E Lisbete, a boa filha, foi tão obediente
que se tornou a Sr.ª Peter Munk sem qualquer resistência. Lisbete não
tem dinheiro, mas leva um presente milagroso para a casa de Peter.
Trata-se de um relógio cuco, que pertence à família desde muitas
gerações. Esse relógio tem uma característica muito particular; não por
acaso as pessoas contam que o Anão de Vidro o deu uma vez a uma pessoa
querida. A particularidade do relógio é a seguinte: funciona como um
verdadeiro relógio cuco da Floresta Negra, batendo de hora em hora. Ao
meio-dia, porém, só dá doze batidas se não tiver uma pessoa malvada na
sala onde está pendurado. Mas se tiver uma pessoa malvada, ele bate
treze vezes. Estamos agora na sala onde está o relógio. O Peter Munk do
Carvão está sentado à mesa com a Lisbete.

\emph{O relógio dá onze batidas.}

LISBETE -- Onze horas? Vou ter que correr e colocar as cenouras no fogo.

PETER -- Cenouras de novo? Que nojo, diabos.

LISBETE -- Mas Peter, é seu prato preferido.

PETER -- Prato preferido! Prato preferido, toda essa comida não tem
graça nenhuma. Agora, se me trouxer um copo grande de conhaque...

LISBETE -- Você não sabe o que o Sr. Padre disse domingo passado, quando
falava dos beberrões?

PETER, \emph{batendo o pé} -- Vamos? Você vai trazer o conhaque ou não?
\emph{Ameaçando}: Ou...

LISBETE, \emph{choramingando} -- Aqui, para fazer sua vontade. Mas isso
não vai acabar bem.

PETER -- Basta que comece bem. A minha vida já é suficientemente triste.
Fico tão irritado quando ouço as pessoas falarem coisas sobre o domingo
ou sobre o bom tempo ou da primavera; estou achando-os totalmente
loucos.

LISBETE -- Você está sentindo dores?

PETER -- Não, mas esse é o problema, não sinto dores, nem alegria. Outro
dia até cortei o dedo e quase não senti nada. Foi quando serrei em
pedaços aquele baú que você ganhou da sua avó como presente de madrinha,
sabe?

\emph{Batidas na porta.}

PETER -- Não se mexa, não responda nada.

\emph{Batem pela segunda vez.}

PETER -- Que ele não ouse entrar sem eu mandar. E não vou mandar.

LISBETE -- Mas por que? Você nem sabe quem é.

PETER -- Um envio de dinheiro não vai ser. Mendigos miseráveis, nada
mais.

\emph{Batem.}

LISBETE -- Entre!

PETER -- Não falei, sua atrevida? Claro que é um mendigo.

MENDIGO -- Por favor, peço que me deem alguma coisa.

PETER -- Vai pedir ao diabo para que você fique com ele!

MENDIGO -- Tenha misericórdia, Senhora, me dê apenas um gole de água.

PETER -- Prefiro esvaziar toda a minha garrafa de conhaque na cabeça
dele em vez de dar um copo d'água.

LISBETE -- Deixe para lá, quero ir buscar para ele um gole de vinho, um
pão e uma pratinha para o caminho.

PETER -- Isso é bem típico seu, sua besta. Você não é capaz de seguir o
raciocínio do seu marido? De repente, até me acha cruel ou sem coração?
Não entende que ponderei tudo com muito critério? Será que você não sabe
o que acontece quando a gente deixa esse tipo de pessoa entrar na casa
da gente? É uma gentalha de mendigos. Uma pessoa conta para a outra.
Eles fazem uma marca na porta. São sinais de bandidos. Depois esperam a
oportunidade e levam tudo o que não está fixo e pregado. Se receber dois
ou três desses rapazes, um ano depois você vai dormir entre suas quatro
paredes peladas.

MENDIGO -- Pessoas ricas como vocês não sabem o quanto a pobreza dói e o
quanto um gole de água fresca faz bem nesse calor.

PETER -- Estou ficando cansado com esse falatório.

\emph{O relógio cuco começa a bater.}

LISBETE -- Céus! Me esqueci das cenouras! E o senhor pode levar tudo que
trago comigo e vai embora.

\emph{As batidas do relógio têm que soar alto e suceder-se bem devagar
para que as palavras da mulher possam ser ouvidas entre a primeira e a
segunda batida.}

PETER, \emph{pensativo, acompanha com uma voz sem ressonância, as
batidas} -- Um, dois, três, quatro, cinco, seis, sete, oito, nove, dez,
onze, doze.

\emph{Silêncio completo, ressoa a décima terceira batida. Ouve-se uma
queda abafada.}

LISBETE -- Meu Deus do céu, o Peter perdeu a consciência. Peter, Peter,
o que você tem? Acorde! \emph{Gemidos, suspiros e choro.}

\emph{Gongo.}

LOCUTOR -- O Peter não apenas perdeu a consciência, mas por pouco não
perdeu também sua vida por causa de sua arrogância e falta de fé. Agora,
depois do relógio dar a décima terceira batida, ele volta a si, cai em
si e resolve fazer seu terceiro e último pedido com o Dono do Tesouro,
desejando seu coração de volta. Vamos ver como isso acaba!

\emph{Gongo.}

PETER:

\begin{quote}
Dono do tesouro na verde floresta de pinheiros,

Já viveste séculos inteiros,

Tua é toda terra que pinheiros tem,

Só crianças de domingo é que te veem.
\end{quote}

ANÃO DE VIDRO, \emph{com uma voz abafada:} O que você quer de mim, Peter
Munk?

PETER -- Ainda tenho um desejo livre, Sr. Dono do Tesouro.

ANÃO DE VIDRO -- Corações de pedra ainda são capazes de desejar alguma
coisa? Você está com tudo que sua maldade exige; dificilmente vou
realizar seu desejo.

PETER -- Mas o senhor me concedeu três desejos; resta ainda um.

ANÃO DE VIDRO -- Se o desejo for tolo, posso negá-lo. Mas, tudo bem,
deixe-me ouvir o que deseja.

PETER -- Tire a pedra morta do meu peito e devolva-me meu coração vivo.

ANÃO DE VIDRO -- Fui eu quem fez o negócio com você? Sou eu Miguel o
Holandês, que presenteia as pessoas com riquezas e corações frios? Lá,
com ele, é que você tem que procurar seu coração.

PETER -- Ele nunca vai me devolver meu coração.

ANÃO DE VIDRO, \emph{depois de uma pausa.} Estou com pena de você, por
mais que você seja mau. Mas, como seu desejo não é tolo, pelo menos não
posso recusar minha ajuda. Consegue guardar um verso?

PETER -- Acredito que sim, Sr. Dono do Tesouro, mesmo tendo esquecido o
seu uma vez.

ANÃO DE VIDRO -- Então repita. Se esquecer, tudo estará perdido: ``Você
não é enviado da Holanda...'' Repita.

PETER -- ``Você não é enviado da Holanda.''

ANÃO DE VIDRO -- ``... Seu Miguel, mas é o inferno que lhe manda.''
Repita.

PETER -- ``Seu Miguel, mas é o inferno que lhe manda.'' Agora consegui,
Sr. Dono do Tesouro, muito bem, certamente se trata de uma fórmula
mágica. Quando Miguel o Holandês ouvir isso, não vai poder me fazer mal
algum.

ANÃO DE VIDRO -- Tudo bem, mas como vai ser?

PETER -- Como vai ser? Nada vai ser. Vou entrar na casa dele e gritar:

\begin{quote}
Você não é enviado da Holanda,

Seu Miguel, mas é o inferno que lhe manda.\footnote{No original alemão,
  há aqui um jogo de palavras entre \emph{Holland} (Holanda) e
  \emph{Höllenland} (inferno). {[}N. dos E.{]}}
\end{quote}

Aí ele não vai poder me fazer mal algum.

ANÃO DE VIDRO -- É bem sua cara. Está certo, ele não vai poder lhe fazer
mal algum. Mas logo depois de falar essas palavras, Miguel o Holandês
vai desaparecer. O Diabo vai saber para onde. Mas você vai olhar para
todos aqueles corações e não vai poder resgatar o seu.

PETER -- Ai meu Deus, como vou fazer então?

ANÃO DE VIDRO -- Isso eu não sei lhe dizer. Até hoje você refletiu muito
pouco na vida. Está mais do que na hora de começar. E agora vou ter que
cuidar dos meus pica-paus nos pinheiros; eles não causam tanta
preocupação quanto as crianças de domingo.

\emph{Gongo.}

LOCUTOR -- Ora, vou ter que dizer uma coisa a vocês: Se é para aguardar,
prefiro aguardar na terra dos homens do que na Terra da Voz. Aqui só tem
neblina. A gente não enxerga nada, só fica aguçando o ouvido, e isso já
venho fazendo durante horas. Porém, na floresta onde mora o Dono do
Tesouro, não há um só galho mexendo, nenhum pica-pau batendo, nenhum
ninho sussurrando. Mas tudo bem, que história é essa, de tanto tédio
acabo fazendo poesia. Agora estou ouvindo um estalo, ou será que é um
sussurro? É a voz do Dono do Tesouro ou a voz do Peter Munk do Carvão?

PETER MUNK DO CARVÃO, \emph{totalmente abatido e triste} -- Peter Munk
do Carvão.

LOCUTOR -- Mas isso não soa muito alegre.

PETER MUNK DO CARVÃO, \emph{totalmente abatido e triste} -- Parece que
você está fazendo o papel do eco aqui na floresta?

PETER MUNK DO CARVÃO, \emph{totalmente abatido e triste} -- Ô!

LOCUTOR -- Mas você não é uma companhia alegre na floresta. E o que
estou ouvindo lá longe? Parece que é a música assombrosa de vidro do
Miguel o Holandês. Mas por que você não responde? Por que você fica
mudo?

PETER MUNK DO CARVÃO, \emph{como acima:} Hum!

LOCUTOR -- Agora as coisas estão ficando confusas demais para mim.
Confusas e inseguras. Não me leve a mal, Sr. Peter, mas agora vou
procurar um novo caminho.

PETER MUNK DO CARVÃO, \emph{como acima:} Adeus! \emph{Ele bate na porta
e grita:} Miguel o Holandês!

\emph{Repete isso três vezes.}

MIGUEL O HOLANDÊS -- Que bom que você veio. Eu também não aguentaria
ficar com a Lisbete, esse muro de lamentações, que desperdiça todo esse
dinheiro com os mendigos. Quer saber de uma coisa? No seu lugar eu faria
outras viagens. Você fica fora por alguns anos e, quem sabe, quando
voltar para casa, a Lisbete já não estará mais.

PETER MUNK DO CARVÃO -- Você adivinhou, Miguel o Holandês, quero ir à
América. Mas para isso vou precisar de dinheiro; é muito longe daqui.

MIGUEL O HOLANDÊS -- Nada mais fácil, Peterzinho, você vai ter o que
precisa. \emph{Ouve-se o barulho das moedas e alguém contando:} 100,
200, 500, 800, 1.000, 1.200. Não são marcos, Peterzinho, só táleres.

PETER MUNK DO CARVÃO -- Miguel, você é mesmo um rapaz porreta, mas, no
fundo, um verdadeiro bandido -- mentindo desse jeito para mim, dizendo
que eu teria uma pedra no peito e você teria o meu coração.

MIGUEL O HOLANDÊS -- Mas não é assim? Você sente mesmo seu coração? Não
está frio como gelo? Você sente medo ou aborrecimento, alguma coisa é
capaz de lhe deixar arrependido?

PETER MUNK DO CARVÃO -- Você apenas fez com que meu coração ficasse
parado, mas eu tenho o mesmo coração de sempre no meu peito; e o
Ezequiel também -- ele me disse que você mentiu para nós. Você não é
homem o suficiente para arrancar o coração do peito sem a gente perceber
e sem risco; você teria que ser um mago.

MIGUEL O HOLANDÊS -- Mas eu lhe garanto que você e o Ezequiel e todas as
pessoas ricas que me procuraram têm um coração frio como você e que
tenho os verdadeiros corações aqui nesse cômodo.

PETER MUNK DO CARVÃO -- Mas com que facilidade você sabe mentir para as
pessoas! Conta isso para outro! Você não acha que, nas minhas viagens,
eu vi dúzias desses truques? Esses corações no seu cômodo são imitações
de cera. Concordo que você é um rapaz muito rico, mas não sabe fazer
magia.

MIGUEL O HOLANDÊS -- Entre e leia todas as etiquetas; aquele lá é o
coração do Peter Munk. Olha como está batendo! Alguém sabe fazer um
coração assim de cera?

PETER MUNK DO CARVÃO -- Ele é de cera sim; um coração de verdade não
bate assim. Tenho o meu ainda no meu peito. Não, fazer mágica você não
sabe.

MIGUEL O HOLANDÊS -- Vou provar que é verdade! Você vai sentir que esse
é seu coração. Aqui, vou colocar seu coração no seu peito! Como está se
sentindo agora?

PETER MUNK DO CARVÃO -- É verdade, você estava certo. Nunca teria
acreditado que isso fosse possível!

MIGUEL O HOLANDÊS -- Não é? E sei fazer magia sim; mas agora quero
colocar a pedra de volta.

PETER MUNK DO CARVÃO -- Devagar, Sr. Miguel! É com toucinho que se pega
os ratos, e desta vez, você é o enganado. Ouça o que tenho a lhe dizer.

\emph{Ele começa balbuciando para depois gritar com cada vez mais
coragem, frequência e velocidade sua fórmula mágica}:

\begin{quote}
Você não é enviado da Holanda,

Seu Miguel, mas é o inferno que lhe manda.
\end{quote}

\emph{Os corações ressoam alto. Gemidos de Miguel o Holandês.
Tempestade.}

PETER MUNK DO CARVÃO -- Agora esse malvado do Miguel o Holandês está se
contorcendo. Mas que tempestade horrível! Estou ficando com medo.
Rápido, para casa, para encontrar com a minha Lisbete.

\emph{Gongo.}

LOCUTOR -- Ah não, até a gente achar alguma coisa nesta Terra da Voz, é
um verdadeiro jogo de cabra-cega. Mas agora estou sentindo claramente,
isso deve ser a fábrica de vidro do Peter Munk do Carvão, e sua mulher
também não deve estar muito longe, pois de quem mais poderia ser essa
voz, senão da querida Lisbete!

LISBETE, \emph{cantando:}

\begin{quote}
Vidros de grave e agudo som,

Por que sozinha estou?

Por que foge Peter, meu amor,

Clandestinamente como um traidor?

Mas sei o que tenho a fazer,

Fraldas finas e sapatinhos tecer,

Para o filho de Peter, e tricotar,

Assim o tempo irá passar.

Vidros de som grave e agudo,

Primeiro a camisa, a meia em segundo,

Quando o bebê ao mundo chegar,

Tudo bem preparado vai encontrar.
\end{quote}

LOCUTOR -- Que coisa, parece pois que o Peter vai ser pai. Assim fica
duplamente injusto que ele passe tanto tempo fora de casa. Mas, para
mim, é uma boa oportunidade. Há quanto tempo que já queria conversar com
a Sr.\textsuperscript{a} Lisbete. Por que só conversaria o tempo todo
com o Peter na Terra da Voz? Mas como faço para chamar a atenção dela?
Não quero simplesmente chamá-la. A minha voz de urso a assustaria, mesmo
porque ela ainda está com a própria voz no ouvido, que soa tão delicada.

\emph{Pequena pausa.}

Já sei o que vou fazer. Apenas vou bater nos vidros.

\emph{Pequena música de vidro}

PETER MUNK DO CARVÃO -- Cá estou!

LISBETE E LOCUTOR -- Quem é?

PETER MUNK DO CARVÃO -- Tenho o meu coração de volta.

LISBETE -- O meu você sempre teve.

LOCUTOR -- Agora quero ir embora. Mas vocês têm que prometer uma coisa:
quando o pequeno Peter Munk do Carvão vier ao mundo, vocês vão escolher
o Dono do Tesouro como padrinho.

\emph{Pequena pausa. Nomes de meses são enumerados.}

Como o tempo passa aqui na Terra da Voz. Lá está o Peter no Morro dos
Pinheiros e diz seus versos.

\emph{Gongo.}

PETER MUNK DO CARVÃO:

Dono do tesouro na verde floresta de pinheiros,

Já viveste séculos inteiros,

Tua é toda terra que pinheiros tem,

Só crianças de domingo é que te veem.

Sr. Dono do Tesouro, escute-me; não quero outra coisa senão pedir que
seja o meu compadre quando meu filhinho nascer!

\emph{Vento.}

Então quero levar essas pinhas como lembrança, já que não quer aparecer.

LOCUTOR -- Crianças! Em que vocês acham que essas pinhas se
transformaram? Em um monte de novos táleres de Baden, e sem nenhum falso
entre eles. Esse foi o presente de padrinho do Anão do Pinheiral para o
pequeno Peter.

-- Agora vocês podem me agradecer. Não vocês, crianças que nos ouviram,
mas você Peter Munk do Carvão e o Dono do Tesouro e o Miguel o Holandês
e todo esse monte de personagens de Hauff\footnote{Benjamin faz aqui um
  jogo de palavras ``\emph{Haufen} \emph{Leute von Hauff''}. {[}N. dos
  E.{]}} que levei para a Terra da Voz a pedido deles e que deixei sãos
e salvos aqui na fronteira de novo.

EZEQUIEL -- Sãos e salvos? Até parece. No meu caso não se pode falar em
``são e salvo'' de maneira alguma, enquanto não receber o meu dinheiro
de volta.

LISBETE -- Credo, Ezequiel o Gordo, você não muda mesmo. Sou eu quem lhe
diz isso, a Lisbete.

LOCUTOR -- Deixe para lá, minha cara Senhora, ele vai receber tudo de
volta, centavo por centavo.

LISBETE -- Sim, Sr. Locutor, e gostaria de expressar-lhe a minha
gratidão pela música de vidro que tanto me alegrou. Pois foi o senhor
que tocou essa melodia graciosa nas garrafas, não foi?

LOCUTOR, \emph{com voz grave} -- Sim, sim.

LISBETE -- Durante um tempo, passei bastante medo quando não havia como
ir para frente e o senhor não sabia mais o caminho na Terra da Voz.

LOCUTOR -- Por favor, chegue mais perto, Sr.\textsuperscript{a} Lisbete.
Dê uma olhada aqui, na página..., onde o Hauff também faz uma pausa
grande. Que coincidência, imagine só, por acaso a nossa pausa aconteceu
exatamente na mesma passagem.

MIGUEL O HOLANDÊS -- Isso que chamo de sorte na má sorte.

LOCUTOR -- A pausa, então, foi o próprio autor que a fez. Por que será?
Essa história é como a cadeia montanhosa da própria Floresta Negra: seu
centro é como um pico do qual se olha os dois lados que descem, a saber,
para o lado do final infeliz e para o lado do final feliz.

CONFUSÃO DE VOZES -- Até a vista, Sr. Dono do Tesouro, minha cara
senhora, Sr. Peter etc.

MIGUEL O HOLANDÊS -- Esperem aí, por favor, fiquem só mais um momento,
meus senhores, por que tanta pressa? Não estou gostando de ter passado
uma impressão tão má. Por isso, queria chamar a atenção para outros
bandidos ainda piores. Leia, por exemplo, ``O navio dos fantasmas'', ``A
mão decepada'' e muitas outras histórias de Hauff, onde rapazes muito
piores que eu contribuem para um final feliz. Mas, tudo bem. Estou vendo
que todo mundo já se foi. Até a próxima!

LOCUTOR -- Até a próxima, Sr. Miguel. Que pessoas simpáticas. Mas agora
estou feliz de estar novamente sozinho no meu escritório. É, eu
pretendia realizar um Programa para Juventude. Será que foi mesmo para a
juventude?

\emph{Gongo.}

\textsc{A BERLIM DEMONÍACA}\footnote{``Das dämonische Berlin'', in GS
  VII-1, pp. 86-92. Tradução de Georg Otte. {[}N. dos E.{]}}\footnote{Palestra
  radiofônica para crianças datada de 25 de fevereiro de 1930. A cena da
  leitura infantil dos contos de E.T.A. Hoffmann, descrita neste texto,
  também aparece em \emph{Infância em Berlim por volta de 1900}. {[}N.
  da O.{]}}

Hoje vou começar com uma lembrança da época dos meus catorze anos.
Naquele tempo, eu era aluno em um internato. Como é usual nesses
institutos, os alunos e os professores se reuniam várias vezes de noite
durante a semana para tocar música, ouvir um discurso ou ler a obra de
um escritor. Numa noite, o professor de música conduzia a ``Capela'' --
esse era o nome dessa reunião noturna. Era um homem pequeno e engraçado,
com uma expressão inesquecível nos seus olhos sérios, com a careca mais
brilhante que já tinha visto, em torno da qual havia uma coroa de cachos
escuros e muito encaracolados. Seu nome é conhecido entre os amadores
alemães de música: ele se chama August Halm. Esse mesmo August Halm
entrou na Capela para ler histórias de E. T. A. Hoffmann para nós,
exatamente daquele escritor sobre o qual quero falar com vocês hoje. Não
me lembro mais qual texto ele leu, mas também não importa. Em
compensação, guardei com exatidão uma única frase de sua apresentação,
que introduziu a leitura. Essa frase caracterizava as obras de Hoffmann,
sua predileção pelo bizarro, excêntrico, fantasmagórico, inexplicável.
Acho que tudo que disse servia para despertar o interesse de nós meninos
pelas histórias que iriam se seguir. Mas depois ele concluiu com a frase
que não esqueci até hoje: ``Na próxima vez, vou contar a vocês para que
alguém escreve histórias desse tipo.'' Estou esperando por essa
``próxima vez'', e, como o bom homem faleceu nesse meio tempo, essa
explicação só chegaria até mim se fosse de uma maneira um tanto
amedrontadora, de modo que prefiro me antecipar a ela tentando cumprir
uma promessa que me fizeram 25 anos atrás.

Se eu quisesse trapacear um pouco, eu poderia facilitar as coisas para
mim. Bastava trocar o ``Para quê?'' pelo ``Por quê?'' e a resposta seria
muito simples. Por que o escritor escreve? Por mil motivos. Porque sente
prazer em inventar alguma coisa; ou porque é tomado por ideias ou
imagens tão impressionantes que só consegue se acalmar depois de
colocá-las no papel; ou porque carrega consigo perguntas e dúvidas pelas
quais encontra um tipo de solução no destino de pessoas fictícias, ou,
simplesmente, porque aprendeu a escrever; ou, e infelizmente é um caso
muito comum, porque não aprendeu nada. Não é difícil dizer por que
Hoffmann escrevia. Ele pertence àqueles escritores que são possuídos
pelas suas personagens: sósias, figuras assustadoras de toda espécie.
Quando os colocava no papel, os via de fato à sua volta; não apenas
quando escrevia, mas em meio à conversa mais inocente à mesa do jantar,
ao tomar vinho ou ponche, e mais de uma vez aconteceu de interromper sua
companhia de mesa com as palavras: ``Desculpe-me por interromper, meu
caro, mas está percebendo lá no canto ao lado direito esse menino
enfezado amaldiçoado aparecendo por debaixo do piso? Observe apenas as
cabriolas que o diabinho está fazendo! Olhe, olhe, agora sumiu! Oh, não
faça cerimônia, meu mais amável pequeno polegar, por favor, fique com a
gente... tenha a bondade de escutar a nossa conversa tranquila... O
senhor nem acredita o prazer que nos daria com sua companhia altamente
agradável... Ah, o senhor voltou... não gostaria de se aproximar um
pouquinho?... como assim?... o senhor gostaria de tomar alguma coisa?...
o que gostaria de dizer?... como?... o senhor se despede... seu criado
obediente.'' Etc. etc. E mal falava essas coisas desvairadas com olhar
fixo no canto, de onde tinha surgido a visão, ele levantava de repente o
olhar, dirigindo-se aos seus companheiros de mesa e pedia para
continuarem sossegadamente.

É assim que Hoffmann é descrito por seus amigos. E nós também nos
sentimos contaminados por esse modo de ser quando lemos histórias como
``A casa abandonada'', ``O morgado'', ``Os sósias'' ou ``A panela
dourada''. Havendo circunstâncias externas favoráveis, o impacto dessas
histórias de fantasma pode chegar a um grau muito surpreendente. Eu
mesmo passei por isso e a circunstância favorável, nesse caso, foi que
os meus pais me proibiram a leitura. Quando era pequeno, só podia ler os
contos de Hoffmann clandestinamente, de noite, quando saíam. Lembro-me
de uma daquelas noites de leitura, quando, sentado sozinho debaixo da
lâmpada da imensa mesa da copa -- era na Rua Carmer --, o silêncio era
completo. Enquanto lia ``As Minas de Falun'', todos os horrores se
reuniam aos poucos, em volta da minha mesa na escuridão, como peixes com
suas bocas mudas, de maneira que meus olhos se fixavam nas páginas do
livro como se fossem uma ilha que me salvava, das quais, no entanto, era
que saíam todos esses horrores. Ou, numa outra vez, era de dia e ainda
me lembro que estava parado diante do armário de livros entreaberto e
pronto para, ao mínimo barulho, jogar o volume de volta. Eu lia ``O
morgado'', com o cabelo em pé e o medo redobrado tanto por causa dos
horrores do livro, quanto pelo perigo de ser pego de surpresa, de modo
que não entendi nada de toda a história.

``O diabo não consegue escrever coisas tão diabólicas'', dizia Heinrich
Heine sobre os contos de Hoffmann. De fato: com o assombroso,
fantasmagórico e amedrontador desses textos anda de mãos dadas algo
satânico. E se tentarmos ir atrás disso, certamente conseguimos passar
da resposta do ``porquê'' dos contos de Hoffmann ao seu misterioso
``para quê''. Como se sabe, o diabo possui, ao lado de muitas outras
peculiaridades, também aquelas da esperteza e do saber. Ora, quem
conhece um pouco os contos de Hoffmann, logo vai entender quando digo
que, nessas histórias, o narrador sempre é um rapaz extremamente
sensível que detecta os espíritos por trás de seus disfarces mais sutis.
É com uma certa teimosia que esse narrador insiste em mostrar que todos
esses honrosos arquivistas, conselheiros de medicina, estudantes,
vendedoras de maçãs, músicos e filhas de boa família não são aquilo que
aparentam, da mesma maneira que o próprio Hoffmann não era o magistrado
meticuloso e pedante, do qual tirava seu ganha-pão. Em outras palavras:
as figuras assombrosas e fantasmagóricas que aparecem nas histórias de
Hoffmann não foram simplesmente inventadas pelo narrador consigo mesmo
no silêncio de seu quarto. Como no caso de muitos grandes escritores,
ele não via o extraordinário pairando livremente no espaço, mas em
pessoas, coisas, casas, objetos, ruas etc. bem definidos. Talvez vocês
saibam que a gente chama de fisiognomonistas aqueles que veem o caráter,
a profissão, ou até mesmo o destino de outras pessoas no rosto, no
andar, nas mãos ou na forma da cabeça destas. Por isso, Hoffmann era
menos um vidente (\emph{Seher}) do que um observador (\emph{Anseher}),
que é a tradução certa para fisiognomonista (\emph{Physiognomiker}) em
alemão. Um dos seus principais objetos de observação era Berlim, a
cidade e as pessoas que moravam nela. Na introdução à ``Casa
abandonada'', que, na verdade, era uma casa na avenida \emph{Unter den
Linden}\footnote{Sob as Tílias (\emph{Unter den Linden}) é a avenida
  central de Berlim. {[}N. do T.{]}}, ele fala, com certo humor amargo,
do sexto sentido que lhe fora dado, isto é: do dom de enxergar em
qualquer fenômeno, seja ele uma pessoa, uma ação ou um acontecimento,
aquela extravagância à qual nós, na nossa vida cotidiana, normalmente
não temos acesso algum. Era sua paixão flanar pelas ruas, contemplar as
figuras que encontrava e, quem sabe, fazer o horóscopo delas. Durante
dias inteiros, ele andava atrás de pessoas que lhe eram desconhecidas,
mas que tinham algo estranho no andar, na sua maneira de se vestir, no
tom de sua voz ou no olhar. Ele se sente em contato permanente com o
suprassensível e, ao invés de perseguir o mundo dos espíritos, é esse
mundo que o persegue. Nessa Berlim tão racional, esse mundo põe-se no
seu caminho ao meio-dia, vai atrás dele em meio ao barulho da Rua König
(\emph{Königstraße}), até chegar ao pouco que restou da Idade Média nas
proximidades da Prefeitura em ruínas, o faz sentir um cheiro misterioso
de rosas e cravos na Rua Grün (\emph{Grünstraße}) e enfeitiça para ele o
ponto de encontro da alta sociedade, \emph{Unter den Linden}. Poderíamos
chamar Hoffmann o pai do romance berlinense, cujos rastros se perderam
mais tarde em generalidades, quando passaram a chamar Berlim de
``capital'', o zoológico de ``parque'', o Spree de ``rio'', até surgir
novamente em nossos dias -- basta pensar em \emph{Berlin
Alexanderplatz}, de Döblin. Uma das suas figuras, entre as quais ele
imagina a si mesmo, diz a outra: ``Você teve um motivo bem definido para
localizar a cena em Berlim e para mencionar ruas e praças. A meu ver,
não há mal nenhum em designar o cenário com exatidão. Além de conferir
ao todo uma aparência de verdade histórica que estimula uma imaginação
preguiçosa, ele também ganha consideravelmente em vivacidade e frescor,
especialmente para alguém que o conheça.''

Certamente eu conseguiria contar para vocês muitas histórias em que
Hoffmann dá provas de sua qualidade de fisiognomonista de Berlim. Eu
poderia nomear as casas que aparecem na sua obra, a começar pelo próprio
apartamento na Rua Charlotte (\emph{Charlottenstraße}), esquina com a
Rua Taube (\emph{Taubenstraße}), até a ``Águia de Ouro'' na Praça
Dönhoff, e Lutter \& Wegner na Rua Charlotte etc. Mas acredito que seja
mais vantajoso investigar a maneira como Hoffmann analisava Berlim e a
impressão que disso restou em seus contos. Ele nunca foi muito amigo da
solidão, nem da natureza livre. O homem, a comunicação com ele, as
observações sobre ele, o mero olhar para as pessoas valiam mais para
Hoffmann do que qualquer outra coisa. Quando dava um passeio no verão,
algo cotidiano na parte da tarde, quando o tempo estava bom, ele sempre
ia para alguma praça pública para encontrar pessoas. Mesmo estando a
caminho, dificilmente havia uma taberna ou uma confeitaria onde não
entrava para ver se tinha pessoas e quem seriam. Mas não se tratava
apenas de procurar nesses lugares por rostos novos que lhe dessem ideias
curiosas: a taberna era antes para ele uma espécie de laboratório
literário, uma sala de experimentos onde testava todas as noites com
seus amigos o enredo emaranhado e os efeitos de seus contos. Pois
Hoffmann não era um autor de romances, mas um contador de histórias e,
até nos livros, muitas delas, talvez a maioria, apresentam um personagem
em cuja boca ele as coloca para serem contadas. No fundo, claro, esse
contador sempre é o próprio Hoffmann, que se senta com seus amigos à
mesa, onde cada um, em sequência, conta o que tem de melhor. Por isso,
um deles nos diz expressamente que Hoffmann nunca foi ocioso na taberna,
como tantas outras pessoas que vemos sentadas, só bicando seu vinho e
bocejando. Pelo contrário: ele ficava olhando à sua volta com seus olhos
de falcão; tudo que observava de ridículo, de diferente e até de
comovente nos clientes da taverna se transformou em um estudo para suas
obras, ou então o colocava -- Hoffmann era um desenhista muito hábil --
com sua pena expressiva no papel. Mas aí quando as pessoas que se
encontravam na taverna não lhe eram agradáveis; quando cabeças limitadas
e mesquinhas na roda o perturbavam! Nesses casos, ele deve ter sido
totalmente insuportável e fazia uso inteiramente temível de sua arte de
fazer caretas, causar constrangimento e assustar pessoas. O cúmulo do
horror, entretanto, eram para ele as chamadas tertúlias estéticas, que
estavam na moda nessa época em Berlim e que eram reuniões de pessoas
espirituosas, porém ignorantes e irrefletidas, muito presunçosas quanto
ao seu interesse por arte e literatura. Uma dessas tertúlias ele
descreveu de forma muito graciosa em suas \emph{Fantasias}.

Chegando agora ao fim, não queremos que alguém nos critique por ter
esquecido a pergunta pelo para quê. Tão pouco a esquecemos que, de forma
imperceptível, até já a respondemos. Para que Hoffmann escreveu essas
histórias? Certamente ele não perseguia conscientemente determinados
fins com essas histórias. Contudo, podemos lê-las como se ele tivesse se
proposto a tais fins. E esses fins não têm como ser outros a não ser
fisiognomônicos, ou seja, para mostrar que essa Berlim rasa, sóbria,
esclarecida e sisuda não reside apenas em seus cantos medievais, ruas
afastadas e casas abandonadas, mas também em seus habitantes
trabalhadores de todas as classes e em todos os bairros cheios de coisas
que atraem um contador de histórias e que basta seguir-lhes os rastros e
observá-las. E, como se Hoffmann realmente quisesse ensinar isso ao seu
leitor com sua obra, uma de suas últimas histórias, que ditou no leito
de morte, na verdade não é outra coisa senão um desses ensinamentos do
olhar fisiognomônico. Essa história se chama \emph{A janela de esquina
do meu primo}.\footnote{Há edição brasileira. Cf. E.T.A. Hoffmann,
  \emph{A janela de esquina do meu primo}.Tradução de Maria Aparecida
  Barbosa. São Paulo: Cosac Naify, 2010. {[}N. do T. e dos E.{]}} O
primo é o próprio Hoffmann, a janela é a janela de esquina do seu
apartamento que dava para a feira do Gendarm (\emph{Gendarmenmarkt}). Na
verdade, essa história é um diálogo. Hoffmann, paralítico, sentado em
sua cadeira de braços, olhando para a feira semanal lá embaixo, instrui
seu primo, que estava de visita, sobre como se poderia rastrear na
roupa, na velocidade e nos gestos das vendedoras e de suas freguesas uma
série de coisas, e fantasiar e inventar mais coisas ainda. E depois de
termos dito tanto em homenagem a Hoffmann, gostaríamos de deixar
registrado por fim algo que os berlinenses nem suspeitam: ou seja, que
ele foi o único escritor que tornou Berlim famosa no exterior e que os
franceses o amavam e liam numa época em que na Alemanha e também em
Berlim nem um cachorro aceitaria um pedaço de pão da sua mão. Hoje isso
mudou, há uma grande quantidade de edições acessíveis e também mais pais
que em meu tempo que permitem aos seus filhos a leitura de Hoffmann.

\textbf{CONTO E CRÍTICA}

\textsc{CONVERSA ASSISTINDO AO CORSO}

\textsc{Ecos do carnaval de Nice}\footnote{``Gespräch über dem
  \emph{Corso}. Nachklänge vom Nizzaer Karneval'', in GS IV-2, pp.
  763-771. Tradução de Georg Otte. {[}N. dos E.{]}}\footnote{Texto
  publicado na \emph{Frankfurter Zeitung} de 24 de março de 1935 sob o
  peseudônimo de Detlef Holz. {[}N. da O.{]}}

Era terça-feira de carnaval em Nice. Em silêncio, eu já havia virado as
costas para o carnaval e dava um passeio até o porto para descansar das
impressões do dia anterior, assistindo ao movimento há muito costumeiro
que sempre acompanha a chegada e a partida dos navios. Meio sonhando,
acompanhava o trabalho dos estivadores que descarregavam o
\emph{Napoleão Bonaparte}, de Ajaccio, quando um tapa no ombro me
surpreendeu.

-- Que coincidência feliz encontrá-lo por aqui, Doutor! Queria mesmo
localizá-lo de qualquer maneira. Quando perguntei pelo senhor no hotel,
já tinha saído.

Era o meu velho amigo Fritjof, que mora em Nice há anos e que toma conta
de mim nas minhas raras aparições nessa cidade, assim como faz com os
forasteiros, aos quais, quando lhe são simpáticos, mostra a cidade velha
e os arredores.

-- Pois, há alguém esperando pelo senhor, ele explicou depois de nos
cumprimentarmos.

-- Mas onde? -- perguntei um tanto desconfiado. Quem?

-- Na cafeteria M., no \emph{Casino Municipal}, como o senhor sabe, onde
se tem a melhor vista para o corso.

Como já disse, eu estava pouco interessado na vista para o corso. Mas a
descrição que Fritjof me dera de um amigo dinamarquês, ao qual havia
prometido me apresentar e que foi o motivo por ter me procurado,
deixou-me curioso.

-- Ele é escultor, disse, um velho companheiro de viagem. Encontrei-o em
1924 na ilha de Capri, em 1926 na de Rodes, em 1927 na de Hiddensee e a
última vez na de Formentera. Ele pertence à estranha espécie de pessoas
que passa a maior parte de sua vida em ilhas e nunca se sente em casa
quando está no continente.

-- No caso de um escultor, esse tipo de vida me parece duplamente
surpreendente, disse eu.

-- Escultor, disse meu amigo, é assim que o chamei. Mas certamente não
se trata de um escultor qualquer. Não acredito que, em algum momento,
ele tenha assumido um serviço. Seus recursos lhe permitem uma vida muito
independente. Aliás, nunca vi qualquer obra dele. Mas, onde o encontrei,
todos falavam dele, principalmente os nativos. Circulava o boato de que
esculpia suas obras diretamente no rochedo, ao ar livre e em regiões
montanhosas distantes.

-- Um artista da natureza, por assim dizer?

-- Na ilha de Rodes o chamavam de ``o bruxo''. Não deve ser tão sério
assim. Mas, com certeza, trata-se de um excêntrico. Aliás, o melhor
mesmo é fingir que não sabe a profissão dele, pois não gosta de falar
sobre si mesmo. Desde que o conheci, há dez anos, só me lembro de uma
única conversa em que tocou nesse assunto. Na época, entendi pouca
coisa, mas ficou claro que tudo que fazia tinha que beirar o gigantesco.
Não sei bem o que achar disso, mas parece que as formações rochosas o
inspiram. Mais ou menos como nos tempos antigos, quando inspiravam a
fantasia dos camponeses ou dos pescadores, que viam nelas deuses,
pessoas ou demônios.

Havíamos atravessado a Praça Massena, que, nesse dia do último cortejo,
se mostrava livre de qualquer tráfego profano e que estava pronta para
receber os carros alegóricos que vinham das ruas laterais.

No primeiro andar da cafeteria, o dinamarquês acenou de uma mesa. Ele
era um homem pequeno, magro, porém com um aspecto agradável, com o
cabelo encaracolado e ligeiramente ruivo. A informalidade da
apresentação provavelmente era proposital da parte de Fritjof, e logo
estávamos sentados em poltronas confortáveis, cada um com seu copo de
Whisky. Um vendedor de jornais com um chapéu pontudo de palhaço dava a
volta pela cafeteria.

-- Cada carnaval tem seu mote, Fritjof explicou, \emph{Le cirque et la
foire} é o mote deste ano, ``A Feira e o Circo''.

-- Nada inábil, disse eu, aproximar as diversões do carnaval àquelas
mais populares.

-- Nada inábil, o dinamarquês repetiu, mas, mesmo assim, talvez um tanto
inadequado. ``A feira e o circo'' -- certamente são coisas próximas ao
ambiente carnavalesco. Mas não seriam próximas demais? O carnaval é um
estado de exceção, descendente dos dias das saturnais, quando o mais
inferior se transmudava em superior e o escravo era servido pelo senhor.
O estado de exceção, portanto, só se destaca claramente em oposição ao
estado normal das coisas. Isso certamente não vale para a feira. Eu
teria preferido outro mote.

-- De onde o senhor tiraria esse mote? -- Fritjof perguntou. Para onde
for, o senhor esbarra no extraordinário, que se transformou no nosso
prato de todos os dias. Sem falar das nossas condições sociais e
econômicas. Basta olhar para o mais próximo: veja o tinhoso lá embaixo,
com seu lápis de um metro atrás da orelha, que representa o ``Cronista
da Feira''. Ele não teria mais semelhança com um boneco de propaganda de
uma fábrica de lápis? Boa parte dessas criaturas gigantescas não
passaria a impressão de que saíram do centro iluminado de uma loja de
departamentos para se juntar a um cortejo de carnaval? Olhe apenas para
o grupo de carros que se aproxima vindo da esquerda! O senhor deve
admitir que representaria muito bem um exército na campanha publicitária
de uma fábrica de sapatos.

-- Aliás, não entendo o que esse grupo pretende representar, falei.

Uma série de carrinhos de mão se aproximava, cada um carregando uma
figura superdimensionada; uma como a outra estava deitada de costas,
esticando uma perna para o alto. Era a única perna que tinham, e no seu
topo havia um pé deformado, largo e chato. Não dava para ver se estava
calçado ou descalço, daquela distância.

-- Também não saberia o que achar delas, respondeu o dinamarquês, se não
tivesse tido contato com eles por acaso, no ano passado em Zurique. Eu
estava fazendo estudos na biblioteca e conheci a famosa coleção de
gravuras avulsas, que é um dos tesouros daquela biblioteca. É de lá que
conheço esses seres fabulosos. Pois é disso que se trata: seres
fabulosos que eram chamados de ``sciapodes'': pés-guarda-sóis. Sabia-se
que viviam no deserto, protegendo-se do sol incandescente com seu único
pé gigantesco. Na Idade Média, provavelmente eram apresentados também
nas feiras -- ou melhor: prometiam apresentá-los aos curiosos, junto com
as pessoas desfiguradas e os milagres da natureza.

Aos poucos, os carros passavam embaixo, seguidos por carros maiores,
puxados por cavalos. Havia o ``carro lotérico'' que se movia sobre seis
rodas da sorte; o ``carro do domador de peixes'', que fazia dançar seu
chicote por cima de pequenas baleias e peixes ornamentais gigantescos de
papel machê; o ``carro do tempo'', que, puxado por uma mula esquálida e
guiado por Cronos, mostrava as estações do verão e do inverno, da Europa
central e ocidental, na forma de senhoras farta e simbolicamente
vestidas.

-- Hoje, disse Fritjof virado para o dinamarquês, esses carros são
apenas estrados em rodas. Mas o senhor deveria tê-los visto antes de
ontem, quando se transformavam em baluartes. Entrincheirada atrás dos
bonecos monstruosos, a tripulação travava uma batalha contra o público
-- contra os meros ``espectadores'', que, nessas ocasiões, se tornam
alvo de todo o rancor que, no decorrer dos dias e dos anos, os
eternamente relegados despertam naqueles que se engajam -- mesmo que
seja apenas como figurantes do carnaval.

-- Os carros em si, disse o dinamarquês com um ar pensativo, simbolizam
alguma coisa. Os carros evocam a ideia do longínquo; é ela, ao meu ver,
que lhes confere seu poder, sua magia, e que um charlatão qualquer sabe
explorar tão bem, quando, para vender seu remédio contra a calvície ou
seu elixir da vida, monta sua venda num carro. Pois o carro é algo que
vem de longe. Tudo que vem de longe, tem um certo mistério.

Ao ouvir essas palavras, tive que pensar num pequeno livro curioso que,
pouco tempo antes, havia encontrado num sebo de Munique, debaixo de uma
pilha de outros livros que diziam respeito às técnicas de transporte e
vinham da administração de uma antiga cavalariça. Era intitulado \emph{O
carro e suas formas no passar dos tempos}. Eu tinha comprado esse livro
por causa de suas gravuras e de seu formato atraente, e raramente o
deixava para trás. Nesse momento, ele também estava comigo. Cansado de
assistir ao espetáculo embaixo, recostei-me na minha poltrona e comecei
a folheá-lo.

Havia reproduções de todo tipo de carro e carruagens e, em um anexo, até
o carro naval -- o \emph{carrus navalis}, do qual muitos acreditam
derivar a origem da palavra tão discutida ``carnaval''. Certamente, essa
derivação deve ser levada mais a sério do que a etimologia caseira dos
monges, que via na palavra uma alusão à Quaresma e a lia como
``\emph{carne vale}!'', ``adeus carne!'' Mais tarde, quando se analisava
as coisas mais de perto, lembraram-se do velho costume de rebatizar as
embarcações em cortejos solenes antes de colocá-las de novo na água após
as tempestades de inverno, e é assim que se descobriu o carro naval
latino.

Fritjof, que estava encostado no peitoril da janela junto com o
dinamarquês, gritava de vez em quando algumas palavras na minha direção:
os nomes das máscaras que passavam embaixo e que havia lido no programa
do cortejo. Algumas das figuras fantásticas que passei a imaginar,
sentado na frente do meu copo, provavelmente podiam concorrer com as que
ficavam balançando lá embaixo -- sem falar do fato que não estavam
deformadas por um número costurado nas costas como aqueles. Assim,
entreguei à minha fantasia a tarefa de imaginar alguma coisa a respeito
do ``domador domado'', dos ``cangurus boxeadores'', do ``vendedor de
castanhas'' ou da ``dama do Maxim''\footnote{Podem ser lidas, aqui,
  alusões a: \emph{O domador domado} (\emph{The tamer tamed}), peça
  teatral do dramaturgo seiscentista inglês, contemporâneo e colaborador
  de Shakespeare, John Fletcher; \emph{O canguru boxeador} (\emph{Das
  boxende Känguruh}), filme mudo de 1895 dirigido por Max Skladanowsky;
  e \emph{A dama do Maxim} (\emph{La dame de chez Maxim}), peça de
  Georges Feydeau encenada em 1899 e adaptada para o cinema em 1912,
  1923 e 1932, apenas para referirmos as adaptações decorridas no tempo
  de vida de Benjamin. {[}N. dos E.{]}}, até que uma música estridente
de uma banda de metais me assustou.

Ela anunciava a chegada do carro-chefe com o Príncipe do Carnaval.

Para fazer jus ao mote do ano, haviam vestido o boneco gigantesco com um
uniforme de domador. Em suas costas, um leão havia encostado suas patas
dianteiras -- o que não o impediu de maneira alguma de dar um sorriso
com seus 32 dentes. Era o sorriso familiar do velho Quebra-Nozes. Mas
senti subitamente a tentação de seguir seus rastros até o canibal
quebra-ossos dos meus livros de infância, que também sorria mostrando
sua dentadura inteira enquanto saboreava sua refeição.

-- O sorriso exagerado desse boneco não é repugnante? -- perguntou
Fritjof a mim, apontando para o boneco cuja cabeça simplória acenava
chegando até a altura da minha poltrona.

-- O exagero, respondi, parece-me ser a alma das figuras carnavalescas.

-- O exagero, replicou o dinamarquês, só nos é repugnante em alguns
casos porque não temos força suficiente para apreendê-lo. A rigor, eu
deveria ter dito que ``não temos inocência suficiente''.

Lembrei das excentricidades que o meu conhecido havia me contado sobre o
trabalho do dinamarquês. Não foi, portanto, sem alguma esperança de
despertar nele a discordância quando, do modo mais natural possível,
disse: ``O exagero é necessário; somente ele se faz crível aos bobos e
desperta a atenção dos desatentos.''

-- Não, disse o dinamarquês, e vi que o havia atingido, as coisas não
são tão simples assim. Ou será que deveria dizer ``são mais simples do
que isso''? Pois faz parte da natureza das coisas, mesmo quando não se
trata de coisas cotidianas. Da mesma maneira que existe um mundo de
cores além do espectro visível, há um mundo de criaturas além dos seres
que conhecemos. Qualquer tradição popular as conhece.

Nesse meio tempo, ele havia se aproximado de mim e, sem interromper sua
fala, sentou-se ao meu lado.

-- Pense nos gigantes e nos anões. Se há uma maneira do corpóreo poder
simbolizar algo espiritual, nada mais significativo do que nessas
criaturas das lendas populares. Há duas esferas de inocência completa, e
estas ficam nas duas fronteiras onde a nossa estatura humana normal,
digamos, se altera em direção ao gigantesco ou ao minúsculo. Pois, tudo
o que é humano carrega culpa. Mas as criaturas gigantescas são
inocentes, e a grosseria de um Gargântua e de um Pantagruel, que, aliás,
fazem parte da dinastia dos príncipes carnavalescos, é apenas uma prova
excessiva disso.

-- E a essa inocência corresponderia, disse eu, a inocência da pequenez?
Isso me faz lembrar a ``Nova Melusina'', de Goethe\footnote{Há tradução
  para o português. Cf. Johann Wolfgang Goethe, \emph{A nova Melusina.
  Novela ou a história de uma caçada}. Tradução de Anneliese Mosch.
  Sintra: Colares Editora, 1997. {[}N. dos E.{]}}, aquela princesa na
caixinha, cujo esconderijo, assim como seu canto mágico e sua natureza
ínfima, sempre me pareceu incorporar da forma mais perfeita o reino da
inocência -- da inocência infantil, quero dizer, que certamente é
diferente daquela do império dos gigantes.

-- Olhem só, Fritjof nos interrompeu da janela, esse grupo inverossímil!

De fato, era um carro estranho que estava passando, ao cair do sol,
diante das arquibancadas. Na frente de uma parede ou de um biombo, nas
quais tinham pendurado algumas pinturas, estavam pintores com paleta e
pincel, e parecia que estavam prestes a terminar sua obra. Mas um grupo
de bombeiros, com a mangueira na mão, os acuou e ameaçou inundar as
obras-primas e seus criadores.

-- Não faço a mínima ideia do que poderia se tratar, confessei.

-- É o ``\emph{Car des pompiers}'', disse Fritjof. Eles chamam um pintor
acadêmico e arrogante de \emph{pompier}, que também é a palavra para
``bombeiro''. Um jogo de palavras em cima de um carro -- pena que é um
caso isolado.

Nesse momento, ainda antes de escurecer, as fachadas em torno da Praça
Massena começaram a se iluminar. A grade, com a qual a haviam cercado e
que estava enfeitada com todo tipo de símbolos da feira e do circo, com
peças de madeira recortadas e pregadas, de repente foi engolida pelo
fogo de lanternas multicores. Onde antes havia um leão, agora tinha uma
jaula com luzes amarelas e em sua silhueta duas lanterninhas
avermelhadas insinuavam os movimentos da boca da fera. E a menina de
madeira, que antes parecia tomar conta de uma barraca, agora tinha se
transformado numa imagem da deusa Astarte. Mais estranho, entretanto, do
que o jogo das luzes na fachada era o que elas tinham a dizer à própria
Praça, pois eram elas que lhe proporcionavam seu verdadeiro destino.
Logo ficou claro que ela fazia parte daquela série grande e nobre de
praças europeias, em forma de sala, que começaram a surgir na Itália e
graças às quais as festas italianas com seus \emph{corsi} e procissões
-- sem falar do carnaval -- se tornaram modelos para a Europa inteira.
Essas praças tinham como função não apenas abrigar, nos dias úteis, as
feiras e reuniões populares, mas de representar, nos dias de festa, a
sala -- uma sala solenemente iluminada debaixo do céu noturno, que não
devia nada ao palácio do duque, com seu revestimento e seu telhado feito
de materiais preciosos. Era uma praça dessas que estávamos olhando
agora, calados.

Depois de um longo intervalo, o dinamarquês dirigiu-se a mim.

-- O senhor falou antes do mundo do delicado e do diminuto que Goethe
trabalhou na ``Nova Melusina'', e achou que esse, ao contrário do mundo
dos gigantes, seria o mundo em que morasse a inocência infantil. Sabe
que tenho lá as minhas dúvidas? A inocência da criança, a meu ver, não
seria uma inocência humana, se não fizesse parte dos dois reinos, tanto
dos gigantes quanto dos anões. Não pense apenas na delicadeza e em quão
comovente é a cena de crianças fazendo uma hortinha na areia ou
brincando com um coelho. Pense também no outro lado -- na grosseria, no
desumano que predomina nos seus livros infantis mais famosos e que não
apenas fez o sucesso de \emph{Max e Maurício} ou do \emph{João
Felpudo},\footnote{\emph{Max und Moritz} (1865), do escritor e
  desenhista Wilhelm Busch, e \emph{Struwwelpeter} (1845), do médico e
  psiquiatra Heinrich Hoffmann, tornaram-se os livros infantis mais
  populares da Alemanha. Há edições brasileiras. Em se tratando do
  primeiro, cf. Wilhelm Busch, \emph{As travessuras de Juca e Chico}.
  Tradução de Claudia Cavalcanti. São Paulo: Iluminuras, 2012. Veja-se,
  igualmente, a conhecida tradução de Olavo Bilac, originalmente
  publicada em 1901 e muitas vezes reeditada, a exemplo de \emph{Juca e
  Chico. História de dois meninos em sete travessuras}. São Paulo: Pulo
  do Gato, 2012. Quanto ao segundo, cf. Heinrich Hoffmann, \emph{João
  Felpudo, ou histórias divertidas com desenhos cômicos do Dr. Heinrich
  Hoffmann}. Tradução de Claudia Cavalcanti. São Paulo: Iluminuras,
  2011, para apenas referirmos uma edição recente. {[}N. do T. e dos
  E.{]}} mas também os tornou úteis. Pois essas características se
apresentam nesses livros em sua inocência. Quero chamá-las de
``canibalescas'', que o senhor também pode detectar nos lábios do
Príncipe Carnaval. O maravilhoso das crianças é que elas conseguem
migrar, sem cerimônia, entre os dois reinos extremos do humano,
permanecendo num ou noutro, sem fazer a mínima concessão ao reino
contrário. Provavelmente é esse descomprometimento que perdemos mais
tarde. Somos muito bem capazes de nos inclinar para o mundo minúsculo,
mas não conseguimos mais nos perder nele. E podemos nos divertir com o
mundo gigantesco, mas sempre com um certo embaraço. As crianças podem
ser tímidas no contato com os adultos, mas se movem entre esses gigantes
lá embaixo como se fossem seus pares. Para nós adultos, porém, pelo
menos uma vez ao ano, o Carnaval deveria ser a oportunidade para nos
comportarmos um pouco como os gigantes -- mais desinibidos e ao mesmo
tempo mais honestos do que somos, dia vai, dia vem.

Um foguete subiu ao céu; um tiro de canhão foi disparado: o sinal para a
queima do 57.º Príncipe Carnaval, de cuja fogueira a última centelha
tinha que se extinguir antes de começar a Quarta-Feira de Cinzas.

\textsc{A MÃO DE OURO}

\textsc{Uma conversa sobre o jogo}\footnote{``Die glückliche Hand. Eine
  Unterhaltung über das Spiel'', in GS IV-2, pp. 771-777. Tradução de
  Georg Otte. {[}N. dos E.{]}}\footnote{Texto provavelmente escrito logo
  depois da publicação de ``Conversa assistindo ao corso'', como uma
  continuação para a \emph{Frankfurter Zeitung}. O manuscrito foi
  assinado com o pseudônimo de Detlef Holz. {[}N. da O.{]}}

-- Pois é, a gente tem que ter mão de ouro, falou o dinamarquês.

-- Eu poderia contar-lhe uma história...

-- Nada de histórias! -- interrompeu o dono do hotel. Quero saber sua
opinião: o senhor acha que, no jogo, tudo depende do acaso ou ainda há
mais alguma outra coisa?

Éramos quatro. Meu velho amigo Fritjof, o romancista; o escultor
dinamarquês, que Fritjof havia me apresentado em Nice; o sabido e
viajado proprietário do hotel, em cujo terraço tomávamos o chá da tarde
-- e eu. Não me lembro mais como o assunto do jogo surgiu. Mal
participava da conversa, pois estava me entregando ao sol da primavera e
ao prazer de ter encontrado aqui, em Saint-Paul, longe de tudo, os meus
amigos de Nice.

A cada dia que passava entendia melhor por que Fritjof havia escolhido
esse lugar para retomar o trabalho no seu romance, que não tinha saído
do lugar enquanto esteve em Nice. Pelo menos foi isso que deduzi do
sorriso indefinido com que respondera naquela cidade semanas atrás a
minha pergunta pelo romance: ``Perdi minha caneta tinteiro''.

Logo depois eu parti; tanto maior foi a minha alegria de rever Fritjof e
seu companheiro dinamarquês aqui -- uma alegria misturada com uma
surpresa: será que Fritjof, um pobre diabo, havia conseguido mesmo se
alojar nesse confortável hotel?

Agora estávamos sentados nesse pequeno refúgio, deixando, enquanto
conversávamos, pairar os nossos olhares sobre os sinais de bandeiras
balançando com o vento em um varal, estendido acima do portal da cidade
ou das árvores escalonadas no vale.

-- Se o senhor quiser ouvir a minha opinião, disse o dinamarquês, nada
depende das coisas que comentamos até agora. Não depende do capital de
jogo, nem dos chamados ``sistemas'', nem do temperamento do jogador --
antes da falta de temperamento.

-- Agora não entendo mesmo o que o senhor quer dizer com isso.

-- Se tivesse acompanhado pessoalmente o que vivenciei no mês passado em
San Remo, o senhor me entenderia imediatamente.

-- Então? -- perguntei intrigado.

-- Eu entrei, começou a contar o dinamarquês, tarde da noite no casino e
me aproximei de uma mesa, onde uma partida de bacará tinha acabado de
começar. Havia um lugar livre, que estava reservado, e os olhares que se
dirigiam a ele davam a entender que alguém estava sendo aguardado. Eu já
queria me informar sobre o cliente que parecia criar uma expectativa tão
grande, quando alguém ao meu redor mencionou seu nome. No mesmo
instante, apoiada pelo atendente do casino de um lado e pelo seu
secretário do outro, a Marchesina Dalpozzo se aproximou da mesa. Pelo
visto, o trajeto do carro até sua poltrona havia custado à velha senhora
muito esforço. Mal chegou e já se afundou na poltrona. Depois de um
tempo, quando o embaralhador de cartas chegou-se até ela e era sua vez
de gerir a banca, abriu sem pressa sua bolsa de mão para tirar uma
pequena matilha de cachorrinhos de porcelana, vidro e jade, seus
mascotes, que distribuiu em torno do si. Mais uma vez, reservando-se
todo o tempo do mundo, enfiou novamente a mão na bolsa e tirou um maço
de notas de mil liras. Deixou ao crupiê o trabalho de contá-las,
distribuiu as cartas, mas, mal passou a última, afundou-se novamente na
poltrona. Ela nem chegou a escutar o pedido de seu parceiro por mais uma
carta para melhorar as chances do seu jogo, pois havia adormecido. Neste
momento, o secretário entrou em cena para, com muito respeito, acordá-la
delicadamente com a mão, que estava visivelmente exercitada nesta
prática. Devagar, a Marchesina virou seus \emph{points} um após o outro.
``Nove para a banca'', disse o crupiê\footnote{No original alemão,
  ``\emph{Neuf à la banque}'', em francês. {[}N. dos E.{]}}; ela havia
ganhado. Mas isso parecia apenas fazê-la adormecer e, por mais que ela
tenha aumentado a soma das notas de mil liras administrando a banca,
quase não houve uma única vez em que o secretário não tivesse que
exortá-la para a sorte.

-- A quem o ama, o Senhor concede o pão enquanto dorme.\footnote{Trata-se
  de uma expressão proverbial em alemão, cuja origem está nos Salmos, e
  que é utilizada ironicamente em referência a pessoas que acumulam
  lucros ``dormindo'', sem trabalhar: ``É inútil madrugar, deitar tarde,
  comendo um pão ganho com suor; a quem ama, o Senhor o concede enquanto
  dorme.'' (Sl 127, 2) {[}N. do T.{]}}

-- Não seria melhor dizer: ``Satanás''? -- observou o dono do hotel
sorrindo.

-- Os senhores sabem, disse Fritjof sem dar qualquer resposta, que já me
perguntei algumas vezes por que o jogo é tão mal-afamado? Claro que não
há nenhum segredo nisso. Suicídios, desvios de dinheiro e o que mais
tiver a origem no jogo não faltam. Mas, como já disse, isso é tudo?

-- Há alguma coisa no jogo que é contra a natureza, disse o dinamarquês.

-- Eu já o vejo muito próximo da natureza, observei. Acho-o tão natural
quanto a incansável e inesgotável esperança na nossa felicidade.

-- Esta é a palavra-chave, respondeu o dinamarquês: ``Fé, amor --
esperança.'' E agora o senhor entende o que sobrou disso!

-- O senhor quer dizer que o assunto não é digno da esperança. ``Vão
dinheiro'' ou algo parecido -- se estou entendendo direito.

-- Mas ele não me entende direito, disse o dinamarquês, virando de
repente as costas para mim e dirigindo-se a Fritjof. O senhor já se
encontrou alguma vez, ele prosseguiu, olhando-o fixamente, no metrô ou
num banco do parque, na proximidade imediata de uma mulher atraente? Mas
realmente na proximidade mais imediata?

-- Quero lhe dizer uma coisa, disse Fritjof: se ela estiver sentada
mesmo muito perto, o senhor dificilmente vai poder fazer uma ideia dela.
Por quê? Porque, quando estivesse muito próximo, seria quase impossível
olhá-la. De qualquer forma, se fosse comigo, acharia isso uma falta de
vergonha.

-- Então o senhor vai me entender tanto melhor se eu retomar agora a
nossa questão: conversamos sobre a esperança. E comparei a esperança a
uma mulher desconhecida jovem e bonita, sendo que captá-la com o olhar,
ou mesmo abordá-la com o olhar, seria falta de pudor.

-- Como assim, perguntei, pois estava perdendo o fio da meada.

-- Eu estava falando da proximidade temporal, disse o dinamarquês. Quero
afirmar que faz muita diferença se cultivo um desejo para o futuro
longínquo ou para o momento. ``O que se deseja para a juventude,
ter-se-á em abundância na velhice'', dizia Goethe. Quanto mais cedo na
vida se tem um desejo, tanto maior a perspectiva dele se realizar... Mas
estou divagando.

-- Provavelmente, o senhor estava querendo dizer, comentou Fritjof, que
aquele que aposta no jogo também tem um desejo.

-- Sim, mas um desejo que o próximo momento deve realizar. E é isso que
o transforma em mal-afamado.

-- É um contexto estranho, disse o dono do hotel, em que coloca o jogo.
E o contrário da bola de marfim na roleta, que acaba rolando em sua
casa, seria a estrela cadente que cai ao longe para realizar um desejo.

-- Sim, o desejo justo, que se dirige para algo longínquo, disse o
dinamarquês.

Ditas essas palavras, houve um silêncio. Mas essas palavras haviam
jogado uma nova luz no velho provérbio ``Azar no jogo, sorte no amor''.
Como se quisesse intervir nas minhas divagações, Fritjof, com ar
pensativo, comentou:

-- Uma coisa é certa: no jogo há estímulos que vão além do ganho.
Algumas pessoas não procurariam no jogo uma luta contra o destino? Ou
uma oportunidade de cortejá-lo? Acredite, no pano verde são acertadas
muitas contas às quais os outros nunca tiveram acesso.

-- Deve ser mesmo uma tentação muito grande pôr à prova a aceitação do
próprio destino.

-- Isso pode ter um fim bastante ambíguo, disse o dono do hotel.
Lembro-me de uma cena que observei em Montevidéu. Quando era jovem,
passei boa parte da minha vida por lá. Montevidéu possui o maior casino
do Uruguai; as pessoas viajam oito horas de Buenos Aires até lá para
passar o \emph{weekend} jogando. Uma noite, estive no casino para
assistir aos jogos. Por precaução, não levei dinheiro nenhum. Na minha
frente, havia dois jovens que jogavam com fervor. Faziam apostas
pequenas, porém frequentes. Mas não davam sorte, e logo um dos dois
tinha perdido tudo. O outro ainda tinha algumas fichas, que, no entanto,
não queria arriscar. Portanto, deram fim ao jogo deles, mas permaneceram
para observar o jogo dos outros. Assim, como muitos outros perdedores,
ficaram parados em silêncio e numa atitude humilde, quando aquele que já
não tinha mais nada se animou de repente e sussurrou ao seu amigo:
``Trinta e quatro!'' O outro se limitou a encolher os ombros. Mas, de
fato, deu trinta e quatro. O nosso vidente, cujo sofrimento,
naturalmente, era grande, fez outra tentativa. ``Sete ou vinte e
oito!'', murmurou para seu companheiro, que sorriu sem emoção. Mas deu
mesmo sete, e o outro começou a se irritar. Quase implorando, sussurrou
``Vinte e dois'', e o repetiu três vezes. Em vão: quando deu vinte e
dois, a casa estava sem aposta. Parecia inevitável que se fizesse uma
cena entre os dois. Mas, justo no momento em que o nosso homem dos
milagres, trêmulo de excitação, queria novamente se virar para o amigo,
este, para não ser mais uma entrave para a felicidade comum, passou-lhe
o restante das fichas. O amigo apostou no quatro. Deu quinze. Apostou no
vinte e sete. Deu zero. E apostou as últimas duas fichas e as perdeu de
uma vez só. Abatidos e reconciliados, os dois desapareceram.

-- Estranho, disse Fritjof. Podemos achar que o fato de segurar as
fichas na mão o teria privado do seu dom de vidente.

-- Da mesma maneira, explicou o dinamarquês, o senhor pode dizer que foi
o dom de vidente que o privou do lucro.

-- É um paradoxo um tanto volátil, repliquei.

-- De maneira alguma, respondeu. Se existe algo como um jogador feliz,
algo como um mecanismo telepático nos envolvidos com o jogo, isso está
no inconsciente. É o saber inconsciente que se põe em movimento quando o
jogador tem sucesso. Quando, porém, se desloca para a consciência, está
perdido para a inervação. Embora o nosso jogador vá ``pensar'' o certo,
``agirá'' errado. Vai ficar parado como muitos perdedores, arrancar os
cabelos e gritar ``Eu sabia!''

-- Na sua opinião, portanto, um jogador sortudo procede de forma
instintiva? Como uma pessoa no momento do perigo?

-- O jogo, o dinamarquês confirmou, é mesmo um perigo produzido
artificialmente. E jogar é, por assim dizer, uma maneira blasfema de pôr
a nossa presença de espírito à prova. Pois, no momento do perigo, o
corpo comunica-se diretamente com as coisas, sem passar pela cabeça. Só
quando estamos a salvo e respiramos aliviados, chegamos a nos dar conta
do que aconteceu. Enquanto estávamos agindo, estávamos à frente do nosso
saber. E o jogo é mal-afamado porque provoca de uma maneira
inescrupulosa aquilo que o nosso organismo tem de mais refinado e
preciso.

Deu-se um silêncio. ``Tem que ter mão de ouro...'', pensei. O
dinamarquês não estava querendo contar uma história a respeito?
Lembrei-o disso.

-- Ah, sim, a história, disse ele sorrindo. Na verdade, já é meio tarde
demais para ela. Aliás, conhecemos o herói dessa história. E todos nós
gostamos dele. Apenas quero revelar que é escritor, pois isso é
relevante, bem que... mas aí já estou quase tirando a graça da história.

Resumindo: o homem estava decidido a tentar sua sorte na Riviera. Não
tinha noção nenhuma dos jogos de azar, tentou um ou outro sistema e
acabou perdendo com todos eles. Depois desistiu dos seus sistemas e
continuou perdendo. Logo seus recursos estavam esgotados, seus nervos
muito mais, e um dia, além de tudo, aconteceu-lhe de perder também sua
caneta tinteiro. Como os senhores sabem, os escritores às vezes são
excêntricos, e o nosso amigo pertence aos mais excêntricos de todos. Ele
precisa ter uma iluminação bem definida sobre sua escrivaninha, um papel
bem definido e um formato bem definido para suas folhas de papel; caso
contrário, ele não consegue trabalhar. Agora os senhores podem
facilmente imaginar o que significava para ele a perda de uma caneta.
Depois de desperdiçar o dia inteiro à procura de uma nova, demos um pulo
no casino. Como nunca jogo, contentei-me em acompanhar o jogo do nosso
amigo. Em pouco tempo, não era apenas eu que o acompanhava, mas esse
homem chamou a atenção de muitos frequentadores do casino, pois ganhava
todos os jogos. Depois de uma hora, partimos para garantir, pelo menos
para aquela noite, o nosso patrimônio líquido. E o dia seguinte também
não deu prejuízo, pois tanto mais passamos em vão a manhã em papelarias,
tanto mais a noite resultou lucrativa. Claro que não se falava mais no
romance, desde que a caneta tinha desaparecido. O nosso amigo,
normalmente um homem assíduo, nem olhava mais para o manuscrito e deixou
de escrever até as cartas mais curtas. Quando o lembrava de alguma
correspondência urgente, ele se esquivava. Os seus apertos de mão se
tornaram escassos; ele evitava carregar o mais leve pacote e mal tinha
força para virar as páginas durante suas leituras. Era como se sua mão
estivesse descansando numa bandagem que só era tirada de noite -- no
casino, onde nunca ficamos por muito tempo. Já tínhamos juntado uma bela
soma, quando, um dia, o porteiro do hotel trouxe a caneta. Alguém a
havia encontrado entre as palmeiras do hotel. O nosso amigo deu uma boa
gorjeta ao porteiro, e no mesmo dia partiu para finalmente escrever seu
romance.

-- Uma bela história, disse o dono do hotel, mas ela prova o quê?

Eu nem queria saber o que a história provava ou não provava, mas me
regozijava em ver meu velho amigo Fritjof, para quem a sorte sorria
muito pouco, feliz da vida na muralha de Saint-Paul, tomando seu chá da
tarde.

\textsc{JAKOB JOB, \emph{NÁPOLES. IMAGENS DE VIAGEM E ESBOÇOS}}

\emph{Zurique: Rascher \& Cia. A.-G. 1928. 255 p. 32 figuras}\footnote{``Jakob
  Job, \emph{Neapel. Reisebilder und Skizzen}. Zürich: Rascher u. Cie.
  A.-G. 1928. 255 S., 32 Abb.'', in GS III, pp. 132-135. Tradução de
  Georg Otte. {[}N. dos E.{]}}\footnote{Resenha crítica publicada no
  periódico \emph{Die Literarische Welt} de 20 de julho de 1928. {[}N.
  da O.{]}}

Amar Nápoles olhando do mar é fácil. No entanto, basta pôr o pé em
terra, descer do trem na estação labiríntica e incandescente, andar num
carro destrambelhado, atravessando nuvens de poeira num pavimento que
descansa tão pouco quanto o Vesúvio, chegando, em vão, numa casa de
hóspedes superlotada, para que a página daquela primeira impressão vire.
Acrescenta-se a isso as experiências do primeiro dia para ver que poucas
pessoas conseguem encarar, sem disfarce, a imagem dessa vida, dessa
existência sem sossego nem sombra. Quem não se privar, ao pisar nesse
solo, de tudo que for associado ao conforto, enfrenta uma batalha fadada
à derrota. Já para os outros, que encontram nessa cidade a face mais
suja, mas também mais passional e horrorizada, da qual jamais a pobreza
olhou para sua libertação, a lembrança dela se resume a uma Camorra.
Para tudo que se conta sobre carteiras furtadas, moças sequestradas e
camas infestadas de percevejos só lhes resta um sorriso impassível. Caso
estes últimos acolham o autor desse livro entre seus pares -- tanto amor
emparelhado com tão pouca compreensão traz certas simpatias por ele --,
então farão a inclusão dele nesta liga depender de um voto de silêncio
eterno, mesmo que seja pelo fato de seu alemão ser o mais incorreto que
se possa imaginar.

Passamos as páginas e encontramos títulos prometedores: ``Camaldoli'',
``Sorrento'', ``Dias de outono em Seiano'', ``Ravello''. Começamos a ler
e o livro, talvez, continue agradando. Pois o que está escrito é tão
irrelevante, tão comovente, tão seco quanto a folha prensada de uma
videira qualquer do Golfo. Podemos nos entregar a sonhos maravilhosos
enquanto estivermos com o livro em mãos. Leio ``Positano'' e me vejo de
novo na rua que faz suas curvas pela cidade. É de noite. Formamos um
pequeno grupo: Ernst Bloch, o filósofo, Tavolato, bom de copo, Alfred
Sohn-Rethel, um dos membros mais jovens da família do pintor
teuto-romano. A lua brilhava no céu, e era uma daquelas noites
meridionais em que sua luz não parece cair sobre o cenário de nossa
existência diurna, mas sim sobre uma terra oposta ou paralela.\footnote{No
  original alemão, \emph{Gegen-Erde} e \emph{Neben-Erde}, neologismos
  criados por Benjamin. {[}N. do T.{]}} Era outra Positano, que
atravessamos. As partes abandonadas dessa grande cidade se destacavam
mais daquelas onde hoje sobrevive o pequeno número de descendentes de
uma população que, nos velhos tempos, já tinha chegado a quarenta mil,
pois, na Idade Média, a cidade era enorme. Eu bem sabia das histórias
que circulavam por aqui, mas não apreciava as histórias penetrantes de
fantasmas, que sempre surgem quando um proletariado de intelectuais
vagantes encontra com nativos primitivos, seja aqui, em Ascona, seja em
Dachau. Portanto, certamente não foi a vontade de aprender a lidar com o
medo,\footnote{Alusão ao título do conto de fada dos irmãos Grimm,
  ``Jemand, der auszog, das Fürchten zu lernen'', ``Alguém que saiu de
  casa para aprender a ter medo''. {[}N. do T.{]}} nem qualquer
interesse sério que tomou conta de mim, quando, de repente, pedi aos
meus companheiros esperarem por mim na rua para que eu desse alguns
passos morro acima e olhasse uma das casas abandonadas que, nesse
momento, se erguiam ao meu lado. Os degraus de pedra eram gigantescos;
sem pressa, subi um após o outro. Devo ter dado trinta passos grandes
dessa maneira; o silêncio era geral e, da rua, ouvia as vozes dos que
estavam à minha espera. Continuei sentindo vontade de prosseguir na
minha subida. Mas logo ficou mais difícil. Sentia como estava me
afastando dos amigos embaixo, apesar de permanecer bastante próximo a
eles, podendo ser ouvido e visto. Fui cercado por um silêncio e uma
solidão preenchidos de acontecimentos. Com cada passo, eu estava
avançando fisicamente para um acontecimento que não conseguia imaginar,
nem conceber e que não queria me tolerar. De repente, parei entre muros
e vãos de janelas, num mato espinhoso de sombras bem delineadas,
projetadas pela lua. Por nada nesse mundo eu teria dado um passo a mais.
Aqui, na presença dos meus companheiros totalmente despersonalizados,
vivi uma experiência do que significa aproximar-se de um círculo mágico.
Retornei.

Essa experiência não é uma simples curiosidade. Qualquer pessoa pode
fazê-la nesse lugar. Por isso, é duplamente necessário preservá-la de
clichês como: ``De noite, andamos numa escuridão assombradora. Parece
que, dos buracos estreitos, dos nichos apertados, das abóbadas
ressoantes, seres fabulosos nos assaltam.'' Essa é a Positano ``tal como
está no livro''. Evidentemente, a aldeiazinha de papel, que o autor
constrói para nós, também não sabe nada das energias que contribuíam
para a construção da famosa torre de Clavel.\footnote{Gilbert Clavel
  (1883-1927), artista plástico e escritor suíço, adepto do movimento
  futurista. {[}N. do T.{]}} No próximo outono, fará um ano que esse
inesquecível excêntrico da Basileia morreu. Um homem que construiu sua
vida no interior da terra, que vivia criativamente nas fundações de sua
torre e que, no grande \emph{carrefour} dos tempos, dos povos e das
classes, que é o Golfo de Sorrento, sabia dar informações como poucos e
que diz mais, em sua breve carta,\footnote{Gilbert Clavel, Carta a Carl
  Albrecht Bernoulli, de 27/08/1927. \emph{Die Annalen}. \emph{Eine
  schweizerische Monatsschrift}, Horgen-Zürich, 1927, pp. 953-955. {[}N.
  de W.B.{]}} da paisagem à sua volta do que o livro inteiro tratado
aqui.

E mais, se o acervo de vivências e saberes é a condição de todos os
relatos de viagem, onde se encontraria, na Europa, um objeto melhor que
Nápoles, que de hora em hora transforma tanto o turista quanto o nativo
numa testemunha que assiste à união da superstição mais antiga com a
mais nova impostura para formar procedimentos úteis, dos quais é o
usufruidor ou a vítima. De que maneira incomparável eles se fundem nas
festas que essa cidade possui em número dez vezes maior, porque cada
bairro comemora seu próprio santo, convidando os outros bairros em seu
dia onomástico. Como seria fácil integrar na descrição dessas festas
conhecimentos sólidos e enriquecedores sobre as localidades e os
costumes da cidade -- um aspecto ao qual o leitor alemão, quase
cinquenta anos depois de Gregorovius e Hehn, teve que aprender a
renunciar. Até o livro em questão retira suas melhores páginas da
descrição das procissões solenes. Mas um único costume entre todos os
dessa festa não teria expresso mais do que a descrição meticulosa,
demasiadamente incolor e imparcial do milagre do sangue de São Januário?
Quando, ao chegar o dia, a multidão fica ajoelhada hora após hora,
rezando fervorosamente dentro e fora da catedral à espera do milagre,
então aqueles entre os napolitanos cuja árvore genealógica remonta à
família do santo, têm o direito de cobrar seus favores aos seus
apadrinhados, com maldições ruidosas e vociferações imperiosas, até o
acenar de um lenço do altar anunciar que o milagre aconteceu, que o
sangue se tornou líquido. Por que não ficamos sabendo da
\emph{Piedigrotta}, do barulhento culto orgiástico da noite do oito de
setembro e das festanças gigantescas para as quais os napolitanos pagam
semanalmente alguns \emph{soldi} em suas mercearias como os nórdicos
pagam o seguro de vida para, ao chegar a época, poder beber e comer além
da medida e além de suas condições. Tradicionalmente, o festim é
encerrado por um frasco de óleo de mamona. Mas o barulho pagão da noite
de \emph{Piedigrotta} se perpetua nas festas cotidianas que o napolitano
comemora com a técnica. Quando chega perto da realização do seu desejo
de adquirir uma motocicleta, experimenta todas que estão ao seu alcance
para ficar com a mais barulhenta. Nunca vou esquecer a inauguração do
metrô, que, durante dias, não podia ser utilizado porque todos os
guichês estavam ocupados pela rapaziada da cidade que superava o barulho
retumbante do trem se aproximando com uma gritaria mais retumbante ainda
e que, durante as viagens, fazia ecoarem os túneis com um uivo
estridente. E até o ``passeio no campo'', a viagem de carro em caravana
para Sto. Elmo ou subindo o Vomero tem que ser banhada em poeira e
estrépitos para proporcionar a devida alegria.

Para tudo isso, o autor do livro não abre as portas. Mesmo assim, aquele
leitor lhe agradecerá, pois, conduzido para o tema, é abduzido de si
mesmo como nesta resenha. E pensando por um momento nas suas fotos
excelentes, podemos nos juntar sem ironia a esse leitor.

ANEDOTAS DESCONHECIDAS DE KANT\footnote{``Unbekannte Anekdoten von
  Kant'', in GS IV-2, pp. 808-815. Tradução de Georg Otte. {[}N. dos
  E.{]}}\footnote{Em uma carta à Scholem, datada de 28 de outubro de
  1931, Benjamin afirma trabalhar numa tentativa fisiognomônica de expor
  a relação entre o enfraquecimento mental de Kant na velhice e sua
  filosofia. Essas anedotas, que foram publicadas anonimamente no
  periódico \emph{Die Literarische Welt}, constituem provavelmente um
  resultado de tal projeto. {[}N. da O.{]}}

As seguintes anedotas são de textos que, como tais, não possuem nenhuma
relação com Kant, isto é, são de almanaques desaparecidos, revistas etc.
Apenas o sexto texto é exceção; ele foi tirado da coletânea
\emph{Immanuel Kant. Sua vida em testemunhos de
contemporâneos}\footnote{Cf. Ludwig Ernst von Borowski, Reinhold
  Bernhard Jachmann e Ehregott Andreas C. Wasianski, \emph{Immanuel
  Kant. Sein Leben in Darstellungen von Zeitgenossen.} Organização de
  Felix Groß. Berlin: Deutsche Bibliothek, s. d. {[}1912{]}. {[}N. do T.
  e dos E.{]}}, que representa um verdadeiro tesouro não apenas para o
leitor de Kant, mas sobretudo para o fisiognomonista. Algumas dessas
histórias sinalizam a postura graças à qual os ensinamentos de Kant,
ainda antes de chegarem à sua completa penetração e apropriação
filosófica, foram percebidos como uma nova potência vital, da qual
ninguém tinha como escapar.

Walter Benjamin

\begin{enumerate}
\def\labelenumi{\Roman{enumi}.}
\item
  \emph{Anedotas desconhecidas}
\end{enumerate}

Uma história na qual Kant é breve

\emph{Seu fâmulo, um teólogo que não conseguia conciliar a filosofia com
a teologia, pediu o conselho de Kant para saber o que deveria ler.}

\emph{Kant: Leia relatos de viagem.}

\emph{O fâmulo: Há coisas na dogmática que não entendo.}

\emph{Kant: Leia relatos de viagem.}

Uma história na qual Kant recorre a uma comparação

\emph{Numa conversa com um filósofo famoso de Königsberg, Kant acabou
falando do belo sexo.}

\emph{``Uma mulher'', disse Kant, ``tem que ser como o relógio de igreja
para fazer tudo pontualmente e no minuto certo, mas também não deve ser
como um relógio de igreja, para não espalhar todos os segredos em voz
alta. Ela tem que ser como um caracol, doméstica, mas também não deve
ser como um caracol, para não carregar tudo sobre o próprio corpo.''}

Um relato que mostra que Kant, já que não conseguia ver nada de positivo
no casamento, pelo menos tirou dele uma bela expressão

\emph{``Vale a regra: não se deve casar, exceto um casal muito digno.''
É o final de um poema de casamento de Michael Richey, Hamburgo, 1741.
Kant gostava de citá-lo diversas vezes quando se tratava de falar de uma
exceção a ser considerada como raridade, independentemente da questão se
o assunto era o casamento ou o celibato ou ainda outra coisa.}

Uma história em que Kant não é elegante

\emph{``Mulheres eruditas'', Kant observou, ``usam seus livros como seu
relógio; elas o carregam para todo mundo ver que têm um relógio, apesar
de normalmente estar parado ou não estar ajustado de acordo com a
posição do sol.''}

Uma história que mostra existirem dois tipos de citação: aquelas que
possuem aspas e aquelas que as receberão mais tarde

\emph{Numa roda de eruditos, a conversa acabou se voltando para a
maioria dos filósofos alemães, sendo que, evidentemente, também se falou
em Kant e suas obras.}

\emph{``Meu Deus'', disse o Conselheiro Oelrich, ``como se pode
vangloriar tanto as obras de Kant? Qualquer um pode escrevê-las numa
ilha deserta -- não contém nem dez citações.''}

Uma história sem a qual ninguém entenderá a \emph{Crítica da faculdade
de julgar}

\emph{Durante um verão mais fresco, quando havia poucos insetos, Kant
tinha percebido uma abundante quantidade de ninhos de andorinhas no
grande armazém de farinha perto do distrito de Lizent}\footnote{Distrito
  mercantil e alfandegário no centro de Königsberg. {[}N. dos E.{]}}
\emph{e encontrado alguns filhotes espatifados no chão. Admirado com
isso, repetiu sua investigação com atenção redobrada e fez uma
descoberta que custou a acreditar: as próprias andorinhas jogavam seus
filhotes dos ninhos. Muito surpreso com esse impulso natural, que, como
se fosse fruto do entendimento, ensinava as andorinhas a sacrificar
alguns para preservar os outros, na falta de alimentação suficiente para
todos, Kant disse: ``Nesse momento, meu pensamento parou e nada mais
pude fazer a não ser cair de joelho e orar.'' Disse isso de uma maneira
indescritível e inimitável. A profunda devoção que ardia no seu rosto, o
tom da voz, suas mãos em posição de oração, o entusiasmo que acompanhava
essas palavras -- tudo isso era único.}

Uma história com algumas palavras novas

\emph{Quando, em uma roda de conversa, falava-se sobre a diversidade dos
caracteres nacionais, Kant descreveu as nações europeias mais
importantes com as seguintes palavras:}

\emph{``Os franceses são polidos, vivos, levianos, volúveis,
libertários. Os ingleses são perseverantes, ativos, avarentos,
orgulhosos e antissociais. Os espanhóis são moderados, orgulhosos,
religiosos, majestosos, ignorantes, cruéis e preguiçosos. Os italianos
são alegres, firmes, afetuosos e assassinos. Os alemães, por fim, são
caseiros, honestos, constantes, fleumáticos, trabalhadores, modestos,
resistentes, hospitaleiros, eruditos, gostam de imitar os outros e
cobiçam títulos.}

\emph{Disso deduz-se'', acrescentou de maneira lacônica, ``que a França
é o país da moda, a Inglaterra o país do humor, a Espanha o país dos
ancestrais, a Itália o país da suntuosidade e a Alemanha o país dos
títulos.''}

Uma história em que Kant dá uma lição a alguns oficiais

\emph{Embora Kant não tivesse o mau hábito de muitos outros eruditos que
consistia em direcionar a conversa sempre para a sua ciência, ele
gostava de falar sobre assuntos que diziam respeito à Filosofia. Alguns
oficiais do quartel sabiam disso. Uma vez estava jantando com o chefe do
batalhão, o conde Henkel. Então alguns senhores resolveram fazer troça
da sua ciência. Iniciaram, portanto, a conversa nesse sentido, mas de
uma forma tão inábil que o filósofo logo percebeu a intenção. Não
notaram que agora era ele quem puxava os fios da conversa quando passou
a falar de cavalos e cachorros, isto é, do assunto preferido desses
senhores. Estes começaram a brigar a respeito e a discussão ficou
acalorada. ``Os senhores pegaram fogo'', disse Kant, ``isso me
surpreende, pois o assunto não é da Filosofia.''}

Um silogismo que não é de Kant, mas sobre Kant

\emph{Uma vez, quando estava lecionando Lógica, Simmel queria mostrar
aos seus ouvintes que duas premissas falsas podem gerar uma conclusão
verdadeira:}

\emph{Premissa maior: Todos os índios usam tranças.}

\emph{Premissa menor: Kant era índio.}

\emph{Conclusão: Logo, Kant usava uma trança.}

\begin{enumerate}
\def\labelenumi{\Roman{enumi}.}
\item
  \emph{Kant como conselheiro do amor}\footnote{Segundo os editores
    alemães das \emph{Gesammelte Schriften}, a segunda parte das
    anedotas, ``II. Kant como conselheiro do amor'', não pode ser
    atribuída com certeza a Benjamin, mas foi certamente elaborada a
    partir de sugestão dele. {[}N. dos E.{]}}
\end{enumerate}

Diferentemente da ``Senhora Cristine'', que, no Jornal da Tarde da
Editora Ullstein, resolve, a cada sábado, os problemas eróticos mais
complicados que as assinantes magoadas e os assinantes desesperados
apresentam, Kant, provavelmente, teve apenas uma vez a oportunidade de
fazer o papel de um conselheiro do amor de uma mulher e seu amor
infeliz. Tratava-se de Maria von Herbert, a irmã de um discípulo
talentoso de Kant, que lhe escreveu em agosto de 1791 uma carta
dilacerante. O primitivismo selvagem e a ortografia totalmente confusa
dessa carta, provavelmente nada de muito surpreendente para o leitor da
época, causa ao leitor de hoje a impressão quase insuportável de uma
derradeira agonia desesperada.

Diante dessa carta, Kant reservou-se um tempo, até a primavera de 1792,
para então responder. O arroubo inesperado das paixões parece tê-lo
impressionado bastante. A resposta que escreveu então certamente pode
ser chamada de a carta mais comovente de um filósofo. Comovente pela
clareza monumental de conteúdo e forma, mas, mais ainda, pela total
ingenuidade na análise das relações entre os sexos e pela total
ignorância quanto às reações do erotismo, que hoje em dia causaria
risadas em qualquer adolescente de quatorze anos: como se a erradicação
dos ``vestígios daquela resistência conforme as regras e baseados em
conceitos da virtude'' pudessem reacender no homem amado um amor uma vez
extinto! A Senhora Cristine entende muito mais do assunto. -- Mas é
exatamente esse muro pétreo e impenetrável, que se ergue entre o mundo
do espírito e o imoralismo da natureza, que parece ser o comentário mais
sublime sobre a figura humana de Kant.

Reproduzimos em seguida a carta completa da mulher e o fragmento
essencial da resposta de Kant.

Sobre essa mulher e sua história de amor, Kant escreve em 17 de janeiro
de 1793 a Ehrhard: ``Ela fracassou no recife do qual eu escapei, talvez
mais por sorte do que por mérito, ou seja, do amor romântico. -- Para
viver um amor ideal, ela se entregou primeiro a um homem que abusou de
sua confiança, e em favor de outro amor ideal ela confessou isso a um
segundo amante.'' Em 11 de fevereiro, Kant envia a carta dessa mulher a
Elisabeth Motherby, finalizando o bilhete que a acompanha com as
seguintes palavras: ``A sorte da educação da senhora dispensa a intenção
de oferecer essa carta como exemplo de alerta para esses desvios de uma
fantasia sublimada, mas ela pode servir assim mesmo para a senhora
sentir essa sorte de uma maneira ainda mais viva.''

Em agosto do ano 1791, Kant recebeu a seguinte carta:

\emph{Grande Kant,}

\emph{Chamo-lhe como um crente clama pela ajuda do seu Deus, clamo por
consolo ou por uma solução na morte, sempre suas razões em suas obras me
satisfaziam para a vida futura, daí refugio-me em você, mas para essa
vida não encontrei nada, nada mesmo, que pudesse substituir meu bem
perdido, pois eu amava uma coisa que, na minha visão, reunia tudo em si,
de maneira que só vivia por ela que era o oposto a todo o restu}
(übrüg)\emph{, pois todas as outras coisas para mim eram futilidades e
todas as pessoas realmente eram para mim como um caldo aguado, acontece
que ofendi essa coisa com uma mentira duradora} (langwirig) \emph{que
lhe revelei agora, mas, para o meu caráter, não vi nada de prejudicial
nisso, pois nunca tive que esconder qualquer erro na minha vida, mas
essa mentira apenas foi motivo suficiente para ele, e o amor dele
desapareceu, ele é um homem honesto, por isso não me nega sua amisade}
(Freindschaft) \emph{e sua fidelidade, mas aquele sentimento intenso que
nos uniu sem fazer qualquer esforço não existe mais, oh meu coração que
explode em mil pedaço} (Stük)\emph{, se já não tivesse lido tanta coisa
do senhor, certamente teria cometido uma violência contra a minha vida,
mas assim a conclusão que tive que tirar de sua tiuria} (Tehorie)
\emph{me segurou, que não deveria morrer por causa dos tormentos da
minha vida, mas que deveria viver por causa da minha existência, agora
coloqe-se} (sich sezen) \emph{no meu lugar e me dê consolo ou me
condene, li a metafísica dos costumes, inclusive o imperativo
categórico, mas não adianta, a minha razão me abandona quando mais
preciso dela, uma resposta lhe imploro ou você mesmo não concegue}
(kanst) \emph{agir conforme seu imperativo estabelesido}
(aufgeseten).\footnote{Com os erros ortográficos cometidos neste trecho,
  procurou-se recriar de forma aproximada em português símiles das
  confusões e incorreções ortográficas em que, conforme explicitado
  previamente no texto, o alemão da referida carta de Maria von Herbert
  incorre. {[}N. dos E.{]}}

Kant respondeu na primavera de 1792:

\emph{Sua carta afetuosa, nascida de um coração que deve ter sido feito
para a virtude e retidão, uma vez que é tão receptivo para uma teoria
das mesmas, que não tem nada de envolvente, leva-me para o lugar onde a
senhora me quer ver, isto é, no seu lugar, para assim refletir sobre o
meio de acalmá-la por vias puramente morais, que são as únicas
verdadeiras...}

\emph{Em primeiro lugar, aconselho que reflita sobre a questão de se as
repreensões que dirige a si mesma por causa de uma mentira, que, aliás,
não usou para esconder qualquer vício, são resultado de uma mera falta
de raciocínio, ou representam uma acusação interior por causa da
imoralidade que reside na própria mentira. No primeiro caso, a senhora
apenas repreende a si mesma pela sinceridade de tê-la revelado, ou seja,
se arrepende de ter cumprido com seu dever (pois é disso que se trata
quando se engana alguém propositalmente -- porém sem prejudicá-lo,
mantendo-o nesse engano durante um tempo -- e quando se livra essa
pessoa desse engano). E por que a senhora se arrepende com essa
revelação? Porque dela resultou a desvantagem, certamente considerável,
de perder a confiança do seu namorado. Ora, esse arrependimento não
contém nada de moral em seu motivo porque não foi gerado pela
consciência do erro, mas pelas suas consequências. Mas se a repreensão
que preocupa a senhora se basear mesmo num julgamento exclusivamente
moral de seu comportamento, seria um mau médico aquela pessoa que lhe
aconselhasse, uma vez que o acontecido não pode mais ser anulado, a
extinguir essa repreensão de sua mente e a munir-se, desde já, de alma
inteira, de uma sinceridade correta. A consciência moral é obrigada a
guardar todas as transgressões como um juiz que não extingue o processo
por causa de delitos já julgados, mas o guarda no arquivo para, em caso
de uma nova acusação por causa de delitos semelhantes, ou ainda
diferentes, reforçar ainda mais o julgamento, de acordo com os critérios
de justiça. Ora, ficar cismando sobre um arrependimento e, depois de ter
decidido a mudar a maneira de pensar, ficar continuamente se
repreendendo por causa de erros antigos irreparáveis torna-se inútil
para a vida. Seria (desde que se tenha certeza de ter melhorado) a
posição fantasiosa de uma autoflagelação merecida que como tantos outros
recursos supostamente religiosos, que consistem na súplica de favores
junto a poderes superiores, sem que se tenha nenhuma necessidade de se
tornar um ser humano melhor, não podem nem ser considerados como parte
da responsabilidade moral.}

\emph{Mas, se essa mudança na maneira de pensar se tornou evidente para
seu namorado -- tendo em vista que a sinceridade fala uma língua
inconfundível --, basta que se tenha tempo suficiente para aniquilar aos
poucos os vestígios daquela resistência do mesmo, que, por sua vez, se
baseia em conceitos morais, e para transformar a frieza em afetos ainda
mais sólidos. Todavia, se este último caso não acontecer, isto significa
que o calor afetivo anterior foi mais físico do que moral e, devido à
natureza passageira do mesmo, teria desaparecido de qualquer maneira com
o passar do tempo. Tal infelicidade ocorre várias vezes na nossa vida e
temos que nos resignar a ela com desprendimento, mesmo porque o valor da
vida, enquanto aquilo que podemos desfrutar de bom, é demasiadamente
valorizado pelas pessoas; mas enquanto aquilo que é apreciado de acordo
com o que podemos fazer de bem, é digno do maior respeito e esmero para
ser conservado e utilizado de forma serena para os melhores fins. --
Nesta resposta, querida amiga, a senhora encontra, portanto, como
costuma acontecer nos sermões, ensinamento, punição e consolo, e peço
que se detenha um pouco mais nos primeiros do que no último, pois quando
os primeiros surtirem efeito, o último e a paz perdida da vida, com
certeza, se encontrarão por si mesmos.}

\textsc{Posfácio}

\textsc{O Crítico e o contador}

Patrícia Lavelle

\textbf{A crítica: entre literatura e filosofia}

A geração de intelectuais alemães que ainda fora à escola num bonde
puxado por cavalos e se formara no rigor sistemático do neokantismo,
acedeu à maioridade intelectual depois da catástrofe representada pela
Primeira Guerra e pelas transformações sociais e econômicas que dela
decorreram. Marcada pela guerra e pela inflação, essa geração se
confrontou com o fracasso dos grandes sistemas do século XIX e da
concepção, de cunho hegeliano, da modernidade como ponto culminante do
progresso da razão na história. O novo interesse pelo tema da vida e da
experiência vivida e o retorno a ideias e formas pré-modernas, evocados
pelo próprio Benjamin em ``Experiência e pobreza'', acompanha assim a
descrença na exposição demonstrativa da reflexão filosófica e o
interesse por novas formas de apresentação do pensamento. Neste sentido,
a obra de Benjamin é emblemática de uma atitude intelectual
compartilhada por outros pensadores de sua geração, como Ernst Bloch ou
Theodor W. Adorno, embora tenha também afinidades com projetos
literários contemporâneos que enfatizam, inversamente, a reflexividade
contida na obra de arte.

Em ``Experiência e pobreza'', ele se refere à aparição de temas e formas
pré-modernas nas ruas das grandes cidades pintadas por Ensor,
qualificando tal efeito como fantasmagoria:

\begin{quote}
Uma nova miséria surgiu com esse monstruoso desenvolvimento da técnica,
sobrepondo-se ao homem. A angustiante riqueza de ideias que se difundiu
entre, ou melhor, sobre as pessoas, com a renovação da astrologia e da
ioga, da \emph{Christian Science} e da quiromancia, do vegetarismo e da
gnose, da escolástica e do espiritualismo, é o reverso dessa miséria.
Porque não é uma renovação autêntica que está em jogo, e sim uma
galvanização. Pensemos nos esplêndidos quadros de Ensor, nos quais uma
grande fantasmagoria enche as ruas das metrópoles: pequenos burgueses
com fantasias carnavalescas, máscaras disformes brancas de farinha,
coroas de folhas de estanho, rodopiam imprevisivelmente ao longo das
ruas. Esses quadros são talvez a cópia da Renascença terrível e caótica
na qual tantos depositaram suas esperanças.\footnote{Walter Benjamin,
  ``Experiência e Pobreza'', in \_\_\_\_\_\_. \emph{Magia e técnica,
  arte e politica}. Tradução de Sergio Paulo Rouanet. São Paulo:
  Brasiliense, 1993, p. 115~(Obras Escolhidas, vol. I); Walter Benjamin,
  ``Erfahrung und Armut'', in \_\_\_\_\_\_. \emph{Gesammelte Schriften}
  {[}daqui em diante: GS{]}, vol. II-1. Frankfurt a. M.: Suhrkamp, 1991,
  p. 214.}
\end{quote}

Podemos comparar as fantasmagorias de Ensor, tal como Benjamin as
apresenta aqui, à sua própria atitude diante do contador de histórias,
compreendido como uma figura arcaica, pré-moderna -- espectro que faz
aparição e deixa traços na modernidade de sua obra. Ora, se em
``Experiência e pobreza'', ele valoriza positivamente a barbárie
moderna, propondo um construtivismo vanguardista capaz de partir do
ponto zero de experiência, o ensaio sobre Leskov, que abre esse volume,
assume um tom inegavelmente nostálgico que parece contradizer a primeira
posição. Entretanto, o novo bárbaro disposto a construir com pouco e o
contador de histórias que encontra na riqueza da experiência
transmissível a matéria de sua arte, são as duas faces de um
mesmo.\footnote{Para uma confrontação entre as teses, apenas
  aparentemente antagônicas, apresentadas nestes dois textos, ver o
  ensaio de Jeanne Marie Gagnebin, ``Não contar mais?'', in
  \_\_\_\_\_\_. \emph{História e narração em W. Benjamin}. São Paulo:
  Perspectiva: Editora da Unicamp, 1994.} Ao evocar o arcaísmo da
narrativa que se inscreve na tradição oral, Benjamin tem o projeto de
construir uma nova forma, profundamente moderna. De fato, ele não
tematiza a narração tradicional apenas teoricamente, no gênero do ensaio
crítico, mas também numa produção ficcional na qual as estratégias
tradicionais da arte de contar histórias são mobilizadas, discutidas e
ironizadas. Como mostra Marc de Launay no estudo que acompanha uma
coletânea de contos de Benjamin por ele traduzidos para o
francês\footnote{Ver ``Préface'', in Walter Benjamin. \emph{N'oublie pas
  le meilleur et autres histoires et récits}. Tradução, apresentação e
  notas por Marc de Launay. Paris: L'Herne, 2012.}, tais textos
ficcionais se caracterizam pelo efeito de choque causado pela evocação
nostálgica e pela brusca denúncia da nostalgia, pelo uso de formas
narrativas tradicionais e por sua ironização pelo contista moderno, que
não pode aderir substancialmente a estas.

Em sua reflexão teórico-literária, esta forma arcaica aparece como uma
fantasmagoria capaz de reatualizar algo que o sistema filosófico do
século XIX negligencia, apontando para um projeto alternativo de
modernidade também ao nível do pensamento. Assim, a interrogação sobre a
arte de contar, que Benjamin ao mesmo tempo teoriza e pratica, pressupõe
a compreensão prévia da fecundidade das passagens entre literatura e
filosofia exploradas por seus trabalhos.

É na perspectiva da relação entre literatura e filosofia que, numa carta
de 1920, endereçada a Ernst Schoen, antigo camarada de estudos, Benjamin
circunscreve o vasto campo da crítica:

\begin{quote}
Muito me interessa, efetivamente, o princípio do grande trabalho
crítico-literário: o campo compreendido entre arte e filosofia
propriamente dita, que compreendo apenas como o pensamento ao menos
virtualmente sistemático. É preciso conceber um princípio perfeitamente
originário de uma forma literária a qual pertencem grandes obras como o
diálogo de Petrarca sobre o desprezo do mundo ou os aforismas de
Nietzsche ou as obras de Péguy. Nestas últimas, por um lado, e por outro
no devir e nas relações de uma jovem pessoa minha conhecida tal questão
se colocou sob meus olhos. Além disso, tornei-me consciente do
fundamento originário e do valor da crítica também em meu próprio
trabalho. Neste sentido, a crítica de arte, cujas fundamentações me
ocuparam, é apenas uma parcela deste vasto domínio.\footnote{Walter
  Benjamin, \emph{Gesammelte Briefe} {[}daqui em diante: GB{]}, vol. II.
  Frankfurt a. M.: Suhrkamp, 2000, p. 71 (252).}
\end{quote}

Ao apresentar a crítica como o princípio de um gênero ou um campo de
pesquisa, Benjamin nele situa seus próprios trabalhos. Deste campo
limítrofe entre literatura e filosofia, a crítica de arte -- que
procurara conceitualizar em sua tese de doutorado sobre o primeiro
romantismo, defendida em 1919, e no ensaio sobre \emph{As afinidades
eletivas} de Goethe, publicado em 1922, -- seria apenas uma parte.
Entretanto, esta parte indica o princípio da zona limítrofe entre
literatura e filosofia que constitui o gênero crítico, pois pressupõe a
correlação entre a dimensão teórica inerente à arte e o elemento
estético no discurso filosófico. O conceito benjaminiano de crítica, que
se inspira diretamente nos românticos, seria fundamentalmente tributário
da estética kantiana, isto é, da correlação descoberta por Kant entre a
produção simbólica do gênio e a apresentação das ideias da razão, entre
criação artística e reflexão teórica.

Na \emph{Critica da razão pura}, Kant estabelece uma rigorosa distinção
entre os conceitos do entendimento, que podem ser conhecidos
objetivamente e determinados através de uma exposição direta,
esquemática, e os conceitos da razão, isto é, as ideias, cuja realidade
não pode ser demonstrada, mas apenas pensada. Ora, se as ideias
racionais não podem ser conhecidas objetivamente, elas também não são
meras ilusões, mas correspondem à esfera do questionamento metafísico
que confere sentido à experiência e ao conhecimento possíveis. É no
entanto na \emph{Crítica da faculdade de julgar} que Kant tematiza o
modo de apresentação próprio às ideias. Este implica uma construção
simbólica na qual a ideia é pensada por analogia com um objeto empírico
qualquer, processo no qual se produz uma afinidade entre as regras de
reflexão sobre esse objeto e a reflexão sobre o conceito racional do
qual ele é apenas o símbolo. Assim, a exposição da esfera do
questionamento filosófico requer, ao menos parcialmente, a intervenção
do gênio artístico.

Definido no § 49 da terceira crítica como a faculdade de criar ``ideias
estéticas'', o gênio corresponde ao talento de articular princípios
racionais a representações da imaginação em configurações sensíveis que,
abrindo perspectivas à perda de vista para o pensamento, constituem o
contrário e o correlato das ideias racionais. Se as ideias estéticas são
representações sensíveis às quais nenhum conceito determinado pode
corresponder, as ideias da razão são conceitos indeterminados aos quais
nenhuma representação sensível pode ser adequada. Assim, ao explicitar a
dimensão estética ou simbólica de toda reflexão filosófica, Kant chama a
atenção também para a dimensão reflexiva da arte, sua abertura ao
pensamento.

A tese de doutorado de Benjamin, que se apoia sobretudo em Friedrich
Schlegel e em Novalis, define o conceito romântico de crítica como
autoconsciência da reflexão que se coloca em forma numa formação
artística. Assim, criticar significa desenvolver a reflexão que já se
encontra na obra, isto é, despertar e completar o pensamento nela
colocado em forma. Segundo a interpretação de Benjamin, para os
românticos, a crítica não é um julgamento sobre a obra, mas um método de
seu acabamento pois deve ultrapassá-la em sua própria reflexão, torná-la
absoluta. Assim, o conceito romântico de crítica de arte se funda sobre
a dimensão reflexiva imanente à formação artística, sua criticabilidade
essencial. Nesta perspectiva, aquilo que os românticos chamavam de
ironia corresponde à exacerbação formal do elemento crítico contido na
própria forma da obra de arte, isto é, à acentuação reflexiva de sua
reflexividade.

O ensaio sobre \emph{As afinidades eletivas} de Goethe funciona como uma
crítica paradigmática na qual Benjamin elabora o seu próprio conceito de
crítica. A terceira parte deste texto é dedicada ao problema da relação
entre arte e filosofia. Segundo Benjamin, as obras de arte são figuras
nas quais aparece o Ideal do problema filosófico definido como o
conceito de uma pergunta inexistente sobre a unidade da filosofia, isto
é, como fundamento unitário de todo questionamento filosófico. De acordo
com ele, o Ideal do problema não se encontra numa multiplicidade de
problemas, mas está encerrado na pluralidade das obras de arte e sua
extração seria tarefa da crítica. Assim, em cada obra de arte verdadeira
pode ser encontrada uma manifestação do Ideal do problema filosófico e a
crítica o apresenta ao se confrontar com o mistério de sua beleza.

\begin{quote}
Diante (...) de todo belo, a ideia do desvelamento converte-se naquela
da impossibilidade de desvelamento. Essa é a ideia da crítica de arte. A
tarefa da crítica de arte não é tirar o envoltório, mas antes elevar-se
à contemplação do envoltório enquanto envoltório.\footnote{Walter
  Benjamin, ``Sobre \emph{As afinidades eletivas} de Goethe'', in
  \_\_\_\_\_\_. \emph{Ensaios reunidos: escritos sobre Goethe}. Tradução
  do ensaio de Mônica Krausz Bornebusch. São Paulo: Editora 34: Livraria
  Duas Cidades, 2009, p. 112; GS I-1, p. 195.}
\end{quote}

A crítica não desvela um conteúdo qualquer que estaria escondido na
forma artística, mas ela revela precisamente a dissolução desta
dicotomia na contemplação do belo.

\begin{quote}
A doutrina kantiana de que o fundamento da beleza é um caráter
relacional impõe portanto, com pleno êxito, as suas tendências
metodológicas numa esfera muito mais elevada do que a psicológica. Toda
beleza, assim como a revelação, conserva em si regras
histórico-filosóficas. Pois a beleza não torna a ideia visível, mas sim
o seu segredo.\footnote{Idem, ibidem, p. 113; GS I-1, p. 195.}
\end{quote}

Tais formulações, que se referem ao conceito do Belo em Kant, se
inscrevem no horizonte inaugurado pela correlação que a \emph{Crítica da
faculdade de julgar} estabelece entre ideia estética e ideia racional,
representação artística e questionamento filosófico. Entretanto,
Benjamin a considera do ponto de vista de uma teoria da linguagem que
compreende a ideia como nome, isto é, como algo que diz respeito à
dimensão simbólica da linguagem. Pois, de acordo com o ``Prefácio
epistemo-crítico'' de seu livro sobre o drama barroco, a cristalização
da reflexão sobre aquelas questões fundamentais, embora sem resposta,
que a filosofia coloca sempre de novo, se apresenta em configurações
discursivas historicamente condicionadas nas quais o elemento conceitual
se articula necessariamente a construções poético-literárias. Nesta
perspectiva, seu segredo, que se enraíza na vida do pensamento, remete à
unidade de forma e conteúdo que também caracteriza a beleza da obra de
arte. A crítica de arte apresenta assim o problema da apresentação do
pensamento em sua relação com a capacidade da linguagem de se articular
poeticamente; problema que, embora tenha sido indicado por Kant, é
negligenciado pela exaustividade demonstrativa do sistema filosófico do
século XIX.

Neste sentido, o conceito de crítica de arte leva-nos a pensar a
fecundidade da interseção entre literatura e filosofia que constitui o
princípio do trabalho crítico-literário evocado por Benjamin. A arte de
contar histórias, abordada tanto no ensaio crítico sobre Leskov quanto
em sua produção ficcional, se inscreve neste horizonte de
problematização da apresentação do pensamento que busca em elementos
arcaicos uma alternativa ao projeto de modernidade baseado na ideia de
progresso racional e na instrumentalização da linguagem no sistema.

Ora, esta forma pré-moderna, já evocada no ensaio sobre \emph{As
afinidades eletivas} através da história contada aos personagens de
Goethe -- conto dentro do romance ao qual Benjamin atribui uma grande
importância, considerando-o como uma chave de leitura para a compreensão
da obra --, nos coloca diante da questão moral. Nesta forma mais arcaica
representada pela narrativa, Benjamin identifica o ``núcleo luminoso''
que se abre à esfera da liberdade da qual os personagens do romance,
cativos das convenções como de uma segunda natureza mítica, estão
excluídos. Neste sentido, a conclusão do conto é significativa: o jovem
enamorado não hesita em mergulhar em águas agitadas para salvar a moça
que, num ato desesperado, joga-se do barco. Ao interpretar esse ato
heroico -- e o episódio no qual o rapaz desnuda o corpo da amada, não
para contemplar sua beleza, mas no intuito de salvar-lhe a vida --,
Benjamin chama a atenção para o abandono da esfera do belo em função de
uma ordem mais alta. Como veremos, tanto sua reflexão teórica sobre a
narração tradicional quanto o recurso ficcional a estratégias próprias à
arte de contar se inscreve na perspectiva do enraizamento de todo pensar
na esfera da liberdade e na interrogação sobre o sentido da ação humana.

\textbf{Nostalgia e modernidade: o ensaio sobre Leskov}

Contemporâneo de Dostoiévski e de Tolstói, Nikolai Leskov viveu e
escreveu na época de apogeu do romance russo do século XIX. Entretanto,
ao caracterizá-lo, logo nas primeiras linhas do ensaio de 1936, Benjamin
o assimila à figura arcaica do contador de histórias, cujas origens
pré-modernas se encontram na tradição oral, na transmissibilidade da
experiência da vida, irremediavelmente perdida no mundo moderno. Assim,
ele situa Leskov num passado anterior ao da modernidade que, no século
XIX, encontra sua expressão literária no romance burguês e sua forma
filosófica no sistema. O anacronismo não é arbitrário -- o contista
russo se serve efetivamente de formas narrativas tradicionais que se
alimentam de um farto material popular e da tipologia do conto de fadas
--, mas indica, entretanto, a orientação mais geral do estudo como
caracterização da figura arcaica do contador de histórias em sua relação
com a modernidade.

O contador é um espectro. Aparição do passado no presente, essa figura
construída em torno de Leskov no ensaio de 1936 não é uma simples
metáfora, mas possui uma certa materialidade que se encarna não apenas
no escritor russo, mas também na obra de Hebel, de Poe ou de Kipling,
entre outros autores citados. Sobre o projeto do ensaio acerca do
contador, que pode ser compreendido como a produção de uma
fantasmagoria, Benjamin se exprime em duas cartas de 1936.

``Tenho me ocupado sobretudo com um estudo sobre Nikolai Leskov no qual
falo menos sobre este grande contista russo do que sobre o tipo do
contador de histórias em geral, sua relação com o romancista e com o
jornalista e seu lento desaparecimento da face da terra'', afirma numa
carta de Paris, endereçada a Werner Kraft.\footnote{GB V, p. 289 (1041).}
Uma semana mais tarde, torna a escrever sobre o assunto, desta vez
dirigindo-se a Adorno: ``Escrevi nos últimos tempos um trabalho sobre
Nikolai Leskov o qual, sem se referir, nem mesmo de longe, à teoria da
arte, contém alguns paralelos com a `perda da aura' no que esta diz
respeito à arte do contador de histórias''.\footnote{GB V, p. 307
  (1047).}

Apresentada como uma forma fundamentalmente aurática cujo declínio no
mundo moderno estaria relacionado à emergência do romance, e sobretudo
da informação, o conto tem um caráter artesanal. Isto significa que se
alimenta da experiência de vida (\emph{Erfahrung}) do contador, cuja
marca se imprime na história contada ``como o oleiro deixa a impressão
de sua mão na argila do vaso''.\footnote{\emph{Supra}, p. ? (GS II-2, p.
  447).} A experiência transmissível da tradição é, segundo Benjamin, a
fonte a que recorreram todos os contadores. O senso prático é uma
característica desta figura cuja autoridade se funda na sabedoria, ``o
lado épico da verdade''.\footnote{\emph{Supra}, p. ? (GS II-2, p. 442).}
Assim, a arte de contar pressupõe a capacidade de aconselhar,
compreendida como o talento para sugerir uma continuação para uma
história que está se desenvolvendo. O declínio da arte de contar estaria
relacionado ao empobrecimento desta experiência de vida que outrora
garantia o valor dos conselhos e se exprimia em provérbios e
ensinamentos morais.

Benjamin caracteriza tal crise através do contraste com o romance e com
a informação. Se os contos podem ser compartilhados e transmitidos
oralmente, a recepção do romance implica a leitura silenciosa pelo
indivíduo isolado. É também a perplexidade diante da singularidade
individual que orienta o romance; este se interroga fundamentalmente
sobre a inefabilidade do sentido de uma vida e sua conclusão corresponde
simbolicamente à morte do personagem. O conto, ao contrário, coloca a
questão moral de tal modo que a história não se acaba com o seu fim, mas
suscita a interrogação sobre o que aconteceu depois, abrindo-se assim a
outras histórias. Segundo Benjamin, o declínio da arte de contar que se
enraíza na tradição oral corresponde à emergência do romance, mas se
acentua com a própria crise deste, associada à importância crescente da
informação que encontra seu espaço nos jornais. Ora, esta se caracteriza
pela proximidade, pela plausibilidade e pela explicação dos fatos. O
saber que vem de longe para se transmitir misteriosamente nos contos não
tem nenhum valor informativo. O interesse da informação está na
proximidade, no aqui e agora de eventos que nunca vem sem numerosas
explicações. Nela, nada é deixado ao julgamento do leitor. Por outro
lado, segundo Benjamin, a ``metade da arte de contar está em despojar de
explicações a história contada''.\footnote{\emph{Supra}, p. ? (GS II-2,
  p. 445).}

A ausência de explicações psicológicas é uma característica fundamental
do conto e remete às suas fontes antigas. Benjamin menciona a propósito
disto a história, contada por Heródoto e comentada por Montaigne, do
faraó egípcio Psaménito. Segundo Benjamin, tal relato conserva suas
forças depois de tanto tempo justamente porque renuncia a nos dizer o
que se passa no coração e no pensamento do rei deposto. É sua concisão,
aliada à ausência completa de explicações psicológicas, que nos deixa
pensativos. Abrindo-se a uma pluralidade de interpretações e comentários
que procuram, em vão, extrair da história uma moral ou um sentido, o
conto não se explica numa frase, mas nos leva à multiplicidade difusa de
representações que constitui aquilo que Hans Blumenberg chama de
``pensatividade''.

Em um pequeno texto intitulado ``Pensatividade''
(\emph{Nachdenklichkeit})\footnote{Hans Blumenberg, ``Pensivité'',
  tradução de Denis Trierweiler, \emph{Cahiers philosophiques} (Dossier
  Blumenberg), n.° 122, 3.° trimestre de 2010, pp. 83-87.}, Blumenberg
evoca uma fábula de Esopo para refletir sobre este estado de espírito ao
qual nos leva o conto. Ele o caracteriza como um momento de hesitação no
qual nos confrontamos confusamente com aquelas questões que não podemos
responder, mas às quais também não devemos renunciar. Entretanto, a
pensatividade se distingue do pensamento porque não conclui, não resolve
nenhum problema, é apenas uma disposição, um espaço de jogo. Neste
sentido, a pensatividade seria, segundo Blumenberg, um adiamento, um
prazo dado contra os resultados banais e decepcionantes que o pensamento
ordenado pode nos dar quando se interroga sobre a vida e a morte, o
sentido e a ausência de sentido, o ser e o nada. Ora, se a filosofia
passa por saber disciplinar com método tais questões -- e por vezes as
proíbe em razão do caráter inatingível de suas respostas --, a
pensatividade suscitada pelo conto estaria em sua origem. Assim, a
conclusão de Blumenberg nos parece extremamente significativa para
compreendermos o recurso à narrativa na perspectiva deste gênero crítico
que Benjamin situa entre literatura e filosofia.

Blumenberg afirma que a filosofia deve conservar, senão renovar, algo da
origem, vinda do mundo da vida, que encontra na pensatividade. Pois,
segundo ele, a filosofia representa uma constatação mais geral de toda
cultura, isto é, que devemos respeitar as questões às quais não podemos
atribuir uma resposta. Ora, ao evitar explicações, o conto nos ensina
tal respeito -- e a palavra respeito é aqui significativa pois nos
remete aos objetos da razão prática. Diante do conto moderno que se
alimenta de fontes pré-modernas, como a fábula ou o conto de fadas, se
descortina a esfera da liberdade na qual esses questionamentos aos quais
não podemos dar uma resposta adequada encontram sua fonte.

No ensaio sobre o contador, Benjamin afirma que o conto de fadas revela
as primeiras medidas tomadas pela humanidade para libertar-se do
pesadelo mítico. Dirigindo-se às origens do homem, o conto de fadas
apresenta a figura do justo que não se identifica ao místico asceta, mas
ao homem simples e ativo, capaz de enfrentar as adversidades com bondade
e astúcia. Benjamin identifica tais elementos na produção de Leskov, o
que lhe permite delinear os contornos da figura do contador,
fantasmagoria na qual ``o justo se encontra consigo mesmo''.\footnote{\emph{Supra},
  p. ? (GS II-2, p. 465).} Segundo ele, o primeiro contador verdadeiro é
e continua sendo o do conto de fadas, que sabia dar um bom conselho e
oferecer sua ajuda na necessidade provocada pelo mito.

\begin{quote}
O conto de fadas ensinou há muito tempo à humanidade e ainda hoje ensina
às crianças que o mais aconselhável é enfrentar o mundo do mito com
astúcia e ousadia. (...) A magia liberadora do conto de fadas não coloca
em cena a natureza de um modo mítico, mas indica a sua cumplicidade com
o ser humano liberado.\footnote{\emph{Supra}, p.? (GS II-2, p. 458).}
\end{quote}

Levando-nos de volta à infância e aos primeiros esforços da humanidade
para libertar-se do mito, o contador de histórias nos conduz à origem da
filosofia na pensatividade -- este espaço de jogo, essa vivência de
liberdade que se abre à esfera da razão prática na qual o questionamento
metafísico se enraíza. Assim, a nostalgia inerente ao ensaio sobre o
contador não diz respeito apenas ao declínio da arte de contar e ao
empobrecimento da experiência que constitui sua fonte, mas concerne
também à pensatividade que se instala no sonho acordado, no devaneio --
esta face interior do tédio que Benjamin associa ao ritmo do trabalho
artesanal. ``O tédio é o pássaro de sonho que choca o ovo da
experiência''\footnote{\emph{Supra}, p.? (GS II-2, p. 446).}, diz ele ao
afirmar que as atividades artesanais que o propiciavam, prestando-se ao
dom de ouvir e à arte de contar, estão em vias de extinção.

\textbf{O conto como gênero crítico}

\begin{quote}
Vivenciamos o surgimento da \emph{short story}. que se emancipou da
tradição oral e não mais permite essa lenta sobreposição de camadas
finas e transparentes que oferece a melhor imagem da maneira pela qual o
conto perfeito vem à luz do dia a partir das camadas acumuladas por suas
diferentes versões.\footnote{\emph{Supra}, p.? (GS II-2, p. 448).}
\end{quote}

Grande parte da produção ficcional de Benjamin se inscreve neste novo
gênero literário que surge, por assim dizer, nas ruínas da arte de
contar. Estes textos curtos, escritos ao longo dos anos vinte e trinta,
utilizam elementos que caracterizam a tradição narrativa, tal como é
apresentada no ensaio sobre Leskov. Entretanto, se o uso de tais
recursos provoca em nós a nostalgia da origem do pensamento na
pensatividade, o recurso à ironia marca a distância entre o contista
moderno e as formas tradicionais das quais ele se serve, criando assim
um efeito de choque. Neste sentido, a barbárie construtivista e a
nostalgia da arte de contar histórias correspondem a dois aspectos de um
mesmo projeto filosófico-literário no qual o recurso ao arcaico visa
encontrar uma alternativa para a imanência radical do mundo moderno.

Encontramos um exemplo disso em ``A sebe de cactos'', onde o personagem
principal, o irlandês O'Brien, encarna a figura do bárbaro moderno,
disposto a começar do zero. O conto, escrito na primeira pessoa, narra o
encontro, em Ibiza, com este excêntrico solitário cujas ocupações -- a
pesca, a caça e a arte de fazer e desfazer nós -- remetem a um mundo
arcaico. Da África, onde convivera com uma tribo primitiva, trouxera uma
bela coleção de máscaras, que fora entretanto furtada, há muitos anos,
por um amigo seu. Assim, o narrador do conto se admira ao contemplar um
impressionante conjunto de máscaras africanas reunidas na casa de
O'Brien, que lhe conta como vieram parar ali numa dessas noites em que o
luar e o tédio estimulam a faculdade de produzir e de perceber
semelhanças. Ele vira pela janela a sebe de cactos ganhar vida e, como
um grupo de guerreiros africanos, avançar usando as máscaras
desaparecidas. Resolve então esculpir suas visões oníricas na madeira.

Mas a história não termina com essa invocação nostálgica, na qual o
arcaico perdido reaparece no arcaísmo do sonho. A ironia fica reservada
a um reencontro posterior com três dessas máscaras numa galeria de arte
de Paris, onde especialistas garantem sua antiguidade e autenticidade.
Como afirma o contista, elas ``inspirariam nossos jovens artistas a
fazer suas próprias tentativas interessantes''.\footnote{\emph{Supra},
  p.? (GS IV-2, p. 754).} Os contos de Benjamin são como essas máscaras
primitivas esculpidas pelo excêntrico O'Brien.

Retorno do passado no presente, a fantasmagoria do contador constitui o
próprio tema de ``O Lenço''. Neste conto, um marinheiro, o Capitão O...,
possui todos os traços que compõem a figura do contador de histórias,
exceto um que aparece como fundamental: a faculdade de contar sua
própria vida. Tal faculdade, que implica a ligação intrínseca entre a
vida do contador e a matéria de suas histórias, corresponde à
transmissão da experiência tradicional, essa \emph{Erfahrung} que não se
confunde com as vivências (\emph{Erlebnisse}) radicalmente individuais e
interiorizadas do homem moderno. Ora, o desenrolar do conto contradirá e
confirmará essa constatação inicial.

Encontrado por acaso durante uma escala, o Capitão O... conta ao
narrador que, há muitos anos, teve como passageira uma moça tão linda
quanto discreta e silenciosa. Um dia, ela deixou cair no convés um lenço
e, quando ele o apanhou, agradeceu seu gesto como se tivesse salvo sua
vida. O marinheiro-contador descreve minuciosamente o lenço, que era
ornamentado com um escudo bordado de estrelas, mas nada diz sobre seus
próprios sentimentos e impressões. Conta apenas que, quando o barco
estava para atracar, a bela passageira precipitou-se sem uma palavra no
mar, justamente no pequeno espaço que restava entre a quilha e o cais. O
perigo era grande e a moça teria sido rapidamente esmagada se, num átimo
de segundo, alguém não tivesse se prontificado a salvá-la. O episódio
faz pensar na pronta decisão do rapaz que salva sua amada das águas
geladas na pequena novela que constitui, segundo Benjamin, o núcleo
luminoso de \emph{As afinidades eletivas} de Goethe. O relato do
marinheiro, no entanto, todo na terceira pessoa, não inclui nenhuma
alusão aos sentimentos, pois não dá lugar a nenhuma explicação
psicológica. Neste ponto, a história do Capitão O... é tão reservada e
discreta quanto a moça, e sua beleza delicada vem justamente dessa
extrema concisão. É a figura arcaica do justo que vemos surgir em sua
simplicidade proverbial.

Na contramão do romance psicológico do início do século XX, a
\emph{short story} de Benjamin nada diz sobre a interioridade dos
personagens. Sua modernidade radical está na evocação irônica e
nostálgica do relato arcaico que apenas expõe, sem nada explicar. Se no
final ficamos sabendo que o herói da história é o próprio contador, isso
se deve a uma única frase em primeira pessoa: ``Quando a segurei assim
(...), ela sussurrou `obrigada' como se eu tivesse lhe estendido um
lenço que caíra ao chão''.\footnote{\emph{Supra}, p.? (GS IV-2, p. 744).}
E a última palavra do conto é confiada justamente a este objeto cuja
presença dispensa maiores explicações: o narrador reconhece o pequeno
escudo bordado no lenço que o capitão agita ao longe, ao despedir-se.

Ao tematizar o próprio contador de histórias, evocando o seu espectro,
Benjamin ironiza a forma narrativa. Penso aqui no conceito romântico de
ironia como um recurso formal, um distanciamento crítico que se inscreve
na própria forma da obra, explicitando a reflexão nela contida e
relativizando o seu caráter condicionado.

Não é por acaso que este conto acerca do contador se abre com a
interrogação sobre o declínio da própria arte de contar. O narrador da
pequena ficção começa por relacionar a morte da narração tradicional com
o desaparecimento das atividades manuais e repetitivas que outrora
deixavam tempo para o tédio. Assim, ao incluir no conto a própria teoria
de seu declínio, distancia-se da simplicidade ``ingênua'' (no sentido de
Schiller) que nos toca no relato de Heródoto, ou mesmo nos contos de
Leskov ou de Hebel. A \emph{short story} do início do século XX, tal
como Benjamin a inventa, é um gênero ``sentimental''. Isto quer dizer
que, nele, a ironização dos procedimentos tradicionais da arte de contar
histórias implica não apenas a consciência de seu fim, e portanto uma
certa nostalgia, como também a reflexão sobre sua reflexividade.
\end{document}