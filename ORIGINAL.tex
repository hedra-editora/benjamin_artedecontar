%!TEX root=./LIVRO.tex 

%\textbf{Coleção Walter Benjamin}

%\textbf{Coordenação científica e editorial}

%Amon Pinho

%\begin{quote}
%\textbf{Supervisão técnica e de tradução}
%\end{quote}

%Francisco De Ambrosis Pinheiro Machado

%\begin{quote}
%\textbf{\textsc{A arte de contar histórias}}

%Walter Benjamin

%\textbf{Organização e posfácio}

%Patrícia Lavelle

%\textbf{Tradução }

%Georg Otte

%Marcelo Backes

%Patrícia Lavelle

%\textbf{Revisão de tradução}

%Francisco De A. P. Machado

%Amon Pinho

%Patrícia Lavelle

%Blima Carvalho Otte
%\end{quote}

%\textbf{Preparação de originais}

%\begin{quote}
%Amon Pinho

%\textbf{ÍNDICE}
%\end{quote}

%\textbf{\textsc{Apresentação}}

%Amon Pinho

%Francisco De Ambrosis Pinheiro Machado

%\textbf{ENSAIO}

%\textsc{O contador de histórias. Considerações sobre a obra de Nikolai
%Leskov}

%\textbf{CONTOS}

%\textsc{O lenço }

%\textsc{A viagem do ``Mascote''}

%\textsc{O anoitecer da viagem}

%\textsc{A sebe de cactos}

%\textsc{Histórias da solidão}

%\textsc{Quatro histórias}

%\textsc{A morte do pai. Novela}

%\textsc{Palácio D\ldots{} Y}

%\textsc{``Inscrito na poeira movediça''. Novela}

%\textsc{O segundo eu. Uma história de final de ano para refletir}

%\textsc{Rastelli conta}

%\textsc{Por que o elefante se chama ``elefante''}

%\textsc{Como o barco foi inventado e por que ele se chama barco}

%\textsc{Uma história estranha, de quando ainda não havia humanos}

%\textsc{Myslowitz -- Braunschweig -- Marselha. História de uma
%embriaguez com haxixe}

%\textbf{O CONTADOR DE HISTÓRIAS NO RÁDIO}

%\textsc{No minuto exato}

%\textsc{Caspar Hauser}

%\textsc{O coração gelado. Peça de rádio baseada em Wilhelm Hauff}

%\textsc{A Berlim demoníaca}

%\textbf{CONTO E CRÍTICA}

%\textsc{Conversa assistindo ao corso. Ecos do carnaval de Nice}

%\textsc{A mão de ouro. Uma conversa sobre o jogo}

%\textsc{Jakob Job, \emph{Nápoles. Imagens de viagem e esboços}}

%\textsc{Anedotas desconhecidas de Kant}

%\textbf{\textsc{Posfácio: O crítico e o contador}}

%Patrícia Lavelle

%\textsc{Apresentação}

%Amon Pinho

%Francisco De Ambrosis Pinheiro Machado

\part{Ensaio}

\chapterspecial{O contador de histórias}{Considerações sobre a obra de Nikolai
Leskov\,\footnote[*]{Tradução de Patrícia Lavelle a partir do
  original alemão ``Der Erzähler. Betrachtungen zum Werk Nikolai
  Lesskows'' em cotejamento com a versão francesa deste mesmo ensaio,
  realizada pelo próprio Benjamin e intitulada ``Le Narrateur.
  Réflexions à propos de l'œuvre de Nicolas Lesskov'', in Walter
  Benjamin, \emph{Gesammelte Schriften} [a partir daqui, \versal{GS}], vol.
  \versal{II}, tomos 2 e 3. Edição de Rolf Tiedemann e Hermann Schweppenhäuser.
  Frankfurt a. \versal{M}.: Suhrkamp, 1991, pp. 438-465 e 1290-1309,
  respectivamente. Escrito entre fins de março/inícios de abril e julho
  de 1936, o ensaio sobre Leskov destinava-se à revista \emph{Orient und
  Okzident}, editada pelo teólogo Fritz Lieb, onde foi publicado em
  outubro do mesmo ano. Provavelmente entre julho e outubro de 1936, ou
  entre este período e 1939, Benjamin trabalhou numa versão francesa,
  porventura para \emph{Europe}, mas o texto não pôde ser publicado pois
  a revista parou de circular. Esta versão francesa, que foi confiada à
  Adrienne Monnier, personagem influente no meio literário parisiense,
  seria publicada somente em 1952, no \emph{Mercure de France}. [\versal{N}.
  \versal{E}.]}}{}

\section{\versal{I}}

Embora seu nome soe familiar, o contador de histórias não está mais
presente entre nós em sua eficácia viva. Ele é para nós algo já
longínquo, e fica cada vez mais distante. Apresentar um Leskov\footnote{Nikolai
  Leskov nasceu em 1831 na província de Oriol e morreu em 1895 em São
  Petersburgo. Ele tinha em comum com Tolstói seus interesses e
  simpatias pelos camponeses e, com Dostoiévski, a sua orientação
  religiosa. Mas, precisamente aqueles escritos cuja expressão é
  fundamental e doutrinária, os romances de sua juventude, podem ser
  considerados como a parte mais efêmera de sua obra. A importância de
  Leskov repousa sobre os contos que pertencem a uma fase mais tardia de
  sua produção. Desde o final da guerra, muitas tentativas foram
  empreendidas no sentido de tornar estes contos conhecidos no círculo
  linguístico alemão. Ao lado das pequenas antologias da Editora
  Musarion e da Editora Georg Müller, encontra-se, principalmente, a
  coletânea em nove volumes da Editora \versal{C}. \versal{H}. Beck. [\versal{N}. de \versal{W.B.}]}
como contador não significa trazê-lo para mais perto de nós, mas
aumentar a distância que nos separa dele. Se o consideramos com um certo
distanciamento, os traços grandes e simples que caracterizam o contador
nele ganham relevo. Melhor dizendo, aparecem tal como um rosto humano ou
um corpo de animal podem aparecer num rochedo para alguém que os examine
a uma boa distância e do ponto de vista adequado. Essa distância e este
ponto de vista são impostos por uma experiência que fazemos quase
diariamente. Ela nos diz que a arte de contar histórias se aproxima de
seu fim. Torna-se cada vez mais raro encontrarmos pessoas que ainda
sabem contar alguma coisa. Cada vez mais frequentemente generaliza-se o
embaraço quando, num grupo, alguém pede uma história. É como se
tivéssemos sido privados de uma faculdade que nos parecia inalienável, e
era a mais segura entre todas: a faculdade de trocar experiências.
%O primeiro parágrafo do ensaio está com um recuo que foge ao padrão, mas não consigo tirar.

Uma causa desse fenômeno é clara: a cotação da experiência caiu. E
parece que continuará a cair infinitamente. Basta olharmos os jornais
para constatarmos que ela atingiu um novo nível, ainda mais baixo, e que não
apenas a imagem do mundo exterior mas também a do mundo moral sofreram
da noite para o dia alterações que nunca antes nos teriam parecido
possíveis. Com a Guerra Mundial, começou a tornar-se manifesto um
processo que desde então não encontrou repouso. Não reparamos que,
quando a guerra acabou, os soldados voltaram mudos dos campos de
batalha? Não mais ricos, mas mais pobres em experiências comunicáveis. O
que se difundiu dez anos depois, com a enxurrada de livros sobre a
guerra, não tinha nada a ver com uma experiência passada de boca em
boca. E não havia nada de estranho nisso. Pois nunca experiências foram
tão fundamentalmente desmentidas quanto a experiência estratégica, pela
guerra de trincheira, a experiência econômica, pela inflação, a
experiência corporal, pela batalha com armamentos pesados e com aviões,
e a experiência ética, pelos detentores do poder. Uma geração que ainda
fora à escola num bonde puxado por cavalos se achava a céu aberto numa
paisagem em que nada permanecera inalterado, a não ser as nuvens, e
abaixo delas, num campo de forças cheio de tensões e explosões
destrutivas, o minúsculo, frágil corpo humano.

\section{\versal{II}}

A experiência que se transmite oralmente é a fonte da qual beberam todos
os contadores de histórias. E entre os que as escreveram, os melhores
são aqueles cujos escritos menos se afastam da fala dos muitos
contadores anônimos. Entre esses últimos, além disso, há dois grupos que
se interpenetram de diversas maneiras. A figura do contador só adquire
sua plena corporeidade se o apresentamos sob os traços de ambos. ``Quem
viaja muito, tem sempre muito o que contar'', diz a voz do povo, que
representa o contador como aquele que vem de longe. Mas também é com
prazer que ouvimos os casos daquele que ficou na sua terra, ganhando
honestamente sua vida, e conhece suas histórias e tradições. Se
quisermos apresentar esses dois grupos através de seus representantes
arcaicos, um se encarna no camponês sedentário e o outro no marinheiro
mercador. De fato, essas duas formas de vida constituíram, de certo
modo, sua própria linhagem de contadores. Cada uma delas manteve algumas
de suas características durante séculos. Assim, entre os contadores
alemães modernos, Hebel e Gotthelf pertencem à primeira enquanto
Sealsfield e Gerstäcker vêm da segunda.\footnote{Johann Peter Hebel
  (1760---1826), filho de camponeses, foi teólogo e professor. Poeta de
  expressão alemã e dialetal, escreveu também contos em prosa. Jeremias
  Gotthelf (1797---1854), pseudônimo do escritor suíço de expressão alemã
  Albert Bitzius. Filho de pastor, foi também pastor além de romancista.
  Procurou descrever o impacto da modernização na sociedade camponesa.
  Charles Sealsfield (1793---1864), pseudônimo de Carl Anton Postl,
  jornalista e escritor de origem austríaca, naturalizado americano.
  Friedrich Gerstäcker (1816---1872), viajante e novelista alemão,
  tornou-se conhecido por seus livros sobre o continente americano.
  [\versal{N}. \versal{T}.]} No entanto, tais linhagens, como foi dito,
correspondem apenas a tipos fundamentais. A real extensão do reino das
narrativas, em toda sua dimensão histórica, não é pensável se não
levamos em conta a íntima interpenetração desses dois tipos arcaicos.
Uma tal interpenetração foi particularmente favorecida pelas corporações
de artesãos da Idade Média. O mestre sedentário e o oficial itinerante trabalhavam juntos no mesmo ateliê, e cada mestre fora oficial itinerante
antes de se estabelecer em sua terra ou no estrangeiro. Se camponeses e
marinheiros foram os antigos mestres da narrativa, o artesanato foi sua
melhor escola. Nele, se associava o saber que vem de longe, trazido para
casa por aquele que viajou muito, com o saber do passado tal como 
é confiado, preferencialmente, ao sedentário.\footnote{Na versão em
  francês, a seguinte passagem foi acrescida ao final do parágrafo: ``É
  assim que se constitui esse personagem do contador que, como bem disse
  Jean Cassou, `dá o tom do relato e de sua realidade, aquele perto do
  qual o leitor \ldots{} gosta de se refugiar fraternalmente e de
  encontrar a medida, a escala dos sentimentos e dos fatos humanos
  normais'.'' [\versal{N}. \versal{T}.]}

\section{\versal{III}}

Leskov está em casa no longínquo, seja este espacial ou temporal. Ele
pertenceu à Igreja Ortodoxa e foi um homem de sinceros interesses
religiosos. Mas ele foi também um sincero opositor da burocracia
eclesiástica. Como suas relações com o funcionalismo secular não eram
melhores, as posições oficiais que obteve não duraram muito. Para sua
produção, o posto de agente russo de uma empresa inglesa, ocupado
por muito tempo, foi provavelmente o mais útil. Viajou pela Rússia
inteira a serviço dessa empresa, e tais viagens formaram tanto sua
inteligência do mundo\footnote{``sua experiência dos homens'', na
  variante da versão francesa. [\versal{N}. \versal{T}.]} quanto seu conhecimento
da situação russa. Desse modo, teve a possibilidade de entrar em contato
com as seitas implantadas no campo, que deixaram traços em seus contos.
Nas lendas russas, Leskov\footnote{``Leskov, que não escondia suas
  simpatias sectárias'', conforme acréscimo feito por Benjamin na
  versão francesa. [\versal{N}. \versal{T}.]} encontrou aliados em sua luta contra
a burocracia ortodoxa. Escreveu uma série de contos lendários em cujo
centro está exposto o justo, raramente um asceta, com mais frequência um
homem simples e ativo, que torna-se um santo da maneira mais natural do
mundo. A exaltação mística não é assunto para Leskov. Por mais que às
vezes se entregasse com prazer ao maravilhoso, mantinha-o, de
preferência, mesmo em questões de piedade, no quadro de uma sólida
naturalidade. Ele via seu modelo no homem acostumado a tratar de coisas
terrenas sem se prender demasiadamente a elas, e adotava, no domínio
secular, uma atitude semelhante a essa. Combina com esse seu modo de ser
o fato de ter começado a escrever tarde, aos 29 anos. Isso se deu durante
suas viagens comerciais. Seu primeiro trabalho publicado se intitulava:
\emph{Por que em Kiev os livros são tão caros?}, uma longa série de escritos
sobre a classe trabalhadora, alcoolismo, os médicos da
polícia e comerciários desempregados precede seus contos.

\section{\versal{IV}}

O interesse prático é um traço característico de muitos contadores
natos. Mais duradouro do que em Leskov, podemos reconhecê-lo por 'exemplo'
em Gotthelf, que dava conselhos de agronomia aos seus camponeses; também
se encontra em Nodier, que se preocupava com os perigos da iluminação a
gás; e Hebel, que transmitia aos seus leitores pequenas informações de
ciência natural na \emph{Caixinha de tesouros},\footnote{Referência à
  coletânea de contos de Johann Peter Hebel (1760---1826), intitulada
  \emph{Caixinha de tesouros do amigo renano das famílias}
  (\emph{Schatzkästlein des Rheinischen Hausfreundes}). [\versal{N}. \versal{T}.]}
pertence igualmente a esse grupo. Tudo isso esclarece sobre as
características da verdadeira narrativa. Ela traz em si, abertamente ou
de modo secreto, sua utilidade. Tal utilidade pode aparecer aqui numa
moral, ali numa recomendação prática, ou ainda num provérbio ou numa
regra de vida -- em cada um desses casos, o contador é um homem que sabe
dar conselhos aos seus ouvintes. Mas se hoje em dia ``saber aconselhar''
começa a soar antiquado aos ouvidos, isso se deve à perda progressiva da
comunicabilidade da experiência. Por isso não sabemos mais dar
conselhos, nem a nós mesmos nem aos outros. Conselho é menos a resposta
a uma pergunta do que uma sugestão de continuação para uma história (que
está se desenrolando). Para poder obtê-lo, é preciso primeiro ser capaz
de contá-la (sem considerar que um ser humano só se abre a um conselho
quando deixa que sua situação se expresse em palavras). Conselho,
entretecido na matéria da vida vivida, é sabedoria. A arte de contar se
aproxima de seu fim pois o lado épico da verdade\label{supra2}, a sabedoria, está se
extinguindo. Mas esse processo vem de longe. E nada seria mais insensato
do que considerá-lo como um ``sinal de decadência'' e ainda menos de
``decadência moderna''. Ao contrário, esse processo é antes um fenômeno
ligado às forças produtivas, seculares e históricas, que expulsam aos
poucos o conto do domínio da palavra viva, ao mesmo tempo que conferem
uma nova beleza ao que está desaparecendo.\footnote{Na variante da
  versão francesa: ``Trata-se mais propriamente de um fenômeno
  consistindo em forças seculares que pouco a pouco separaram o contador
  do domínio da palavra viva para confiná-lo na literatura. Esse
  fenômeno nos tornou, ao mesmo tempo, mais sensíveis à beleza de um
  gênero que se vai.'' [\versal{N}. \versal{T}.]}

\section{\versal{V}}

O primeiro indício de um processo que culmina no declínio da
narrativa é o aparecimento do romance no início da época moderna. O que
distingue o romance do conto (e da epopeia num sentido estrito) é sua
ligação essencial com o livro. A difusão do romance só se tornou
possível com a invenção da imprensa. A transmissão oral, patrimônio da
épica, é de natureza diferente daquela que caracteriza o romance. O que
distingue o romance em comparação com todas as outras formas de prosa --
contos de fada, lendas, e mesmo novelas -- é que nem provém da tradição
oral nem a ela conduz. Isso o distingue em primeiro lugar do conto. O
contador de histórias tira o que ele conta da sua própria experiência ou
da que lhe foi relatada por outros. E ele, por sua vez, a transforma em
experiência para aqueles que escutam sua história. O romancista
isola-se. O lugar de nascimento do romance é o indivíduo em sua solidão:
aquele que não é capaz de expor suas preocupações mais altas de modo
exemplar, é ele mesmo carente de conselhos e não sabe dá-los. Escrever
um romance significa exacerbar o incomensurável na apresentação de uma
vida humana. De dentro da plenitude da vida, e através da apresentação
de tal plenitude, o romance aprofunda a ausência de conselho dos
viventes. O primeiro grande livro desse gênero, \emph{Dom Quixote},
ensina como a grandeza de alma, a coragem e a generosidade de um dos
mais nobres heróis -- o próprio \emph{Dom Quixote} -- foram
completamente abandonadas pelo bom conselho e não contém mais a menor
centelha de sabedoria.\footnote{Na versão francesa, o capítulo termina
  em ``\emph{la moindre étincelle de sagesse}'', ``a menor centelha de
  sabedoria''. A sequência final do capítulo, que consta no original
  alemão, foi eliminada pelo autor. [\versal{N}. \versal{T}.]} Quando ao longo dos
séculos, aqui e ali, procurou-se incutir ensinamentos no romance -- e
talvez da maneira mais consistente n'\emph{Os Anos de peregrinação de
Wilhelm Meister} --, tais tentativas sempre acabaram por transformar a
própria forma do romance. Por outro lado, o romance de formação não
contraria de modo algum a estrutura fundamental do romance. Integrando o
processo de vida da sociedade no desenvolvimento de uma pessoa, ele
justifica as ordens que condicionam o primeiro do modo mais fraco. Sua
legitimação encontra-se em desaprumo com sua realidade. Precisamente no
romance de formação, o que é insuficiente torna-se evento.
% O livro do Goethe normalmente é traduzido como 'Os anos de aprenzidado...'mantem peregrinação?

\section{\versal{VI}}

Devemos pensar a mutação das formas épicas em ritmos comparáveis aos das
transformações que se operaram na superfície terrestre ao longo de
milhares de séculos. Dificilmente outras formas de comunicação humana se
desenvolveram mais lentamente e se perderam mais lentamente. O romance,
cujos primórdios remontam à Antiguidade, levou centenas de anos até
encontrar na burguesia nascente os elementos de que precisava para
florescer. Quando esses elementos entraram em cena, o conto começou
lentamente a se refugiar no domínio do arcaico; se apropriou de diversas
maneiras do novo conteúdo, embora não fosse propriamente condicionado
por ele. Por outro lado, reconhecemos que com o apogeu da burguesia --
da qual a imprensa constitui um dos mais importantes instrumentos no
capitalismo avançado -- apareceu uma forma de comunicação que, embora
suas origens sejam muito antigas, nunca antes influenciara a forma
épica. Agora o faz. E torna-se visível que ela se opõe ao conto de modo
não menos estranho, mas muito mais ameaçador que o romance, além de
provocar uma crise no próprio romance. Essa nova forma de comunicação é
a informação.

Villemessant, o fundador do \emph{Figaro}, descreveu a essência da
informação com uma célebre fórmula: ``Para os meus leitores'', dizia,
``o incêndio num sótão do \emph{Quartier Latin} é mais importante do que
uma revolução em Madri''. Essa expressão mostra de modo claro e sucinto
que a informação sobre os acontecimentos próximos encontra hoje 
audiência muito maior do que a mensagem que vem de longe. A mensagem que
vinha de longe -- seja espacialmente, de terras estrangeiras, seja
temporalmente, da tradição -- dispunha de uma autoridade que lhe
fornecia validade mesmo onde não fora submetida ao controle. Mas a
informação reclama verificação imediata. Em primeiro lugar, ela precisa
ser ``compreensível em si e para si''. Frequentemente não é mais exata
do que relatos dos séculos anteriores. Entretanto, enquanto esses podiam
lançar mão do maravilhoso, a informação deve soar plausível. Por isso
ela é irreconciliável com o espírito do conto. Se a arte de contar
tornou-se hoje rara, a difusão da informação desempenhou um papel
decisivo em tal situação.

Cada manhã nos informa acerca das novidades do globo terrestre. E mesmo
assim somos pobres em histórias dignas de nota. A razão é que nenhum
fato mais nos atinge sem estar cercado de explicações. Em outras
palavras: com isso, quase nada mais do que acontece vem a favor do
conto, quase tudo se torna informação. A metade da arte de contar está em
despojar de explicações a história contada\label{supra3}. Leskov é mestre nisso (basta
pensar em textos como ``A fraude'' ou ``A águia branca'').\footnote{Ambos
  se encontram traduzidos para o português, tendo sido editados na
  coletânea intitulada \emph{A Fraude e outras histórias}. Tradução de
  Denise Sales. São Paulo: Editora 34, 2012. Também pela Editora 34
  foram publicados, de Nikolai Leskov, o volume de contos \emph{Homens
  interessantes e outras histórias} (tradução de Noé Oliveira Policarpo
  Polli, 2012) e a novela \emph{Lady Macbeth do Distrito de Mtzensk}
  (tradução de Paulo Bezerra, 2009). [\versal{N}. \versal{T}.]} Ele
conta o extraordinário, o maravilhoso com a maior exatidão, mas o
encadeamento psicológico dos acontecimentos não é imposto ao leitor
que fica livre para interpretar a coisa como a entende, e com isso o
que é contado atinge uma amplitude que a informação não tem.

\section{\versal{VII}}

Leskov frequentou a escola dos antigos. O primeiro contador de histórias
grego foi Heródoto. No décimo quarto capítulo do terceiro livro de suas
\emph{Histórias}, encontra-se um relato que muito nos ensina. É sobre
Psaménito. Quando o rei egípcio Psaménito foi vencido e aprisionado pelo
rei da Pérsia Cambises, este decidiu humilhar seu prisioneiro. Ele
ordenou que Psaménito fosse levado até a rua na qual passaria o cortejo
triunfal dos persas. E, assim, fez com que o prisioneiro ainda visse sua
filha, reduzida à condição de serva, indo buscar água no poço com um
jarro. Enquanto todos os egípcios bradavam e se lamentavam com esse
espetáculo, só Psaménito ficou silencioso e imóvel, com os olhos baixos;
e quando, logo depois, assistiu seu filho ser conduzido em cortejo para
execução pública, permaneceu igualmente imóvel. Mas ao reconhecer na
fila dos prisioneiros um de seus antigos servidores, um homem velho e
depauperado, então bateu na cabeça com os punhos e deu sinais da mais
profunda tristeza.

Essa história mostra como é o verdadeiro conto. A informação tira seu
valor do instante em que é nova. Ela vive apenas desse instante, deve
render-se a ele e explicar-se nele sem perda de tempo. O conto funciona
de outro modo: não se desgasta. Ele guarda em si mesmo suas forças
reunidas e longo tempo depois ainda é capaz de se desenvolver. Assim,
Montaigne retornou à história do rei egípcio e se perguntou: ``Por que ele
se lamenta apenas quando vê seu servidor?'' Montaigne respondeu: ``Como já
estava repleto de tristeza, a menor sobrecarga bastou para arrebentar
todos os diques''.\footnote{Cf. Michel de Montaigne, \emph{Essais},
  Livro \versal{I}, capítulo \versal{II}. Paris: Gallimard, 1962. (Col. Bibliothèque de la
  Pléiade). [\versal{N}. \versal{T}.]} Isso disse Montaigne. Mas poderíamos também
dizer: ``O rei não se comove com o destino da realeza pois é o seu
próprio destino''. Ou: ``Muito do que nos comove na cena teatral não nos
comove na vida: esse servidor é para o rei apenas um ator''. Ou ainda:
``A maior dor fica represada e só vem à tona quando ocorre uma
distensão. A visão desse servidor foi a distensão.'' -- Heródoto não
explica nada. Seu relato é dos mais secos. É por isso que essa história
do Antigo Egito ainda desperta espanto e reflexão depois de milhares de
anos. Ela se assemelha àquelas sementes que, durante milênios, ficaram
fechadas hermeticamente nas câmaras das pirâmides e ainda hoje conservam
o seu poder de germinar.\footnote{Por volta de 1928---1929, Benjamin
  escreveu um texto curto intitulado ``Arte de contar histórias''
  (``Kunst zu erzählen'', em \emph{Imagens de Pensamento},
  \emph{Denkbilder}, \versal{GS} \versal{IV}-1, pp. 436---438), que já abordava a história
  de Psaménito e explorava a plurivocidade de sua extrema concisão.
  Entre os manuscritos do início dos anos trinta, encontramos também
  notas sobre Psaménito, contendo diversas interpretações propostas por
  seus amigos e conhecidos:

  ``Interpretação de [Franz] Hessel~: `o rei não se emociona com a
  sorte da família real, pois é a sua própria.'

  Interpretação de Benjamin: `a dor se desencadeia mais facilmente sob
  um pretexto menos importante do que sua causa. É uma grande tampa
  sobre uma pequena panela. Ou então ela evita até o pretexto e
  privilegia o choque. É um tal choque que desencadeia as lágrimas em
  Proust depois da morte de sua avó querida: o gesto de se abaixar para
  abotoar as botinas.'

  Interpretação de Asja [Lacis]: `Muitas coisas que nos emocionam no
  teatro nos deixam indiferentes na vida; esse velho é apenas um ator
  para o rei.'

  Interpretação de Montaigne: `Ocorre que estando além disso já pleno e
  cheio de tristeza, a menor sobrecarga arrebentou as barreiras da
  paciência.'

  Observação de Martin-Gueillot: `Se Psaménito tivesse vivido em nossos
  dias, todos os jornais nos teriam ensinado que ele preferia seu
  empregado a seus filhos.' (\emph{Nouvelle Revue Française}, 1928, p.
  696).'' Cf. \versal{GS} \versal{IV}-2, p. 1011. [\versal{N}. \versal{T}.]}

\section{\versal{VIII}}

Não há nada que imprima mais duradouramente as histórias na memória do
que essa recatada concisão que as afasta da análise psicológica. E
quanto mais naturalmente o contador renuncia aos detalhes psicológicos,
mais facilmente sua história encontra lugar na memória do ouvinte, mais
perfeitamente junta-se à sua própria experiência, e assim mais prazer
ele terá um dia, próximo ou longínquo, em recontá-la. Esse processo de
assimilação, que ocorre em camadas profundas, exige um estado de
distensão que se torna cada vez mais raro. Se o sono é o ponto mais alto
da distensão física, o tédio corresponde ao ponto mais alto da distensão
espiritual. O tédio é o pássaro de sonho que choca o ovo da experiência\label{supra6}.
O menor farfalhar de folhas o afasta. Seus ninhos, as atividades
intrinsecamente ligadas ao tédio, não encontram mais lugar nas cidades e
estão se tornando raras também no campo. Com isso, perde-se o dom de
escutar e desaparece a comunidade dos que escutam. Contar histórias é
sempre a arte de contá-las novamente, arte que se perde quando as
histórias não são mais conservadas. Ela se perde pois ninguém mais tece
nem fia enquanto ouve histórias. Quanto mais o ouvinte se esquece de si
mesmo, mais profundamente aquilo que escuta se imprime nele. Onde o
ritmo do trabalho o toma inteiramente, ele ouve as histórias de tal
maneira que o dom de contá-las lhe vem espontaneamente. Assim foi tecida
a rede na qual está contido o dom de contar. Assim essa rede se desfaz
hoje por todos os lados, depois de ter sido tecida há milhares de anos,
em torno das mais antigas formas de artesanato.

\section{\versal{IX}}

O conto, tal como prosperou longo tempo na esfera do artesanato --
artesanato camponês, marítimo e depois urbano --, é ele mesmo uma forma
artesanal de comunicação. Ele não visa transmitir o puro ``em si'' da
coisa como uma informação ou um relatório. Ele mergulha a coisa na vida
daquele que a relata para em seguida daí retirá-la. O contador deixa sua
marca no conto, assim como o oleiro deixa a impressão de sua mão na
argila do vaso. \label{supra}Os contadores têm sempre a tendência a começar suas
histórias com uma apresentação das circunstâncias em que tomaram
conhecimento do que irão em seguida contar, isso quando não preferem
dizer que o vivenciaram pessoalmente. Leskov inicia ``A
fraude''\footnote{Cf. Nikolai Leskov, \emph{A Fraude e outras
  histórias}, ed. cit. [\versal{N}. \versal{T}.]} com a descrição de uma viagem de
trem na qual teria ouvido de um outro viajante os acontecimentos que
passa em seguida a contar; ou pensa no enterro de Dostoiévski, no qual
travou conhecimento com a heroína de seu conto ``A propósito de \emph{A
Sonata a Kreutzer}''\footnote{Idem, ibidem. [\versal{N}. \versal{T}.]}; ou faz
alusão à reunião de um grupo de leitura no qual falou-se dos fatos que
relata em ``Homens interessantes''.\footnote{Cf. Nikolai Leskov,
  \emph{Homens interessantes e outras histórias}, ed. cit. [\versal{N}. 
  \versal{T}.]} Assim, deixa sua marca naquilo que conta de diversos modos,
seja como aquele que o viveu, seja como aquele que o relata.

Essa arte artesanal de contar, o próprio Leskov a considerava como um
ofício. ``A literatura'', diz em suas cartas, ``não é para mim uma arte
livre, mas um ofício artesanal''. Não é de se estranhar que ele tenha se
sentido ligado ao artesanato e hostil à técnica industrial. Tolstói, que
desse ponto de vista devia compreendê-lo, indica este aspecto do talento
narrativo de Leskov quando o aponta como o primeiro a ``denunciar a
deficiência do progresso industrial (\ldots{}). É estranho que tanta gente
leia Dostoiévski (\ldots{}). Por outro lado, simplesmente não entendo porque
Leskov não é lido. É um escritor verídico''.\footnote{Citado por Erich
  Müller, ``Nikolai Semjonowitsch Lesskow. Sein Leben und Wirken'',
  in~Nikolai Lesskow, \emph{Am Ende der Welt}. München: \versal{C}. \versal{H}. Beck, 1927,
  p. 313. (\emph{Gesammelte Werke}, v. \versal{IX}) [\versal{N}. \versal{T}.]} Numa
história maliciosa e petulante, ``A pulga de aço'', a meio caminho entre
a lenda e a farsa, Leskov louva, através dos ourives de Tula, o
artesanato de sua terra. Sua obra prima, a pulga de aço, chega aos olhos
de Pedro, o Grande, e o convence de que os russos não precisam se
envergonhar diante dos ingleses.

A imagem\footnote{Na variante da versão francesa: ``lei''. [\versal{N}. 
\versal{T}.]} espiritual dessa esfera artesanal da qual provém o contador
nunca foi descrita de modo mais significativo do que por Paul Valéry. Ao
falar de coisas perfeitas na natureza, pérolas imaculadas, vinhos
maduros e profundos, de criaturas verdadeiramente completas, ele
reconhece nelas ``a valiosa obra de uma longa cadeia de causas
sucessivas e semelhantes''.\footnote{Paul Valéry, ``Les broderies de
  Marie Monnier'', \emph{Pièces sur l'art}, in~\emph{Œuvres}, tomo \versal{II}.
  Edição de Jean Hytier. Paris: Gallimard, 1960, p. 1244. (Col.
  Bibliothèque de la Pléiade). [\versal{N}. \versal{T}.]} Mas a acumulação de tais
causas encontraria seu limite de tempo apenas na perfeição. ``Outrora'',
continua Paul Valéry, ``o homem imitava esta paciência da natureza.
Iluminuras, marfins inteiramente entalhados à perfeição, pedras duras
perfeitamente polidas e distintamente gravadas; lacas ou pinturas
obtidas através da sobreposição de uma série de camadas finas e
translúcidas (\ldots{}) -- todas essas produções de um esforço tenaz e pleno
de autorrenúncia estão desaparecendo, e passou o tempo em que o tempo
não contava. O homem de hoje não cultiva mais o que não pode ser
abreviado''.\footnote{Idem, ibidem. [\versal{N}. \versal{T}.]} De fato, conseguiu
até abreviar o conto. Vivenciamos o surgimento da \emph{short story},
que se emancipou da tradição oral e não mais permite essa lenta
sobreposição de camadas finas e transparentes que oferece a melhor
imagem da maneira pela qual o conto perfeito vem à luz do dia a partir
das camadas acumuladas por suas diferentes versões\label{supra7}.\footnote{``Fomos
  testemunhas do nascimento da \emph{short story}. Ela abala o prestígio
  do conto, o qual consiste em religar as gerações de contadores entre
  si.'', conforme variante da versão francesa. [\versal{N}. \versal{T}.]}

\section{\versal{X}}

Valéry conclui suas considerações com a seguinte frase: ``Pode-se dizer
que o enfraquecimento nos espíritos da ideia de eternidade coincide com
a aversão crescente por tarefas demoradas''.\footnote{Paul Valéry, ``Les
  broderies de Marie Monnier'', \emph{op. cit.}, p. 1244. [\versal{N}. \versal{T}.]}
A ideia de eternidade teve sempre na morte sua fonte mais poderosa. Se
essa ideia se enfraqueceu, isso significa que o rosto da morte também se
modificou. Essa modificação demonstra ser a mesma que reduziu a
comunicabilidade da experiência na medida em que a arte de contar
chegava ao fim.

Se seguimos o decorrer dos séculos precedentes, percebemos a que ponto a
ideia da morte perde a onipresença e a força plástica que encontrava na
consciência coletiva. Em suas últimas etapas, esse processo se acelerou.
No decurso do século~\versal{XIX}, a sociedade burguesa, com suas instituições
econômicas e sociais, públicas ou privadas, realizou um feito acessório
que, inconscientemente, foi talvez seu objetivo principal: dar às
pessoas a possibilidade de se esquivar à visão dos moribundos. O ato de
morrer, outrora o mais público e o mais exemplar da vida individual
(lembremo-nos aqui das imagens da Idade Média nas quais a cama do
moribundo vira um trono diante do qual se aglomera o povo que entra
pelas portas abertas de sua casa), subtraiu-se aos poucos da atenção dos
vivos no decorrer da época moderna. Outrora não havia casa, por vezes
nem mesmo quarto, onde ninguém tivesse morrido (a Idade Média tinha
também a intuição espacial do sentimento temporal evocado por aquela
inscrição num relógio solar de Ibiza: \emph{Ultima multis}.) Hoje em dia
os burgueses vivem em espaços depurados de qualquer morte, primeiros
moradores da eternidade, e, quando chegam perto do fim, são depositados
por seus herdeiros em sanatórios e hospitais. Ora, não apenas o
conhecimento e a sabedoria de um ser humano, mas sobretudo sua vida
vivida -- e essa é a matéria da qual as histórias são feitas -- assumem
uma primeira forma transmissível no leito do moribundo. Assim como no
interior da pessoa, no momento final de sua vida, uma sequência de
imagens se põe em movimento -- constituídas das visões de si, sob as
quais, sem se dar conta, encontrou-se consigo mesma --, também se revela
de repente, em seus gestos e olhares, o inesquecível que atribui a tudo
o que lhe diz respeito essa autoridade que mesmo o mais miserável dos
moribundos possui aos olhos dos vivos à sua volta. É essa autoridade que
está na origem do que foi contado.

\section{\versal{XI}}

A morte é a sanção de tudo o que o contador de histórias pode contar. É
a morte que lhe confere sua autoridade. Em outras palavras: suas
histórias remetem à história natural. Isso se exprime de forma exemplar
numa das mais belas passagens do incomparável Johann Peter Hebel, que se
encontra em ``Reencontro inesperado'',\footnote{Na versão francesa,
  ``Reencontro inesperado dos Amantes. [\versal{N}. \versal{T}.]} conto
incluído n'\emph{A caixa de tesouros do amigo renano das famílias}. Este
relato começa com o noivado de um jovem operário que trabalha nas minas
de Falun.\footnote{``(\ldots{}) nas minas de Falun na Suécia'', consoante o
  acréscimo pontual da versão francesa. [\versal{N}. \versal{T}.]} Na véspera do
casamento, no fundo de sua galeria subterrânea, a morte dos mineiros o
surpreendeu. Sua noiva permanece fiel depois de sua morte e vive ainda
um longo tempo. Um dia, quando já é uma velhinha de idade avançada, um
cadáver é trazido à luz do fundo da galeria abandonada. Impregnado de
vitríolo de ferro, escapou da decomposição e ela pôde reconhecer o seu
noivo. Depois desse reencontro, também ela foi chamada pela morte.
Quando Hebel, no curso de seu conto, sentiu a necessidade de tornar
palpável a longa série de anos que separa o começo do fim, ele o fez com
as seguintes frases: ``Entrementes a cidade de Lisboa foi destruída por
um terremoto, e a guerra dos sete anos passou, e o Imperador Francisco~\versal{I}
morreu, e a ordem dos jesuítas foi dissolvida, e a Polônia foi
repartida, e a Imperatriz Maria Teresa morreu, e Struensee foi
executado, a América tornou-se livre e as forças reunidas da França e da
Espanha não puderam conquistar Gibraltar. Os turcos aprisionaram o
General Stein na gruta dos veteranos, na Hungria, e o Imperador José
morreu também. O rei Gustavo da Suécia conquistou a Finlândia russa, e a
Revolução Francesa e a longa guerra começaram, e o Imperador Leopoldo~\versal{II}
também foi para a cova. Napoleão conquistou a Prússia, e os ingleses
bombardearam Copenhague, e os camponeses semearam e ceifaram. O moleiro
moeu, e os ferreiros forjaram, e os mineiros cavaram a terra em busca de
filões metálicos em seu canteiro subterrâneo. Ora, quando os mineiros de
Falun, no ano de 1809\ldots{}''\footnote{Johan Peter Hebel, \emph{Sämtliche
  Werke}, v. \versal{III}. Karlsruhe: Müller, 1834, p. 188. [\versal{N}. \versal{T}.]}
Nunca um contador introduzira seu relato na história natural de forma
mais profunda do que Hebel nessa cronologia. Leia-se atentamente: a
morte nela reaparece de modo tão regular quanto o homem com a foice nos
cortejos das procissões que desfilam ao meio-dia em torno do relógio da
catedral.

\section{\versal{XII}}

Cada estudo de uma certa forma épica deve levar em consideração a
relação dessa forma para com a historiografia. Mais ainda, podemos até
nos perguntar se a historiografia não apresenta o ponto de indiferença
criadora entre todas as formas épicas. Nesse caso, a história escrita
estaria para as formas épicas assim como a luz branca para as cores do
espectro. Seja como for, entre todas as formas épicas não há nenhuma que
tenha sido tão indiscutivelmente incorporada à luz pura e incolor da
história quanto a crônica. E na larga gama colorida da crônica, as
diferentes maneiras de contar se escalonam como as nuances de uma e
mesma cor. O cronista é o contador da história. Lembremo-nos ainda da
passagem de Hebel, que tem o tom da crônica, e notaremos sem dificuldade
a diferença entre aquele que escreve a história, o historiador, e aquele
que a conta, o cronista. O historiador deve de uma maneira ou de outra
explicar os fatos que o ocupam; ele não poderia de modo algum se
contentar em expô-los como amostras do curso do mundo. É justamente o
que faz o cronista, e especialmente os seus representantes clássicos, os
cronistas medievais, que foram os precursores dos historiógrafos
modernos. Na base de sua narrativa da história encontra-se o plano
divino da salvação, que é insondável, e com isso se desembaraçaram de
antemão do ônus de uma explicação demonstrável. Essa cede lugar à
interpretação, a qual não se preocupa de modo algum em encadear com
precisão fatos determinados, mas limita sua tarefa a descrever como
eles se inserem na trama insondável do curso do mundo.

Se o curso do mundo é condicionado por uma história sagrada ou por uma
história natural não faz nenhuma diferença. No contador de histórias, a
figura do cronista conservou-se transformada e, por assim dizer,
secularizada. Leskov está entre aqueles cuja obra dá testemunho de modo
particularmente claro desse estado de coisas. O cronista, com sua
orientação para a história sagrada, e o contador, com sua orientação
para a história profana, fundam-se ambos em sua obra, de tal modo que em
muitos contos é difícil decidir se o fundo sobre o qual eles se
destacam é a trama dourada de uma concepção religiosa ou a trama
colorida de uma visão secularizada do curso das coisas. Pensemos no
conto ``A Alexandrita'', que leva os leitores a ``priscas eras, quando
tanto as pedras nas entranhas da terra quanto os planetas nas alturas
celestes, todos eles se preocupavam com o destino do homem, e não
atualmente, quando até nos céus há desgosto e sob a terra restou a
indiferença fria pelo destino dos filhos dos homens e de lá não chegam
vozes nem obediência. Todos os planetas, de novo descobertos, já não
recebiam mais nenhuma atribuição nos horóscopos; há também muitas pedras
novas, e todas são medidas, pesadas, comparadas em termos de peso
específico e densidade, mas depois nada profetizam, não são úteis em
nada. O seu tempo de falar ao homem já virou passado''.\footnote{Nikolai
  Leskov, ``Alexandrita'', in \emph{A Fraude e outras histórias}, ed.
  cit., p. 160. [\versal{N}. \versal{T}.]}

Como se vê, não é possível caracterizar o curso do mundo de modo
unívoco, tal como o ilustra essa história de Leskov. É ele determinado
pela história sagrada ou pela história natural? É certo apenas que,
enquanto tal, o curso do mundo é estranho a toda categoria propriamente
histórica. Foi-se a época, diz Leskov, em que o homem podia acreditar
viver em uníssono com a natureza. Schiller chamou essa era de o tempo da
poesia ingênua. O contador permanece fiel a ela e seu olhar não se
desvia do relógio diante do qual avança a procissão das criaturas, onde
a morte tem seu lugar, como líder ou como a última e miserável retardatária.

\section{\versal{XIII}}

Raramente nos damos conta de que a relação ingênua do ouvinte com o
contador é determinada pelo interesse em guardar o que foi contado.\footnote{Neste capítulo, Benjamin utiliza termos que cobrem o campo
  semântico da memória (\emph{Erinnerung}, \emph{Gedächtnis},
  \emph{Eingedenken}) e não têm equivalente imediato em português. O
  próprio autor encontra dificuldade em transpô-los para o francês,
  preferindo traduzir tanto \emph{Erinnerung} quanto \emph{Gedächtnis}
  por ``memória'' e introduzir uma distinção entre a \emph{souvenance},
  que caracteriza o romance, e o \emph{souvenir}, que concerne ao conto,
  distinção que não aparece no original alemão. [\versal{N}. \versal{T}.]} O
que importa ao ouvinte isento é assegurar-se da possibilidade da
repetição. A memória é a faculdade épica por excelência. Somente graças
a uma memória abrangente, a épica pode apropriar-se do curso das coisas,
por um lado, e por outro resignar-se com sua perda irreparável, com o
poder da morte. Não é de se espantar que para um homem simples do povo,
tal como Leskov o imaginou um dia, o tsar, enquanto soberano do cosmos
no qual suas histórias acontecem, possui a memória mais vasta. ``Nosso
imperador e toda a sua família'', diz ele, ``tem efetivamente uma memória
prodigiosa.''

Mnemosine, aquela que recorda, era entre os gregos a musa do gênero
épico. Esse nome nos reconduz a uma bifurcação na história do mundo. Com
efeito, se aquilo que é registrado pela recordação -- a historiografia
-- expõe a indistinção criadora entre os diversos gêneros épicos (assim
como a grande prosa expõe a ausência de diferenciação criadora entre as
diversas métricas poéticas), sua forma mais antiga, a epopeia
propriamente dita, contém, por força de um tipo de indistinção, o conto
e o romance. Quando, no decorrer dos séculos, o romance começou a surgir
no seio da epopeia, viu-se claramente que o elemento inspirador da
poesia épica, a recordação, nele aparecia de um modo diferente daquele
do conto.

A \emph{recordação}\footnote{\emph{Erinnerung}, no original alemão;
  \emph{mémoire}, na versão francesa. [\versal{N}. \versal{T}.]} estabelece a
cadeia da tradição que transmite os acontecimentos de geração em
geração. Ela é a musa da épica em geral e preside todas as variedades do
gênero épico. Entre essas, encontramos em primeiro lugar aquela
encarnada pelo contador. Ela tece a rede formada por todas as histórias.
Uma está ligada à outra, como mostraram todos os grandes contadores, e
principalmente os orientais. Em cada um deles vive uma Sherazade, que em
cada passagem de sua história lembra-se de outra. Esta é uma
\emph{memória}\footnote{\emph{Gedächtnis}, no original alemão;
  \emph{mémoire}, na versão francesa. [\versal{N}. \versal{T}.]} épica e a musa
inspiradora do conto. É preciso opor a ela um outro princípio
inspirador, o do romance, o qual, ainda indiferenciado daquele que é
próprio ao conto, também habita o gênero épico em sentido
estrito.\footnote{Na variante da versão francesa: ``Encontra-se um
  elemento análogo, mas intrinsecamente diferente, na base do romance. E
  como no que diz respeito ao conto, podemos avançar, para o romance,
  que primitivamente, isto é, na epopeia, formava apenas um germe na
  unidade indivisa do gênero épico.'' [\versal{N}. \versal{T}.]} Às vezes, já
podemos pressenti-lo nas epopeias. E isso ocorre antes de tudo nas
invocações solenes das musas que abrem os poemas homéricos. O que se
anuncia em tais passagens é a memória perpetuadora\footnote{\emph{Verewigende
  Gedächtnis}, no original alemão; \emph{souvenance éternisante}, na
  versão francesa. [\versal{N}. \versal{T}.]} do romancista em oposição à memória
divertida\footnote{\emph{Kurzweilige Gedächtnis}, no original alemão;
  \emph{souvenir passe"-temps}, na versão francesa. [\versal{N}. \versal{T}.]} do
contador. A primeira diz respeito a \emph{um} herói, a \emph{uma}
odisseia ou a \emph{uma} guerra; a segunda concerne a \emph{múltiplos}
fatos dispersos. Em outras palavras: a \emph{Rememoração}\footnote{\emph{Eingedenken},
  no original alemão; \emph{souvenance}, na versão francesa. Tanto o
  termo alemão quanto o francês conotam o desenrolar do processo em
  oposição aos seus efeitos pontuais, conotação que procuramos preservar
  na tradução. [\versal{N}. \versal{T}.]}, musa do romance, surge ao lado da
memória\footnote{\emph{Gedächtnis}, no original alemão; \emph{souvenir},
  na versão francesa. [\versal{N}. \versal{T}.]}, musa do conto, depois que a
unidade de sua origem na Recordação\footnote{\emph{Erinnerung}, no
  original alemão; \emph{mémoire}, na versão francesa. [\versal{N}. \versal{T}.]}
se rompeu com o declínio da epopeia.

\section{\versal{XIV}}

``Ninguém'', disse Pascal, ``morre tão pobre que não deixe alguma
coisa''. Com certeza, deixa ao menos recordações -- só que essas nem
sempre encontram herdeiros. O romancista recolhe esse legado, e
raramente sem uma profunda melancolia. Pois com a soma por ele recebida,
ocorre o mesmo que se pode dizer de uma morta num romance de Arnold
Bennett: ``ela não tinha efetivamente vivido''. Devemos o esclarecimento
mais importante sobre esse aspecto das coisas a Georg Lukács, que viu no
romance ``a forma do desenraizamento transcendental''. Do mesmo modo, de
acordo com Lukács, o romance é a única forma de arte que inclui o tempo
entre seus princípios constitutivos. ``O tempo'', diz ele na
\emph{Teoria do Romance}, ``só pode tornar-se constitutivo quando o
homem deixou de estar ligado a uma pátria transcendental (\ldots{}). No
romance, separam-se sentido e vida, e com isso também o essencial e o
temporal; pode-se quase dizer que toda a ação interna do romance não é
nada mais do que uma luta contra o poder do tempo (\ldots{}). E desse
(\ldots{})\footnote{A passagem suprimida neste ponto da citação, feita por
  Benjamin, da obra referida de Lukács, e sem a qual a inteligibilidade
  fica parcialmente comprometida, é: ``sentimento maduro e resignado''
  (\emph{resigniert-mannbaren Gefühl}). Cf. Georg Lukács, \emph{Die
  Theorie des Romans. Ein geschichtsphilosophischer Versuch über die
  Formen der großen Epik}. 9. ed. Darmstadt; Neuwied: Luchterhand,
  1984, p. 110 [1. ed., Berlim, 1920]; e também Georg Lukács,
  \emph{A teoria do romance. Um ensaio histórico-filosófico sobre as
  formas da grande épica}. Tradução de José Marcos Mariani de Macedo.
  São Paulo: Editora 34: Duas Cidades, 2009, p. 130. [\versal{N}. \versal{E}.]}
emergem as vivências temporais autenticamente épicas (\ldots{}): a
esperança e a recordação. (\ldots{}) Apenas no romance (\ldots{}) aparece uma
recordação criadora que atinge o fundamento do seu objeto e o transforma
(\ldots{}). A dualidade da interioridade e do mundo exterior'' só ``pode ser
superada pelo sujeito quando esse percebe a unidade (\ldots{}) de toda a sua
vida (\ldots{}) resumida na recordação da corrente vital do passado (\ldots{}). A
percepção que apreende essa unidade (\ldots{}) torna-se compreensão
divinatoriamente intuitiva do sentido da vida inatingido e, portanto,
inexprimível''.\footnote{Cf. Georg Lukács, \emph{Die Theorie des
  Romans}, ed. cit., pp. 107---115; e Georg Lukács, \emph{A teoria do
  romance}, ed. cit., pp. 128---136\emph{.} [\versal{N}. \versal{E}.]}

``O sentido da vida'' é de fato o eixo em torno do qual gira o romance.
O questionamento desse sentido não é, entretanto, nada mais do que a
expressão simples do desaconselhamento do leitor quando este se vê
inserido no âmago da vida escrita. De um lado, ``o sentido da vida'';
de outro, ``a moral da história'': com essas fórmulas, romance e conto
se confrontam; elas permitem distinguir completamente o índice histórico
do estatuto das duas formas artísticas. Se o primeiro modelo perfeito de
romance é o \emph{Dom Quixote}, \emph{Educação Sentimental} pode ser
considerado como o último. Nas palavras que concluem esse romance, o
sentido, que caracterizava o início do declínio da época burguesa em seu
modo de agir e omitir, se deposita como o resíduo\footnote{``(\ldots{}) resíduo do
  vinho'', na variante da versão francesa. [\versal{N}. \versal{T}.]} no fundo do
copo da vida. Amigos de juventude, Frédéric e Deslauriers rememoram sua
amizade de juventude. Uma pequena história lhes vem à cabeça, eles se
lembram do dia em que, cheios de constrangimento e clandestinamente,
estiveram na casa de tolerância de sua cidade natal apenas para oferecer
à patroa um buquê de flores colhidas no jardim. ``Isso deu numa história
que três anos depois ainda não tinha sido esquecida. Eles a contaram a
si mesmos prolixamente, cada um completando as recordações do outro, e
quando finalmente acabaram: `É o que tivemos de melhor!', disse
Frédéric. -- `Sim, talvez seja mesmo! Isso aí é o que tivemos de
melhor!', disse Deslauriers''. Com tal reconhecimento, o romance chega
ao seu fim; fim que lhe é próprio no sentido mais estrito do que a
qualquer conto. Não há, de fato, nenhum conto em que a questão ``o que
aconteceu depois?'' perca seus direitos. O romance, ao contrário, não
pode esperar dar o menor passo além daquela fronteira onde, ao escrever
a palavra ``fim'' na parte inferior da página, convida o leitor a ter em
mente, por pressentimento, o sentido da vida.

\section{\versal{XV}}

Quem ouve uma história se encontra na companhia do contador; mesmo quem
a lê participa dessa companhia. Por outro lado, mais do que qualquer
outro, o leitor de um romance é solitário (pois até mesmo aquele que lê
um poema tende a emprestar sua voz às palavras, visando ouvintes
possíveis). E nessa solidão que lhe é própria, o leitor do romance se
apodera da matéria lida de modo mais possessivo do que todos os demais.
Ele está disposto a se apropriar dela inteiramente e, de certo modo,
devorá"-la. Sim, ele a destrói, consumindo-a como faz o fogo com a lenha
na lareira. A tensão que atravessa o romance se assemelha muito à
corrente de ar que alimenta a chama e reanima o seu jogo.

É um material seco que alimenta o interesse ardente do leitor. -- O que
isto significa? ``Um homem que morre aos trinta e cinco anos é'', como
disse uma vez Moritz Heimann, ``a cada instante de sua vida, um homem
que morre aos trinta e cinco anos''. Nada é mais duvidoso do que essa
sentença. Mas somente porque o autor se engana quanto ao tempo verbal. A
verdade é que um homem que morreu aos trinta e cinco anos aparecerá na
\emph{rememoração}, em cada instante de sua vida, como um homem que
morre aos trinta e cinco anos. Em outras palavras: a sentença, que não
tem sentido para a vida real, torna-se irrefutável para a vida
recordada. Nada apresenta melhor a essência do personagem de romance. O
``sentido'' de sua vida -- é o que a frase nos diz -- só se mostra a
partir de sua morte. Ora, o leitor de romances procura efetivamente
seres humanos nos quais pode ler o ``sentido da vida''. Ele precisa de
antemão ter certeza de que poderá, de um jeito ou de outro, assistir à
morte deles. Se necessário, a morte no sentido figurado: o fim do
romance. Ainda melhor quando isso ocorre no sentido próprio. Como o
protagonista indica que a morte já o espera, isto é, uma morte bem
determinada, em um ponto bem determinado? Eis a questão que alimenta o
interesse devorador do leitor por aquilo que ocorre no romance.

O romance, portanto, não é significativo graças a um ensinamento
qualquer que um destino alheio nos apresentaria, mas porque, através da
chama que o devora, esse destino alheio nos transmite um calor que não
podemos tirar da nossa própria vida. O que prende o leitor ao romance é
a esperança de aquecer sua vida gelada em uma morte sobre a qual ele lê.

\section{\versal{XVI}}

``Leskov'', escreve Gorki, ``é o escritor mais profundamente (\ldots{})
enraizado no povo e o mais isento de toda influência estrangeira''. O
grande contador terá sempre suas raízes no povo, e em primeiro lugar nas
camadas artesanais. Entretanto, assim como essas englobam, nos múltiplos
estágios de seu desenvolvimento econômico e técnico, os elementos
camponês, marítimo e urbano, também os conceitos nos quais a soma de
suas experiências se cristaliza para nós possuem múltiplas gradações 
(sem falar na considerável contribuição dos mercadores para a arte de
contar; eles não precisaram tanto aumentar o conteúdo instrutivo dos
contos\footnote{``(\ldots{}) através de seus relatos de terras longínquas'',
  consoante acréscimo efetuado por Benjamin na versão francesa. [\versal{N}. 
  \versal{T}.]}, quanto refinar as estratégias destinadas a captar a atenção
dos ouvintes.\footnote{``De fato, não vemos entre os contistas árabes o
  ouvinte tornar"-se cliente de um contador de histórias?'', segundo
  acréscimo da versão francesa. [\versal{N}. \versal{T}.]} Deixaram marcas
profundas no ciclo de histórias das \emph{Mil e uma noites}). Em suma,
sem desconsiderar o papel elementar que a narrativa desempenha na
economia doméstica da humanidade, os conceitos através dos quais podemos
colher os seus frutos são de uma grande diversidade. O que, em Leskov,
deve ser compreendido de modo mais tangível em um sentido religioso,
parece se ajustar por si mesmo, em Hebel, às perspectivas pedagógicas do
Iluminismo, surge em Poe como tradição hermética e encontra um último
asilo em Kipling, no campo de ação de marinheiros e soldados coloniais
britânicos. Por outro lado, todos os grandes contadores de história têm
em comum a facilidade com a qual descem e sobem os degraus de sua
experiência, como numa escada. Uma escada cuja base desce até as
profundezas da terra e cujo topo se perde nas nuvens é a imagem de uma
experiência coletiva para a qual o mais profundo choque de cada
experiência individual, a morte, não representa nem um escândalo nem uma
barreira.

``E se não morreram, vivem até hoje'', diz o conto de fadas. O conto de
fadas, que ainda hoje é o primeiro conselheiro das crianças, pois foi
outrora o primeiro da humanidade, sobrevive secretamente no conto. O
primeiro contador verdadeiro é e continua sendo o dos contos de fadas.
Esse conto sabia trazer um bom conselho, onde nada era mais difícil de
se encontrar, e onde a necessidade era a mais urgente, a \emph{sua}
ajuda era a que estava mais próxima. Essa necessidade era a do mito. O
conto de fadas nos informa sobre as primeiras tentativas da humanidade
em sacudir para fora o pesadelo que o mito depositou em seu peito. A
figura do bobo nos mostra como a humanidade ``faz-se de boba'' contra o
mito; a do irmão caçula nos mostra como suas chances aumentam com o
distanciamento em relação à época mítica primeva; a figura daquele que
sai de casa para aprender o medo nos mostra que podemos tornar
transparentes as coisas que tememos; a figura do esperto nos mostra que
as questões do mito são tão simples quanto as da esfinge; as figuras dos
bichos que vêm ajudar a criança do conto nos mostram que a natureza
prefere muito mais associar-se ao homem do que comprometer-se com o
mito. O conto de fadas ensinou há muito tempo à humanidade e ainda hoje
ensina às crianças que o mais aconselhável é enfrentar o mundo do mito
com astúcia e ousadia (deste modo, o conto de fadas polariza a coragem,
a saber, dialeticamente, em astúcia e em ousadia).\footnote{Benjamin faz,
  nesta passagem, um jogo de palavras com \emph{Mut} (``coragem''),
  \emph{Übermut}, que quer dizer ``alegria exuberante'', ``animação
  excessiva que chega às raias da insolência ou da ousadia'', e um termo
  insólito, \emph{Untermut}, que aparece aqui como equivalente
  a~\emph{List}, ``astúcia''. [\versal{N}. \versal{T}.]} A magia liberadora do
conto de fadas não coloca em cena a natureza de um modo mítico, mas
indica a sua cumplicidade com o ser humano liberado\label{supra5}. O homem maduro
concebe essa cumplicidade apenas ocasionalmente, isto é, quando está
feliz; para a criança, ela aparece pela primeira vez nos contos de fadas
e provoca sua felicidade.

\section{\versal{XVII}}

Poucos contadores tiveram uma afinidade tão profunda com o espírito dos
contos de fadas quanto Leskov. Trata-se de uma tendência reforçada pelos
dogmas da Igreja Ortodoxa grega. Nessa dogmática, a especulação de
Orígenes sobre a apocatástase -- a admissão de todas as almas no paraíso
--, que foi descartada pelo catolicismo romano, desempenha um papel
significativo. Leskov foi muito influenciado por Orígenes. Planejou
traduzir sua obra \emph{Sobre os Princípios}.\footnote{Há tradução para
  o português. Cf. Orígenes. \emph{Tratado sobre os Princípios}.
  Tradução de João Eduardo Pinto Basto Lupi. São Paulo: Paulus, 2012.
  (Coleção Patrística, v. 30) [\versal{N}. \versal{E}.]} De acordo com as
crenças populares russas, interpretou a ressurreição menos como uma
transfiguração do que como um desencantamento (num sentido parecido com
o dos contos de fadas). Essa interpretação de Orígenes constitui a base
do conto ``O peregrino encantado''. Aqui, como em muitas outras
histórias de Leskov, trata-se de um misto de conto de fadas e lenda, não
muito diferente do misto de conto de fadas e saga de que fala Ernst
Bloch numa passagem em que retoma a seu modo a nossa distinção entre
mito e conto de fadas. Segundo Bloch, um ``misto de conto de fadas e
saga inclui, de modo figurado, um elemento mítico que atua de modo
estático e encantatório, embora não fora do ser humano. Assim, na saga,
personagens de tipo taoista são `míticos', em particular os muito
antigos, como o casal Filemon e Baucis: redimidos, como nos contos de
fadas, apesar de repousarem como natureza. E certamente esse tipo de
relação existe também no Tao menos acentuado de Gotthelf; às vezes, ele
priva a saga do local encantado, salva a luz da vida, a luz propriamente
humana da vida que arde tranquilamente, por dentro e por
fora''.\footnote{Citação de Ernst Bloch, \emph{Erbschaft dieser Zeit}.
  Frankfurt a. \versal{M}.: Suhrkamp, 1962. (\emph{Gesammtausgabe}, v. 4) [\versal{N}.
  \versal{T}.]} Redimidas, ``como nos contos de fadas'', são aquelas
criaturas que abrem o cortejo da criação de Leskov: os justos. Pavlin,
Figura, o artista cabeleireiro, o domador de ursos, a sentinela
prestativa -- todos aqueles que personificam a sabedoria, a bondade, a
consolação do mundo amontoam"-se em torno do contador. Estão todos
incontestavelmente impregnados da imago de sua mãe. De acordo com a
descrição de Leskov, ``ela tinha a alma tão boa que era incapaz de fazer
mal a qualquer ser humano, ou mesmo aos animais. Não comia carne nem
peixe tal sua compaixão por todas as criaturas vivas. Meu pai tinha o
costume de reprovar-lhe, de vez em quando, tal atitude. Mas ela
respondia: `Eu mesma criei esses bichinhos, são como meus filhos. Não
posso comer meus próprios filhos!' Mesmo na casa dos vizinhos não comia
carne. `Eu os vi vivos', dizia, `são meus conhecidos. Não posso comer
meus conhecidos'.''

O justo é ao mesmo tempo o porta-voz da criatura e sua mais alta
personificação. Em Leskov, ele possui um traço maternal que se eleva às
vezes até o mítico (colocando assim, é verdade, em risco a pureza do
conto de fadas). Exemplar neste sentido é o personagem principal de seu
conto ``Kótin, o provedor, e Platonida''. Este personagem, o camponês
Pizónski, é bissexuado. Durante doze anos, foi criado pela mãe como
menina. Seu lado feminino amadureceu ao mesmo tempo que sua
masculinidade, e sua dupla sexualidade tornou-se um símbolo do
Homem-Deus.\footnote{Cf. Nikolai Leskov, ``Kótin, o provedor, e
  Platonida'', in \emph{A Fraude e outras histórias}, ed. cit., pp.
  10---11. [\versal{N}. \versal{T}.]}

Nesse símbolo, Leskov acredita poder atingir o apogeu da criatura e, ao
mesmo tempo, estabelecer uma ponte entre os mundos terrestre e
supraterrestre. Pois estes homens cuja potência vem da terra, estas
figuras masculinas maternais, que reiteradamente tomam posse da arte
fabuladora de Leskov, foram subtraídas à escravidão das pulsões sexuais
no pleno florescimento de suas forças. Mas nem por isso personificam um
ideal propriamente ascético; ao contrário, a temperança desses justos
tem um caráter tão pouco privado que torna-se, na ordem das paixões, o
polo oposto ao do furor sexual que o contador pintou em \emph{Lady
Macbeth do Distrito de Mtzensk}.\footnote{Cf. Nikolai Leskov, \emph{Lady
  Macbeth do Distrito de Mtzensk}, ed. cit. [\versal{N}. \versal{T}.]} Se a
envergadura entre Pavlin e essa esposa de negociante permite medir a
extensão do mundo das criaturas, então Leskov também sondou, na
hierarquia de suas criaturas, a sua profundeza.

\section{\versal{XVIII}}

A hierarquia do mundo das criaturas, que atinge com o justo o seu ponto
mais alto, desce em múltiplos graus até as profundezas do inanimado. A
propósito disso, uma circunstância particular deve ser levada em conta:
essa totalidade do mundo das criaturas não se expressa tanto na voz
humana quanto naquela que pode ser nomeada de acordo com o título de um
de seus contos mais significativos: ``A voz da natureza''. Esse conto é
sobre o pequeno funcionário Filip Filipovitch que se esforça por todos
os meios para hospedar em sua casa um marechal de campo de passagem em
sua cidadezinha. Ele consegue o que queria. O hóspede, que se espantara
a princípio com o convite insistente do funcionário, com o tempo pensa
reconhecer nele alguém que já havia encontrado antes. Mas quem? Não
consegue lembrar-se. E o estranho é que o dono da casa recusa-se a
facilitar o reconhecimento. Em vez disso, consola a alta personalidade
um dia depois do outro, dizendo que ``a voz da natureza'' acabará por
lhe falar claramente. Isso dura até que um dia, pouco antes de
prosseguir sua viagem, quando, atendendo ao pedido público do anfitrião
durante um jantar, o hóspede deve dar sua permissão para que ele possa
fazer soar ``a voz da natureza''. Então a dona da casa se afasta. Ela
``voltou com uma grande trompa de cobre, reluzentemente polida, e
entregou-a ao marido. Ele pegou a trompa, encostou o bocal aos lábios e
transformou-se inteiro num minuto. Foi só ele inflar as bochechas e sair
um ribombo vibrante para o marechal de campo gritar: -- Estou
reconhecendo, irmão, agora estou reconhecendo, você é aquele músico do
regimento de caçadores, que, por sua honestidade, enviei para vigiar um
intendente trapaceiro. -- Exatamente, meu príncipe -- respondeu o
anfitrião. -- Não queria eu lembrar-lhe disso, então a própria natureza
o fez''.\footnote{Nikolai Leskov, ``A voz da natureza'', in \emph{A
  Fraude e outras histórias}, ed. cit., p. 100. [\versal{N}. \versal{T}.]} A
maneira pela qual o sentido profundo dessa história se esconde atrás de
sua aparente tolice nos dá uma ideia do humor magnífico de Leskov.

Este humor confirma-se na mesma história de um modo ainda mais discreto.
Ouvimos que o pequeno funcionário foi enviado ``por sua honestidade''
para ``vigiar um intendente trapaceiro''. Isso é dito no final, na cena
do reconhecimento. Logo no início da história, porém, ouvimos o seguinte
sobre o anfitrião: ``Os moradores locais, todos eles, conheciam esse
homem e sabiam que o seu título não era alto, já que não era funcionário
civil nem militar, mas só encarregado do pequeno depósito local da
intendência ou do comissariado e, junto com as ratazanas, roía torradas
do erário e lambia botas, tendo conseguido, com a roedeira e a lambição,
uma casa bonitinha, de madeira (\ldots{})''.\footnote{Idem, ibidem, p. 90.
  [\versal{N}. \versal{T}.]} Como se vê, essa história mostra a tradicional
simpatia do contador pelos trapaceiros e malandros. Toda literatura
burlesca é testemunha dessa simpatia que não se desmente nem na mais
alta arte: Hebel é acompanhado do modo mais fiel, entre todas as suas
personagens, por \emph{Zundelfrieden}, \emph{Zundelheiner} e o ruivo
\emph{Dieter}.\footnote{Alusão a uma série de contos de Hebel nos quais
  os dois irmãos Zundel -- Frieden e Heiner -- se associam ao ruivo
  Dieter, seu antigo camarada de escola, para cometerem diversos tipos
  de furtos, malandragens e farsas. Embora não passem de pequenos
  bandidos, os três comparsas são apresentados com simpatia pelo autor.
  [\versal{N}. \versal{T}.]} E, claro, também para Hebel o justo desempenha o papel
principal no \emph{theatrum mundi}. Mas como ninguém está propriamente à
altura desse papel, ele passa de um para outro. Ora é o vagabundo, ora o
judeu avarento, ora é o imbecil que surge para desempenhá-lo. Trata-se
sempre apenas de um estágio provisório, que varia de um caso a outro,
uma improvisação moral. Hebel é um casuísta. Não se solidariza de
nenhuma maneira com algum princípio, mas também não recusa nenhum deles,
pois todos podem um dia servir de instrumento ao justo. Podemos comparar
essa atitude com a de Leskov. ``Reconheço'', diz ele em ``A propósito de
\emph{A Sonata a Kreutzer}'', ``que, nas minhas reflexões, entra muito
mais senso prático do que filosofia abstrata e moral elevada; contudo
sou inclinado a pensar como penso''.\footnote{Nikolai Leskov, ``A
  propósito de \emph{A Sonata a Kreutzer}'', in \emph{A Fraude e outras
  histórias}, ed. cit., pp. 176---177. [\versal{N}. \versal{T}.]} É verdade, por
outro lado, que as grandes catástrofes morais que ocorrem em seu mundo
estão para os incidentes morais de Hebel como o grande curso silencioso
do rio Volga para a tagarelice de um pequeno riacho que faz girar a roda
do moinho. Entre os contos históricos de Leskov, há vários em que as
paixões agem de modo tão avassalador quanto a cólera de Aquiles ou o
ódio de Hagen.\footnote{Personagem da mitologia escandinava e, depois,
  da mitologia germânica, responsável pelo assassinato do herói
  Siegfried. [\versal{N}. \versal{T}.]} É surpreendente como, nesse autor, o mundo
às vezes pode tornar-se sombrio e com qual majestade o mal é capaz de
empunhar seu cetro. Leskov claramente conheceu estados de espírito
próximos de uma ética antinomista -- e esse deve ser um de seus poucos
pontos de contato com Dostoiévski. As naturezas elementares de seus
\emph{Contos dos velhos tempos}\footnote{``Erzählungen aus der alten
  Zeit'', conforme o original alemão. Na edição
  em nove volumes das \emph{Gesammelte Werke} (Obras Reunidas) de
  Nikolai Leskov, que a editora \versal{C}. \versal{H}. Beck fez publicar entre 1924 e
  1927, utilizada por Benjamin e por ele referida em nota inicial ao
  presente ensaio, o título correspondente é o do quarto
  volume, \emph{Geschichten aus alter Zeit}. (\emph{Histórias dos velhos tempos}).
  [\versal{N}. \versal{E}.]} vão até o fim em sua paixão sem escrúpulos. E esse
fim era justamente o ponto no qual para os místicos a perversão
consumada transforma-se bruscamente em seu contrário, tornando-se
santidade.

\section{\versal{XIX}}

Quanto mais baixo Leskov desce na escala das criaturas, mais abertamente
sua concepção de mundo se aproxima da mística. Aliás, como mostraremos,
há boas razões para se dizer que tal característica pertence à própria
natureza do contador de histórias. Certamente, são raros os que se
aventuraram nas profundezas da natureza inanimada e, na literatura
narrativa moderna, não há muitos casos em que a voz do contador anônimo,
anterior a toda escrita, ressoe tão claramente como na história de
Leskov ``A Alexandrita''. Trata-se de uma pedra, o piropo. A pedra
corresponde à camada mais baixa da criação. Mas para o contador, ela se
liga diretamente à mais alta. Ele tem o dom de vislumbrar, nesta pedra
semipreciosa, o piropo, uma profecia da natureza petrificada, inanimada,
sobre o mundo histórico, no qual ele mesmo vive. Este mundo é o de
Alexandre~\versal{II}. O contador -- ou melhor, o homem a quem ele atribui seu
próprio saber -- é um lapidador, de nome Wenzel, que atingiu em seu
ofício a mais alta perfeição imaginável. Podemos compará-lo aos ourives
de Tula e dizer, de acordo com Leskov, que o artesão perfeito tem acesso
às câmaras mais internas do reino das criaturas. É uma encarnação do
homem piedoso. Vejamos o que é dito sobre ele: ``de repente, agarrou-me
pelo anel com a alexandrita, que agora, sob a luz, estava
vermelha\footnote{Na tradução do russo para o alemão de que fez uso
  Benjamin, há uma variação relativamente à tradução do russo para o
  português, aqui adotada, que importa destacar. De modo que onde nesta
  figura ``(\ldots{}) alexandrita, que agora, sob a luz, estava
  vermelha'', naquela se lê ``(\ldots{}) alexandrita, que, como se sabe,
  com iluminação artificial irradia um brilho vermelho''. [\versal{N}. \versal{E}.]}, e pôs-se a gritar: --- (\ldots{}) Vejam só, eis aqui aquela pedra
russa profética (\ldots{})! Siberiana astuta! O tempo todo estava verde como
a esperança, mas agora, com a aproximação do anoitecer, banhou-se de
sangue. Desde priscas eras ela é assim, mas escondeu-se o tempo todo, no
seio da terra, e permitiu que a encontrassem apenas no dia da maioridade
do tsar Aleksandr, quando um grande feiticeiro, mago, bruxo, foi à
Sibéria procurar por ela. --- O senhor está falando asneiras --
interrompi. --- Essa pedra não foi encontrada por um feiticeiro, foi por
um cientista: Nordenskiöld! --- Feiticeiro! Estou dizendo: feiticeiro! --
pôs-se a gritar Wenzel bem alto. --- Veja só que pedra! Nela a manhã é
verde e a noite, sangrenta\ldots{} É o destino, é o destino do nobre tsar
Aleksandr! -- E o velho Wenzel voltou-se para a parede, apoiou a cabeça no
braço e\ldots{} pôs-se a chorar''.\footnote{Nikolai Leskov, ``Alexandrita'',
  in \emph{A Fraude e outras histórias}, ed. cit., pp. 164---165. [\versal{N}. \versal{T}.]}

Não podemos apreender mais diretamente o sentido desse importante conto
do que com o auxílio de algumas palavras escritas por Paul Valéry sobre
outro assunto, bem diferente.

``A observação do artista'', disse ele em suas considerações sobre um
artista, ``pode atingir uma profundidade quase mística. Os objetos
iluminados por ela perdem seus nomes: sombras e claridades formam
sistemas e problemas bem particulares, que não dizem respeito a nenhuma
ciência, que não se relacionam com nenhuma prática, mas que recebem toda
sua existência e seu valor de certos acordos singulares entre a alma, o
olho e a mão de alguém, nascido para, dentro de si, apreendê-los e
evocá-los''.\footnote{Paul Valéry, ``Autour de Corot'', \emph{Pièces sur
  l'art}, in~\emph{Œuvres}, tomo~\versal{II}. Edição de Jean Hytier. Paris:
  Gallimard, 1960, p. 1318. (Col. Bibliothèque de la Pléiade). [\versal{N}. \versal{T}.]}

Tais palavras estabelecem uma estreita relação entre alma, olho e
mão.\footnote{``Relação de colaboração que determina todo trabalho
  artesanal'', conforme acréscimo efetuado na versão francesa. [\versal{N}. 
  \versal{T}.]} Interagindo, determinam uma prática com a qual não estamos mais
acostumados. O papel da mão na produção tornou-se mais restrito e o
lugar que ela ocupava no contar histórias foi deixado de lado (pois
contar histórias não é de modo algum, do ponto de vista sensível, apenas
um trabalho da voz. No autêntico contar, a mão atua decisivamente,
apoiando o que é dito de diversos modos, com seus gestos aprendidos por
experiência no trabalho). A antiga coordenação de alma, olho e mão, que
aparece nas palavras de Valéry, é artesanal, e é ela que encontramos
onde quer que a arte de contar esteja em seu domínio. Sim, podemos ir
além e nos perguntar se a relação do contador com sua matéria, a vida
humana,\footnote{``a experiência humana'', segundo variação da versão
  francesa. [\versal{N}. \versal{T}.]} não seria ela própria uma relação
artesanal? Se sua tarefa não consiste em elaborar, de um modo sólido,
útil e único, a matéria crua da experiência, seja da sua própria ou da
alheia? Trata-se aqui de uma reelaboração da qual o provérbio nos dê
talvez mais facilmente uma ideia, se o concebemos como o ideograma de
uma narração. Provérbios são, por assim dizer, ruínas que ocupam o lugar
de antigas histórias nas quais uma moral cresce em torno de um gesto,
como a hera numa muralha.

O contador de histórias pode assim ser considerado como um mestre ou
como um sábio. Ele sabe aconselhar -- não em alguns casos, como o
provérbio, mas em muitos,\footnote{Na variante da versão francesa: ``em
  todos os casos''. [\versal{N}. \versal{T}.]} como o sábio. Pois lhe é dado
recorrer a toda uma vida (vida que não inclui apenas sua própria
experiência, mas também uma boa parte da experiência alheia. O contador
assimila ao que tem de mais intimamente seu aquilo que aprendeu por
ouvir dizer). Seu talento é poder contar sua vida; sua dignidade é poder
contá-la \emph{por inteiro}. O contador é o homem que poderia deixar a
mecha de sua vida consumir-se completamente na doce chama de sua
narração. Daí vem essa atmosfera\footnote{``esse halo'', na variante da
  versão francesa. [\versal{N}. \versal{T}.]} incomparável que, em Leskov como em
Hauff, em Poe como em Stevenson, cerca o contador. O contador de
histórias é a figura na qual o justo se encontra consigo mesmo\label{supra4}.

\part{Contos}

\chapter{O lenço\footnote[*]{``Das Taschentuch'', in Walter Benjamin,
  \emph{Gesammelte Schriften} [a partir daqui, \versal{GS}], vol. \versal{IV}-2.
  Edição de Rolf Tiedemann e Hermann Schweppenhäuser. Frankfurt a. \versal{M}.:
  Suhrkamp, 1991, pp. 741-745. Tradução de Marcelo Backes. Escrito
  provavelmente entre abril e junho de 1932, durante uma estadia em
  Ibiza, o conto foi publicado no \emph{Frankfurter Zeitung} em novembro
  de 1933. O ensaio sobre Leskov, de 1936, retoma vários temas e
  formulações contidas neste texto ficcional. [\versal{N}. \versal{E}.]}}

Por que a arte de contar histórias está chegando ao fim? -- eu já me
fizera esta pergunta muitas vezes enquanto me entediava, sentado com
outros convidados durante uma noite inteira em torno de uma mesa.
Naquela tarde, porém, quando estava em pé no convés de passeio do
``Bellver'', ao lado da cabine do timoneiro, e buscava com meu excelente
binóculo todos os aspectos da imagem inigualável que Barcelona oferecia
do alto do navio, acreditei ter encontrado a resposta para ela. O sol
descia sobre a cidade e parecia derretê-la. Tudo que era vida havia se
recolhido nas gradações cinza claro entre a copa das árvores, o cimento
das construções e a rocha das montanhas distantes. O ``Bellver'' é um
navio a motor, belo e espaçoso, ao qual se gostaria de creditar um
encargo maior do que o de abastecer o pequeno trânsito insular para as
Baleares. E, de fato, sua imagem pareceu se encolher quando, no dia
seguinte, o vi esperando pela viagem de volta no molhe de Ibiza, pois eu
imaginara que de lá ele tomaria seu curso até as Ilhas Canárias. Assim,
eu estava parado, e voltava meus pensamentos para o capitão O\ldots{}, do
qual eu me despedira há algumas horas, o primeiro e talvez último
contador de histórias que encontrei em minha vida. Pois, conforme disse,
a arte de contar está chegando ao fim. E, quando me lembrava das várias
horas em que o capitão O\ldots{} havia passeado de um lado a outro no convés
traseiro, olhando de quando em vez para a distância, ocioso, então eu já
sabia: quem jamais se entedia não é capaz de contar. O tédio, porém, não
tem mais espaço em nosso agir. As atividades, que se uniram a ele
secreta e intimamente, estão se extinguindo. E também por isso o dom de
contar histórias está chegando ao fim: não se tece e não se fia mais,
não se constrói nem se aplaina objetos enquanto se ouve histórias. Em
resumo: histórias precisam de trabalho, ordem e disciplina para vingar.

Contar histórias, na verdade, não é apenas uma arte, é muito mais uma
dignidade, se é que não é, como no Oriente, um ofício. Contar termina em
uma sabedoria, assim como por outro lado a sabedoria muitas vezes se
revela numa narrativa. O contador de histórias é, portanto, alguém que
sempre sabe dar conselhos. E, para recebê-los, é preciso que também se
conte algo a ele. Mas nós sabemos apenas suspirar a respeito de nossas
preocupações, nos lamentar, e não contar. E, lembrando um terceiro
aspecto, pensei no cachimbo do capitão: o cachimbo que ele batia quando
começava a contar algo e batia quando se calava, mas, entre um e outro,
quando se fazia necessário, deixava se apagar calmamente. O cachimbo
tinha uma piteira de âmbar, mas sua cabeça era de chifre e adornada com
pesados trabalhos em prata. Ele o herdara do avô, e acredito que esse
cachimbo era o talismã do contador de histórias. Pois também por causa
disso não há mais nada interessante a ouvir, porque as coisas não duram
mais do modo correto. Quem um dia usou um cinto de couro por tanto tempo
até ele se desintegrar em pedaços, sempre haverá de encontrar algo
interessante: em algum momento, no decorrer do tempo, uma história se
prendeu a ele. O cachimbo do capitão já devia conhecer muitas delas.

Assim eu sonhava quando, lá embaixo, bem no fundo, nas docas, apareceu
um homem atarracado com o rosto mais maciço que já esteve enfiado
debaixo de um quepe de capitão: o capitão O\ldots{}, em cujo navio de carga
eu havia chegado pela manhã. Quem está acostumado a partidas solitárias
de cidades estranhas sabe ou então é capaz de avaliar o que significa o
aparecimento de um rosto conhecido, ainda que não seja um dos mais
familiares, em tais momentos, quando a partida que em breve ocorrerá
tira do caminho todas as ponderações de uma conversa mais longa, mas ao
mesmo tempo também coloca à sua disposição um chapéu qualquer, uma mão,
um lenço, nos quais o olhar desabrigado pode encontrar seu ninho antes
de se perder na superfície imensa do mar. Eis, pois, que ali estava o
capitão, como se eu o tivesse chamado com meus pensamentos. Ele havia
saído de casa aos quinze anos, cruzado por aí durante três anos em um
navio-escola o Pacífico e o Atlântico, e mais tarde chegado a um vapor
americano do Lloyd, que ele no entanto -- por que motivo era
desconhecido -- abandonara logo em seguida. Mais do que isso eu não
consegui descobrir. Sobre sua vida parecia pairar uma sombra, e ele
também não falava com prazer a respeito. E, com tudo isso, no fundo lhe
parecia faltar o que é a coisa mais maravilhosa no contador: o fato de
poder contar sua vida, deixando que essa mecha se consuma nas chamas
suaves da narrativa. Como quer que seja, sua vida parecia ser pobre se
comparada com a do navio que ele sabia fazer viver em todas as suas
ripas e vigas. E assim estava ele, ali à minha frente, quando pela manhã
deixei o navio. Eu conhecia tão bem tanto seu ano de construção e suas
tarifas, seu espaço de carga e sua tonelagem, quanto os salários dos
grumetes e as preocupações dos oficiais. Sim, que tempos eram aqueles em
que o transporte de cargas ainda era feito por veleiros, onde o capitão,
ele mesmo, contratava os fretes nos portos! Nessa época ainda imperava a
velha sentença irônica: ``Abandonar as viagens marítimas e entrar em um
navio a vapor''. Hoje, porém\ldots{} e então se seguiam, na maior parte das
vezes, algumas frases a partir das quais se podia deduzir como também
aqui as necessidades econômicas haviam mudado as coisas.

Em tais oportunidades o capitão O\ldots{}, de quando em vez, dizia também uma
palavra sobre a política. Mas jamais o vi com um jornal. Tornou-se
inesquecível para mim sua resposta quando, certo dia, eu conduzi a
conversa ao assunto. ``Dos jornais'', disse ele, ``não se pode saber
absolutamente nada; é que as pessoas querem explicar tudo para a
gente''. E, de fato: já não é metade da arte de relatar, manter o relato
livre de explicações? E os antigos por acaso não são exemplares nisso,
eles que por assim dizer apresentavam o acontecido a seco, deixando
antes que escorresse dele tudo que era fundamentação psicológica e
opinião? Suas próprias histórias, de qualquer modo, deve-se admitir, se
mantinham livres de explicações supérfluas, sem que, conforme me
parecia, perdessem algo por causa disso. Até havia algumas mais
estranhas entre elas, mas nenhuma que confirmasse aquela peculiaridade
tanto quanto a seguinte, sobre a qual ainda recairia, nesta tarde junto
ao molhe de Barcelona, o mais surpreendente reflexo.

``Tudo aconteceu'', assim me contou o capitão na altura de Cádiz, ``há
muitos anos, em uma das minhas primeiras viagens aos Estados Unidos,
quando eu ainda era um oficial bem jovem. Estávamos sete dias em
alto-mar e, ao meio-dia do dia seguinte, chegaríamos a Bremerhaven. À
hora costumeira, eu fazia minha ronda no convés de passeio, trocava
algumas palavras aqui e ali com os passageiros. Foi quando fiquei de
repente estupefato; a sexta espreguiçadeira da fila estava vazia. Um
sentimento de angústia se manifestou dentro de mim, mesmo que eu tenha,
acho, passado por ela ainda mais angustiado nos dias anteriores, quando
me voltava com uma saudação muda para a jovem mulher que costumava
contemplar o vazio à sua frente sobre ela, imóvel, com as mãos cruzadas
sob a nuca. A jovem mulher era muito bonita, mas tão chamativa quanto
sua beleza era também sua discrição. Esta ia tão longe que só raras
vezes tinha-se oportunidade de ouvir sua voz -- a voz mais maravilhosa
da qual consigo me lembrar --, frágil e rouca, sombria e metálica. Certa
vez, quando lhe entreguei um lenço que havia caído ao chão -- ainda hoje
sei como seu emblema me impressionou: um brasão tripartido com três
estrelas em cada um dos campos --, ouvi-a dizer seu `obrigada' com uma
expressão que era como se eu tivesse salvado sua vida. Desta vez, pois,
terminei minha ronda e já estava a ponto de procurar pelo médico do
navio, a fim de saber se a dama por acaso não estava doente, quando de
repente um redemoinho de franjas brancas me envolveu. Levantei os olhos
e vi como aquela que eu dava por perdida, apoiada sobre o parapeito do
convés solar, seguia com os olhos, ausente, um bando de bilhetes e
papéis com os quais o vento e as ondas brincavam. No meio-dia seguinte
-- eu assumira meu posto no convés e inspecionava as manobras de aportar
--, meu olhar cruzou mais uma vez com o da estranha de passagem. O navio
estava a ponto de ancorar e, vagarosamente, a quilha se aproximava do
cais no qual havíamos amarrado a popa. Reconhecia-se com nitidez a
feição dos que esperavam; febril, a estranha os media. O baixar das
âncoras havia ocupado minha atenção quando, de uma hora para outra, se
ergueu um grito em várias vozes. Eu me virei e no mesmo instante vi que
a estranha havia desaparecido; no movimento da multidão se podia
perceber que ela havia se precipitado abaixo. Qualquer tentativa de
salvação era inútil. Mesmo que se conseguisse desligar as máquinas
naquele mesmo instante -- o casco do navio não estava mais do que três
metros distantes do cais, e seu movimento era impossível de ser detido.
Quem ficasse no meio, estava perdido. Então aconteceu o improvável:
houve alguém que fez a tentativa formidável. Era possível vê-lo, cada um
dos músculos distendidos, as sobrancelhas virando uma só, como se
quisesse fazer mira, saltar da grade e, enquanto o vapor foi se
aproximando em todo seu comprimento a estibordo -- para horror de todos
os que acompanhavam o espetáculo --, a bombordo apareceu, sem que a
princípio se o notasse, pois ninguém olhava para aquele lado, o homem,
salvo, e em seu braço a moça, na superfície da água. Ele de fato fizera
mira -- exatamente, aproveitando todo seu peso e se precipitando sobre a
outra, arrancando-a consigo para as profundezas, para em seguida
reaparecer por baixo da quilha do navio --, trazendo-a para a
superfície. `Quando a segurei assim', disse ele mais tarde a mim, `ela
sussurrou `obrigada' como se eu tivesse lhe estendido um lenço que caíra
ao chão'.''\label{supra9}
%não entendi o significado de 'sobre ela' na linha 1336 "o vazio a sua frente sobre ela"

Eu ainda tinha no ouvido a voz com a qual o contador havia pronunciado
estas últimas palavras. Se eu quisesse lhe dar a mão ainda uma vez, não
havia tempo a perder. Quando eu já fazia menção de descer correndo as
escadas que levavam até ele, percebi como os armazéns, barracas e gruas
recuavam lentamente. Estávamos zarpando. O binóculo diante dos olhos,
deixei Barcelona passar à minha frente pela última vez. Depois o baixei
devagar até o cais. Ali estava o capitão, em meio à multidão; ele devia
ter acabado de me notar. Saudando, levantou a mão, eu acenei com a
minha. Quando pus o binóculo outra vez diante dos olhos, ele havia
desdobrado um lenço e o abanava. Nítido, vislumbrei o emblema em um dos
cantos: um brasão tripartido com três estrelas em cada um dos campos.

\chapter{A viagem do Mascote\footnote[*]{``Die Fahrt der Mascotte'', in \versal{GS} \versal{IV}-2, pp.
  738-740. Tradução de Marcelo Backes. Finalizado e corrigido juntamente
  com ``O Lenço'' e ``O anoitecer da viagem'', este conto data
  provavelmente de 1932, quando Benjamin realizou uma viagem marítima às
  Ilhas Baleares, na Espanha. [\versal{N}. \versal{E}.]} }

Esta é uma dessas histórias que se costuma ouvir no mar, para a qual o
casco do navio é a caixa de ressonância perfeita e o socar das máquinas
o melhor acompanhamento, e diante das quais não se deve insistir para
saber de onde elas vêm.

Conforme contou meu amigo, o sinaleiro de bordo, tudo se deu depois do
final da guerra, quando alguns donos de navios passaram a querer trazer
de volta à pátria veleiros, navios de salitre, que haviam sido
surpreendidos por uma catástrofe no Chile. A questão jurídica era
simples; os navios continuavam sendo propriedade alemã, e agora se
tratava apenas de preparar a tripulação e os homens necessários para
assumir o controle sobre eles em Valparaíso ou Antofagasta. Havia
marinheiros suficientes esperando para serem engajados nos portos. Mas
havia um detalhe problemático na questão toda. Pois como se poderia
transportar as equipes até o lugar? Uma coisa era certa: eles só
poderiam subir à bordo como passageiros e entrariam em serviço apenas no
lugar determinado. Por outro lado, também estava claro que se tratava de
pessoas das quais dificilmente se lograria conseguir alguma coisa com os
poderes que um capitão tem sobre seus passageiros, e com certeza não em
tempos nos quais o clima da Revolta de Kiel ainda se fazia sentir nos
ossos dos marinheiros.

Ninguém sabia disso melhor do que os hamburgueses e, portanto, o mesmo
acontecia com a equipe de comando do navio de quatro mastros
``Mascote'', que consistia em uma elite de oficiais decididos e com
experiência marítima. Eles viam naquela viagem uma aventura na qual
poderiam colocar sua pele em risco. E, uma vez que o homem inteligente
age de modo precavido, não confiaram apenas na própria coragem. Muito
antes, avaliaram cada um dos homens que seriam engajados com a maior
exatidão. Porém, se entre os escolhidos mesmo assim havia um cara alto
cujos documentos não estavam completamente regulares e cuja constituição
física também deixava um pouco a desejar, seria precipitado creditar
isso ao desleixo dos comandantes. Por que era assim, logo ficará claro.

Ainda não haviam se afastado cinquenta milhas de Cuxhaven quando se
fizeram notar coisas que prenunciavam perspectivas bem ruins para a
viagem. No convés e nas cabines, e até mesmo nas escadarias, se
instalavam desde bem cedo até bem tarde as mais diferentes reuniões e
círculos, e, antes de Helgoland, já havia três clubes de jogo em pleno
funcionamento, um ringue de boxe permanente e um palco para amantes,
cuja visita não era recomendável a pessoas escrupulosas. Na sala dos
oficiais, cujas paredes da noite para o dia foram ornadas com desenhos
drásticos, os homens dançavam \emph{jimmy} uns com os outros à tarde, e
no salão de carga havia se estabelecido uma bolsa de valores de bordo,
cujos membros negociavam com notas de dólares, binóculos, nus
fotográficos, facas e passaportes ao clarão de lâmpadas de bolso. Em
resumo, o navio era uma Magic City navegante, e até seria possível
acreditar que todas as maravilhas da vida no porto poderiam ser
arrancadas da terra -- ou, antes, das vigas do navio --, até mesmo sem
mulheres.

O capitão, um daqueles homens do mar que unem o pouco saber escolar a
muita esperteza de vida, permaneceu senhor de seus nervos mesmo sob
circunstâncias tão desconfortáveis, e não os perdia nunca, nem mesmo
quando certa tarde -- deve ter sido na altura de Dover --, Frieda, uma
moça bem crescidinha, mas de muita má fama, do bairro de Sankt Pauli, em
Hamburgo, apareceu no deque traseiro, com um cigarro na boca. Sem dúvida
alguma, havia pessoas a bordo que sabiam onde ela estivera enfiada até
então, e as mesmas também sabiam das medidas que deveriam ser tomadas
caso fossem feitos preparativos na parte de cima para afastar os
passageiros em excesso.

O movimento à noite se tornou ainda mais chamativo a partir de então.
Mas não se estaria no ano de 1919 se, além de todos os outros
divertimentos, também a política não se juntasse a eles. Fizeram-se
ouvir vozes que queriam ver naquela expedição o princípio de uma nova
vida em um novo mundo; outros viam se aproximar o instante, há muito
almejado, em que seriam acertadas as contas com os dominantes.
Inequivocamente, soprava um vento mais agudo. E em breve também se
descobriria de onde ele vinha: havia ali um certo Schwinning, um homem
alto de postura frouxa, que trazia o cabelo ruivo dividido por uma
risca, e do qual apenas se sabia que havia trabalhado como acompanhante
de passageiros em várias linhas marítimas, além de saber tudo, até mesmo
sobre os segredos profissionais de contrabandistas finlandeses de
combustível.

No princípio, ele se mantivera afastado, mas agora podia ser encontrado
a cada passo, onde quer que fosse. Quem ouvia o que ele dizia era
obrigado a admitir que se tratava de um agitador de peso. E quem seria
capaz de não ouvi-lo quando ele envolvia um ou outro em uma conversa
briguenta em voz alta no ``bar'', na qual sua voz sobrepujava o disco
tocando, ou quando distribuía informações precisas junto ao ``ringue'',
mesmo sem ser perguntado, sobre as opções partidárias dos lutadores. E
assim ele trabalhava, enquanto a massa se entregava a suas distrações,
incansável em sua politização, até que por fim uma assembleia noturna
recompensou seus esforços ao nomeá-lo diretor de um conselho de
marinheiros.

Com a entrada no Canal do Panamá as eleições adquiriram ímpeto. E não
havia pouca coisa a votar: uma comissão de alimento, uma coluna de
controle, um secretariado de bordo, um tribunal político -- em resumo,
um aparato grandioso foi posto em pé, sem que houvesse o menor
desentendimento com o comando do navio. Tanto maior era a frequência, no
entanto, com que aconteciam discórdias no interior da direção
revolucionária, e elas eram tanto mais aborrecidas na medida em que, se
vistas as coisas com mais exatidão, no fundo todos pertenciam à referida
direção. Quem não tinha posto, por certo poderia esperá-lo da seguinte
assembleia da comissão, e assim não passava um dia sem que houvesse
dificuldades a esclarecer, votações a examinar, resoluções a tomar.
Quando enfim o comitê de ação havia definido em todos os seus detalhes o
plano para um ataque surpresa -- em duas noites, às onze horas, o
comando seria dominado e em seguida tomado o curso oeste em direção a
Galápagos --, o ``Mascote'' já tinha, sem o saber, Callao às costas.
Mais tarde, os balizamentos provaram estar falsificados. Mais tarde
ainda, mais exatamente na manhã seguinte, quarenta e oito horas antes do
motim planejado e cuidadosamente preparado, o navio de quatro mastros
adentrou o molhe de Antofagasta como se nada houvesse acontecido.

Foi isso que disse meu amigo. A segunda guarda chegou ao fim. Nós
entramos na casa de mapas e instrumentos onde o cacau já esperava por
nós nas fundas xícaras de pedra\ldots{} Eu fiquei em silêncio e me ocupava em
fazer um verso acerca do que ouvira. Mas o sinaleiro, no exato momento
em que daria seu primeiro gole, estacou de repente e olhou para mim por
sobre a borda de sua xícara. ``Deixa estar!'', disse ele. ``Na época nós
também não estávamos em cena. Mas quando, três meses mais tarde, no
prédio da administração, em Hamburgo, dei de cara com Schwinning que --
com um grosso Virginia entre os lábios -- acabava de sair do escritório
do chefe\ldots{} então eu compreendi com exatidão o que havia sido a viagem
do `Mascote'.''

\chapter{O anoitecer da viagem\footnote[*]{``Der Reiseabend'', in \versal{GS} \versal{IV}-2, pp.
  745-748. Tradução de Marcelo Backes. Um primeiro esboço deste conto
  encontra-se no conjunto de notas \emph{Espanha 1932} sob o seguinte
  titulo: ``Sobre a honestidade dos nativos e seu contrário. Duas
  histórias''. A outra história planejada não chegou a ser escrita ou se
  perdeu. O presente texto foi redigido, provavelmente, no mesmo ano.
  Quanto ao conjunto referido de notas (cf. \versal{GS} \versal{VI}, pp. 446-464),
  Benjamin o escreveu em sua viagem de abril a junho de 1932, quando foi
  de Hamburgo a Barcelona em um navio cargueiro, percorrendo um trajeto
  de 11 dias e ficando amigo do capitão. Depois, em um navio a vapor do
  correio, seguiu para Ibiza, onde ficou por três meses. [\versal{N}. \versal{E}.]} }

A economia na ilha é arcaica. Eles não colhem os cereais com máquinas,
mas com foices. Em algumas regiões, as mulheres os recolhem à mão e aí
não resta um colmo sequer. Depois de colhido, eles são levados à eira,
onde um cavalo, arreado e tangido pelo camponês, que fica parado no meio
do lugar, debulha os grãos das vagens, pisando com seus cascos. Há
sessenta anos ainda não se conhecia pão ali; o principal alimento era o
milho. E ainda hoje se irriga os campos à maneira antiga, com rodas
d'água que são movidas por mulas. Vacas, há apenas um punhado delas na
ilha. Alguns dizem que é devido ao pasto; no entanto, Dom Rosello,
deputado e comerciante de vinhos, que representa o progresso ali, diz:
por causa do atraso dos moradores. Não faz nem tanto tempo assim que
alguém, quando chegava a Ibiza, podia ficar sabendo pelo primeiro que
encontrava em seu caminho: agora temos tantos estrangeiros na ilha. E é
dessa época que vem a seguinte história, que se contou à mesa de Dom
Rosello:

Um estrangeiro, que depois de ficar vários meses na ilha conseguiu
angariar amizades e confiança, vê o último dia de sua permanência
chegando. Acontece que é um dia abrasivo de tão quente e, quando ele
termina seus preparativos de viagem, decide se livrar o mais rápido
possível da preocupação com suas coisas, a fim de desfrutar ainda duas
horas do entardecer à sombra fresca, no terraço de um comerciante de
vinhos ibicense. No navio lhe prometem cuidar de sua bagagem, incluindo
seu casaco, e, visivelmente aliviado, o estrangeiro vai até o dono da
\emph{tienda}, para o qual ele é assaz bem-vindo, mesmo estando em
mangas de camisa. Sem esforço, o dono da \emph{tienda} chega logo com as
primeiras \emph{copitas} de um Alicante corriqueiro. Mas quanto mais o
tempo avança para ele em meio à bebida, tanto mais difícil lhe parece se
tornar a despedida, sobretudo uma despedida tão prosaica e fortuita.
Ocorrem-lhe perguntas sobre a história dos belos galgos, descendentes
dos cães do faraó, que perambulam sem dono pela ilha, sobre os velhos
costumes de seduzir e raptar mulheres, a respeito dos quais jamais
conseguiu saber algo mais detalhado, sobre a origem daqueles nomes
estranhos que os pescadores usam para designar as montanhas, e que são
bem diferentes dos nomes que as mesmas montanhas têm na língua dos
camponeses. Na hora certa, ele se lembra de, às vezes, ter ouvido o
proprietário daquela pequena \emph{tienda} falar como se fosse uma
autoridade em todas as questões que dizem respeito à crônica do lugar.
No último momento, ele ainda gostaria de garantir informações sobre isso
e aquilo, e de superar assim a solidão da noite que entra. Ele pede uma
garrafa do melhor vinho e, enquanto o dono saca a rolha diante de seus
olhos, uma conversa já se desenrolou entre os dois. Eis que o
estrangeiro nas últimas semanas conheceu a hospitalidade fanática dos
moradores da ilha, de modo suficiente para saber que é preciso estipular
de antemão e com boa antecipação a hora de lhes oferecer alguma coisa.
Assim, a primeira coisa que ele faz é convidar o dono a beber com ele, e
nesse ponto ele se mantém firme também à segunda e à terceira garrafa,
tanto mais que enquanto isso consegue, com boas maneiras, anotar em seu
caderno, na forma de palavras-chave, uma ou outra dessas informações. E
enquanto vai folheando o caderno ao clarão da vela, acaba dando de cara
-- ele também é um pouco desenhista -- com esboços feitos nos primeiros
dias, logo após sua chegada. Ali está o cego com a coxa crua de uma
cabra ou de um carneiro, que anda sempre pelas ruas, guiado por um
garoto; em outra folha, os perfis vivazes dos muros, que em nenhum
momento tocam qualquer medida de referência mais exata; e, em seguida, a
escadaria de azulejos com as cifras misteriosas, com a qual ele deu de
cara logo no princípio, ao procurar por moradia. O dono da taverna
olhava por sobre o seu ombro com interesse. É claro que ele conhece a
história da coxa de carneiro: ele mesmo se engajou na municipalidade, a
fim de conceder ao cego a permissão para fazer uma loteria precária e
distribuir números cujo único prêmio é aquela coxa. E os azulejos
misteriosamente cifrados, ele mesmo ainda vira em uma rua na qual eles
anunciavam os números das casas. Mais do que isso: ele também sabe o que
significam as cruzes brancas aos pés de algumas casas, que
proporcionaram ao estrangeiro um quebra-cabeças e tanto. Elas são uma
espécie de altar de descanso. Por toda parte onde aparecem, representam
um dos pontos nos quais as procissões estacam de repente ao seguirem
pelas ruas. E agora o estrangeiro se lembra vagamente de ter visto algo
semelhante em aldeias da Vestfália. Entrementes esfriou; o dono da
taverna faz questão de dar ao hóspede um de seus próprios casacos, e a
última garrafa é aberta. Mas voltemos às anotações do estrangeiro; em
que passagem há, nas novelas italianas de Stendhal, um tema comparável a
esse tema típico de Ibiza: o da moça casadoira, cercada de pretendentes
em um dia de feriado, mas com o pai estipulando rigorosamente a duração
das conversas com todos os candidatos; uma hora, uma hora e meia no
máximo, ainda que sejam trinta rapazes ou mais, de modo que cada um
deles é obrigado a resumir tudo o que quer dizer em poucos minutos\ldots{}
Uma boa metade da garrafa ainda está à espera quando a sirene ecoa até o
lugar em que eles estão. É o Ciudad de Mahón, há dez minutos de
distância, chamando para a partida do porto onde se encontra, com a
bagagem do estrangeiro já a bordo. Sobre os telhados, a luz de seu
mastro aparece na escuridão do céu. Que não resta mais muito tempo para
saudações, o próprio dono da taverna admite, de modo que estende, sem
grandes relutâncias e conforme o combinado, a conta ao estrangeiro.
Este, no entanto, se assusta, antes mesmo de lançar os olhos sobre ela.
Seu dinheiro se foi. Rápido como um raio, ele lança um olhar ao dono da
taverna. O rosto singelo deste expressa consternação. Impossível que ele
esteja com o envelope e as cédulas que este contém. Com as mesuras mais
distintas, ele pede que o estrangeiro não dê importância ao fato. De
resto, já lhe parecera pouco adequado ser convidado do homem em sua
própria casa. E, quanto ao dinheiro, estaria por certo na jaqueta que se
encontrava a bordo. Para o estrangeiro, no entanto, isso representa
apenas meio consolo. As cédulas das quais sente falta não são de pequeno
valor, e também não são poucas. A bordo, o pior se confirma. O casaco
está vazio, e ele agora sabe o que pensar da louvada honradez dos
moradores da ilha. Diante da alternativa de suspeitar do dono da taverna
ou do marinheiro que cuidou de suas coisas, ele se decide, durante a
noite insone em sua cabine, pela última alternativa. Mas se enganou. Foi
o dono da taverna que pegou o dinheiro. Mal chegou em casa, recebe a
prova disso na feição do seguinte telegrama: ``Dinheiro na jaqueta que o
senhor usou quando esteve aqui. Instruções seguem''.

``No que diz respeito ao telegrama'', disse Dom Rosello, que ouvira com
um sorriso de concessão no rosto, ``com certeza foi o primeiro que ele
mandou em sua vida toda.''

``Sim, mas e o que isso tem a ver?''

``Sei muito bem'', replicou ele, ``onde o senhor quer chegar. Na
intocabilidade dos nativos. Na idade de ouro. Os lugares-comuns de
Rousseau. Há sete anos foram abertas as portas da prisão que se
localizava em um castelo mouro, e de fato não se precisava mais dela.
Mas o senhor sabe por quê? Vou dizê-lo com as palavras do velho guarda,
que na época tivemos de demitir: `A nossa gente\ldots{}, ela agora já
andou tanto pelo vasto mundo. E acabou aprendendo a diferenciar entre o
bem e o mal'. O contato com o mundo incentiva a moralidade. Isso é
tudo''.

\chapter{A sebe de cactos\footnote[*]{``Die Kaktushecke'', in \versal{GS} \versal{IV}-2, pp. 748-754.
  Tradução de Marcelo Backes. Este conto foi publicado em 8 de janeiro
  de 1933 no suplemento literário do \emph{Vossischen Zeitung.} Ele
  corresponde à versão resumida de um texto mais longo que infelizmente
  se perdeu. [\versal{N}. \versal{E}.]} }

O primeiro estrangeiro que veio até nós, em Ibiza, foi um irlandês,
O'Brien. Isso agora faz mais ou menos vinte anos, e o homem na época já
estava na casa dos quarenta. Antes de se aposentar aqui conosco, ele
viajara pelo mundo inteiro. Na juventude, vivera por muito tempo como
fazendeiro no leste da África, foi um grande caçador e laçador, mas era
sobretudo um tipo esquisito como nunca conheci outro igual. Ele se
mantinha longe dos círculos instruídos, dos padres, dos funcionários da
magistratura, e até mesmo com os nativos mantinha apenas alguns contatos
esparsos. Mesmo assim sua memória continua viva entre os pescadores
ainda hoje, e sobretudo por causa da sua maestria com os nós. De resto,
sua timidez parecia apenas em parte a consequência de sua natureza;
experiências adversas com pessoas próximas terminaram por completar a
dose.

Na época, eu não consegui descobrir muita coisa além do fato de que um
amigo, ao qual ele havia confiado sua única e valiosa propriedade,
simplesmente desaparecera com ela. Tratava-se de uma coleção de máscaras
negras, que ele havia adquirido com os próprios nativos em seus anos de
África. E de resto elas não trouxeram sorte àquele que se apropriara
delas. Ele morrera em um incêndio de navio, levando consigo a coleção de
máscaras que o acompanhara a bordo.

O'Brien se estabeleceu em sua finca, bem alto acima da baía. Quando
tinha algum trabalho em vista, no entanto, seu caminho sempre o levava
ao mar. Ali ele se ocupava da pesca, fazia as armadilhas de cana
baixarem cem metros ou mais na água, onde as lagostas passeiam sobre o
fundo rochoso do mar, ou saía às tardes tranquilas para lançar suas
redes, que doze horas depois eram retiradas. Além disso, ele continuava
gostando de capturar animais terrestres, e tinha relações suficientes
com amadores e cientistas na Inglaterra para apenas raramente ficar sem
encargos de arrumar algum pássaro empalhado, espécies raras de besouros,
gecos ou borboletas. Na maior parte do tempo, contudo, ele se ocupava
das lagartixas. Lembre-se dos terrários que na época, primeiro na
Inglaterra, acabaram por se estabelecer nos cantos de cacto dos
\emph{boudoirs} ou nos jardins de inverno. Lagartixas começaram a se
tornar artigos da moda, e nossas Baleares em pouco tempo ficaram tão
conhecidas entre os comerciantes de animais quanto o foram no passado
entre os chefes de legiões romanas por suas catapultas. Pois ``balea''
significa catapulta.

O'Brien, eu já disse, era um tipo esquisito. Acho que desde o fato de
caçar lagartixas até o modo de cozinhar e inclusive o de dormir e
pensar, ele não fazia nada do jeito que os outros costumam fazer. No que
dizia respeito a alimentos, ele dava pouco valor a vitaminas, calorias e
coisas do tipo. Tudo que era de comer, ele costumava dizer, era cura ou
envenenamento, e entre ambas as coisas não havia nada de intermediário.
Aquele que comia, portanto, deveria ver sempre a si mesmo como uma
espécie de convalescente, caso quisesse se alimentar de modo correto. E
logo se podia ouvir dele uma lista inteira de alimentos, dos quais
alguns eram mais adequados ao comportamento sanguíneo, os outros ao
colérico, outros ainda ao fleumático, e por fim outros ao melancólico,
se mostrando curativos para eles por incorporarem as devidas substâncias
complementares e atenuantes.

Coisa bem semelhante sucedia com o sono; ele tinha a respeito do assunto
sua própria teoria dos sonhos, e afirmava ter conhecido entre os
pangwes, uma tribo negra do interior africano, o meio infalível de
manter distante de si os pesadelos e os rostos torturantes que retornam
durante o sono. Seria preciso apenas, antes de ir dormir à noite,
invocar -- como os pangwes fazem em suas cerimônias -- a imagem
assustadora, pois assim se ficaria livre dela durante a noite inteira.
Ele chamava a isso de vacina do sonho.

E, por fim, o pensamento -- como ele agia em relação ao pensamento eu
ficaria sabendo certa tarde, quando estávamos em um barco, em meio à
água, para retirar as redes que haviam sido lançadas no dia anterior. A
pesca foi miserável. Nós havíamos recolhido quase toda a rede
praticamente vazia, quando algumas das malhas ficaram presas a um recife
e, apesar de todo o cuidado, acabaram se rasgando ao ser retiradas.

Eu enrolei minha capa de chuva, coloquei-a no meu barco e me estendi no
chão. O tempo estava encoberto, o ar tranquilo. Em pouco tempo caíram
alguns pingos de chuva, e a luz, que nesta terra solicita todas as
coisas tão violentamente do alto do céu, se afastou para devolvê-las à
terra.

Quando me levantei, meu olhar caiu sobre ele. Ele ainda segurava sua
rede nas mãos, mas elas já descansavam; o homem estava como que ausente.
Estranhando, eu o contemplei mais detidamente; seu rosto não tinha
expressão e não mostrava idade; em torno da boca cerrada, brincava um
sorriso. Eu peguei meu par de remos; alguns golpes nos levaram sobre a
água tranquila.

O'Brien levantou os olhos.

``Agora ela vai pegar de novo'', disse ele, e examinou, forçando
bastante, os novos nós na rede. ``Mas também se trata de um flamengo
duplo.''

Sem compreender, eu olhei para ele.

``Um flamengo duplo'', repetiu ele. ``Olhe bem, ele pode servir também
para a pesca com linha.''

E, ao dizê-lo, tomou um pedaço de barbante, dobrou uma de suas pontas e
o envolveu em si mesmo três, quatro vezes, até que ele se tornasse o
eixo de uma espiral cujos verticilos de um só movimento se juntaram em
nó.

``Na verdade'', prosseguiu ele, ``esse nó é apenas uma variação do nó
duplo das galés e, em todo caso, enlaçado ou não, preferível ao nó
carpinteiro''. Tudo isso ele acompanhava de volteios rápidos e laçadas.
Eu sentia vertigens.

``Quem dá de primeira esse nó'', concluiu ele, ``conseguiu chegar bem
longe, e pode enfim descansar. E digo isso literalmente: descansar
mesmo, pois dar nós é uma arte da yoga; talvez o mais maravilhoso dos
meios para se descontrair. E só podemos aprendê-lo através do exercício
e da repetição -- não quando já estamos na água, mas em casa, com toda a
calma e paciência, no inverno, sobretudo se estiver chovendo. Melhor
ainda nos momentos em que se está angustiado ou preocupado. O senhor não
acredita quantas vezes encontrei soluções para perguntas que me
importunavam fazendo esse exercício.''

Por fim, ele prometeu me dar aulas nessa matéria, introduzindo-me em
todos os seus mistérios: dos nós cruzados aos de tecelão e inclusive os
de amortecimento e de Hércules.

Mas isso acabou não acontecendo; pois logo depois passou a ser visto
cada vez mais raramente no mar. Primeiro, ele ficava três, quatro dias
longe dele, depois semanas inteiras. Ninguém tinha a menor ideia do que
O'Brien fazia enquanto isso. Murmuravam acerca de uma ocupação secreta.
Sem dúvida alguma ele havia descoberto uma nova paixão.

Passaram-se alguns meses até estarmos outra vez juntos no barco. Nessa
ocasião, a pesca foi mais abundante e, quando por fim encontramos uma
grande truta do mar em seu anzol, O'Brien me convidou para ir visitá-lo
na noite seguinte para um pequeno jantar.

Depois de concluída a refeição, O'Brien disse, abrindo uma porta:
``Minha coleção, da qual o senhor com certeza já ouviu falar''.

Por certo eu já ouvira falar da coleção de máscaras de negros,
entretanto sabia apenas que elas haviam naufragado.

Mas eis que ali estavam penduradas, vinte ou trinta peças, no quarto
vazio, sobre paredes brancas. Eram máscaras de expressões grotescas, que
revelavam sobretudo uma severidade levada ao cômico, uma recusa
completamente inexorável de tudo que era desmedido. Os lábios superiores
abertos, as estrias abobadadas que haviam se tornado a fenda das
pálpebras e sobrancelhas pareciam expressar algo como asco infinito
contra aquele que se aproximava, até mesmo contra tudo o que se
aproxima, enquanto os cimos empilhados dos adornos na testa e os
reforços das mechas de cabelos entrançadas se destacavam como marcas que
anunciavam os direitos de um poder estranho sobre aquelas feições. Para
qualquer dessas máscaras que se olhasse, em lugar nenhum sua boca
parecia destinada, como quer que fosse, a emitir sons; os lábios grossos
e entreabertos, ou então bem cerrados, eram cancelas instaladas antes ou
depois da vida, como os lábios dos embriões ou os dos mortos.

O'Brien havia ficado para trás.

``Esta aqui'', disse ele de repente atrás de mim e como se falasse
consigo mesmo, ``foi a primeira que reencontrei''.

Quando me virei, ele estava parado diante de uma cabeça alongada, lisa,
de ébano negro, que mostrava um sorriso. Era um sorriso por assim dizer
do princípio ao fim, que no fundo parecia um ruminar do sorriso por trás
dos lábios cerrados. De resto, aquela boca jazia bem profunda, como
também o semblante inteiro não parecia mais do que o rebento monstruoso
da testa formidavelmente abaulada, que descia abaixo em arcos
inexoráveis, interrompidos apenas pelos círculos redondos e solenes dos
olhos, que se destacavam como que de um escafandro.

``Essa foi a primeira que reencontrei. E eu também poderia dizer como
foi ao senhor.''

Eu apenas olhei para ele. Com as costas, ele se apoiou contra a janela
baixa, e em seguida principiou:

``Se o senhor olhar para fora, verá a sebe de cactos diante de seus
olhos. É a maior de toda a região. Observe o tronco, como está
amadeirado até bem no alto. Nele o senhor pode reconhecer a idade da
sebe; pelo menos cento e cinquenta anos. Era uma noite como a de hoje,
só que a lua brilhava. Lua cheia. Não sei se o senhor já se deu conta do
efeito da lua nessa região, pois sua luz não parece cair sobre o cenário
de nossa existência diurna, mas sim sobre uma terra oposta ou
paralela.\footnote{Esta passagem figura igualmente numa resenha de 1928,
  constante do presente volume, em que Benjamin trata do livro de Jakob
  Job, intitulado \emph{Nápoles. Imagens de viagem e esboços}.
  \emph{Vide} página\,\pageref{supra10}, adiante. [\versal{N}. \versal{E}.]} Eu
havia passado aquele entardecer inteiro diante dos meus mapas marítimos.
O senhor precisa saber que meu cavalo de batalha é melhorar os mapas do
Ministério da Marinha britânico, o que é ao mesmo tempo uma fama
conquistada de modo bem barato, pois onde ocupo um novo lugar com minhas
nassas também acabo fazendo sondagens. Eu havia, pois, identificado o
lugar correto de algumas colininhas no mar e pensado como seria bonito
se me eternizassem lá embaixo, nas profundezas, dando a uma delas o meu
nome. E em seguida fui para a cama. O senhor por certo há de ter visto
que minhas janelas estão cobertas por cortinas; na época elas ainda não
estavam, e a lua avançava sobre a minha cama, enquanto eu estava
deitado, insone. Eu começara outra vez minha brincadeira predileta de
dar nós. Acho que já falei disso ao senhor uma vez. Isso vai acontecendo
assim que dou um nó complicado em pensamentos, logo o ponho comigo mesmo
de lado e acabo conseguindo dar um segundo, outra vez em pensamentos.
Então o primeiro volta a me ocupar a mente. Só que dessa vez não preciso
atá-lo, mas sim desatá-lo. É claro que tudo depende de manter, com toda
a precisão, a forma do nó na memória, sobretudo o primeiro não pode se
confundir com o segundo. Volto a fazer esses exercícios, nos quais
realmente consegui adquirir um bocado de conhecimento, sempre que tenho
ideias na cabeça e não encontro nenhuma solução, ou cansaço nos membros
e não encontro sono. Em ambos os casos o resultado é o mesmo:
descontração.

Dessa vez, porém, minha maestria não me ajudou em nada, pois quanto mais
eu me aproximava da solução, tanto mais próximo ficava também o clarão
ofuscante da lua em minha cama. Então resolvi fugir para um outro
método. Passei em revista todas as sentenças, os enigmas, as canções e
ditados que eu havia aprendido aos poucos na ilha. Isso já estava dando
mais certo. Sentia minhas contrações internas cedendo, quando meu olhar
caiu sobre a sebe de cactos. Um antigo versinho de zombaria me veio à
memória: `Buenas tardes chlumbas figas.' O jovem campônio diz `Boa
tarde' aos figos do cacto, arranca sua faca e, como se diz, dá-lhe um
talho da espinha até o traseiro.

Mas a época dos frutos do cacto já havia passado há tempo. A sebe estava
pelada; suas folhas ora espetavam o vazio, inclinadas, ora se
amontoavam, cascas grossas que esperavam em vão pela chuva.

`Nenhuma cerca, mas muitos espectadores olhando por cima dela', foi o
que me passou pela cabeça.

Pois, entrementes, parecia ter ocorrido uma transformação com aquela
sebe. Era como se aqueles lá fora olhassem todos para o clarão que agora
envolvia toda a minha cama; como se ali houvesse um bando dependurado,
prendendo a respiração, em meus olhares. Uma confusão de escudos
erguidos, clavas e machados de guerra. E, ao adormecer, eu reconheci de
repente o método através do qual aquelas figuras ali fora mantinham"-me
em xeque. Eram máscaras que se erguiam em minha direção!

Assim o sono acabou tomando conta de mim. Na manhã seguinte, porém, eu
não me permiti ficar em paz. Peguei uma faca e em seguida me tranquei
durante oito dias com o bloco do qual acabou surgindo a máscara que está
pendurada aqui. As outras surgiram uma após a outra, e sem que eu algum
dia tenha perdido mais um olhar sequer para a sebe de cactos. Não quero
dizer que são todas parecidas com as minhas máscaras de antes; mas
poderia jurar que nenhum conhecedor seria capaz de diferenciá"-las
daquelas que há anos estiveram em seu lugar.''

Foi isso que me contou O'Brien. Nós ainda papeamos por um momento,
depois eu fui embora.

Algumas semanas mais tarde, ouvi que O'Brien havia se trancado outra vez
com um trabalho misterioso e se tornara inacessível para todo mundo.
Jamais voltei a vê"-lo, pois logo em seguida ele morreu.

Por muito tempo não pensara mais nele quando, para minha surpresa,
descobri certo dia três máscaras de negros numa caixa de vidro de um
comerciante de arte parisiense na Rue La Boétie.

``Posso'', disse eu, voltando-me para o diretor da casa, ``parabenizá-lo
de coração por essa aquisição incrivelmente bela?''

``Vejo com prazer'', foi a resposta, ``que o senhor sabe honrar a
qualidade! Vejo ademais que o senhor é um conhecedor! As máscaras que o
senhor com razão admira não são mais do que uma pequena amostra da
grande coleção cuja exposição estamos preparando no momento!''

``E eu poderia pensar, meu senhor, que essas máscaras certamente
inspirariam nossos jovens artistas a fazer suas próprias tentativas
interessantes.\label{supra8}''

``É o que eu espero, inclusive!\ldots{} Aliás, se o senhor se interessar mais
de perto pelo assunto, posso fazer com que cheguem até o senhor, do meu
escritório, os pareceres de nossos maiores conhecedores de Haia e de
Londres. O senhor haverá de ver que se trata de objetos de centenas de
anos. De dois deles eu diria até que de milhares de anos.''

``Ler esses pareceres de fato me interessaria muito! Eu poderia
perguntar a quem pertence essa coleção?''

``Ela pertence ao espólio de um irlandês. O'Brien. O senhor com certeza
jamais ouviu seu nome. Ele viveu e morreu nas Ilhas Baleares.''

\chapter{Histórias da solidão\footnote[*]{``Geschichten aus der Einsamkeit '', in \versal{GS}
  \versal{IV}-2, pp. 755-757. Tradução de Marcelo Backes. De acordo com o
  testemunho de Gershom Scholem, essas histórias teriam sido escritas
  entre 1932 e 1933. Não há certeza quanto à ordem da série. [\versal{N}. \versal{E}.]} }

\section{O muro}

Eu vivia há alguns meses em um ninho nas rochas, na Espanha. Muitas
vezes eu me propusera a sair para explorar os arredores, envolvidos por
uma coroa de cumes perigosos e formações florestais de pinheiros bem
escuras. No meio, havia algumas aldeias escondidas; a maior parte
recebera nomes de santos, que poderiam muito bem habitar aquela região
paradisíaca. Mas era verão; o calor fazia com que eu adiasse a cada dia
meu propósito, e mesmo ao passeio preferido até a colina dos moinhos de
vento, que eu via da minha janela, acabei renunciando. De modo que
restou-me o perambular costumeiro pelas ruelas estreitas e sombreadas,
em cuja rede jamais se encontra o mesmo nó do mesmo jeito. Certa tarde,
acabei encontrando, em minhas errâncias, uma loja de bugigangas na qual
podiam ser comprados cartões postais. De qualquer modo havia alguns na
vitrine, um deles com a foto da muralha de um dos lugarejos, uma das
muitas que encontramos neste canto do mundo. Eu jamais vira uma muralha
semelhante, no entanto. O fotógrafo havia captado toda a sua magia, e
ela serpenteava através da paisagem como uma voz, como um hino por todos
os séculos de sua duração. Eu prometi não comprar aquele cartão antes de
ter visto pessoalmente o muro que era apresentado nele. Não falei a
ninguém do meu propósito, e podia fazê-lo facilmente pois o cartão me
conduzia ao lugar certo com sua assinatura: ``S. Vinez''. Por certo, eu
não sabia nada sobre um santo Vinez. Mas saberia mais acerca de um São
Fabiano, de um São Romano ou de um São Sinfório, como eram chamados os
outros lugares da região? Ainda que meu guia de viagem não apontasse o
nome, isso não queria dizer nada, a princípio. A região era habitada por
camponeses, e marinheiros faziam suas marcações segundo ela: ambos,
porém, tinham nomes diferentes para os mesmos lugares. De modo que
busquei o auxílio de mapas mais antigos e, como isso também não me levou
mais adiante, consegui um mapa de navegação. Em pouco tempo essa
pesquisa já me deixava fascinado, e teria sido contra a minha honra
buscar a ajuda ou o conselho de um terceiro em um estágio tão avançado
das minhas pesquisas. Eu acabara de passar mais uma vez uma hora sobre
meus mapas quando um conhecido, nativo do lugar, convidou-me para um
passeio ao entardecer. Ele queria me levar para a colina fora da cidade,
da qual os moinhos de vento há muito desativados me cumprimentaram
tantas vezes por sobre a copa dos pinheiros. Quando chegamos lá em cima,
começou a escurecer, e nós descansamos para esperar a lua, a cujo
primeiro raio nós nos pusemos a caminho de casa. Saímos de um bosque de
pinheiros. Ali estava, à luz da lua, próximo e inconfundível, o muro
cuja imagem me acompanhava há dias, e sob sua proteção a cidade para a
qual estávamos voltando. Eu não disse palavra, mas logo me separei de
meu amigo\ldots{} Na tarde seguinte, dei de cara sem querer com minha loja de
bugigangas. O cartão postal ainda estava pendurado na vitrine. Sobre a
porta, no entanto, li em uma placa que antes não havia percebido,
escrito em letras vermelhas, ``Sebastiano Vinez''. O pintor havia
acrescentado um pão-de-açúcar e uma fatia de pão.

\section{O cachimbo}

Durante um passeio na companhia de um casal com o qual fizera amizade,
passei pelas proximidades da casa que eu habitava na ilha. Tive vontade
de acender meu cachimbo. Como não o encontrei ao tocar no bolso, num
gesto habitual, pareceu-me que era a oportunidade adequada para ir
buscá-lo no quarto, onde haveria de estar sobre a mesa. Com uma breve
explicação, pedi ao amigo que se adiantasse com sua mulher, enquanto eu
procurava o desaparecido. Dei meia volta; mas eu mal havia me afastado
dez passos quando senti, vasculhando de novo, que o cachimbo estava em
meu bolso. Foi assim que os outros me viram retornar para junto deles em
menos de um minuto, soltando nuvens de fumaça pelo cachimbo. ``Não é que
ele estava realmente sobre a mesa'', expliquei eu, seguindo um capricho
incompreensível. No olhar do homem apareceu algo que o fez se parecer
com alguém que despertava e, depois de um sono profundo, ainda não
conseguira descobrir muito bem onde estava. Nós seguimos adiante, e a
conversa retomou seu curso normal. Um pouco mais tarde, eu a reconduzi
ao \emph{intermezzo}. ``Como é possível'', perguntei eu, ``que o senhor não
tenha percebido nada? O que eu afirmei era completamente impossível''.

``Isso com certeza'', respondeu o homem depois de uma breve pausa. ``Eu até
quis dizer algo. Mas então pensei comigo: no fundo deve estar certo. Por
que ele haveria de mentir pra mim?''

\section{A luz }

Eu estava pela primeira vez sozinho com minha amada e em um povoado
desconhecido. Esperava diante do meu alojamento, que não era o dela. Nós
ainda queríamos fazer um passeio noturno. À espera, andei para cima e
para baixo na rua do povoado. Então vi ao longe, entre as árvores, uma
luz. ``Essa luz'', foi o que pensei comigo, ``nada diz àqueles que todas
as noites a têm diante do olhos. Ela deve pertencer a um farol ou a uma
fazenda. Para mim, que não sou daqui, no entanto, ela diz muito''. E com
isso dei meia volta para seguir novamente pela rua do povoado. Assim
continuei por algum tempo e, sempre que eu me voltava, depois de alguns
instantes, a luz entre as árvores atraía meu olhar. Mas então aconteceu
que ela ordenou que eu parasse. Isso foi pouco antes da minha amada me
reencontrar. Eu me desviara outra vez, e reconheci: a luz que eu
vislumbrara junto à terra era a da lua, que aos poucos havia subido
acima das copas longínquas das árvores.

\chapter{Quatro histórias\footnote[*]{``Vier Geschichten'', in \versal{GS} \versal{IV}-2, pp. 757-761.
  Tradução de Marcelo Backes. Esse conjunto de quatro histórias foi
  publicado em 5 de agosto de 1934 no \emph{Prager Tagblatt}. Outras
  publicações, parciais, apareceram no \emph{Kölnischen Zeitung}, em
  12/07/1933; no \emph{Frankfurter Zeitung}, em 05/09/1934; e no
  \emph{Basler Nachrichten}, em 26/09/1935. Uma tradução dinamarquesa de
  ``A assinatura'' foi publicada em 16/09/1934 num periódico de
  Copenhagen. ``A assinatura'' e ``O desejo'' aparecem também no ensaio
  de Benjamin sobre Franz Kafka, de 1934. E histórias semelhantes a
  estas duas são narradas por Ernst Bloch, respectivamente, em
  \emph{Rastros} (\emph{Spuren}, 1930) e \emph{Através do deserto}
  (\emph{Durch die Wüste}, 1923). [\versal{N}. \versal{E}.]} }

\section{O alerta }

Junto a um lugar de excursão, não muito distante de Tsingtao, havia uma
formação rochosa que se distinguia por sua localização romântica e pelas
paredes íngremes com que se precipitava nas profundezas. Essa formação
rochosa era visitada por muitos amantes em seus momentos felizes, os
quais, depois de terem admirado a paisagem nos braços de suas namoradas,
voltavam na companhia das mesmas até uma hospedaria próxima. Essa
hospedaria ia muito bem. Ela pertencia ao senhor Ming.

Então certo dia um amante, que havia sido abandonado, teve a ideia de
pôr um fim em sua vida justamente ali onde ele a desfrutara com mais
entusiasmo e, não muito longe da hospedaria, precipitou-se da rocha para
as profundezas. Esse amante inventivo encontrou imitadores, não demorou
muito e essas formações rochosas ficaram tão famigeradas com seu
cemitério de crânios quanto famosas como mirante. O estabelecimento do
senhor Ming, no entanto, sofria com a nova fama; nenhum cavalheiro
poderia ousar levar sua dama a um lugar onde a todo instante poderia
chegar uma ambulância. Os negócios do senhor Ming iam de mal a pior, e
não lhe restou outra coisa a fazer senão refletir.

Ele se trancou certo dia em seu quarto. Quando voltou a aparecer, foi
até a estação elétrica que se localizava nas proximidades. Depois de
poucos dias, um arame contornava a extremidade da romântica formação
rochosa. Sobre uma tabuleta pendurada a ele, era possível ler:
``Atenção! Alta tensão! Perigo de morte!'' Desde então, os candidatos a
suicida evitaram aquela região, e os negócios do senhor Ming voltaram a
florescer como antes.

\section{A assinatura }

Potemkin sofria de pesadas depressões, que sempre voltavam mais ou menos
regularmente, durante as quais ninguém podia se aproximar dele e o
acesso a seu quarto era estritamente proibido. Na corte, esse sofrimento
não era mencionado, e se sabia, sobretudo, que qualquer alusão ao fato
suscitaria o desagrado fatal da czarina Catarina. Uma dessas depressões
do chanceler se mostrou extraordinariamente longa. Problemas sérios
foram a consequência; nos registros, se amontoavam autos cujas demandas
a czarina exigia que fossem solucionadas, ainda que isso se mostrasse
impossível sem a assinatura de Potemkin. Os altos funcionários não
sabiam mais o que fazer.
%czar, em 'O contador de histórias', aparece grafado no russo, tsar. Padroniza as ocorrências da palavra ou pode deixar diferindo, uma vez que são textos diferentes? 

Por essa época, e em razão de um acaso, o pequeno e insignificante
funcionário de chancelaria Schuwalkin acabou adentrando a antessala do
palácio do chanceler, onde os conselheiros de Estado, como de costume,
haviam se reunido, lamentando e se queixando. ``O que há, Excelências?
Como posso servir a Vossas Excelências?'', observou o zeloso Schuwalkin.
Explicaram-lhe o caso, e lamentaram não poder fazer uso absolutamente
nenhum de seus serviços. ``Se não for nada além disso, meus senhores'',
respondeu Schuwalkin, ``podem deixar os autos comigo. Eu peço que seja
assim''. Os conselheiros de Estado, que nada tinham a perder, se
deixaram convencer, e Schuwalkin tomou, com a pilha de pastas debaixo do
braço, o caminho até o quarto de Potemkin, atravessando galerias e
corredores. Sem bater, e até mesmo sem parar um momento sequer, ele
baixou a maçaneta. O quarto não estava trancado. No lusco-fusco,
Potemkin estava sentado em sua cama, roendo as unhas, vestia um pijama
puído. Schuwalkin foi até a escrivaninha, mergulhou a pena na tinta e,
sem perder tempo em dizer uma palavra sequer, empurrou-a até a mão de
Potemkin, colocando o primeiro documento sobre seus joelhos. Lançando um
olhar ausente para o intruso, como se estivesse dormindo, Potemkin
assinou; depois uma segunda vez; e assim com todas as pastas. Quando a
última estava devidamente assinada, Schuwalkin deixou, sem quaisquer
circunstâncias e do mesmo jeito que viera, os aposentos, com seu dossiê
debaixo do braço.

Triunfante e sacudindo os documentos, Schuwalkin adentrou a antessala.
Os conselheiros de Estado se precipitaram ao encontro dele e arrancaram
os papéis de suas mãos. Todos se curvaram esbaforidos sobre eles.
Ninguém disse uma palavra; o grupo ficou pasmo. O pequeno funcionário da
chancelaria se aproximou mais uma vez, e mais uma vez perguntou, zeloso,
pelo motivo do espanto dos senhores. Então também seu olhar caiu sobre a
assinatura. Tanto aquele quanto todos os outros documentos estava
assinado com: Schuwalkin, Schuwalkin, Schuwalkin\ldots{}

\section{O desejo }

Certa noite, em uma aldeia hassídica, ao fim do \emph{shabat}, estavam
sentados os judeus em uma taverna pobre. Eram todos do lugar, exceto um,
a quem ninguém conhecia, um bem pobre e esfarrapado, encolhido ao fundo,
à sombra da lareira. As conversas haviam ido de um lado a outro. Então
alguém teve a ideia de perguntar o que cada um deles desejaria caso
tivesse direito a um desejo. Um queria dinheiro, outro um genro, o
terceiro um banco de carpinteiro novo, e assim os desejos seguiram a
roda.

Todos haviam tomado a palavra, restava apenas o mendigo no canto da
lareira. Contra a vontade e hesitando, ele cedeu às perguntas: ``Eu
gostaria de ser um rei muito poderoso e reinar sobre um país gigante e
deitar à noite e dormir em meu palácio, e que na fronteira o inimigo
invadisse o reino e, antes que escurecesse, a cavalaria teria avançado
até diante do meu castelo e não haveria resistência, e, acordando
assustado do meu sono, sem tempo nem mesmo de me vestir, em mangas de
camisa, eu teria de me pôr em fuga e seria acossado por montanhas e
vales e pela floresta e pelas colinas, sem paz, dia e noite, até que
enfim chegasse aqui, a esse banco, salvo nesse vosso cantinho. Isso eu
desejaria.''

Sem compreender, os outros se entreolharam. ``E o que você conseguiria
com tudo isso?'', perguntou um deles.

``Uma camisa'', foi a resposta.

\section{O agradecimento}

Beppo Aquistapace era empregado de um banco nova-iorquino. O humilde
homem vivia apenas para seu trabalho. Em quatro anos de serviços
prestados, ele faltara no máximo três vezes, e jamais sem uma desculpa
plausível. Por isso, chamou obrigatoriamente a atenção ao faltar certo
dia sem nada anunciar. Quando também no dia seguinte não chegaram nem o
homem nem sua desculpa, o senhor McCormik, chefe de pessoal, lançou uma
série de palavras questionadoras no escritório de Aquistapace. Mas
ninguém soube lhe dar informações. O desaparecido tinha poucas relações
com seus colegas; circulava com italianos, que como ele eram oriundos
das baixas classes sociais. E foi justamente esta circunstância que
Aquistapace invocou em uma carta, na qual, depois de uma semana, deu
informações sobre seu destino ao senhor McCormik.

Essa carta veio de uma cela da detenção. Nela, Aquistapace se dirigia a
seu chefe com palavras tão ponderadas quanto urgentes. Um acontecimento
lamentável no bar que costumava frequentar, acontecimento no qual ele
aliás não tivera absolutamente nenhuma participação, havia determinado
sua detenção. Ainda hoje ele não era capaz de mencionar o motivo que
levara a uma briga de faca entre seus conterrâneos. Lamentavelmente,
houvera uma vítima. E eis que ele não conhecia ninguém a não ser o
senhor McCormik para nomear como avalista de sua boa fama\ldots{} Este não
apenas tinha um certo interesse no trabalho fiel e cumpridor de seu
dever do preso, mas inclusive relações que tornaram fácil para ele dar
uma palavrinha a favor do outro na hora e no lugar adequados.
Aquistapace estava detido há apenas dez dias quando voltou a assumir seu
trabalho no banco.

Depois de fechado o escritório, ele anunciou sua presença junto a
McCormik. Encabulado, Aquistapace se encontrava em pé diante de seu
chefe. ``Senhor McCormik'', principiou ele, ``não sei como posso
agradecer ao senhor. Ao senhor, e apenas ao senhor, devo o fato de ter
sido libertado. Acredite em mim, nada me deixaria mais alegre do que me
mostrar reconhecido ao senhor. Lamentavelmente, sou um homem pobre. E'',
acrescentou ele com um sorriso humilde, ``que eu também não ganho nenhum
tesouro no banco, o senhor sabe melhor do que ninguém. Mas, senhor
McCormik'', concluiu ele em voz firme, ``uma coisa posso lhe garantir:
se algum dia o senhor estiver em uma situação em que a eliminação de um
terceiro poderia ser útil, peço que se lembre de mim. Comigo o senhor
pode contar''.

\chapterspecial{A morte do pai}{Novela\,\footnote[*]{``Der Tod des Vaters. Novelle'', in \versal{GS} \versal{IV}-2, pp.
  723-725. Tradução de Marcelo Backes. Benjamin se refere à redação
  deste conto numa carta datada de 7 de junho de 1913 e endereçada ao
  seu colega de estudos Herbert Belmore. Escrito no período de
  militância no Movimento da Juventude (\emph{Jugendbewegung}), o texto
  permaneceu inédito até a publicação póstuma. [\versal{N}. \versal{E}.]}}{}



Durante a viagem, ele evitou tornar claro o sentido daquele telegrama:
``Venha imediatamente. Mudança para pior''. Ao anoitecer, havia deixado
o lugar em que estava, na Riviera, em meio ao tempo ruim. As lembranças
o envolviam como as luzes matinais que caem sobre um frequentador de bar
que chega atrasado: doces e vergonhosas. Indignado, ele ouvia os ruídos
da cidade, em cujo meio-dia adentrava. Estar humilhado lhe parecia a
única resposta às perseguições da terra natal. Gorjeando, porém, ele
sentia a volúpia das horas perdidas junto a uma mulher casada.

Ali estava seu irmão. E, como um choque elétrico que descia por seus
quadris, ele odiava aquele homem vestido de preto. Cumprimentou-o às
pressas com um olhar melancólico. Um carro já estava pronto. A viagem
começou matraqueante. Otto balbuciou uma pergunta, mas a lembrança de um
beijo o arrebatou.

De repente, nas escadarias do prédio, estava a criada, e ele desabou
quando ela pegou sua mala pesada. Ele ainda não vira sua mãe, mas o pai
estava vivo. Ali estava ele, sentado junto à janela, inchado, em sua
cadeira de braços\ldots{} Otto foi até ele e lhe deu a mão. ``Você não vai me
dar um beijo, Otto?'', perguntou o pai em voz baixa. O filho se jogou
sobre ele, correu para fora -- parou na sacada e berrou para a rua.
Ficou cansado de tanto chorar, e se lembrou, sonhando, da época em que
entrou no colégio, dos anos de comerciante, da viagem para os Estados
Unidos.

``Senhor Martin''. Ele estava tranquilo e agora se sentia envergonhado
por seu pai ainda estar vivo. Assim que soluçou mais uma vez, a criada
botou a mão sobre o seu ombro. Mecanicamente, ele viu: uma pessoa
saudável e loura, a refutação do doente que ele havia tocado. Ele se
sentiu em casa.

A biblioteca que Otto usou nas duas semanas de sua estadia ficava no
bairro mais movimentado da cidade. Todas as manhãs ele trabalhava três
horas num texto que deveria lhe conceder o título de Doutor em Economia.
À tarde, ele também ia até lá para estudar as revistas de arte
ilustradas. Ele amava a arte e lhe dedicava muito tempo. Naquelas salas,
não ficava sozinho. Se entendia muito bem com o digno funcionário que
lhe emprestava e depois recebia os livros que ele devolvia. Quando
levantava os olhos da obra que estava lendo, franzindo a testa e perdido
em pensamentos, não eram poucas as vezes em que encontrava uma cabeça
conhecida dos tempos do primário.

A solidão daqueles dias, que jamais era ociosa, lhe fazia bem, depois de
nas últimas semanas na Riviera pôr cada um dos seus nervos a serviço de
uma mulher sensual. À noite, na cama, ele procurava por detalhes do
corpo dela, ou então agradava-lhe enviar, em belas ondas, sua
sensualidade cansada até onde ela estava. Pensava nela raramente. Quando
se encontrava sentado diante de uma mulher no bonde, apenas distendia as
sobrancelhas de modo significativo, com expressão vazia, um gesto com o
qual implorava solidão inacessível em troca da doce inércia.

A azáfama em torno do moribundo era bem regular na casa, e nem sequer
lhe importava. Certa manhã, no entanto, o acordaram mais cedo do que de
costume e o levaram até diante do cadáver de seu pai. O quarto estava às
claras. Diante da cama, a mãe jazia desmaiada. O filho, porém, sentiu
tanta força que a agarrou por baixo dos braços e disse com voz firme:
``Levante-se, mamãe''. Nesse dia, ele foi para a biblioteca como sempre.
Seu olhar, quando passava pelas mulheres, estava ainda mais vazio e
firme do que de costume. Apertou a pasta, na qual havia dois maços com
as folhas de seu trabalho, junto ao corpo quando subiu à plataforma do
bonde.

De qualquer modo, ele trabalhou mais inseguro desde aquele dia. Percebeu
defeitos, problemas fundamentais que até então simplesmente ignorara
começaram a ocupá-lo. Quando encomendava livros, perdia de repente
qualquer medida e objetivo. Pilhas inteiras de revistas o envolviam, nas
quais buscava com um detalhismo estúpido os dados mais desimportantes.
Se interrompia a leitura, jamais era abandonado pela sensação
equivalente a de uma pessoa que usa roupas largas demais. Quando jogou
os torrões de terra na tumba de seu pai, vislumbrou o nexo entre o
discurso fúnebre, a sequência infinita de conhecidos e a própria falta
de ideias. ``Tudo isso já foi assim tantas vezes. Como isso é típico''.
E, quando saiu das proximidades do túmulo, se misturando à multidão em
luto, a dor de sua alma já se tornara como uma coisa que simplesmente se
carrega consigo por aí, e seu rosto parecia mais largo devido à
indiferença. As conversas em voz baixa entre a mãe e o irmão o
incomodavam quando eles estavam sentados à mesa a três. A criada loura
trazia a sopa. Despreocupadamente, Otto levantava a cabeça e olhava para
seus olhos castanhos, desamparados.

Assim Otto ainda conseguia, com uma certa frequência, embelezar para si
mesmo o medo mesquinho daqueles dias de luto. Certa vez beijou a criada
-- à noite -- no corredor. A mãe recebia sempre palavras calorosas
quando estava sozinha com ele; na maior parte das vezes, contudo, ela
discutia assuntos de negócios com o irmão mais velho.

Quando ele, num desses meios-dias, voltou da biblioteca, teve a ideia de
viajar. Pois o que ele ainda tinha a fazer por ali? Importava era
estudar.

Ele se encontrava sozinho em casa, e assim entrou no escritório de seu
pai como de hábito fazia. Ali, sobre o divã, o falecido passara suas
últimas horas de sofrimento. As cortinas haviam sido baixadas, porque
estava quente, e pelas frestas aparecia o céu. A criada veio e colocou
anêmonas sobre a escrivaninha. Otto estava apoiado junto ao divã e,
quando ela passou, puxou-a para si sem fazer ruído. Uma vez que ela
pressionava o corpo ao dele, eles se deitaram juntos. Depois de algum
tempo, ela o beijou e se levantou sem que ele a segurasse.

Ele viajou dois dias mais tarde. Deixou a casa bem cedo. Ao lado dele
seguia a criada com a mala, e Otto lhe contava da cidade universitária e
da faculdade. Mas, na despedida, ele apenas deu a mão a ela, pois a
estação ferroviária estava cheia. ``O que meu pai haveria de dizer?'',
ele pensou, enquanto se recostava e bocejava, afastando o derradeiro
sono de seu corpo.

\chapter{Palácio \versal{D}\ldots{}\versal{Y}\footnote[*]{``Palais \versal{D}\ldots{}\versal{y}'', in \versal{GS} \versal{IV}-2, pp.
  725-728. Tradução de Marcelo Backes. Conto publicado no periódico
  \emph{Die Dame}, em junho de 1929. [\versal{N}. \versal{E}.]} }

Nos anos de mil oitocentos e setenta e cinco a mil oitocentos e oitenta
e cinco, o Barão \versal{X} costumava chamar a atenção no Café de Paris, e,
quando se pedia aos estranhos de alguma distinção para atentarem ao
Conde de Caylus, ao marechal Fécamts, ao cavaleiro Raymond Grivier e
também a ele, o barão, não era por causa de sua elegância, sua origem,
suas conquistas esportivas, mas simplesmente por reconhecimento, sim,
por admiração à fidelidade que este havia dedicado ao estabelecimento
por tantos anos. Uma fidelidade que ele mais tarde demonstraria a alguém
bem diferente, e ademais bem pouco usual. Mas é disso, justamente, que
trata essa história.

Ela principia, mais precisamente, com a herança que durante trinta anos
sempre deveria ter sido dada, e aliás justamente dada, ao barão, e
finalmente também lhe foi dada em setembro de mil oitocentos e oitenta e
quatro. Na época, o herdeiro não estava muito distante de seu
quinquagésimo aniversário e há tempos já não era mais um \emph{bon}
\emph{vivant}. E por acaso o havia sido algum dia? Às vezes até se fazia
a pergunta. Se então alguém poderia afirmar que jamais dera de cara uma
única vez com o nome do barão na crônica escandalosa de Paris, e mesmo
na boca dos frequentadores de clube mais inescrupulosos e das cocotes
mais famigeradas jamais ouvira qualquer alusão a ele, não se poderia
duvidar de outro que dissesse: o barão em suas calças ajustadas, com a
gravata \emph{lavallière} larga era mais do que uma figura mundana; em
seu semblante havia algumas rugas que revelavam um conhecedor de
mulheres que pagou por sua sabedoria. De modo que, até aquele momento, o
barão permanecera sendo um mistério, e ver em suas mãos aquela herança
vultosa, esperada há tanto tempo, despertou em seus amigos, além de uma
benevolência desprovida de inveja, a curiosidade mais discreta e
maliciosa. O que nenhum papo junto à lareira, nenhuma garrafa de
Borgonha haviam conseguido -- erguer o véu que encobria aquela vida --,
eles acreditavam poder esperar da riqueza repentina.

Depois de dois ou três meses, no entanto, a opinião de todos era
unânime: a decepção não poderia ter sido mais completa. Nada, nem mesmo
uma sombra havia mudado nas vestes, no humor, na distribuição do tempo,
e até mesmo nos gastos e na moradia do barão. Ele continuava sendo o
indolente distinto, para o qual o tempo parecia tomado até a borda como
ao mais mesquinho dos amanuenses, ele continuava, ao sair do clube,
sendo recebido pela \emph{garçonnière} da Avenue Victor Hugo, e jamais
amigos que queriam acompanhá-lo até em casa à noite foram despedidos com
desculpas. Sim, parecia que o dono da casa mantinha a banca aberta até
às cinco horas da manhã, e ainda mais tempo se fosse em seu quarto de
visitas, no espaço onde outrora ficava um magnífico armário Chippendale
que desde sempre por lá estivera, sentado diante de uma mesa verde que
passou a ocupar o lugar. O barão costumava ter sorte no jogo -- isso se
sabia pelas raras vezes em que ele aparecera mais cedo para se sentar à
mesa verde. Mas agora nem mesmo os jogadores mais contumazes podiam
deixar de vivenciar as sequências de sorte que o inverno de mil
oitocentos e oitenta e quatro acabou lhe trazendo. Elas perduraram por
toda a primavera e assim continuaram quando o verão invadiu, com seus
lagos de sombra, os bulevares. Como foi que o barão se tornou um homem
pobre em setembro? Pobre não; mas exatamente tão flutuante, indefinível
entre pobre e rico como já era antes, e apenas mais pobre por já não ter
a expectativa de uma grande herança. Tanto que começou a se impor
limites, visitava o clube apenas para uma xícara de chá ou uma partida
de xadrez. E ninguém ousava fazer alguma pergunta. O que, ademais,
deveria parecer questionável em uma existência que decorria em seu
âmbito estreito e mundano diante dos olhos de todo mundo, da cavalgada
matinal, do exercício de florete e do almoço até a hora em que o sino
batia, às quinze para as seis, quando ele deixava o Café de Paris para
duas horas mais tarde jantar em boa companhia no Delaborde? No
intervalo, ele nem sequer tocava o baralho. E mesmo assim aquelas duas
horas do dia lhe custaram toda a fortuna.

Como isso aconteceu, soube-se em Paris apenas anos mais tarde, quando o
barão já havia se retirado sabe-se lá para onde -- o que o nome de uma
propriedade rural aristocrática localizada em terras lituanas distantes
acrescentaria aqui? --, e, em certa manhã chuvosa, um de seus amigos, em
meio às perambulações esquecidas da vida, estacou assustado, ele mesmo
não sabia por que motivo no primeiro momento: se devido a uma visão ou a
uma ideia. Na verdade, devido a ambas. Pois o monstro que descia
balançando diante dele, sobre os ombros de três transportadores, a
escada do Palácio \versal{D}\ldots{}\versal{y} era aquele valioso móvel Chippendale que certo
dia havia cedido lugar à mesa de jogo que tanta sorte lhe trouxera. O
armário era maravilhoso, e não podia ser confundido com nenhum outro.
Mas o amigo não o reconheceu apenas nisso. Balançando do mesmo jeito e
abalado na estrutura de seus ombros largos na época, à despedida, também
apareceram pela última vez e em seguida desapareceram as costas
formidáveis de seu proprietário diante dos que acenavam na plataforma da
estação. Às pressas, o desconhecido se acotovelou passando pelos
carregadores, e subindo os degraus baixos, entrou pela grande porta
aberta e ficou parado, quase sentindo vertigens, no gigantesco saguão
vazio. Diante dele se erguia, em espirais, uma escada que levava ao
primeiro andar, e sua rampa maciça não era mais do que um único relevo
em mármore sem fim: faunos, ninfas; ninfas, sátiros; sátiros, faunos. O
novato voltou a se controlar e investigou os corredores, as suítes dos
quartos. Por todo o lado, bocejavam paredes vazias ao encontro dele.
Nenhum rastro de moradores até o \emph{boudoir} também abandonado, mas
suntuoso, e tomado de peles e travesseiros, divindades em jade e
recipientes de incenso, vasos luxuosos e \emph{gobelins}. Uma leve
camada de pó encobria tudo. Aquela soleira nada tinha de convidativo, e
o estranho quis recomeçar a busca outra vez quando, por trás dele, uma
bela moça, ainda jovem, vestindo uniforme de aia, fez menção de adentrar
o ambiente. E ela, a única que tinha alguma familiaridade com o que ali
se passara, contou:

Fazia um ano que o barão havia alugado aquele palácio de seu
proprietário, um duque montenegrino, por uma soma inacreditavelmente
alta. Ainda no dia da assinatura do contrato, ela tivera de começar seus
afazeres, que por duas semanas consistiram em vigiar o pessoal de
serviço e receber entregadores. Em seguida, vieram novas instruções,
prescrições bem parcas, mas inflexíveis, cuja maior parte dizia respeito
ao cuidado com as flores que ainda haviam deixado um pouco de seu
perfume no quarto diante do qual os dois agora estavam parados. As
outras coisas diziam respeito apenas a uma das ordens, a última, e era
justamente esta que a moça acreditava estar vinculada ao pagamento
fabuloso, que lhe foi então prometido. ``Dia sim, dia não, nem um minuto
antes, nem um minuto depois das seis, aparecia'', prosseguiu ela, ``o
barão na escadaria, para subir até a grande porta vagarosamente. E
jamais ele vinha sem um buquê enorme nas mãos''. Mas qual era a
sequência assumida por orquídeas, lírios, azaleias, crisântemos, e em
que relação se encontravam com a época do ano, isso não havia ficado
claro. Ele tocava a campainha. A porta se abria. A aia, justamente ela,
de quem ficamos sabendo tudo isso, abria para receber as flores e a
pergunta que era a palavra-chave para seu serviço mais secreto:

``A honorável senhora se encontra em casa?''

``Lamento'', respondia-lhe a aia, ``a honorável senhora deixou a casa há
pouco.''

Pensativo, o amante principiava o caminho de volta logo depois, para no
dia seguinte continuar sua espera no palácio abandonado.

Assim ficou-se sabendo como a riqueza, que por tantas vezes serve ao
objetivo ordinário de fomentar fulgores amorosos alheios, nessa única
vez levou os de seu proprietário às derradeiras chamas.

\chapterspecial{``Inscrito na poeira movediça''}{Novela\footnote[*]{``Dem Staub,
  dem beweglichen, eingezeichnet. Novelle'', in \versal{GS} \versal{IV}-2, pp. 780-787.
  Tradução de Georg Otte. Conto escrito provavelmente em 1929. [\versal{N}.
  \versal{E}.]}\footnote[*]{Verso do ``Divã ocidental-oriental'', de Johann
  Wolfgang von Goethe. [\versal{N}. \versal{T}.]}}{}

Lá estava ele. Sempre estava lá nessa hora. Mas não dessa maneira. Esse
homem imóvel, que costumava manter o olhar fixo ao longe, hoje estava
olhando para baixo. E mesmo assim, não parecia fazer diferença, pois,
nesse último caso, ele também não estava vendo nada. Mas a bengala de
mogno com seu punho de prata não se encontrava, como de costume, ao lado
dele encostada no banco. Ele a segurava, a conduzia, deixando que
deslizasse sobre a areia: ``\versal{O}'' -- pensei numa fruta; ``\versal{L}'' -- parei;
``\versal{I}'' -- fiquei envergonhado como ao fazer algo proibido. Vi que não
escrevia isso como alguém que quisesse ser lido, mas os signos se
entrelaçavam e, como quisessem incorporar-se um ao outro, seguiram,
quase se sobrepondo aos outros, ``\versal{MPIA}'', sendo que o primeiro começou a
desaparecer quando os últimos surgiram. Aproximei-me; isso tampouco o
fez levantar o olhar -- ou deveria dizer: ``acordar''? --, de tão
acostumado que ele estava comigo.

--- Fazendo cálculos de novo? Perguntei, fazendo-me de desentendido. Pois
eu sabia que seu ócio estava totalmente voltado para os custos
fantásticos de longas viagens que se estendiam de Samarcanda à Islândia,
que ele nunca faria. Será que ele jamais tinha deixado o país -- a não
ser naquela viagem secreta, claro, que tinha feito para fugir à
lembrança de um amor de juventude, um amor selvagem e, como não se
cansavam de assegurar, indigno e vergonhoso por Olímpia, cujo nome
acabou de desenhar em seu devaneio.

--- Estou pensando na minha rua. Ou em você, se prefere, pois acaba dando
na mesma. A rua em que uma palavra sua se tornou tão viva quanto nenhuma
outra que ouvi desde então, ou antes. Trata-se daquilo que você me falou
uma vez em Travemünde, isto é: que cada aventura de viagem, para se
poder mesmo contá-la, deve girar, em última instância, em torno de uma
mulher, ou, pelo menos, em torno do nome de uma mulher. Pois, querendo
ou não, esse seria o ponto de partida do qual careceria o fio condutor
do vivido para poder passar de uma mão à outra. Você estava certo, mas
quando estava subindo aquela rua calorenta, não tive ainda como imaginar
de que maneira estranha e por que, depois de alguns segundos, os meus
próprios passos nessa rua, que ecoava abandono, pareciam me chamar como
uma voz. As casas em volta tinham pouco a ver com aquelas que fizeram a
fama dessa cidadezinha do sul da Itália. Sem ser suficientemente velha
para ser decaída, nem suficientemente nova para ser convidativa, era um
conjunto dos caprichos do limbo da arquitetura. Venezianas fechadas
reforçavam a mudez das fachadas cinzentas, e parecia que a glória do Sul
havia se retirado totalmente para as sombras, que se acumulavam por
debaixo das escoras anti-terremoto e dos arcos das ruelas laterais. Cada
passo me afastava mais de tudo que tinha sido o motivo da minha visita;
deixei a pinacoteca e a catedral para trás. Dificilmente, eu teria
encontrado força para mudar o rumo, mesmo se não tivesse sido matéria
para novos devaneios a visão de braços vermelhos de madeira, uma espécie
de suporte de candelabro que, como só agora percebo, cresciam dos muros
em ambos os lados a intervalos regulares. Digo ``matéria para
devaneios'' justamente porque não conseguia entender, nem procurava
explicar, como os restos de uma iluminação tão arcaica podiam ter
sobrevivido numa cidade nas montanhas, que, apesar de tudo, tinha
canalização e eletricidade. Por isso, também não me surpreendi quando,
alguns passos à frente, encontrei echarpes, cortinas, xales ou
passadeiras que, pelo visto, tinham acabado de lavar. Algumas lanternas
com papel amassado na frente de vidros turvos completavam a imagem,
nestas casas, de pobreza e administração decadente. Nesse momento,
gostaria de perguntar a alguém como voltar para o centro por outra via,
pois estava cansado dessa rua, também por estar tão vazia de gente. Foi
justamente por isso que tive que desistir da minha intenção e, quase
humilhado e subjugado, tive que retornar pelo mesmo caminho. Decidido a
não ter que arcar com o prejuízo do tempo perdido, mas também para pagar
por aquilo que considerava como uma derrota, abri mão do almoço e, o que
era mais amargo, do repouso, de modo que, depois de uma breve escalada
por escadas íngremes, encontrei-me na praça da catedral.

Se aquilo que me cercava, pouco tempo antes, era a ausência angustiante
de pessoas, agora era uma solidão que fez com que me sentisse livre. E,
com isso, o meu humor mudou por completo. Nessa hora, nada teria sido
pior para mim do que ser abordado ou apenas ser notado. De uma só vez,
fui devolvido ao meu destino de viajante, à aventura solitária, e de
novo vi na minha frente o momento quando, acima da Marina Grande, não
muito longe de Ravello, dei-me conta desse destino pela primeira vez e
de forma dolorosa. Desta vez também havia uma montanha em volta, mas no
lugar dos despenhadeiros rochosos que levam de Ravello ao mar, havia o
flanco de mármore da catedral e, no lugar dos declives cheios de neve,
inúmeros santos de pedra pareciam fazer sua romaria até nós homens.
Quando segui o cortejo com os olhos, vi que a fundação do edifício
estava à vista. Haviam cavado um corredor que, depois de vários degraus,
levava perpendicularmente a uma porta de bronze debaixo da terra, que
não estava trancada. Não sei por que passei clandestinamente por essa
porta lateral subterrânea; talvez fosse apenas o medo que nos acomete
quando visitamos pessoalmente um dos sítios mil vezes reproduzidos e
descritos, e que tentei evitar por esse desvio. Entretanto, se havia
achado que entraria no escuro de uma cripta, fui amplamente castigado
pelo meu esnobismo. Como se não bastasse o fato de ser esse espaço, na
verdade, a sacristia pintada de branco, cuja iluminação pelas janelas
superiores era ofuscante, além de tudo, era ocupado por um grupo de
turistas para o qual o sacristão estava contando, pela centésima ou
milésima vez, uma daquelas histórias em que ressoa o eco das moedas de
cobre que iria receber pela centésima ou milésima vez. Lá estava ele,
imponente e corpulento, ao lado do pedestal no qual se concentrava a
atenção dos ouvintes. Um capitel pelo visto muito antigo, porém muito
bem conservado, no estilo do gótico antigo, estava fixado nele com
grampos de ferro. Enquanto falava, segurava um lenço. Tudo levava a
supor que fosse por causa do calor, sendo que, de fato, o suor escorria
pela sua testa. Mas, longe de se enxugar, o sacristão apenas o passava
distraidamente no bloco de pedra, como uma empregada doméstica que, em
meio a uma conversa constrangedora com seu senhorio, entre uma palavra e
outra, passa, seguindo um velho hábito, o pano de limpeza pela estante
ou pelo console. A constituição autodestrutiva que cada pessoa que viaja
sozinha já sentiu voltou a tomar conta de mim, de maneira que deixei
chegar aos meus ouvidos as explicações do sacristão.

``Há dois anos'', esse era o teor não exatamente literal das suas
informações, passadas pausadamente, ``havia entre os moradores um homem
que, mediante as formas mais ridículas de blasfêmia e lascívia, fez com
que a cidade fosse assunto de todas as conversas. Pelo resto de sua
vida, ele pagou pelo seu deslize e fazendo penitência ainda quando o
prejudicado, isto é, Deus, talvez o tivesse perdoado há muito tempo. Ele
era canteiro e, depois de trabalhar durante dez anos na manutenção da
catedral, graças ao seu talento, foi promovido a diretor de todo o
projeto de restauração. Era um homem na melhor idade, de natureza
imperiosa, sem família, nem parentes, quando caiu na rede da cocote mais
bonita e despudorada jamais vista na boemia do balneário vizinho. Talvez
a natureza terna e fechada desse homem tenha impressionado a mulher. De
qualquer forma, não se tem notícia de que ela tivesse prestado seus
favores a outra pessoa na região. Na época, ninguém desconfiava do preço
que ele teria que pagar, e toda a história nem teria vindo à tona se o
departamento de fiscalização da construção de Roma não tivesse vindo
inesperadamente para inspecionar a famosa obra de restauração. Entre os
superiores, havia um jovem arqueólogo, petulante, porém com bons
conhecimentos, que havia se especializado no estudo dos capitéis do
século~\versal{XIV}. Ele estava prestes a ampliar seus planos de escrever um
livro monumental a respeito de um estudo sobre `Um capitel do púlpito na
catedral de \versal{V}\ldots{}' e havia anunciado sua visita ao nosso chefe da
\emph{opera del duomo}. O mesmo, mais de dez anos depois de suas
melhores noites, levava uma vida na mais completa solidão, sendo que a
época do brilho e da autoafirmação havia acabado há muito tempo. No
entanto, o resultado que o jovem pesquisador levou para casa desse
encontro foi tudo menos um ensinamento da história do estilo, e sim uma
informação que não guardou para si. As autoridades acabaram tendo
conhecimento dos seguintes fatos: O amor que a cocote havia dedicado ao
seu galã não era obstáculo para ela, talvez antes um estímulo, para
exigir um preço satânico em troca dos seus favores. Ela queria ver seu
\emph{nom de guerre}, o nome comercial que essas mulheres usam de acordo
com a mais antiga tradição, gravado numa pedra da catedral, o mais
próximo do Santíssimo. O amante resistia, mas suas forças se esgotaram
e, num belo dia, ele começou, na presença da própria prostituta, o
trabalho naquele capitel do gótico antigo, que ficava por debaixo de um
capitel mais velho e corroído, até acabar como corpo de delito na mesa
dos seus juízes clericais. Todo esse processo, entretanto, levou muitos
anos e, quando todas as formalidades foram cumpridas e todas as atas
reunidas, era tarde demais. Quem estava olhando para sua obra era um
velho frágil e meio demente e ninguém acreditava em dissimulação quando
olhava para aquela cabeça que, em outros tempos, impunha respeito. Agora
ficava inclinada, com a testa enrugada, para o emaranhado de arabescos,
tentando em vão depreender dele o nome que, anos incontáveis antes, nele
havia escondido.''
%não entendo o sentido da frase 'ele pagou pelo seu deslize e fazendo penitência', linha 2539. Talvez 'pagou pelo seu deslize fazendo penitência' ou 'pagou pelo seu deslize e fez penitência' ficasse mais compreensível, embora mude o sentido (que não sei qual é)

Com surpresa observei como, eu mesmo não sei por quê, aproximei-me do
capitel. Mas antes de poder estender a minha mão em direção à pedra,
senti a do sacristão no meu ombro. Com boa vontade e uma certa surpresa,
ele tentou entender os motivos do meu interesse. Eu, todavia, na minha
insegurança e no meu cansaço, balbuciei a coisa mais sem sentido que
pudesse ter dito: ``Colecionador''. E, nesse estado, voltei para casa.

Se o sono, como dizem, não é apenas uma necessidade física do organismo,
mas também uma força compulsiva que o inconsciente exerce no consciente
para que este saia de cena, cedendo o lugar às pulsões e às imagens, o
esgotamento que me assaltou talvez tivesse significado mais do que
normalmente significaria numa cidade nas montanhas do Sul da Itália ao
meio-dia. Seja como for, eu sonhei, sei que sonhei com o nome. Mas não
da maneira que ele estava gravado na pedra, escondido de mim, mas
conduzido para outro reino, ao mesmo tempo enaltecido, desencantado e
mais nítido, e no emaranhado variado de capins, folhas e flores, as
letras que, na época, causavam as batidas mais dolorosas do meu coração,
abanavam e tremiam na minha direção. Quando acordei, eram oito horas.
Hora de jantar e de me perguntar como iria passar o resto da noite. A
minha sesta de várias horas me proibia de terminar o dia cedo e não
tinha dinheiro, nem disposição, para gastá-lo com qualquer aventura.
Depois de dar alguns passos sem rumo, cheguei numa praça livre, o
\emph{Campo}. Já estava escurecendo. Algumas crianças ainda estavam
brincando em torno de uma fonte. Esse lugar, proibido para todos os
veículos, onde nunca mais haveria reuniões, apenas feiras, desempenhava
seu papel vivo como grande praça de banho e de jogos das crianças. Por
isso, ela era ao mesmo tempo o local preferido para carrinhos com doces,
nozes ou melancias, sendo que dois ou três deles estavam presentes,
acendendo aos poucos suas tochas. Um brilho destacou-se nas proximidades
do último, que havia atraído alguns ociosos e crianças. Ao chegar mais
perto, reconheci instrumentos de sopro. Sou um \emph{flâneur} atento.
Qual vontade ou qual desejo proibido haviam me impedido de reparar
naquilo que não escapava nem ao menos atento. Alguma coisa estava
acontecendo nessa rua, em cujo fim me encontrei novamente, sem
suspeitá-lo. Ao contrário do que achava, as passadeiras de seda que
estavam penduradas nas janelas não eram roupas estendidas para secar, e
por que logo aqui e em nenhum outro lugar do país a velha iluminação
teria sido mantida? A banda de música começou a se movimentar, entrando
na rua que em pouco tempo se encheu de pessoas. E agora ficou claro que
a riqueza, quando chega perto dos pobres, apenas dificulta a fruição
daquilo que é seu: a luz das velas e o fogo das tochas travavam uma luta
feroz contra os feixes amarelos das lâmpadas elétricas que se projetavam
no pavimento e nas paredes das casas. Fui o último a sair atrás da
banda. Haviam preparado tudo para receber o cortejo na frente de uma
igreja. Aqui os lampiões e as lâmpadas incandescentes estavam mais
próximos do que em qualquer outro lugar, e da multidão em festa
desprendeu-se o fluxo ininterrupto dos devotos para perder-se nas dobras
da cortina que escondia a abertura do portão.

Parei a uma certa distância desse centro iluminado de vermelho e verde.
A multidão que agora preenchia a rua por completo não era uma massa
incolor. Era a população bem delimitada e estreitamente interligada do
bairro e, uma vez que era um bairro da pequena burguesia, não se via
pessoas das classes mais altas, muito menos estrangeiros. Da maneira em
que fiquei parado junto ao muro, eu, pela roupa e pelo aspecto,
normalmente deveria ter chamado a atenção das pessoas. Mas nessa
multidão, curiosamente, ninguém olhava para mim. Será que ninguém
reparava ou será que este homem totalmente perdido nessa rua tomada pelo
calor e pelo canto, em que havia me transformado cada vez mais, parecia
pertencer a eles? Ao pensar nisso, enchi-me de orgulho; uma grande
felicidade tomou conta de mim. Não entrei na igreja e queria, satisfeito
de ter desfrutado a parte profana da festa, tomar o rumo de casa junto
com os primeiros saciados e muito antes das crianças que iriam cair de
sono, quando meu olhar esbarrou numa das placas de mármore com que as
cidades pobres dessa região envergonham as placas de rua do mundo
restante. A luz das tochas a inundava, parecia pegar fogo. Mas, bem
delineadas e ardentes, as letras saltaram do seu centro, formando
novamente o nome que, da pedra transformada em flor, da flor
transformada em fogo, procurava me pegar de forma cada vez mais quente e
devoradora. Tomando a decisão irrevogável de voltar para casa, iniciei o
meu retorno e fiquei feliz de encontrar uma ruela pequena que prometia
ser um atalho considerável. Em todas as partes, a vida já estava se
retirando e a rua principal, que pouco tempo atrás ainda estava cheia de
vida e onde tinha que estar o meu hotel, não me parecia ser apenas mais
sossegada, mas também mais estreita. Enquanto ainda refletia sobre as
leis que associam imagens acústicas e óticas, uma música de %"ao"?
longe e alta
chocou-se contra o meu ouvido e, com os primeiros toques, fui atingido
pelo relâmpago da iluminação: aqui, então, acontece o grande evento. Por
isso havia tão poucas pessoas, cidadãos, naquela rua. Aqui iria
acontecer o grande concerto da tarde de \versal{V}\ldots{}, onde todo sábado os
moradores se reúnem. Uma nova cidade ampliada, até com uma história mais
rica e movimentada, estava de repente na frente dos meus olhos. Redobrei
os meus passos, virei uma esquina e, mais uma vez, parei, imobilizado de
estupor, na frente daquela rua que havia me atraído violentamente, como
puxado por um laço, eu, como ouvinte atrasado e solitário, para quem a
banda apresentou sua última música, a mais perdida de todas.''

Aqui o meu amigo interrompeu. De repente, parecia que sua história havia
fugido. E apenas os lábios que, ainda há pouco, estavam falando, me
acenaram com um longo sorriso. Eu, no entanto, olhava para os signos
que, diluídos na poeira, estavam inscritos aos nossos pés. E o indelével
verso passou majestosamente pelo arco dessa história como por um portão.

\chapterspecial{O segundo eu}{Uma história de final de ano para refletir\,\footnote[*]{``Das
  zweite ich. Eine Sylvestergeschichte zum Nachdenken'', in \versal{GS} \versal{VII}-1,
  pp. 296-298. Tradução de Marcelo Backes. Conto escrito entre 1930 e o
  início de 1933, aproximadamente; publicação póstuma. [\versal{N}. \versal{E}.]}}{}

Krambacher é um funcionário de bem baixo nível e além disso um homem
``sem reboques'', conforme ele garante às locadoras de seus quartos
mobiliados, que troca a cada quatro a seis semanas. Durante semanas,
ele refletiu onde poderia passar a noite de final de ano. Mas todos os
arranjos acabaram se desfazendo; com seu último dinheiro, ele arrumou
duas garrafas de ponche. A partir das 9 horas, principiou um banquete
solitário, sempre na esperança de que a campainha irá tocar, de que
alguém irá procurá-lo e lhe fazer companhia.

A esperança é desiludida. Pouco antes das 11, ele se prepara para sair:
está com medo da solidão em sua biboca. Nós seguimos seu passo um tanto
sinistramente animado pelas ruas noturnas. Percebe-se nele que bebeu.
Talvez ele nem sequer ande, talvez apenas sonhe que esteja andando. Essa
suposição pode surgir fugidiamente no leitor.

Krambacher vem por uma ruela bem fora de mão. Uma lâmpada sombria chama
sua atenção. Um lugar dúbio com movimento na noite de final de ano? Mas
por que tão silencioso? Ele se aproxima, não há rastro de que se trate
de um estabelecimento: com letras de madeira, apagadas, está escrito
sobre uma vitrine branqueada e sem transparência, da qual vem a luz
leitosa: \versal{PANORAMA IMPERIAL}.

Ele quer passar sem parar, mas um bilhete sujo na vitrine o faz estacar:
Hoje! Apresentação de gala! \emph{Viagem pelo ano velho!} Krambacher
fica parado, abre a porta timidamente, toma coragem, uma vez que não
encontra ninguém, e entra. Ali está o panorama imperial. Agora ele é
descrito com suas 32 cadeiras em roda. Sobre uma dessas cadeiras o
proprietário, um italiano viúvo, Geronimo Cafarotti, dormindo. Quando o
cliente se aproxima, ele se levanta de um salto.

Grande torrente discursiva. De suas palavras, se pode ouvir que noite a
noite a casa ficava lotada e sem mais vagas; que hoje, coincidentemente,
estava pouco visitada, apesar da programação de gala; mas ele sabia que
alguém acabaria por vir: a pessoa certa. Enquanto obriga o visitante a
se sentar sobre uma banqueta diante de dois buracos pelos quais poderia
espiar, ele explica:

Aqui o senhor conhecerá alguém bem estranho, verá um homem que não tem
nenhuma semelhança com o senhor: seu segundo eu\ldots{} O senhor passou a
noite fazendo autoacusações, tem complexos de inferioridade, se sente
tolhido, faz censuras a si mesmo por não seguir seus impulsos. Pois bem,
o que são esses impulsos? É a pressão do segundo eu no trinco da porta
que conduz para dentro de sua vida. E agora o senhor haverá de
reconhecer por que foi que sempre manteve esta porta tão trancada,
porque %"por que"? Me parece que a segunda parte do período não responde a primeira, mas a complementa. As duas leituras são possíveis, porém
se tolheu tanto e não seguiu seus impulsos.

A viagem pelo ano velho começa. Doze imagens, para cada uma delas uma
pequena legenda; além disso as explicações do velho, que escorrega de
uma cadeira para outra. As imagens:

\begin{quote}
\quad\,
\emph{O caminho que pretendias tomar }

\emph{A carta que pretendias escrever }

\emph{O homem que pretendias salvar }

\emph{O lugar que pretendias ocupar }

\emph{A mulher que pretendias seguir }

\emph{A palavra que pretendias ouvir }

\emph{A porta que pretendias abrir }

\emph{O traje que pretendias usar }

\emph{A pergunta que pretendias fazer }

\emph{O quarto de hotel que pretendias ter }

\emph{O livro que pretendias ler }

\emph{A oportunidade que pretendias aproveitar }
\end{quote}

Em algumas das imagens, o segundo eu pode ser visto, em outras apenas as
situações nas quais ele pretendia enredar o primeiro. As imagens são
descritas, como elas se livram do lugar em que estão com um pequeno
tilintar para permitir que a seguinte se aproxime, e como elas, mal
voltaram a se aquietar, tremendo, dão lugar a uma nova. O último tinido
é suplantado pelo reboar dos sinos do ano novo. Krambacher desperta com
o copo de ponche vazio nas mãos, sentado em sua cadeira.

\chapter{Rastelli conta\ldots{}\footnote[*]{``Rastelli erzählt\ldots{}'', in \versal{GS} \versal{IV}-2, pp.
  777-780. Tradução de Marcelo Backes. Conto escrito provavelmente entre
  setembro e outubro de 1935, publicado no \emph{Neue Zürcher Zeitung}
  em 6 de novembro de 1935. [\versal{N}. \versal{E}.]} }
Ouvi essa história de Rastelli, o malabarista inigualável, inesquecível,
que a contou certa noite em seu camarim.

Era uma vez, principiou ele, nos velhos tempos, um malabarista. Sua fama
se espraiara pelo globo terrestre, levada pelas caravanas e pelos navios
mercantes, e certo dia também Mohammed Ali Bei, que na época imperava
sobre os turcos, ouviu falar dele. E eis que enviou seus mensageiros aos
quatro cantos do mundo com a missão de convidar o mestre a vir a
Constantinopla, a fim de que ele pudesse se convencer em sua própria e
imperial pessoa das habilidades artísticas do homem. Mohammed Ali Bei
teria sido um príncipe imperioso e até mesmo cruel de quando em vez, e
dele se contava, inclusive, que, a seu aceno, um cantor que buscara seus
ouvidos mas não encontrara seus aplausos, havia sido jogado ao mais
profundo dos cárceres. Mas também sua generosidade era conhecida, e um
artista que o satisfazia podia contar com uma grande recompensa.

Depois de alguns meses, o mestre chegou à cidade de Constantinopla. Ele
não chegou sozinho, no entanto, ainda que não anunciasse seu
acompanhante em altos brados. E isso muito embora pudesse alcançar
honras especiais com ele na corte do sultão. Qualquer um sabe que os
déspotas do oriente têm um fraco por anões. O acompanhante do mestre era
justamente um anão, ou, mais exatamente, um criado anão. E um tão
excepcionalmente suave, uma criaturinha tão delicada e rápida, que com
certeza não teria encontrado outra igual na corte do sultão. O mestre
manteve esse anão escondido, e tinha seu bom motivo para tanto. É que
ele trabalhava de um modo um pouco diferente do de seus colegas. Estes,
conforme se sabe, frequentaram a escola chinesa, e lá aprenderam a lidar
com bastões e pratos, com espadas e fogos. Mas nosso mestre não buscava
sua honra no número e na variedade dos requisitos, e sim a mantinha com
um único número, que ainda por cima era o mais simples e que se
destacava pura e exclusivamente por sua grandiosidade incomum. E esse
número era com uma bola, uma única bola. Essa bola lhe trouxera sua fama
mundial, e de fato não havia nada que se comparasse com os milagres que
fazia com ela. Àqueles que acompanhavam a brincadeira do mestre chegava
a parecer que ele estava lidando com um ser vivo, ora dócil ora
renitente, ora suave ora zombeteiro, ora atento ora distraído, mas
jamais com uma coisa morta. Os dois pareciam habituados um ao outro, e
sequer pareciam conseguir viver separados tanto em tempos bons quanto em
tempos difíceis. E ninguém conhecia o segredo da bola. O anão, esse elfo
flexível, ficava sentado dentro dela. Depois de muitos anos de
exercício, ele soubera se adequar a cada um dos impulsos e a cada um dos
movimentos de seu senhor, e agora brincava com as molas localizadas no
interior da bola com tanta desenvoltura como se estivesse tocando as
cordas de uma viola. Para fugir a qualquer suspeita, os dois jamais se
deixavam ver um ao lado do outro, e senhor e ajudante também nunca
moravam sob o mesmo teto em suas viagens.

O dia ordenado pelo sultão chegara. Um estrado envolvido por cortinas
havia sido instalado na sala da meia-lua, lotada pelos dignitários do
soberano. O mestre fez uma mesura diante do trono e levou uma flauta aos
lábios. Depois de algumas melodias de prelúdio, ele passou a um
\emph{stacatto} em cujo ritmo a grande bola se aproximou aos saltos,
vinda dos bastidores. De repente, ela havia tomado lugar sobre os ombros
de seu dono, para em seguida não mais sair de perto dele. Ela brincava
em torno de seu senhor, adulava-o, acariciava-o. Este, porém, deixara
sua flauta de lado e, como se nada soubesse a respeito de seu visitante
peculiar, começara uma dança lenta que teria sido um prazer acompanhar
se a bola não tivesse cativado os olhos de todos. Assim como a Terra
gira em torno do Sol e ao mesmo tempo em torno de si mesma, também a
bola girava em torno do dançarino, sem esquecer nisso de sua própria
dança. Da cabeça aos pés, não havia lugar que a bola não tocasse, e cada
um desses lugares se tornava seu próprio parque de diversões ao passar
voando. Ninguém pensaria em pedir música para aquela ciranda muda. E
isso porque os dois davam as deixas um ao outro da maneira mais
harmônica: o mestre à bola e a bola ao mestre, conforme o pequeno
ajudante escondido já dominava com precisão depois de tantos anos de
exercício.

E assim continuou por muito tempo, até que em dado momento, em um giro
do dançarino, a bola, impulsionada para longe ao mesmo tempo, rolou ao
encontro da rampa na qual bateu e junto à qual ficou saltitando,
enquanto o mestre se compunha. Pois agora se aproximava o grande final.
O mestre voltou a pegar a flauta. Primeiro pareceu que ele quisesse
acompanhar com música baixa e cada vez mais baixa os saltos de sua bola,
que ficavam cada vez mais fracos. Mas então a flauta se fez dona da
situação. A respiração daquele que soprava se fez mais forte e, como se
soprasse nova vida à sua bola com o novo e mais vigoroso modo de tocar,
os saltos da mesma foram ficando aos poucos mais e mais altos, enquanto
o mestre começou a levantar o braço para, depois de alcançar
relaxadamente a altura do ombro, esticar o dedo mindinho sempre a tocar,
ao que a bola, obedecendo a um último e longo trinado, com um único
salto, se pôs imóvel no chão.

Sussurros de admiração percorreram as fileiras do público, e o sultão
convidou, ele mesmo, ao aplauso. O mestre, porém, concedeu uma
derradeira prova de sua arte ao aparar em pleno voo o saco pesado, cheio
de ducados, que lhe foi lançado por ordens vindas de cima.

Pouco depois ele saiu do palácio, para esperar por seu fiel anão em uma
saída distante. Foi então que um mensageiro apareceu diante dele, se
acotovelando entre os guardas. ``Procurei o senhor por toda a parte'',
disse ele, dirigindo-se ao mestre. ``Mas o senhor acabou deixando seu
alojamento antes da hora, e não permitiram que eu entrasse no palácio.''
Com essas palavras, ele apresentou uma carta que trazia a letra do anão.
``Meu caro mestre, peço que não se irrite comigo'', estava escrito nela.
``Mas hoje o senhor não poderá se mostrar diante do sultão. Estou doente
e não conseguirei deixar meu leito.''

Conforme o senhor pode ver, acrescentou Rastelli depois de uma pausa,
nossa casta não é de ontem e nós também temos nossa história -- ou pelo
menos nossas histórias\ldots{}

\chapter{Por que o elefante se chama ``elefante''\,\footnote[*]{``Warum der Elefant
  `Elefant' heißt'', in \versal{GS} \versal{VII}-1, pp. 298-299. Tradução de Marcelo
  Backes. Conto não publicado durante a vida do autor. Provavelmente
  escrito em setembro de 1933, quando Benjamin retornava a Paris depois
  de uma temporada em Ibiza. [\versal{N}. \versal{E}.]} }

Era uma vez. Vivia por aí um homem que se chamava Elefante; mas na época
ainda nem sequer se conhecia o elefante como ele é hoje, isso foi há
vários milhares de anos. E, de repente -- todas as pessoas se admiraram
muito --, apareceu um animal por aí que nem sequer nome tinha, e o homem
o viu e, uma vez que tinha um nariz curto e era assim meio parecido com
um homem, o levou consigo e o animal ficou com ele.

E o animal estava com ele. Ele pegou um pedaço de madeira, não muito
longo, mas pesado, e o jogou para que o animal fosse buscá-lo. E uma vez
que o animal não tinha mãos com as quais pudesse pegar o pedaço de
madeira, tentou pegá-lo com o nariz.

Mas o nariz era curto demais, e isso deu muito trabalho ao animal. E uma
vez que o tentou por diversas vezes, repetindo o gesto seguidamente -- e
isso demorou um bocado! --, o nariz foi ficando cada vez mais longo e
mais longo com as tentativas.

Isso do nome acontecera já antes, quando o nariz ainda era curto. Pois
uma vez que o animal estava com o homem que se chamava Elefante, as
pessoas também o chamaram de elefante.

E agora o nariz já estava tão comprido que ele podia pegar o pedaço de
madeira com facilidade. E tudo ia bem, e o nariz ficava cada vez maior.
E hoje ele é tão grande e grosso e tem esse nariz-mão comprido -- sim,
este é justamente o nosso elefante. E esta é a história.

\chapter{Como o barco foi inventado \\e por que ele se chama ``barco''\,\footnote[*]{``Wie
  das Boot erfunden wurde und warum es Boot heißt'', in \versal{GS} \versal{VII}-1, p.
  299. Tradução de Marcelo Backes. Assim como ``Por que o elefante se
      chama `elefante'\,'', este conto foi redigido provavelmente em setembro
  de 1933, quando Benjamin retornava de Ibiza à Paris. Não conheceu
  publicação em tempo de vida do autor. [\versal{N}. \versal{E}.]} }

Antes de todos os outros humanos vivia um que se chamava Barco. Ele foi
o primeiro humano, pois antes dele existia apenas o anjo, que havia se
rebaixado e se transformado em um humano; e esta é uma outra história.

O homem Barco quis, pois, ir para a água -- na época havia bem mais água
do que hoje, disso você precisa saber. Então ele atou tábuas ao redor de
seu corpo usando cordas, uma tábua comprida debaixo da barriga, e isso
era a quilha. E tomou um gorro pontudo de tábuas que ficava, quando ele
estava deitado na água, na parte da frente -- e isso se tornou a proa. E
atrás ele esticou uma perna e guiou com ela.

Assim ele se deitou na água e guiou e remou com os braços e traçou sua
rota com o gorro de tábuas, uma vez que era pontudo, com toda a
facilidade através da água. Sim, foi assim; o homem Barco, o primeiro
homem, havia feito um barco de si mesmo, com o qual se podia navegar.

E por isso -- não é verdade? O que me parece até bem claro --, porque
ele próprio era o barco, chamou aquilo que estava fazendo de ``barco''.
E por isso o barco se chama ``barco''.

\chapter{Uma história estranha, \\de quando ainda não havia humanos\,\footnote[*]{``Eine komische Geschichte, als es noch keine
  Menschen gab'', in \versal{GS} \versal{VII}-1, p. 300. Tradução de Marcelo Backes. Como
  os dois contos imediatamente precedentes, este também teve publicação
  póstuma, e foi escrito provavelmente em 1933, no mês de setembro,
  quando Benjamin regressava a Paris desde Ibiza. [\versal{N}. \versal{E}.]} }

Na época a Terra ainda não era firme e tudo era um pântano, como massa
molhada. Existia apenas uma única árvore, que era gigantesca e sabia
correr -- é que as primeiras árvores sabiam correr como animais. A
árvore gigantesca saiu a passear, e de repente começou a correr,
justamente à beira do pântano mais profundo, e caiu com um catrapus
formidável dentro da água.

E no mesmo instante tudo ficou firme, a massa ficou bem dura, e por toda
parte na terra havia pedras em torrões e paus, de modo que o homem --
que ainda não existia -- não teria podido caminhar simplesmente, porque
teria se machucado demais.

Então o anjo se transformou, pela primeira vez rebaixando-se, e tinha
asas de ferro e examinou a terra. E então o Deus mais uma vez borrifou
muita coisa molhada sobre a terra, de modo que tudo se tornou pântano e
oceano e mar de novo.

Mas tudo secou ao sol e então passou a estar liso em vários lugares. Mas
agora também havia montanhas -- porque o grande borrifar havia lavado a
areia e aberto sulcos e dobraduras --, ou seja, montanhas. Quando eu
borrifo, o resultado são apenas pequenos sulcos e mares, quando Deus
borrifa, aparecem montanhas.

E o anjo, que então caminhava ali por baixo, deixou suas asas derreterem
e logo elas sumiram, e o anjo era como um humano. Mas continuava
havendo torrões sobre a terra -- uma gororoba de ovo, tudo colava.

Disso se fizeram os humanos -- e por primeiro o senhor que se chamava
barco. Eles se fizeram -- simplesmente passaram a ser, e o anjo, que
também se tornara humano, precisava apenas ficar olhando. Eles se
fizeram conforme seu aspecto.

Então os homens construíram molhes e botaram muitos monumentos e humanos
de ferro de asas bem abertas sobre eles. Mas isso foi bem mais tarde,
pouco tempo antes de inventarem as lâmpadas.

\chapterspecial{Myslowitz --- Braunschweig --- Marselha}{História de uma embriaguez com haxixe\,\footnote[*]{``Myslowitz
  - Braunschweig -- Marseille. Die Geschichte eines
  Haschisch-Rausches'', in \versal{GS} \versal{IV}-2, pp. 729-737. Tradução de Georg Otte.
  No final dos anos 20, Benjamin realizou, com a ajuda de um amigo
  médico, uma série de experiências de absorção de haxixe. As
  associações e fantasias que surgiam sob o efeito da droga eram
  anotadas em ``protocolos'' que foram, por vezes, utilizados como
  material para a redação de textos teóricos ou ficcionais. Este conto
  está relacionado ao protocolo de 29 de setembro de 1928, que também
  serviu para a redação de \emph{Haxixe em Marselha}. Publicado na
  revista \emph{\versal{UHU}}, n.° 7, caderno 2, em novembro de 1930. [\versal{N}. \versal{E}.]}}{}

Esta história não é minha. Não quero me deter na questão de saber se o
pintor Eduard Scherlinger, que vi pela primeira e última vez naquela
noite, quando a contou, era um grande contador de histórias ou não,
porque, nesta época dos plágios, sempre há alguns ouvintes que nos
atribuem uma história mesmo quando explicamos que ela apenas foi
relatada de forma fidedigna. Mas eu a escutei numa noite em um dos
poucos lugares que há em Berlim para contar e escutar histórias, na casa
Lutter \& Wegener. Era agradável sentar em torno da mesa no nosso
pequeno grupo, mas as conversas já haviam se dispersado e só ressurgiam
de modo escasso e abafado entre duas ou três pessoas, sem serem notadas
pelos demais.

De repente, numa situação cujas circunstâncias eu nunca mais
vivenciaria, meu amigo, o filósofo Ernst Bloch, saiu com a frase de que
não haveria ninguém que, em algum momento de sua vida, não tivesse
chegado a um fio de cabelo da oportunidade de se transformar em
milionário. Todo mundo riu. Achamos que fosse um dos seus paradoxos. Mas
aí aconteceu uma coisa curiosa: quanto mais tempo dedicamos a essa
afirmação, tanto mais interesse tivemos em debatê-la, para ver como um
depois do outro se tornou pensativo quando se lembrou do momento de sua
vida em que quase tocou nesse milhão. Dentre várias histórias notáveis
que vieram à tona encontra-se, portanto, a de Scherlinger, hoje
desaparecido. Na medida do possível, vou reproduzi-la com suas próprias
palavras.

``Quando, após a morte do meu pai,'' ele começou, ``herdei um patrimônio
nada desprezível, precipitei a minha partida para a França. Fiquei feliz
principalmente pelo fato de conhecer, ainda antes do fim dos anos 20, a
cidade natal de Monticelli, a quem devo tudo na minha arte, sem falar de
outras coisas que Marselha representava para mim. Deixei a minha herança
no pequeno banco particular que, durante décadas, havia atendido meu pai
a contento. O chefe júnior não chegou a ser meu amigo, mas eu tinha um
excelente contato com ele. Assim, o mesmo garantiu que, durante a minha
longa ausência, daria uma atenção especial ao meu patrimônio e que me
informaria imediatamente sobre qualquer possibilidade favorável de
aplicação.

--- Você apenas teria que deixar um código conosco, ele concluiu.

Olhei para ele sem entender.

--- Nós só podemos executar ordens telegráficas se nos protegermos contra
qualquer abuso. Suponha que lhe mandemos uma mensagem telegráfica e o
telegrama caia nas mãos de outra pessoa. Protegemo-nos contra as
consequências combinando com você um nome secreto que sirva de
assinatura de suas ordens telegráficas.

Entendi e fiquei perplexo por um momento. Não é mesmo muito fácil
disfarçar-se com um nome falso como se fosse uma peça de roupa. Há
milhares e milhares de opções; a ideia de o nome ser totalmente
indiferente paralisa a escolha, que se torna ainda mais estática por uma
sensação -- escondida e que mal se tornou um pensamento: quão
incalculável tal escolha, e quão plena de graves consequências. Como um
jogador de xadrez, que se encontra num beco sem saída e que gostaria
muito de deixar tudo como está, mas acaba sendo obrigado a mover uma
peça, acabei dizendo `Braunschweiger'. Não conhecia ninguém com esse
nome -- aliás, nem conhecia a cidade que deu origem a ele.

Depois de um descanso de quatro semanas em Paris, cheguei num dia muito
abafado de julho na Gare Saint Louis em Marselha. Meus amigos haviam me
recomendado o hotel Regina, não muito longe do porto. Gastei apenas o
tempo necessário para me instalar, para testar o abajur da mesa de
cabeceira e as torneiras para depois dar um passeio. Como era o meu
primeiro nesta cidade, esse passeio tinha que obedecer à minha velha
regra de viajante, que consistia em explorar antes de mais nada a
periferia em sua extensão, ao contrário da maioria dos passantes que,
mal chegam, já se comprimem desajeitadamente no centro da cidade
estrangeira. Logo ficou claro para mim o quanto justamente aqui essa
regra era válida. Nunca a primeira hora trouxe-me mais ganhos do que
essa entre os portos internos e as docas, os armazéns, os bairros da
pobreza e os redutos dispersos da miséria. Sabemos que a periferia é o
estado de exceção da cidade, o terreno em que se trava ininterruptamente
a grande batalha decisiva entre a cidade e o campo. Em lugar algum ela é
mais acirrada do que entre Marselha e a paisagem da Provença. É a luta
corpo-a-corpo de postes de telefones contra agaves, arame farpado contra
palmeiras espinhosas, correntes de neblina em vielas mal cheirosas
contra a sombra úmida dos plátanos em campos ensolarados, escadarias de
pouco fôlego contra colinas imponentes. A longa Rue de Lyon é o caminho
de pólvora que Marselha cavou na paisagem para fazê-la explodir em
Saint-Lazare, Saint-Antoine, Arenc, Septèmes e cobrir com estilhaços de
granada de todas as línguas dos povos e das empresas: \emph{Alimentation
Moderne}, \emph{Rue de Jamaïque}, \emph{Comptoir de la Limite},
\emph{Savon Abat-Jour}, \emph{Minoterie de la Campagne}, \emph{Bar du
Gaz}, \emph{Bar Facultatif}. E por cima de tudo a poeira densa, composta
de sal marinho, calcário e minerais brilhantes. Depois caminhei em
direção ao cais mais externo, usado apenas pelos grandíssimos vapores
ultramarinos, sob os raios ardidos de um sol forte que estava se pondo.
Andei entre as fundações muradas da cidade velha do lado esquerdo e os
morros ou as pedreiras sem vegetação do lado direito, ao \emph{Pont
Transbordeur}, esse polígono quadrado que os fenícios isolaram do mar
como uma praça grande e que se ergue no final do porto velho. Se, mesmo
nos subúrbios mais populosos, eu havia seguido meu caminho sozinho, aqui
me sentia integrado no cortejo de marinheiros que se divertiam, de
estivadores voltando para casa e donas de casa passeando. Cortejo este
que, repleto de crianças, se movia ao longo dos bares e dos bazares,
para depois se perder nas ruas laterais, sendo que a grande artéria
principal, a avenida do comércio, da bolsa e dos estrangeiros, \emph{La
Cannebière}, só foi alcançada por alguns marujos ou \emph{flâneurs} como
eu. Através de todos os bazares, de um lado do porto até o outro, se
estende a serra dos `Souvenirs'. Forças sísmicas empilharam essa
elevação de vidros, calcário e esmalte, na qual frascos de tinta, barcos
a vapor, âncoras, colunas de mercúrio e sereias são imbricados. Para
mim, todavia, a pressão de mil atmosferas, sob a qual todo esse mundo de
imagens se comprime, ergue-se e se escalona parecia ser a mesma força
que se experimenta nas mãos duras dos marinheiros quando, depois de uma
longa jornada, tocam coxas e peitos de mulheres. E a volúpia que nas
caixinhas de conchas faz surgir do mundo de pedras um coração de veludo
vermelho ou azul para ser espetado de agulhas e broches, a mesma que,
nos dias do pagamento, sacode as ruelas.

Mergulhado nesses pensamentos eu havia, há muito tempo, deixado para
trás a \emph{Cannebière}. Sem prestar muita atenção, tinha passado por
baixo das árvores da \emph{Allée de Meilhan}, e das grades das janelas
do \emph{Cours Puget}. Isso, até que o acaso, que continuou guiando os
meus primeiros passos na cidade, levou-me à \emph{Passage de Lorette}, a
sala mortuária da cidade, no pátio estreito, onde, na companhia
sonolenta de alguns homens e mulheres, o mundo inteiro parecia ter se
encolhido numa única tarde de domingo. Algo do luto que, até hoje, amo
na luz das pinturas de Monticelli, tomou conta de mim. Creio que, nessas
horas, quando vivenciadas por um estrangeiro, alguma coisa se comunica a
ele que, normalmente, só os nativos sentem. Pois a infância é o detector
das fontes da melancolia e, para conhecer o luto de cidades tão
gloriosas, a pessoa tem que ter sido criança nelas.

Seria um belo arranjo romântico'', disse Scherlinger com um sorriso,
``se agora relatasse como, num bar mal afamado qualquer, consegui o
haxixe através de um árabe, que poderia ter sido fogueiro num cargueiro
ou também carregador. Mas esse arranjo não tem nenhuma utilidade para
mim, pois, talvez, eu me assemelhasse mais a esses árabes do que aos
estrangeiros cujos caminhos conduzem a esses bares, pelo menos no
sentido de que eu também levo comigo haxixe nas minhas viagens. Não
penso que, depois, no meu quarto, tivesse sido o desejo subalterno de
escapar à minha tristeza que fez com que, por volta das sete horas da
noite, eu fumasse o haxixe. Era antes a tentativa de me subjugar por
inteiro à mão mágica da cidade que havia me tomado silenciosamente pelo
pescoço. Como já disse, o entorpecente não era novidade para mim, mas,
seja pelas minhas depressões quase cotidianas em casa, seja a companhia
escassa ou ainda os locais inadequados, nunca havia me sentido acolhido
na comunidade dos iniciados, cujos testemunhos todos eram-me familiares,
dos \emph{Paraísos artificiais} de Baudelaire até o \emph{Lobo da
Estepe} de Hermann Hesse. Deitei na minha cama, comecei a ler e a fumar.
A janela à minha frente dava, logo abaixo, para uma das ruas escuras e
estreitas do bairro portuário, que são como a linha de corte de uma faca
no corpo da cidade. Apreciei a certeza incondicional de permanecer
escondido de centenas de milhares de pessoas dessa cidade, onde ninguém
me conhecia, sem ser incomodado e totalmente entregue aos meus
devaneios.

Mas o haxixe demorou a fazer efeito. Já haviam passado 45 minutos e
comecei a desconfiar da qualidade da droga. Ou será que a havia guardado
tempo demais comigo? De repente, alguém bateu com força na minha porta.
Nada era mais inexplicável. Levei um susto mortal, mas não tive
disposição nenhuma de abrir a porta; apenas perguntei o que era, sem
alterar minimamente a minha posição.

--- Há um senhor que quer lhe falar, disse o empregado do hotel.

--- Mande-o subir, eu disse. Não tive a presença de espírito, ou então, a
coragem de perguntar pelo nome.

Com o coração batendo, fiquei apoiado no encosto da cama, olhando fixo
para a fresta aberta da porta, até um uniforme aparecer nela. `O senhor'
era um mensageiro do telégrafo.

--- Proposta comprar 1000 Royal Dutch sexta-feira primeiro câmbio mande
de acordo.

Olhei para o relógio: eram oito horas. Um telegrama em regime de
urgência teria como chegar na manhã do dia seguinte ao escritório
berlinense do meu banco. Despedi o mensageiro com uma gorjeta. Fiquei
alternando entre a inquietação e o desprazer. Inquietação sobre o fato
de ser molestado logo naquele momento por um negócio bancário e um
caminho a ser feito; e desprazer sobre a demora em sentir qualquer
efeito da droga. Pareceu-me ser o mais indicado pegar logo o caminho do
correio central, que, pelas minhas informações, aceitava telegramas até
meia-noite. Diante da segurança com que meu conselheiro de confiança me
atendia, não havia dúvida de que eu concordaria. Entretanto, fiquei
preocupado com a ideia de que pudesse esquecer o código combinado caso o
haxixe, inesperadamente, começasse a surtir efeito. Sendo assim, era
melhor não perder tempo.

Enquanto descia a escada, lembrava da última vez em que havia fumado
haxixe -- já fazia alguns meses -- e que não tinha como matar a fome
atormentadora que havia me assaltado no quarto. Comprar uma barra de
chocolate era o mais aconselhável. De longe, uma vitrine com vidros de
balas, papel brilhante de alumínio e belos confeitos empilhados me
chamou a atenção. Entrei na loja e fiquei parado. Não havia ninguém. Mas
isso era menos surpreendente do que as poltronas bastante estranhas,
cujo aspecto me deu a entender, pelo bem ou pelo mal, que, em
Marselha, se toma chocolate em poltronas altas como tronos que se
pareciam mais com cadeiras cirúrgicas. Nesse momento, o dono da loja,
vestido com um avental branco, veio correndo do outro lado da rua e mal
tive tempo, dando altas risadas, para escapar de sua oferta de fazer a
minha barba ou de cortar meu cabelo. Só então ficou claro para mim que o
haxixe já tinha começado a fazer efeito e, se a transformação de
caixinhas de pó de arroz em vidros de balas, estojos de níquel em barras
de chocolate e perucas em rocamboles não me tivessem mostrado isso, as
minhas próprias risadas teriam sido alerta suficiente. Pois o
entorpecimento começa com essas risadas ou então com um riso mais
silencioso e interno, mas ainda mais prazeroso. E agora o reconheci
também pela ternura infinita do vento que movia, no outro lado da rua,
as franjas das marquises.

Logo depois se fizeram sentir as exigências de tempo e espaço do fumante
de haxixe. Como se sabe, tomam dimensões absolutamente reais. Para quem
consome haxixe, o palácio de Versalhes não é grande demais e a
eternidade não é demasiadamente longa. E diante dessas dimensões
gigantescas da vivência interior, da duração absoluta e do espaço
incomensurável, um humor maravilhoso acompanha, junto com aquele sorriso
feliz, tanto mais prazerosamente o caráter questionável de todo ser.
Além disso, senti uma leveza e uma determinação no passo que transformou
o chão de terra irregular e cheio de pedras da grande praça, que estava
atravessando, numa estrada mestra pela qual eu, caminhante valente,
passei durante a noite. No final dessa praça, no entanto, ergueu-se uma
construção feia na forma de um galpão simétrico, com um relógio
iluminado no frontão: o correio.

Só agora chamo essa construção de feia; na época, não teria achado isso.
Não apenas porque nada é feio quando consumimos haxixe, mas sobretudo
porque o correio desperta em mim uma sensação profunda de gratidão --
esse correio escuro que estava aguardando, me aguardando, que, em todos
os seus compartimentos e suas cavidades, estava disposto a acolher e
passar para frente a inestimável concordância que faria de mim um homem
rico. Não conseguia tirar dele meu olhar e até sentia o quanto teria
perdido se tivesse me aproximado demasiadamente, deixando de ver o todo,
sobretudo o relógio-lua brilhante.

Nesse momento, bem no lugar certo, mesas e cadeiras de um bar pequeno e,
nesse caso, com certeza mal afamado, foram colocadas na escuridão.
Embora suficientemente distante do bairro dos apaches\footnote{No
  original alemão, \emph{Apachenviertel}, no sentido já obsoleto de
  bairro submundano, frequentado por indivíduos de hábitos desregrados,
  não raro envolvidos em atividades ilegais, afeitos à vadiagem, às ruas
  e à vida noturna deste tipo de bairro nas grandes cidades. [\versal{N}. \versal{E}.]}, não havia
fregueses burgueses; na melhor das hipóteses havia, ao lado do
proletariado portuário propriamente dito, algumas famílias proprietárias
de boutiques na vizinhança. Foi nesse pequeno bar que tomei lugar.
Naquela direção, era o último que, na minha opinião, ainda era acessível
sem perigo, e que, sob o efeito do haxixe, havia avaliado com a mesma
segurança com que se consegue, em estado de profundo cansaço, encher um
copo de água até a beirada sem derramar uma única gota -- algo que não
se consegue em hipótese alguma com os sentidos despertos.

Mas, mal o haxixe sentiu meu descanso, ele começou a soltar sua magia
com uma agudeza primitiva que não havia sentido antes, nem iria mais
sentir depois disso. Pois ele fez com que me tornasse um
fisiognomonista. Logo eu, que normalmente não estou em condições de
reconhecer conhecidos distantes, de guardar traços faciais na memória,
fiquei obstinadamente atento aos rostos ao meu redor e que teria evitado
normalmente por dois motivos: não teria desejado atrair seus olhares,
nem teria suportado sua brutalidade. Entendi de repente como, para um
pintor -- não havia acontecido isso com Leonardo e muitos outros? -- a
fealdade era o verdadeiro reservatório da beleza, ou melhor, o cofre do
seu tesouro, a montanha escarpada que esconde todo o ouro da beleza que
fulgurava das rugas, dos olhares, dos traços. Lembro-me especialmente do
rosto de um homem infinitamente animalesco e vulgar, do qual me atingiu
repentinamente a ``dobra da renúncia'' de forma assustadora. Eram
principalmente os rostos dos homens que me tocavam. Começou também o
velho jogo de identificar em cada face nova algum conhecido. Muitas
vezes eu lembrava do nome, outras vezes não; a ilusão passou, como
passam as ilusões no sonho, isto é, não de forma envergonhada ou
comprometida, mas pacífica e amável, como um ser que cumpriu sua
obrigação. Meu vizinho, porém, um pequeno burguês pela sua postura,
mudou constantemente a forma, a expressão e a plenitude do seu rosto. O
corte de cabelo e os óculos de armação preta o deixaram ora severo, ora
bondoso. Disse para mim mesmo que não é possível alguém mudar com tanta
rapidez, mas não adiantou. E ele já havia passado por muitas vidas
quando, de repente, transformou-se num aluno de ginásio numa cidade
pequena da Europa oriental. Ele tinha um quarto de estudos bonito e
cultivado. Perguntei para mim mesmo: `de onde esse jovem tirou tanta
cultura? Qual será a profissão do seu pai? Comerciante de têxteis ou de
cereais?' De repente eu soube que se tratava de Myslowitz. Levantei o
olhar e, nesse momento, vi realmente, no final da praça, não, mais longe
ainda, no final da cidade, o ginásio de Myslowitz e o relógio da escola
que -- será que parou? Ele não tinha avançado -- mostrava que era pouco
depois das onze. As aulas já deviam ter começado. Mergulhei inteiramente
nesse quadro, não alcançava mais o fundo. As pessoas que, pouco tempo
atrás -- ou será que foram duas horas atrás -- haviam-me encantado
totalmente, estavam como que apagadas. `De um século para outro, as
coisas se tornam mais estranhas', pensei. Hesitei muito em tomar o
vinho. Era uma meia garrafa de Cassis, um vinho seco que havia pedido.
Um pedaço de gelo boiava no copo. Não sei quanto tempo fiquei com as
imagens que o povoavam. Mas quando olhava para a praça vi que ela tendia
a mudar com cada pessoa que chegava, como se essa pessoa representasse
uma figura que, bem entendido, não tinha nada a ver com aquela que a
olhava, mas antes com o olhar com que os grandes retratistas do século~\versal{XVII} 
colocam, de acordo com o caráter das pessoas de nobreza, essas
pessoas numa galeria de colunas ou na frente de uma janela, para
destacá-las dessa galeria ou dessa janela.

De repente, um susto me despertou do mais profundo devaneio. Tudo estava
claro em mim e só pensei numa coisa: o telegrama. Tinha que mandá-lo
imediatamente. Para me manter completamente acordado, pedi um café puro.
Aí começou uma espera de meia eternidade, até o garçom aparecer com a
xícara. Ávido pelo café, peguei-o, o odor subiu pelo meu nariz. No
entanto, quando estava próximo dos meus lábios, a minha mão parou --
para a minha própria surpresa ou de tanta surpresa, quem saberia
dizê-lo? De uma só vez entendi claramente a pressa instintiva do meu
braço e me dei conta do odor envolvente do café, só agora lembrei o que
transforma essa bebida, para qualquer fumante de haxixe no auge da
fruição: nada mais do que o fato de aumentar o efeito da droga. Por isso
quis parar, e acabei parando. A xícara não encostou na boca. Mas também
não tocou a superfície da mesa. Ela ficou flutuando na minha frente,
sustentada pelo meu braço que começou a ficar insensível, segurando-a
como um emblema, uma pedra sagrada ou um osso, rígido e morto. Meu olhar
caiu nas dobras geradas por minha calça de praia, a reconheci, dobras do
albornoz. Caiu na minha mão, a reconheci, uma mão bronzeada, etíope, e
enquanto os meus lábios continuavam rigorosamente colados um ao outro,
recusando-se igualmente à bebida e à palavra, do seu interior surgia um
sorriso, um sorriso arrogante, africano, sardanapalesco, o sorriso do
homem que está prestes a entender tudo que está por trás do curso do
mundo e dos destinos e para quem não há mais nenhum segredo nas coisas e
nos nomes. Vi a mim mesmo sentado lá, bronzeado (\emph{braun}) e em
silêncio (\emph{schweigend}). Braunschweiger. O Sésamo do nome, que
escondia no seu interior todas as riquezas, havia se aberto. Com um
sorriso de compaixão infinita tive que pensar, pela primeira vez, nos
moradores da cidade de Braunschweig, que vegetam humildemente em sua
cidadezinha da Alemanha central, sem saber das forças mágicas que
receberam junto com seu nome. Nesse momento ocorreram-me, como um coral,
solene e enfaticamente com suas badaladas da meia noite, todas as torres
das igrejas de Marselha.

Escureceu e o bar foi fechado. Fiquei vagueando pelo cais, lendo um nome
depois do outro dos barcos que lá estavam atracados. Nisso, uma alegria
incompreensível tomou conta de mim e fiquei sorrindo para o rosto de
todos os nomes de meninas da França. Marguerite, Louise, Renée, Yvonne,
Lucile -- pareceu-me que o amor que havia sido prometido a esses barcos
mediante seus nomes era maravilhoso, bonito e comovente. Ao lado do
último barco havia um banco de pedra -- `banco', disse para mim mesmo e
desaprovei o fato de ele não ter seu nome escrito em letras douradas em
fundo preto. Este foi o último pensamento claro que tive naquela noite.
O próximo foi-me trazido pelos jornais da manhã, quando acordei num
banco do cais ao sol quente do meio-dia:

`Alta sensacional da \emph{Royal Dutch}.'

Nunca'', encerrou o narrador, ``me senti tão ressoante, claro e solene
depois de uma embriaguez.''